\documentclass[12pt,t]{beamer}
\usetheme{Malmoe}
\usepackage[utf8]{inputenc}
\usepackage{amsmath}
\usepackage{amsfonts}
\usepackage{amssymb}
\usepackage{pgfpages}
\pgfpagesuselayout{resize to}[letterpaper,landscape,border shrink=5mm]
\newenvironment{amatrix}[1]{%
  \left[\begin{array}{@{}*{#1}{c}|c@{}}
}{%
  \end{array}\right]
}
%\geometry{landscape,paper=letterpaper}
\beamertemplatenavigationsymbolsempty
\date{}
\author{Math 1410 Linear Algebra}
\title{Week 2: Solving systems of equations}
%\setbeamercovered{transparent} 
%\setbeamertemplate{navigation symbols}{} 
%\logo{} 
%\institute{} 
%\date{} 
%\subject{} 
\begin{document}

\begin{frame}
\titlepage
\end{frame}

%\begin{frame}
%\tableofcontents
%\end{frame}

\begin{frame}{Systems of linear equations - Examples}
\begin{itemize}
 \item One equation, two variables: $2x-3y = 6$
 \item Two equations, two variables:
\begin{align*}
 x&-3y=4\\
3x&+y=6
\end{align*}
 \item One equation, three variables: $x+2y-3z=6$
 \item Two equations, three variables:
\begin{align*}
 -x&+y+4z=8\\
 2x&-y+z=6
\end{align*}
\item etc.
\end{itemize}
\end{frame}
\begin{frame}\frametitle{Systems of linear equations - Definitions}
 \begin{definition}
  A {\bf system} of $m$ linear equations in $n$ variables is a collection of equations of the form
\begin{align*}
 a_{11}x_1+a_{12}x_2+\cdots + a_{1n}x_n & = b_1\\
 a_{21}x_1+a_{22}x_2+\cdots + a_{2n}x_n & = b_2\\
 \vdots \hspace{60pt} \vdots \hspace{48pt} & \hspace{6pt}\vdots\\
 a_{m1}x_1+a_{m2}x_2+\cdots + a_{mn}x_n & = b_m
\end{align*}
If we can find at least one solution, the system is {\bf consistent}. If there is no solution to the system, we say that the system is {\bf inconsistent}.
 \end{definition}

\end{frame}
\begin{frame}\frametitle{2 equations, 2 variables}
 \begin{example}
  Solve the system
\begin{align*}
 3x-2y& = 4\\
 -2x+5y& = -2
\end{align*}

 \end{example}
\end{frame}
\begin{frame}\frametitle{Verifying a solution}
 Previous example: the solution to the system
\begin{align*}
    3x-2y& = 4\\
 -2x+5y& = -2                                               
\end{align*}
is $x=\dfrac{16}{11}, y = \dfrac{2}{11}$.

\end{frame}

\begin{frame}\frametitle{2 equations, 2 variables, no solution}
 \begin{example}
  Solve the system
\begin{align*}
 x-2y& = 4\\
 -2x+4y& = 0
\end{align*}

 \end{example}

\end{frame}
\begin{frame}\frametitle{2 equations, 3 variables}
 \begin{example}
  Find all solutions to the system
\begin{align*}
 2x-4y-z &= -6\\
 3x+y+2z &= 4
\end{align*}

 \end{example}

\end{frame}

\begin{frame}\frametitle{Elementary operations}
 For more than two variables/equations, a systematic approach is needed. There are three basic manipulations we use to attempt to solve a system. These are the \alert{elementary operations}.

\begin{definition}
 The {\bf elementary operations} on a system of linear equations are as follows:
\begin{enumerate}
 \item Change the places of any two equations.
 \item Multiply both sides of any equation by a (non-zero) constant.
 \item Add any multiple of one equation to another.
\end{enumerate}

\end{definition}


\end{frame}
\begin{frame}\frametitle{3 equations, 3 variables}
 \begin{example}
  Solve the system
\begin{align*}
 x +3y-2z &= 4\\
 x  -y \phantom{ + 22z} &= -2\\
 3x -4y+z  &= 0
\end{align*}

 \end{example}

\end{frame}
\begin{frame}\frametitle{The augmented matrix}
 Solving the system
\begin{align*}
 x +3y-2z &= 4\\
 x  -y \phantom{ + 22z} &= -2\\
 3x -4y+z  &= 0
\end{align*}
gets messy - there are lots of variables to keep track of. An {\bf augmented} matrix is a way to keep track of the same information, without writing down the variables. The above system is represented by the matrix
\[
 \begin{amatrix}{3}
  1&3&-2&4\\1&-1&0&-2\\3&-4&1&0
 \end{amatrix}
\]

\end{frame}
\begin{frame}\frametitle{Row operations}
 Each elementary operation for our system of equations corresponds to an {\bf elementary row operation} for the resulting augmented matrix:
\begin{itemize}
 \item Exchange the order of two equations $\rightsquigarrow$\\
\alert{Interchange two rows.}
 \item Multiply both sides of an equation by a constant $\rightsquigarrow$\\
\alert{ Multiply a row by a constant.}
 \item Add a multiple of one equation to another $\rightsquigarrow$\\
\alert{ Add a multiple of one row to another.}
\end{itemize}
Each operation produces a new matrix that represents an equivalent system of equations. If we simplify the matrix, we simplify the system of equations.
\end{frame}
\begin{frame}\frametitle{Row operations by example}
\begin{itemize}
 \item Exchange Row 1 and Row 2:
\[
 \begin{amatrix}{3}
  1&3&-2&4\\1&-1&0&-2\\3&-4&1&0
 \end{amatrix}\xrightarrow[]{R_1\leftrightarrow R_2}
\begin{amatrix}{3}
  1&-1&0&-2\\1&3&-2&4\\3&-4&1&0
 \end{amatrix}
\]
\item Multiply Row 3 by $\frac{1}{3}$:
\[
 \begin{amatrix}{3}
  1&3&-2&4\\1&-1&0&-2\\3&-4&1&0
 \end{amatrix}\xrightarrow[]{R_3\rightarrow \frac{1}{3}R_3}
\begin{amatrix}{3}
   1&3&-2&4\\1&-1&0&-2\\1&-\frac{4}{3}&\frac{1}{3}&0
 \end{amatrix}
\]
\item Add $-1$ times Row 1 to Row 2:
 \[
\begin{amatrix}{3}
  1&3&-2&4\\1&-1&0&-2\\3&-4&1&0
 \end{amatrix}\xrightarrow[]{R_2\rightarrow R_2-R_1}
\begin{amatrix}{3}
  1&3&-2&4\\0&-4&2&-6\\3&-4&1&0
 \end{amatrix}
 \]
\end{itemize}
\end{frame}
\begin{frame}\frametitle{Example Revisited 1}
 Applying elementary operations to our system
\begin{align*}
 x +3y-2z &= 4\\
 x  -y \phantom{ + 22z} &= -2\\
 3x -4y+z  &= 0
\end{align*}
gave us the simpler system
\begin{align*}
 x - y \phantom{ +22z} & = -2\\
\phantom{x } + y - z & = -6\\
\phantom{x + y} +z & = 15.
\end{align*}
Note that we can reduce further to
\begin{align*} x \phantom{+y + z}& = 7\\
 \phantom{x + } y \phantom{ +z}& = 9\\
\phantom{x + y + } z & = 15 
\end{align*}
\end{frame}
\begin{frame}\frametitle{Example Revisited 2}
 Let's work instead with the augmented matrix using row operations:
\[
  \begin{amatrix}{3}
  1&3&-2&4\\1&-1&0&-2\\3&-4&1&0
 \end{amatrix}
\]
\end{frame}
\begin{frame}\frametitle{Another Example}
 Use Gaussian elimination to solve the system
\begin{align*}
 2x +\phantom{1}y + 5z +w & = 4\\
 x+3y\phantom{+3z}-2w & = 0\\
 \phantom{4x +}2y -\phantom{2}z + w & = 1
\end{align*}

\end{frame}
\begin{frame}\frametitle{Terminology}
\begin{itemize}
 \item  An (augmented) matrix is in \alert{row echelon form} if:
\begin{enumerate}
 \item Any rows of zeros are at the \alert{bottom}
 \item The first entry in any non-zero row is a 1 (the ``leading 1'')
 \item Every leading 1 is to the \alert{right} of any leading 1s above it.
\end{enumerate}
\item An (augmented) matrix that is in row-echelon form is in \alert{reduced} row-echelon form if every leading 1 is the \alert{only} non-zero entry in its column.
\item A column containing a leading 1 is called a \alert{pivot column}. The corresponding variable is called a \alert{pivot}, or \alert{basic variable}.
\item Variables corresponding to columns without a leading 1 are called \alert{free variables}.
\end{itemize}
\end{frame}
\begin{frame}\frametitle{Examples}
 \[
  \begin{amatrix}{3}
   1 &-1& 2& 0\\
   0 & 0& 1&2
  \end{amatrix}\quad
  \begin{bmatrix}
   0 & 1 & \ast & \ast\\
   0 & 0 & 1 & \ast\\
   0 & 0 & 0 & 0
  \end{bmatrix}\quad
  \begin{amatrix}{3}
   1 & 0 & 2 & -1\\
   0 & 0 & 1 & 3 \\
   0 & 2 & -1 & 0
  \end{amatrix}
 \]

\vspace{0.5in}

\[
 \begin{bmatrix}
  0 & 0 & 0\\
  0 & 0 & 0
 \end{bmatrix}\quad
\begin{amatrix}{3}
 1 & 0 & 2 & -14\\
 0 & 1 & -5 & 3
\end{amatrix}\quad
\begin{bmatrix}
 1 & 0 & 0 & 2\\
 0 & 0 & 1 & -1\\
 0 & 0 & 0 & 0
\end{bmatrix}
\]
\end{frame}

\begin{frame}\frametitle{Gaussian elimination: Computer version}
 Gaussian elimination is an algorithm for reducing an (augmented) matrix to (reduced) row-echelon form.
\begin{enumerate}
 \item If the matrix contains only zeros, stop.
 \item If not, find the first column containing a non-zero entry. (Call this entry $a$.)
 \item Move the row containing $a$ to the top of the matrix.
 \item Multiply the row by $1/a$ to create a leading 1.
 \item By subtracting multiples of the first row from the rows below it, make every entry below the leading 1 zero.
 \item Repeat steps 1-5 for the matrix consisting of all remaining rows.
\end{enumerate}
\end{frame}
\begin{frame}\frametitle{Gaussian elimination: Human version}
 When solving by hand it's convenient to avoid introducing fractions until we have to.

\alert{Modify the algorithm:} proceed as before, but:
\begin{enumerate}
 \item Find the first non-zero column. If it has a 1 or -1, move that row to the top. (Multiply by -1 if necessary.)
 \item If none of the entries are a 1 or -1, check to see if subtracting a multiple of one row from another will create a 1.
\end{enumerate}
\begin{example}
 Given $\displaystyle \begin{amatrix}{3}-2 & 5 & 0 & -3\\7 & 0 & 2 & 1\end{amatrix}$, consider adding 3 times row 1 to row 2. This gives the equivalent matrix
\[
 \begin{amatrix}{3}
  -2 & 5 & 0 & -3\\1 & 15 & 2 & -8
 \end{amatrix}.
\]
\end{example}
\end{frame}
\begin{frame}\frametitle{Examples and Exercises}

Solve the following systems of equations:
\begin{equation}
\begin{array}{rrrrrrrrr}
 x_1 &-& 2x_2 &-& x_3 &+& 3x_4 &=& 1\\
 2x_1&-& 4x_2 &+& x_3 & & &=& 5\\
 x_1 &-& 2x_2 &+& 2x_3 &-& 3x_4 &=& 4
\end{array}
\end{equation}

\bigskip

\begin{equation}
 \begin{array}{rrrrrrr}
  x & & &+& 10z &=& 5\\
  3x &+& y &-& 4z &=& -1\\
  4x &+& y &+& 6z &=& 1
 \end{array}
\end{equation}

\bigskip

\begin{equation}
 \begin{array}{rrrrrrr}
  3x &+& 4y &+& z &=& 1\\
2x &+& 3y &+& &+& 0\\
4x &+& 3y &-& z &=& -2
 \end{array}
\end{equation}
\end{frame}

\begin{frame}\frametitle{An ``application''}
 \begin{example}
  Determine the values of $a$, $b$, and $c$ such that the parabola $y=ax^2+bx+c$ passes through the points $(-1,8)$, $(0,4)$, and $(2,2)$.
 \end{example}

\end{frame}
\begin{frame}\frametitle{More notation}
 Given a system 
\begin{align*}
 a_{11}x_1+a_{12}x_2+\cdots + a_{1n}x_n & = b_1\\
 a_{21}x_1+a_{22}x_2+\cdots + a_{2n}x_n & = b_2\\
 \vdots \hspace{60pt} \vdots \hspace{48pt} & \hspace{6pt}\vdots\\
 a_{m1}x_1+a_{m2}x_2+\cdots + a_{mn}x_n & = b_m
\end{align*}
denote
\[
 A = \begin{bmatrix}a_{11} & a_{12} & \cdots & a_{1n}\\a_{21} & a_{22} & \cdots & a_{2n}\\ \vdots & \vdots & & \vdots\\a_{m1} & a_{m2} & \cdots & a_{mn}\end{bmatrix} \text{ and } B = \begin{bmatrix}b_1\\b_2\\ \vdots \\b_m\end{bmatrix}
\]

\end{frame}
\begin{frame}\frametitle{Structure of a solution}
\begin{itemize}
 \item  With notation as on the previous page, write $(A|B)$ for the augmented matrix of our system.
 \item Use row operations to reduce $(A|B)$ to an augmented matrix $(A'|B')$ in (reduced) row echelon form.
 \item Cases:
\begin{enumerate}
 \item The matrix $(A'|B')$ has a row of the form $\begin{amatrix}{4}0&0&\cdots &0&1\end{amatrix}$.

\bigskip

 \item Every column in $A'$ contains a leading 1.

\bigskip

 \item $A$ has $n$ columns and $A'$ has $k$ leading 1s, with $k<n$.                                                                                 
\end{enumerate}

\bigskip

\bigskip 

\item Note: The number of leading 1s is equal to the number of non-zero rows.
\end{itemize}
\end{frame}
\begin{frame}\frametitle{A slightly harder problem}
 \begin{example}
  Find a condition on the numbers $a$, $b$, and $c$ such that the following system of linear equations is consistent. When that condition is satisfied, find all solutions (in terms of $a$, $b$, and $c$).
\[
 \begin{array}{rrrrrrr}
  x_1 &+& 3x_2 &+& x_3 &=& a\\
 -x_1 &-& 2x_2 &+& x_3 &=& b\\
 3x_1 &+& 7x_2 &-& x_3 &=& c
 \end{array}
\]
 \end{example}
\end{frame}
\begin{frame}\frametitle{Vector form of a solution}
 A \alert{column vector} is an object of the form $C = \begin{bmatrix}c_1\\c_2\\ \vdots \\ c_k\end{bmatrix}$. For example, we could have
\[
 A = \begin{bmatrix}1\\-3\\5\end{bmatrix}\quad B = \begin{bmatrix}b_1\\b_2\end{bmatrix} \quad \text{ or } C=\begin{bmatrix} 7\\ -5\\ 3.72\\0\end{bmatrix}.
\]
We say two column vectors are \alert{equal} if each corresponding entry is equal. Instead of writing $x_1 = 2, x_2 = -4, x_3 = 1$, we can write
\[
 \begin{bmatrix}x_1\\x_2\\x_3\end{bmatrix} = \begin{bmatrix}2\\-4\\1\end{bmatrix}.
\]

\end{frame}
\begin{frame}\frametitle{Vector form, with parameters}
 We saw earlier that the solution to
\[
 \begin{array}{rrrrrrrrr}
 x_1 &-& 2x_2 &-& x_3 &+& 3x_4 &=& 1\\
 2x_1&-& 4x_2 &+& x_3 & & &=& 5\\
 x_1 &-& 2x_2 &+& 2x_3 &-& 3x_4 &=& 4
\end{array}
\]
was $x_1 = 2+2s-t, x_2=s, x_3 = 1+2t, x_4 = t$. In vector form we write
\[
 \begin{bmatrix}x_1\\x_2\\x_3\\x_4\end{bmatrix} = \begin{bmatrix}2\\0\\1\\0\end{bmatrix}+s\begin{bmatrix}2\\1\\0\\0\end{bmatrix}+t\begin{bmatrix}-1\\0\\2\\1\end{bmatrix}
\]

\end{frame}





\end{document}