\documentclass[11pt,t]{beamer}
\usetheme{Malmoe}
\usepackage[utf8]{inputenc}
\usepackage{amsmath}
\usepackage{amsfonts}
\usepackage{amssymb}
\usepackage[all,cmtip]{xy}
\usepackage{pgfpages}
\pgfpagesuselayout{resize to}[letterpaper,landscape,border shrink=5mm]
\newenvironment{amatrix}[1]{%
  \left[\begin{array}{@{}*{#1}{c}|c@{}}
}{%
  \end{array}\right]
}
\DeclareMathOperator{\rank}{rank}
\newcommand{\R}{\mathbb{R}}
%\geometry{landscape,paper=letterpaper}
\beamertemplatenavigationsymbolsempty
\date{}
\author{Math 1410 Linear Algebra}
\title{Elementary Matrices and Inverses}
%\setbeamercovered{transparent} 
%\setbeamertemplate{navigation symbols}{} 
%\logo{} 
%\institute{} 
%\date{} 
%\subject{} 
\begin{document}
\begin{frame}
\titlepage
\end{frame}

%\begin{frame}
%\tableofcontents
%\end{frame}
\begin{frame}\frametitle{Matrix Inverses}
Recall:
\begin{itemize}
\item The \alert{inverse} of an $n\times n$ matrix $A$ satisfies
\[
AA^{-1}=A^{-1}A = I_n.
\]
\item Inverse only possible for \alert{square} matrices.
\item Inverse of $A$ exists if and only if $\rank A = n$.
\end{itemize}
Algorithm:
\begin{enumerate}
\item Begin with $[A|I_n]$.
\item Apply elementary row operations until $A$ is in RREF.
\item If $\rank A=n$, result will be $[I_n|A^{-1}]$.
\item If $\rank A<n$, there will be a row of zeros on the left, and $A^{-1}$ does not exist.
\end{enumerate}
\end{frame}
\begin{frame}
\frametitle{Example 1}
$A = \begin{bmatrix}2&4\\-3&4\end{bmatrix}$

\end{frame}
\begin{frame}
\frametitle{Example 2}
$A = \begin{bmatrix}1&3&-2\\2&0&2\\-1&-2&2\end{bmatrix}$

\end{frame}
\begin{frame}
\frametitle{Example 3}
$A = \begin{bmatrix}2&0&7\\-1&3&5\\0&12&34\end{bmatrix}$
\end{frame}
\begin{frame}
\frametitle{A system of equations}
Solve the system 
\[
\begin{array}{ccccccc}
x&+&3y&-&2z&=&3\\
2x& & &+&2z&=&-2\\
-x&-&2y&+&2z&=&4
\end{array}
\]
\end{frame}
\begin{frame}
\frametitle{Elementary matrices}
\begin{definition}
An \alert{elementary matrix} is an $n\times n$ matrix obtained from $I_n$ by a \alert{single} elementary row operation.
\end{definition}

\begin{example}

\vspace{2in}

\end{example}
\end{frame}
\begin{frame}
\frametitle{Type 1: use row operation $R_i\leftrightarrow R_j$}.
(These are also called permutation matrices)

\end{frame}
\begin{frame}
\frametitle{Type 2: use row operation $R_i\to kR_i$}
Here, $k\neq 0$ is a scalar.

\end{frame}
\begin{frame}
\frametitle{Type 3: use row operation $R_i\to R_i+kR_j$}


\end{frame}

\begin{frame}
\frametitle{Inverse of an elementary matrix}
From the above examples, we see:
\begin{quote}
\alert{Multiplying $A$ on the left by an elementary matrix has the same effect as performing the corresponding elementary row operation on $A$.}
\end{quote}
Now, given an elementary matrix $E$, can we find a matrix $F$ such that $EF=FE=I_n$? Yes, and it's another elementary matrix!

\bigskip

Reason: every elementary row operation is \alert{reversible}.
\end{frame}
\begin{frame}
\frametitle{Examples}


\end{frame}
\begin{frame}
\frametitle{Row-echelon form revisited}
Let $A$ be an $m\times n$ matrix with RREF $R$. We know we can obtain $R$ from $A$ be a series of elementary row operations; call them $RO_1, RO_2, \ldots, RO_k$ 

For each operation there is a corresponding elementary matrix $E_1, E_2, \ldots, E_k$. We get:
 \[
 \xymatrix{
 A \ar[r]^{RO_1} & A_1 \ar[r]^{RO_2} \ar@{=}[d] & A_2 \ar[r]^{RO_3} \ar@{=}[d] & \cdots \ar[r]^{RO_k}  & R \ar@{=}[d]\\
 & E_1A & E_2(E_1A) & \cdots & UA
 }
 \]
 where $U=E_kE_{k-1}\cdots E_2E_1$ is a product of elementary matrices. Note also that $A=U^{-1}R$, where
 \[
 U^{-1} = (E_kE_{k-1}\cdots E_2E_1)^{-1} = E_1^{-1}E_2^{-1}\cdots E_{k-1}^{-1}E_k^{-1}.
 \]
\end{frame}
\begin{frame}
\frametitle{Elementary matrices and inverses}
Now, suppose that $A$ is invertible. Then we know we can perform a series of elementary row operations as follows:
\[
A \to A_1 \to A_2 \to \cdots \to A_{k-1} \to A_k=I_n
\]
with corresponding elementary matrices $E_1, E_2,\ldots, E_k$. Then:
\begin{align*}
A_1 & = E_1A\\
A_2 & = E_2A_1 = (E_2E_1)A\\
A_3 & = E_3A_2 = (E_3E_2E_1)A\\
 \vdots & \phantom{abcde} \vdots\\
A_k & = E_kA_{k-1} = (E_kE_{k-1}\cdots E_2E_1)A 
\end{align*}
In other words, $(E_k\cdots E_1)A = I_n$, which suggests $A^{-1} = E_k\cdots E_1$.
\end{frame}
\begin{frame}
\frametitle{Example}
Consider $A = \begin{bmatrix}1&-3\\3&2\end{bmatrix}$
\end{frame}
\begin{frame}
\frametitle{Example}
Consider $B = \begin{bmatrix}1&0&-2\\-1&2&1\\0&3&-2\end{bmatrix}$.
\end{frame}
\begin{frame}
\frametitle{Row-echelon form, one more time}
\begin{theorem}
Let $A$ and $B$ be any matrices (not necessarily square) such that $AB=I_n$. Then the RREF of $A$ does not have a row of zeros.
\end{theorem}
\begin{proof}
Let $R$ be the RREF of $A$. Then $R=U^{-1}A$ as above. Now consider

\bigskip

$R(BU)$

\vspace{1in}

\end{proof}
\end{frame}
\begin{frame}
\frametitle{Uniqueness of inverses, again}
Recall that $A^{-1}$ must satisfy $AA^{-1}=A^{-1}A=I_n$. As long as $A$ is a \alert{square} matrix, checking one of these is enough:
\begin{theorem}
If $A$ is an $n\times n$ matrix and $B$ is an $n\times n$ matrix such that $AB=I_n$, then $BA=I_n$, and thus $A$ and $B$ are invertible, with $B=A^{-1}$.
\end{theorem}
\begin{proof}
Since $AB=I$, the RREF $R$ of $A$ does not have a row of zeros, so $R=U^{-1}A=I$

\vspace{1in}
\end{proof}
\end{frame}
\end{document}