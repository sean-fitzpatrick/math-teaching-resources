\documentclass[12pt,t]{beamer}
\usetheme{Malmoe}
\usepackage[utf8]{inputenc}
\usepackage{amsmath}
\usepackage{amsfonts}
\usepackage{amssymb}
\usepackage{pgfpages}
\pgfpagesuselayout{resize to}[letterpaper,landscape,border shrink=5mm]
\newenvironment{amatrix}[1]{%
  \left[\begin{array}{@{}*{#1}{c}|c@{}}
}{%
  \end{array}\right]
}
\DeclareMathOperator{\rank}{rank}
%\geometry{landscape,paper=letterpaper}
\beamertemplatenavigationsymbolsempty
\date{}
\author{Math 1410 Linear Algebra}
\title{Rank and Homogeneous systems}
%\setbeamercovered{transparent} 
%\setbeamertemplate{navigation symbols}{} 
%\logo{} 
%\institute{} 
%\date{} 
%\subject{} 
\begin{document}

\begin{frame}
\titlepage
\end{frame}

%\begin{frame}
%\tableofcontents
%\end{frame}

\begin{frame}\frametitle{Notation}
 Given a system 
\begin{align*}
 a_{11}x_1+a_{12}x_2+\cdots + a_{1n}x_n & = b_1\\
 a_{21}x_1+a_{22}x_2+\cdots + a_{2n}x_n & = b_2\\
 \vdots \hspace{60pt} \vdots \hspace{48pt} & \hspace{6pt}\vdots\\
 a_{m1}x_1+a_{m2}x_2+\cdots + a_{mn}x_n & = b_m
\end{align*}
denote
\[
 A = \begin{bmatrix}a_{11} & a_{12} & \cdots & a_{1n}\\a_{21} & a_{22} & \cdots & a_{2n}\\ \vdots & \vdots & & \vdots\\a_{m1} & a_{m2} & \cdots & a_{mn}\end{bmatrix},\, X = \begin{bmatrix}x_1\\x_2\\\vdots\\ x_n\end{bmatrix}, \text{ and } B = \begin{bmatrix}b_1\\b_2\\ \vdots \\b_m\end{bmatrix}
\]
\end{frame}
\begin{frame}\frametitle{Structure of a solution}
\begin{itemize}
 \item  With notation as on the previous page, write $(A|B)$ for the augmented matrix of our system.
 \item Use row operations to reduce $(A|B)$ to an augmented matrix $(A'|B')$ in (reduced) row echelon form.
 \item Cases:
\begin{enumerate}
 \item The matrix $(A'|B')$ has a row of the form $\begin{amatrix}{4}0&0&\cdots &0&1\end{amatrix}$.

\bigskip

 \item Every column in $A'$ contains a leading 1.

\bigskip

 \item $A$ has $n$ columns and $A'$ has $k$ leading 1s, with $k<n$.                                                                                 
\end{enumerate}

\bigskip

\bigskip 

\item Note: The number of leading 1s is equal to the number of non-zero rows.
\end{itemize}
\end{frame}
\begin{frame}\frametitle{Rank}
 \begin{definition}
 The \alert{rank} of a matrix $A$ is the number of leading ones in the row-echelon form of $A$.
 \end{definition}
 \begin{theorem}
 Let $(A|B)$ denote the augmented matrix of a system of $m$ linear equations in $n$ variables. Then:
 \begin{enumerate}
 \item If $\rank A <\rank (A|B)$, then the system is inconsistent.
 \item If $\rank A = \rank(A|B) = n$, then the system has a unique solution.
 \item If $\rank A =\rank(A|B)= r<n$, then the system has infinitely many solutions, with $n-r$ parameters.
 \end{enumerate}
 \end{theorem}
 Note: $\rank A\leq \min\{m,n\}$. If our system is consistent and $n>m$, then we will have infinitely many solutions.
\end{frame}
\begin{frame}\frametitle{Examples}
In each case, find $\rank A$, $\rank (A|B)$, and the solution of the corresponding system of equations:
\begin{equation}
\begin{amatrix}{3}
1 & -2 & 3 & 0\\ -2 & 0 & 1 & 4\\ 0& -8 & 14 & 8
\end{amatrix}
\end{equation}
\begin{equation}
\begin{amatrix}{4}
-2 & 8 & 4 & 0 & 6\\0 & 3 & -2 & 1 & -5\\ -1& 0 & 2 & 6 & 7\\ 0 & 0 & 1 & -1 & 4
\end{amatrix}
\end{equation}
\end{frame}
\begin{frame}\frametitle{Column vectors}
 A \alert{column vector} is an object of the form $C = \begin{bmatrix}c_1\\c_2\\ \vdots \\ c_k\end{bmatrix}$. For example, we could have
\[
 A = \begin{bmatrix}1\\-3\\5\end{bmatrix}\quad B = \begin{bmatrix}b_1\\b_2\end{bmatrix} \quad \text{ or } C=\begin{bmatrix} 7\\ -5\\ 3.72\\0\end{bmatrix}.
\]
We say two column vectors are \alert{equal} if each corresponding entry is equal. Instead of writing $x_1 = 2, x_2 = -4, x_3 = 1$, we can write
\[
 \begin{bmatrix}x_1\\x_2\\x_3\end{bmatrix} = \begin{bmatrix}2\\-4\\1\end{bmatrix}.
\]

\end{frame}
\begin{frame}\frametitle{Algebra of column vectors}
We allow two operations on column vectors: \alert{addition}, and \alert{scalar multiplication}. 
\[
\text{Addition: } X+Y = 
\begin{bmatrix}
x_1\\x_2\\ \vdots \\ x_n
\end{bmatrix}
+\begin{bmatrix}
y_1\\y_2\\ \vdots\\y_n
\end{bmatrix}
=\begin{bmatrix}
x_1+y_2\\x_2+y_2\\ \vdots\\x_n+y_n
\end{bmatrix}
\]

\[
\text{Scalar multiplication: } cX = 
c\begin{bmatrix}
x_1\\x_2\\  \vdots\\x_n
\end{bmatrix}
=\begin{bmatrix}
cx_1\\cx_2\\ \vdots \\cx_n
\end{bmatrix}
\]
\end{frame}
\begin{frame}\frametitle{Examples}

\end{frame}
\begin{frame}\frametitle{Vector form of a system}
Given a system of $m$ equations in $n$ variables, let $A$, $X$, and $B$ be as before. Let
\[
A_1 = \begin{bmatrix}
a_{11}\\a_{21}\\\vdots \\a_{m1}
\end{bmatrix}, A_2 = \begin{bmatrix}
a_{12}\\a_{22}\\\vdots\\a_{m2}
\end{bmatrix}, \ldots, A_n = \begin{bmatrix}
a_{1n}\\a_{2n}\\\vdots \\a_{mn}
\end{bmatrix}
\]
denote the columns of $A$. If we define $AX = x_1A_1+x_2A_2+\cdots +x_nA_n$, then we can write our system compactly as
\[
AX=B.
\]
\end{frame}
\begin{frame}\frametitle{Example}
Re-write the system of equations as a single matrix equation:
\[
 \begin{array}{rrrrrrrrr}
 x_1 &-& 2x_2 &-& x_3 &+& 3x_4 &=& 1\\
 2x_1&-& 4x_2 &+& x_3 & & &=& 5\\
 x_1 &-& 2x_2 &+& 2x_3 &-& 3x_4 &=& 4
\end{array}
\]
\end{frame}
\begin{frame}\frametitle{Solutions in vector form}
Solving our system: row-reducing gives
\[
 \begin{amatrix}{4}
  1&-2&-1&3&1\\2&-4&1&0&5\\1&-2&2&-3&4
 \end{amatrix}\to \cdots \to
 \begin{amatrix}{4}
  1&-2&0&1&4\\0&0&1&-2&1\\0&0&0&0&0
 \end{amatrix}.
\]
Leading variables: \\
Parameters:\\
Solution: \\
Vector form:
\[
 X = \begin{bmatrix}\hspace{3in}
\]
\end{frame}
\begin{frame}\frametitle{Homogeneous systems}
A homogeneous system of equations is a system of the form
\begin{align*}
a_{11}x_1+a_{12}x_2+\cdots + a_{1n}x_n & = 0\\
 a_{21}x_1+a_{22}x_2+\cdots + a_{2n}x_n & = 0\\
 \vdots \hspace{60pt} \vdots \hspace{48pt} & \hspace{6pt}\vdots\\
 a_{m1}x_1+a_{m2}x_2+\cdots + a_{mn}x_n & = 0
 \end{align*}
That is, all the constants $b_i$ on the right-hand side are zero.\\
Matrix form: $AX=0$.\\
Note: homogeneous systems are always consistent. (Why?)

\bigskip

Homogeneous system: relationship among the variables.\\
\hspace{1in}$\Rightarrow$ unique solution not desirable.

\end{frame}
\begin{frame}\frametitle{Example}
Solve the homogeneous system
\[
\begin{array}{rrrrrrrrr}
x&+&3y&-&2z&+&w &=& 0\\
2x&-&y&+&5z & & &=&0\\
& &14y&-&2z&+&w &=&0
\end{array}
\]
\end{frame}
\begin{frame}\frametitle{Linear combinations}
\begin{definition}
We say that a column vector $Y$ is a \alert{linear combination} of column vectors $X_1, X_2, \ldots, X_k$ if $Y$ can be written in the form
\[
Y = c_1X_1 + c_2X_2 + \cdots c_kX_k
\]
for constants $c_1, c_2, \ldots, c_k$.
\end{definition}
Given a homogeneous system $AX=0$ in $n$ variables, with $\rank A = r$, we will have parameters $t_1, t_2, \ldots, t_k$, where $k=n-r$. The general solution is then of the form
\[
X_h = t_1X_1+t_2X_2+\cdots + t_kX_k.
\]
{\em Note:} in Lyryx the vectors $X_1,\ldots X_k$ are referred to as \alert{basic solutions}.
\end{frame}
\begin{frame}\frametitle{General solution - nonhomogeneous case}
Let's return to the case of a general system $AX=B$. Suppose:
\begin{enumerate}
\item The system is consistent.
\item We have the general solution $X_h$ to the homogeneous system $AX=0$.
\item We have a \alert{particular} solution (no parameters) $X_p$ to $AX=B$.
\end{enumerate}
Then the general solution to $AX=B$ is given by $X=X_h+X_p$.
\end{frame}
\begin{frame}\frametitle{Example}
Let's return to an earlier example: the system
\[
 \begin{array}{rrrrrrrrr}
 x_1 &-& 2x_2 &-& x_3 &+& 3x_4 &=& 1\\
 2x_1&-& 4x_2 &+& x_3 & & &=& 5\\
 x_1 &-& 2x_2 &+& 2x_3 &-& 3x_4 &=& 4
\end{array}
\]
\end{frame}
\end{document}