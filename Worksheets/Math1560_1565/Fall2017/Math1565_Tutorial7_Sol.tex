\documentclass[12pt]{article}
\usepackage{amsmath}
\usepackage{amssymb}
\usepackage[letterpaper,top=0.8in,bottom=1in,left=0.75in,right=0.75in,centering]{geometry}
%\usepackage{fancyhdr}
\usepackage{enumerate}
%\usepackage{lastpage}
\usepackage{multicol}
\usepackage{graphicx}

\reversemarginpar

%\pagestyle{fancy}
%\cfoot{}
%\lhead{Math 1560}\chead{Test \# 1}\rhead{May 18th, 2017}
%\rfoot{Total: 10 points}
%\chead{{\bf Name:}}
\newcommand{\points}[1]{\marginpar{\hspace{24pt}[#1]}}
\newcommand{\skipline}{\vspace{12pt}}
%\renewcommand{\headrulewidth}{0in}
\headheight 30pt

\newcommand{\di}{\displaystyle}
\newcommand{\abs}[1]{\lvert #1\rvert}
\newcommand{\len}[1]{\lVert #1\rVert}
\renewcommand{\i}{\mathbf{i}}
\renewcommand{\j}{\mathbf{j}}
\renewcommand{\k}{\mathbf{k}}
\newcommand{\R}{\mathbb{R}}
\newcommand{\aaa}{\mathbf{a}}
\newcommand{\bbb}{\mathbf{b}}
\newcommand{\ccc}{\mathbf{c}}
\newcommand{\dotp}{\boldsymbol{\cdot}}
\newcommand{\bbm}{\begin{bmatrix}}
\newcommand{\ebm}{\end{bmatrix}}                   
                  
\begin{document}


\author{Instructor: Sean Fitzpatrick}
\thispagestyle{empty}

\begin{center}
\emph{University of Lethbridge}\\
Department of Mathematics and Computer Science\\
{\bf MATH 1565 - Tutorial \#7 Solutions}\\
\end{center}
%\skipline \skipline \skipline \noindent \skipline
%Last Name:\underline{\hspace{350pt}}\\
%\skipline
%First Name:\underline{\hspace{348pt}}\\
%\skipline
%Student Number:\underline{\hspace{322pt}}\\
%\skipline
\begin{enumerate}
  \item Find the equation of the tangent line at the point $(1,1)$ for the curve \points{5}
  \[
  (x^2+y^2)^2=4xy.
  \]
 
 \medskip
 
 \textbf{Solution:} Taking the derivative of both sides with respect to $x$, we obtain:
 \[
 2(x^2+y^2)(2x+2y\frac{dy}{dx})=4y+4x\frac{dy}{dx}
 \]
 Since we are only interested in the value of $\frac{dy}{dx}$ at the point $(1,1)$, we set $x=1$ and $y=1$ in the above, obtaining
 \[
 4(2+2\frac{dy}{dx})=4+4\frac{dy}{dx}.
 \]
 Solving for $\frac{dy}{dx}$, we find that our slope is 
 \[
 m=\frac{4-8}{8-4}=-1.
 \]
 The equation of the line is therefore $y-1=-1(x-1)$, or $y=-x+2$.
 
 \medskip
 
 If you solved first for $\frac{dy}{dx}$ in terms of $x$ and $y$, you should have found (after cancelling a lot of 4s)
 \[
 \frac{dy}{dx} = \frac{y-x^3-xy^2}{y^3+x^2y-x}.
 \]
 Putting $x=1$ and $y=1$ at this stage still returns a slope of $m=-1$.
 
 \medskip
 
 
 
 \item The function $f(x)=\dfrac{1}{x^2+1}$, with domain $[0,\infty)$, is one-to-one. \points{2} Compute the value of $(f^{-1})'(1/2)$.
 
 Hint: It is not necessary to find $f^{-1}(x)$. Note that $f(1)=1/2$.
 
\medskip

\textbf{Solution:} We use the formula $(f^{-1})'(x) = \dfrac{1}{f'(f^{-1}(x))}$, with $f(x) = \dfrac{1}{x^2+1} = (x^2+1)^{-1}$.

We find 
\[
f'(x) = -1(x^2+1)^{-2}(2x) = \frac{-2x}{(x^2+1)^2}.
\]
Since $f(1) = \frac{1}{2}$, we have $f^{-1}(1/2)=1$, and thus
\[
f'(f^{-1}(1/2)) = f'(1) = \frac{-2(1)}{(1^2+1)^2} = -\frac{1}{2}.
\]
We conclude that $(f^{-1})'(1/2) = \dfrac{1}{f'(1)} = -2$.

\bigskip
 
 \item Compute the derivative of $f(x) = \tan^{-1}(x^3)$, and $g(x) = \cosh^{-1}(x)$. (For $g(x)$, see handout.) \points{3}
 
 \bigskip
 
 For the derivative of $f(x)$, we use the known result $\dfrac{d}{du}(\tan^{-1}(u)) = \dfrac{1}{1+u^2}$ and the Chain Rule to obtain
 \[
 f'(x) = \frac{1}{1+(x^3)^2}\frac{d}{dx}(x^3) = \frac{3x^2}{1+x^6}.
 \]
 
 For the derivative of $g(x)$, there are two possible approaches. For the first approach, we let $y=g(x) = \cosh^{-1}(x)$, so that $\cosh(y)=x$. Taking the derivative of both sides of the equation $\cosh(y)=x$ with respect to $x$, we obtain
 \[
 \sinh(y)\frac{dy}{dx}  = 1, \quad \text{ so } \quad g'(x) = \frac{dy}{dx} = \frac{1}{\sinh(y)}.
 \]
 To express $g'(x)$ in terms of $x$, we note that
 \[
 \cosh^2(y)-\sinh^2(y) = 1,
 \]
 so $\sinh^2(y) = \cosh^2(y)-1 = x^2-1$. This gives us $\sinh(y) = \sqrt{x^2-1}$, so we find
 \[
 g'(x) = \frac{1}{\sqrt{x^2-1}}.
 \]
 
 Note: when we define $g(x) = \cosh^{-1}(x)$, we start with the function $h(x) = \cosh(x)$, restricted to $x\geq 0$, since $\cosh$ is not one-to-one if we also allow $x<0$. This gives us a domain of $[0,\infty)$ and range of $[1,\infty)$ for $h$, so the domain of $g(x)$ is $[1,\infty)$, and the range is $[0,\infty)$.

Since $y=\cosh^{-1}(x)\geq 0$, we have $\sinh(y)\geq 0$, since $\sinh(t)\geq 0$ when $t\geq 0$. This is why we can take the positive square root above.

\bigskip

The alternative approach is to first derive (or look up) an explicit expression for $\cosh^{-1}(x)$ in terms of known functions.

If $y=\cosh{^-1}(x)$, then $x = \cosh(y) = \dfrac{e^y+e^{-y}}{2}$.

Multiplying this equation by $2e^y$ and rearranging, we obtain
\[
(e^y)^2-2xe^y+1=0,
\]
which is quadratic in $e^y$. The quadratic formula gives us
\[
e^y = \frac{2x\pm\sqrt{4x^2-4}}{2} = x\pm \sqrt{x^2-1}.
\]
Now, recall that given the domain and range of $g(x)$, we have $x\geq 1$ and $y\geq 0$, so $e^y\geq 1$. It's easy to check that the negative square root gives values between 0 and 1, so we take the positive square root, and then solve for $y$, giving us
\[
y=g(x) = \ln(x+\sqrt{x^2-1}).
\]
We can then find $g'(x)$ using the chain rule:
\begin{align*}
g'(x) & = \frac{1}{x+\sqrt{x^2-1}}\left(1+\frac{1}{2}(x^2-1)^{-1/2}(2x)\right)\\
 & = \frac{1}{x+\sqrt{x^2-1}}\left(1+\frac{x}{\sqrt{x^2-1}}\right)\\
 & = \frac{1}{x+\sqrt{x^2-1}}\left(\frac{\sqrt{x^2-1}+x}{\sqrt{x^2-1}}\right)\\
 & = \frac{1}{\sqrt{x^2-1}},
\end{align*}
as before.
 
\end{enumerate}
\end{document}