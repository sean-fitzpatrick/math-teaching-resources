\documentclass[12pt]{article}
\usepackage{amsmath}
\usepackage{amssymb}
\usepackage[letterpaper,top=0.85in,bottom=1in,left=0.75in,right=0.75in,centering]{geometry}
%\usepackage{fancyhdr}
\usepackage{enumerate}
%\usepackage{lastpage}
\usepackage{multicol}
\usepackage{graphicx}

\reversemarginpar

%\pagestyle{fancy}
%\cfoot{}
%\lhead{Math 1560}\chead{Test \# 1}\rhead{May 18th, 2017}
%\rfoot{Total: 10 points}
%\chead{{\bf Name:}}
\newcommand{\points}[1]{\marginpar{\hspace{24pt}[#1]}}
\newcommand{\skipline}{\vspace{12pt}}
%\renewcommand{\headrulewidth}{0in}
\headheight 30pt

\newcommand{\di}{\displaystyle}
\newcommand{\abs}[1]{\lvert #1\rvert}
\newcommand{\len}[1]{\lVert #1\rVert}
\renewcommand{\i}{\mathbf{i}}
\renewcommand{\j}{\mathbf{j}}
\renewcommand{\k}{\mathbf{k}}
\newcommand{\R}{\mathbb{R}}
\newcommand{\aaa}{\mathbf{a}}
\newcommand{\bbb}{\mathbf{b}}
\newcommand{\ccc}{\mathbf{c}}
\newcommand{\dotp}{\boldsymbol{\cdot}}
\newcommand{\bbm}{\begin{bmatrix}}
\newcommand{\ebm}{\end{bmatrix}}                   
                  
\begin{document}


\author{Instructor: Sean Fitzpatrick}
\thispagestyle{empty}
\vglue1cm
\begin{center}
\emph{University of Lethbridge}\\
Department of Mathematics and Computer Science\\
{\bf MATH 1565 - Tutorial \#11}\\
\end{center}
%\skipline \skipline \skipline \noindent \skipline
%Last Name:\underline{\hspace{350pt}}\\
%\skipline
%First Name:\underline{\hspace{348pt}}\\
%\skipline
%Student Number:\underline{\hspace{322pt}}\\
%\skipline



\vspace{0.1in}

\vspace*{\fill}

\begin{quote}
Print your name and student number clearly in the space above. 

\medskip

Complete the problems on the back of this page to the best of your ability. If there is a problem you especially desire feedback on, please indicate this. 

\medskip

It is recommended that you work out the details on scrap paper before writing your solutions on this page.
\end{quote}
\newpage
%\thispagestyle{empty}

\begin{enumerate}
  \item Calculate the following Taylor polynomials:
  \begin{enumerate}
  \item For $f(x)=e^{x^2}$, degree 4, about $x=0$. \points{4}
  
  \vspace{2in}
  
  \item For $g(u)=e^u$, degree 2, about $u=0$. \points{2}
  
  \vspace{1in}
  
  (What happens if you put $u=x^2$ in your answer for part (b)?)
  
  \bigskip
  
  \bigskip
  
  \end{enumerate}
  \item Calculate the following antiderivatives:
  \begin{enumerate}
  \item The antiderivative $F$ of $f(x) = \dfrac{1}{1+x^2}$ such that $F(1) = \pi$. \points{3}
  
  \vspace{2in}
  
  \item $\di \int (x^3-3\sqrt{x}+4)\,dx$\points{3}
\end{enumerate}    
  \newpage
  
  \item Estimate the area under $f(x) = 4-3x^2$, for $0\leq x\leq 1$, using 3 rectangles and:
  \begin{enumerate}
  \item Left endpoints. \points{3}
  
  \vspace{2.5in}
  
  \item Right endpoints. \points {3}
  
  \vspace{2in}
  
  \end{enumerate}

\item Given that 
\[
\int_1^4 f(x)\,dx = 4, \int_1^6 f(x)\,dx = 7, \int_1^4 g(x)\,dx = -3, \text{ and } \int_4^6 g(x)\,dx = 1,
\]
compute:
\begin{enumerate}
\item $\di \int_4^6 f(x)\,dx$\points{2}

\vspace{1in}

\item $\di \int_1^6 (f(x)+g(x))\,dx$ \points{2}
\end{enumerate}
\end{enumerate}
\end{document}