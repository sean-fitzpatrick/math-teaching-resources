\documentclass[12pt]{article}
\usepackage{amsmath}
\usepackage{amssymb}
\usepackage[letterpaper,top=1in,bottom=1.25in,left=0.75in,right=0.75in,centering]{geometry}
%\usepackage{fancyhdr}
\usepackage{enumerate}
%\usepackage{lastpage}
\usepackage{multicol}
\usepackage{graphicx}

%\reversemarginpar

%\pagestyle{fancy}
%\cfoot{}
%\lhead{Math 1560}\chead{Test \# 1}\rhead{May 18th, 2017}
%\rfoot{Total: 10 points}
%\chead{{\bf Name:}}
\newcommand{\points}[1]{\marginpar{\hspace{24pt}[#1]}}
\newcommand{\skipline}{\vspace{12pt}}
%\renewcommand{\headrulewidth}{0in}
\headheight 30pt

\newcommand{\di}{\displaystyle}
\newcommand{\abs}[1]{\lvert #1\rvert}
\newcommand{\len}[1]{\lVert #1\rVert}
\renewcommand{\i}{\mathbf{i}}
\renewcommand{\j}{\mathbf{j}}
\renewcommand{\k}{\mathbf{k}}
\newcommand{\R}{\mathbb{R}}
\newcommand{\aaa}{\mathbf{a}}
\newcommand{\bbb}{\mathbf{b}}
\newcommand{\ccc}{\mathbf{c}}
\newcommand{\dotp}{\boldsymbol{\cdot}}
\newcommand{\bbm}{\begin{bmatrix}}
\newcommand{\ebm}{\end{bmatrix}}                   
                  
\begin{document}


\author{Instructor: Sean Fitzpatrick}
\thispagestyle{empty}
\begin{center}
\emph{University of Lethbridge}\\
Department of Mathematics and Computer Science\\
{\bf MATH 1565 - Tutorial \#1}\\
\end{center}
%\skipline \skipline \skipline \noindent \skipline
%Last Name:\underline{\hspace{350pt}}\\
%\skipline
%First Name:\underline{\hspace{348pt}}\\
%\skipline
%Student Number:\underline{\hspace{322pt}}\\
%\skipline



\vspace{0.1in}

\vspace*{\fill}

\begin{quote}
Print your name and student number clearly in the space above. 

\medskip

Complete the problems on the back of this page to the best of your ability. If there is a problem you especially desire feedback on, please indicate this. 

\medskip

Tutorial worksheets are graded for completeness, not correctness. Mistakes are not penalized as long as you've made progress towards the solution.
\end{quote}
\newpage
\thispagestyle{empty}
\begin{enumerate}
 \item  \textbf{Using the definition} of $\sinh(x)$, show that
 \[
 \sinh(x+y) = \sinh(x)\cosh(y)+\cosh(x)\sinh(y).
 \]
 
 \vspace{4.5in}
 
 \item Show that $\sin(\tan^{-1}(x)) = \dfrac{x}{\sqrt{1+x^2}}$.
\end{enumerate}

\end{document}