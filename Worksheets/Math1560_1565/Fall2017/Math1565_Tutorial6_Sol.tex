\documentclass[12pt]{article}
\usepackage{amsmath}
\usepackage{amssymb}
\usepackage[letterpaper,top=0.8in,bottom=1in,left=0.75in,right=0.75in,centering]{geometry}
%\usepackage{fancyhdr}
\usepackage{enumerate}
%\usepackage{lastpage}
\usepackage{multicol}
\usepackage{graphicx}

\reversemarginpar

%\pagestyle{fancy}
%\cfoot{}
%\lhead{Math 1560}\chead{Test \# 1}\rhead{May 18th, 2017}
%\rfoot{Total: 10 points}
%\chead{{\bf Name:}}
\newcommand{\points}[1]{\marginpar{\hspace{24pt}[#1]}}
\newcommand{\skipline}{\vspace{12pt}}
%\renewcommand{\headrulewidth}{0in}
\headheight 30pt

\newcommand{\di}{\displaystyle}
\newcommand{\abs}[1]{\lvert #1\rvert}
\newcommand{\len}[1]{\lVert #1\rVert}
\renewcommand{\i}{\mathbf{i}}
\renewcommand{\j}{\mathbf{j}}
\renewcommand{\k}{\mathbf{k}}
\newcommand{\R}{\mathbb{R}}
\newcommand{\aaa}{\mathbf{a}}
\newcommand{\bbb}{\mathbf{b}}
\newcommand{\ccc}{\mathbf{c}}
\newcommand{\dotp}{\boldsymbol{\cdot}}
\newcommand{\bbm}{\begin{bmatrix}}
\newcommand{\ebm}{\end{bmatrix}}                   
                  
\begin{document}
\begin{center}
\textbf{Math 1565 Tutorial \#6 Solutions}
\end{center}

\begin{enumerate}
  \item State the domain and range of $f(x) = \cos^{-1}(x)$. 
 
\bigskip

The domain of $f$ is $[-1,1]$. The range of $f$ is $[0,\pi]$.

\bigskip
 
 
 \item Evaluate the following limits:
 \begin{enumerate}
 \item $\di \lim_{x\to 4}\frac{\sqrt{x}-2}{x^2-16} = \lim_{x\to 4}\frac{(\sqrt{x}-2)(\sqrt{x}+2)}{(x-4)(x+4)(\sqrt{x}+2)} = \lim_{x\to 4}\frac{1}{(x+4)(\sqrt{x}+2)}  = \frac{1}{32}. $ 
 
 \vspace{1cm}
 
 \item $\di \lim_{\theta\to 0}\frac{\tan(5x)}{\sin(3x)} = \lim_{\theta\to 0}\frac{\sin(5x)}{5x}\frac{3x}{\sin(3x)}\frac{1}{\cos(5x)}\frac{5}{3} = (1)(1)\frac{1}{1}\cdot\frac{5}{3}=\frac{5}{3}.$
 \end{enumerate}
 
 \vspace{1cm}
 
 \item List all horizontal and vertical asymptotes for the function $f(x) = \dfrac{5x^3-4x+6}{x^3-4x}$. 

\bigskip

Since the degree of the top and bottom are the same, the horizontal asymptote can be found by comparing coefficients of top powers: we find $y=5$.

The vertical asymptotes occur where the denominator is zero. Since $x^3-4x=x(x-2)(x+2)$ we have zeros at $x=0, 2, -2$. None of these are also zeros for the numerator, so $x=0$, $x=2$, and $x=-2$ are vertical asymptotes.

\bigskip

 \item Prove that there exists a real number $x$ such that $\cos(x)=x$. (Hint: IVT)
 
 \bigskip
 
Consider $f(x)=\cos(x)-x$. This is a continuous function, since it's the sum of continuous functions. We also note that $f(0)=1-0=1>0$, while $f(\pi/2) = 0-\pi/2 = -\pi/2 <0$.

Thus, by the Intermediate Value Theorem, there must exist some $x\in (0,\pi/2)$ such that $f(x)=0$, and the result follows. 
\newpage

\item Given $f(x) = \dfrac{1}{x-3}$, determine $f'(2)$ \textbf{using the definition of the derivative}. 

\bigskip

We have
\begin{align*}
f'(2) & = \lim_{h\to 0}\frac{f(2+h)-f(2)}{h}\\
& = \lim_{h\to 0}\frac{1}{h}\left(\frac{1}{(2+h)-3}-\frac{1}{2-3}\right)\\
& = \lim_{h\to 0}\frac{1}{h}\left(\frac{1}{h-1}+1\right)\\
& = \lim_{h\to 0}\frac{1}{h}\left(\frac{1+(h-1)}{h-1}\right)\\
& = \lim_{h\to 0}\frac{1}{h-1} = -1.
\end{align*}

\bigskip

\item Evaluate the derivatives of the following functions:
\begin{enumerate}
\item $f(x) = 4x^{11}-3x^{2/3}+5\ln(x)+e^\pi$

\[
f'(x) = 44x^{10}-2x^{-1/3}+\frac{5}{x}.
\]

\item $g(x) = e^{3x}\cos(5x)$

\[
g'(x) = 3e^{3x}\cos(5x)-5e^{3x}\sin(5x).
\]

\item $h(x) = \dfrac{x^5-4x^3}{x^2}$

\medskip

Since $h(x) = x^3-4x$, we have $h'(x) = 3x^2-4$. Or you can do it the hard way using the quotient rule.

\medskip


\item $k(x) = \sec(\ln(x^5+3x^2))$

\[
k'(x) = \sec(\ln(x^5+3x^2))\tan(\ln(x^5+3x^2))\cdot \frac{5x^4+6x}{x^5+3x^2}.
\]

\end{enumerate}
\end{enumerate}
\end{document}