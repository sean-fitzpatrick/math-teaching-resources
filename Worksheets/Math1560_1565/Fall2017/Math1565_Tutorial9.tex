\documentclass[12pt]{article}
\usepackage{amsmath}
\usepackage{amssymb}
\usepackage[letterpaper,top=0.85in,bottom=1in,left=0.75in,right=0.75in,centering]{geometry}
%\usepackage{fancyhdr}
\usepackage{enumerate}
%\usepackage{lastpage}
\usepackage{multicol}
\usepackage{graphicx}

\reversemarginpar

%\pagestyle{fancy}
%\cfoot{}
%\lhead{Math 1560}\chead{Test \# 1}\rhead{May 18th, 2017}
%\rfoot{Total: 10 points}
%\chead{{\bf Name:}}
\newcommand{\points}[1]{\marginpar{\hspace{24pt}[#1]}}
\newcommand{\skipline}{\vspace{12pt}}
%\renewcommand{\headrulewidth}{0in}
\headheight 30pt

\newcommand{\di}{\displaystyle}
\newcommand{\abs}[1]{\lvert #1\rvert}
\newcommand{\len}[1]{\lVert #1\rVert}
\renewcommand{\i}{\mathbf{i}}
\renewcommand{\j}{\mathbf{j}}
\renewcommand{\k}{\mathbf{k}}
\newcommand{\R}{\mathbb{R}}
\newcommand{\aaa}{\mathbf{a}}
\newcommand{\bbb}{\mathbf{b}}
\newcommand{\ccc}{\mathbf{c}}
\newcommand{\dotp}{\boldsymbol{\cdot}}
\newcommand{\bbm}{\begin{bmatrix}}
\newcommand{\ebm}{\end{bmatrix}}                   
                  
\begin{document}


\author{Instructor: Sean Fitzpatrick}
\thispagestyle{empty}
\vglue1cm
\begin{center}
\emph{University of Lethbridge}\\
Department of Mathematics and Computer Science\\
{\bf MATH 1565 - Tutorial \#9}\\
\end{center}
%\skipline \skipline \skipline \noindent \skipline
%Last Name:\underline{\hspace{350pt}}\\
%\skipline
%First Name:\underline{\hspace{348pt}}\\
%\skipline
%Student Number:\underline{\hspace{322pt}}\\
%\skipline



\vspace{0.1in}

\vspace*{\fill}

\begin{quote}
Print your name and student number clearly in the space above. 

\medskip

Complete the problems on the back of this page to the best of your ability. If there is a problem you especially desire feedback on, please indicate this. 

\medskip

It is recommended that you work out the details on scrap paper before writing your solutions on this page.
\end{quote}
\newpage
%\thispagestyle{empty}

\begin{enumerate}
  \item Consider the function $f(x)=x^4\ln(x)$.
  \begin{enumerate}
  \item State the domain of $f$ \points{1}
  
  \vspace{1cm}
  
  \item Determine any $x$-intercepts. (If you answered (a) correctly, you'll know why I didn't ask for a $y$-intercept). \points{1}
  
  \vspace{2.5cm}
  
  \item Determine $\di \lim_{x\to 0^+}f(x)$. (Use l'Hospital's Rule.) \points{3}
  
  \vspace{4.5cm}
  
  \item Compute $f'(x)$. \points{2}
  
  \vspace{3.5cm}
  
  \item Construct a sign diagram for $f'$. On what intervals is $f$ increasing/decreasing? \points{2}
  
  \vspace{3cm}
  
  \item Classify any critical numbers found in part (e) as local maxima, minima, or neither. \points{1}
  
  \newpage
  
  \item Compute $f''(x)$.\points{2}
  
  \vspace{3.5cm}
  
  \item Construct a sign diagram for $f''$.  On what intervals is the graph of $f$ concave up/down? \points{2}
  
  \vspace{3cm}
  
  \item Determine any inflection points on the graph of $f$. \points{1}
  
  \vspace{2cm}
  
  \item Sketch the graph of $f$. Your graph should reflect your results in parts (a) - (i) above. Label any intercepts, critical points, and inflection points.\points{3}
  \end{enumerate}
\end{enumerate}
\end{document}