\documentclass[12pt]{article}
\usepackage{amsmath}
\usepackage{amssymb}
\usepackage[letterpaper,top=0.85in,bottom=1in,left=0.75in,right=0.75in,centering]{geometry}
%\usepackage{fancyhdr}
\usepackage{enumerate}
%\usepackage{lastpage}
\usepackage{multicol}
\usepackage{graphicx}

\reversemarginpar

%\pagestyle{fancy}
%\cfoot{}
%\lhead{Math 1560}\chead{Test \# 1}\rhead{May 18th, 2017}
%\rfoot{Total: 10 points}
%\chead{{\bf Name:}}
\newcommand{\points}[1]{\marginpar{\hspace{24pt}[#1]}}
\newcommand{\skipline}{\vspace{12pt}}
%\renewcommand{\headrulewidth}{0in}
\headheight 30pt

\newcommand{\di}{\displaystyle}
\newcommand{\abs}[1]{\lvert #1\rvert}
\newcommand{\len}[1]{\lVert #1\rVert}
\renewcommand{\i}{\mathbf{i}}
\renewcommand{\j}{\mathbf{j}}
\renewcommand{\k}{\mathbf{k}}
\newcommand{\R}{\mathbb{R}}
\newcommand{\aaa}{\mathbf{a}}
\newcommand{\bbb}{\mathbf{b}}
\newcommand{\ccc}{\mathbf{c}}
\newcommand{\dotp}{\boldsymbol{\cdot}}
\newcommand{\bbm}{\begin{bmatrix}}
\newcommand{\ebm}{\end{bmatrix}}                   
                  
\begin{document}


\author{Instructor: Sean Fitzpatrick}
\thispagestyle{empty}
\vglue1cm
\begin{center}
\emph{University of Lethbridge}\\
Department of Mathematics and Computer Science\\
{\bf MATH 1560 - Tutorial \#3}\\
Monday, January 29
\end{center}
\skipline \skipline \skipline \noindent \skipline

\vspace*{\fill}



Some additional practice (copy these into your notes but do not submit anything):
\begin{enumerate}
\item Compute the derivatives of the following functions using the product rule:
\begin{multicols}{3}
\begin{enumerate}
\item $f(x) = x^2\cos(x)$
\item $g(x) = \sec(x)\tan(x)$
\item $h(x) = \sqrt{x}(x^2+1)$ 
\end{enumerate}
\end{multicols}

\item Compute the derivatives of the following functions using the quotient rule:
\begin{multicols}{3}
\begin{enumerate}
\item $f(x) = \dfrac{\sin(x)}{x^2+1}$
\item $g(x) = \dfrac{x^3-2x^2+5x}{x^4+e^x}$
\item $h(x) = \dfrac{x^8+\sqrt[3]{x}}{x^3}$
\end{enumerate}
\end{multicols}
Note: Problems 1(c) and 2(c) can be done without the product or quotient rule, respectively. Do you see how?
\end{enumerate}




\newpage
%\thispagestyle{empty}

\vglue12pt

  \begin{enumerate}
    \item Let $f(x) = \dfrac{1}{\sqrt{x}}$. Compute $f'(1)$ \textbf{using the definition of the derivative}.\\
    (Note: in the definition, one can set $x=1$ at the very beginning, or at the end. Which is going to be less work?)
    
    \vspace{3in}
    
    \item Compute the derivative of $f(x) = 6\sqrt{x^3}-7\sin(x)+4e^x +\pi.$
    
    \vspace{1in}
    
    \item Compute the derivative of $f(x)=x^3(x-1)^2$ by (a) using the product rule, and (b) first multiplying everything out. Confirm that your answers agree. Which method do you prefer? (What if you need to find where $f'(x)=0$?)
    
   \newpage
    
    \item Compute the derivative of $g(x) = \dfrac{x^7-3x^5+2x}{x^3}$ by\\ (a) using the quotient rule, and (b) simplifying first. Which method do you prefer?
    
\vspace{2.5in}
    
    \item For the following functions, find all values of $x$ such that $f'(x)=0$:
    \begin{enumerate}
    \item $f(x) = x^5-15x^3$
    
    \vspace{1.75in}
    
    \item $f(x) = x^{5/3}-5x^{2/3}$
    
    \vspace{1.75in}
    
    \item $f(x) = \dfrac{1-x^2}{1+x^2}$
    \end{enumerate}
    
    
  \end{enumerate}
\end{document}