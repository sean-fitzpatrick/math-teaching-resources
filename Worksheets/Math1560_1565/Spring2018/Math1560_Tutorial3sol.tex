\documentclass[12pt]{article}
\usepackage{amsmath}
\usepackage{amssymb}
\usepackage[letterpaper,top=0.85in,bottom=1in,left=0.75in,right=0.75in,centering]{geometry}
%\usepackage{fancyhdr}
\usepackage{enumerate}
%\usepackage{lastpage}
\usepackage{multicol}
\usepackage{graphicx}

\reversemarginpar

%\pagestyle{fancy}
%\cfoot{}
%\lhead{Math 1560}\chead{Test \# 1}\rhead{May 18th, 2017}
%\rfoot{Total: 10 points}
%\chead{{\bf Name:}}
\newcommand{\points}[1]{\marginpar{\hspace{24pt}[#1]}}
\newcommand{\skipline}{\vspace{12pt}}
%\renewcommand{\headrulewidth}{0in}
\headheight 30pt

\newcommand{\di}{\displaystyle}
\newcommand{\abs}[1]{\lvert #1\rvert}
\newcommand{\len}[1]{\lVert #1\rVert}
\renewcommand{\i}{\mathbf{i}}
\renewcommand{\j}{\mathbf{j}}
\renewcommand{\k}{\mathbf{k}}
\newcommand{\R}{\mathbb{R}}
\newcommand{\aaa}{\mathbf{a}}
\newcommand{\bbb}{\mathbf{b}}
\newcommand{\ccc}{\mathbf{c}}
\newcommand{\dotp}{\boldsymbol{\cdot}}
\newcommand{\bbm}{\begin{bmatrix}}
\newcommand{\ebm}{\end{bmatrix}}                   
                  
\begin{document}


\author{Instructor: Sean Fitzpatrick}
\thispagestyle{empty}
\vglue1cm
\begin{center}

{\bf MATH 1560 - Tutorial \#3 Solutions}
\end{center}

\textbf{Additional practice problems:}
\begin{enumerate}
\item Compute the derivatives of the following functions using the product rule:
\begin{enumerate}
\item $f(x) = x^2\cos(x)$, $f'(x)=2x\cos(x)-x^2\sin(x)$

\item $g(x) = \sec(x)\tan(x)$, $g'(x) = (\sec(x)\tan(x))\tan(x)+\sec(x)(\sec^2(x))=\sec(x)\tan^2(x)+\sec^3(x)$.
\item $h(x) = \sqrt{x}(x^2+1)$, $h'(x) = \frac12 x^{-1/2}(x^2+1)+\sqrt{x}(2x)$.

Alternatively, since $h(x) = x^{1/2}(x^2+1)=x^{5/2}+x^{1/2}$, we have $h'(x) = \frac52 x^{3/2}+\frac12 x^{-1/2}$. These answers agree, since 
\[
\frac12 x^{-1/2}(x^2+1)+\sqrt{x}(2x) = \frac12 x^{3/2}+\frac12 x^{-1/2}+2x^{3/2} = \frac52 x^{3/2}+\frac12 x^{-1/2}.
\]
\end{enumerate}


\item Compute the derivatives of the following functions using the quotient rule:
\begin{enumerate}
\item $f(x) = \dfrac{\sin(x)}{x^2+1}$, $f'(x) = \dfrac{\cos(x)(x^2+1)-2x\sin(x)}{(x^2+1)^2}$.
\item $g(x) = \dfrac{x^3-2x^2+5x}{x^4+e^x}$, $g'(x) = \dfrac{(3x^2-4x+5)(x^4+e^x)-(x^3-2x^2+5x)(4x^3+e^x)}{(x^4+e^x)^2}$.
\item $h(x) = \dfrac{x^8+\sqrt[3]{x}}{x^3}$, $h'(x) = \frac{(8x^7+\frac13 x^{-2/3})(x^3)-3x^2(x^8+\sqrt[3]{x})}{x^6}$.

Alternatively, we can divide both terms in the numerator by $x^3$, giving us $h(x) = x^5+x^{-8/3}$ (note $\sqrt[3]{x}=x^{1/3}$ and $1/3-3=1/3-9/3=-8/3$), so $h'(x)=5x^4-\frac83 x^{-11/3}$. I'll leave it as an exercise to confirm that result is equivalent to the one above.
\end{enumerate}

\end{enumerate}




\newpage
%\thispagestyle{empty}

\textbf{Assigned problems}

  \begin{enumerate}
    \item Let $f(x) = \dfrac{1}{\sqrt{x}}$. Compute $f'(1)$ \textbf{using the definition of the derivative}.\\
    (Note: in the definition, one can set $x=1$ at the very beginning, or at the end. Which is going to be less work?)
    
\medskip

We have
\begin{align*}
f'(1) &= \lim_{h\to 0}\frac{f(1+h)-f(1)}{h} \tag{definition of the derivative}\\
& = \lim_{h\to 0}\frac{\dfrac{1}{\sqrt{1+h}}-1}{h} \tag{putting in the function; note $f(1)=1$}\\
& = \lim_{h\to 0}\frac{1}{h}\left(\frac{1-\sqrt{1+h}}{\sqrt{1+h}}\right) \tag{common denominator}\\
& = \lim_{h\to 0}\frac{1-(1+h)}{h\sqrt{1+h}(1+\sqrt{1+h})} \tag{rationalizing the numerator}\\
& = \lim_{h\to 0}\frac{-h}{h\sqrt{1+h}(1+\sqrt{1+h})} \tag{simplifying the numerator}\\
& = \lim_{h\to 0}\frac{-1}{\sqrt{1+h}(1+\sqrt{1+h})} \tag{cancelling $h$ top and bottom}\\
& = \frac{-1}{\sqrt{1}(1+\sqrt{1})} = -\frac12 
\end{align*}

If you decided to do everything in terms of $x$ (perhaps because it would be good practice in case a future question asked for $f'(x)$ instead of $f'(1)$, your work would look like the following:
\begin{align*}
f'(x) & = \lim_{h\to 0}\frac{f(x+h)-f(x)}{h} = \lim_{h\to 0}\frac{\dfrac{1}{\sqrt{x+h}}-\dfrac{1}{\sqrt{x}}}{h}\\[5pt]
& = \lim_{h\to 0}\frac{1}{h}\left(\frac{\sqrt{x}-\sqrt{x+h}}{\sqrt{x}\sqrt{x+h}}\right)\\[5pt]
& = \lim_{h\to 0}\frac{(\sqrt{x}-\sqrt{x+h})(\sqrt{x}+\sqrt{x+h})}{h\sqrt{x}\sqrt{x+h}(\sqrt{x}+\sqrt{x+h})}\\[5pt]
& = \lim_{h\to 0}\frac{x-(x+h)}{h\sqrt{x}\sqrt{x+h}(\sqrt{x}+\sqrt{x+h})}\\[5pt]
& = \lim_{h\to 0}\frac{-h}{h\sqrt{x}\sqrt{x+h}(\sqrt{x}+\sqrt{x+h})}\\[5pt]
& = \lim_{h\to 0}\frac{-1}{\sqrt{x}\sqrt{x+h}(\sqrt{x}+\sqrt{x+h})}\\[5pt]
& = \frac{-1}{\sqrt{x}\sqrt{x}(\sqrt{x}+\sqrt{x})}=-\frac{1}{2(\sqrt{x})^3}.
\end{align*}
If we now wanted to know the value of $f'(1)$, we would have $f'(1)=-\dfrac{1}{2(\sqrt{1})^3}=-\dfrac12$.
    
    \item Compute the derivative of $f(x) = 6\sqrt{x^3}-7\sin(x)+4e^x +\pi.$
    Noting that $\sqrt{x^3} = x^{3/2}$, we have
    \[
    f'(x) = 9x^{1/2}-7\cos(x)+4e^x.
    \]
    
    
    
    \item Compute the derivative of $f(x)=x^3(x-1)^2$ by (a) using the product rule, and (b) first multiplying everything out. Confirm that your answers agree. Which method do you prefer? (What if you need to find where $f'(x)=0$?)
    
   \begin{enumerate}
   \item Using the product rule,
   \[
   f'(x) = 3x^2(x-1)^2+2x^3(x-1).
   \]
   \item Multiplying things out,
   \[
   f(x) = x^3(x^2-2x+1)=x^5-2x^4+x^3, \text{ so } f'(x) = 5x^4-8x^3+3x^2.
   \]
   Both methods take about the same amount of work. However, if we want to solve the equation $f'(x)=0$, we need to first factor $f'(x)$. In the first case, we can easily identify the common factors $x^2$ and $x-1$, so
   \[
   f'(x) = x^2(x-1)(3(x-1)+2x) = x^2(x-1)(5x-3),
   \]
   and we find that $f'(x)=0$ for $x=0, 1$, or $3/5$.
   In the second case, the common factor of $x^2$ is present; factoring it out gives us $f'(x) = x^2(5x^2-8x+3)$, and it remains to factor the quadratic. Of course, we know from the above that it must factor as $5x^2-8x+3=(5x-3)(x-1)$, but that might not be so obvious otherwise.
   \end{enumerate}
    
    \item Compute the derivative of $g(x) = \dfrac{x^7-3x^5+2x}{x^3}$ by\\ (a) using the quotient rule, and (b) simplifying first. Which method do you prefer?
    
\begin{enumerate}
\item Applying the quotient rule directly,
\[
g'(x) = \dfrac{(7x^6-15x^4+2)x^3-(x^7-3x^5+2x)(3x^2)}{x^6}.
\]
\item If we first divide each term by $x^3$, we get
\[
g(x) =x^4-3x^2+2x^{-2}, \text{ so } g'(x) = 4x^3-6x-4x^{-3}.
\]
(Personally, I think I'd rather do the second approach. I'll leave it to you to simplify the answer in part (a) and confirm that it matches the one in part (b).)
\end{enumerate}
    
    \item For the following functions, find all values of $x$ such that $f'(x)=0$:
    \begin{enumerate}
    \item $f(x) = x^5-15x^3$

    We have
    \[
    f'(x) = 5x^4-45x^2=5x^2(x^2-9)=5x^2(x-3)(x+3),
    \]
    so $f'(x)=0$ for $x=0, 3$ and $-3$.
    
    \item $f(x) = x^{5/3}-5x^{2/3}$
    
    Since
    \[
    f'(x) = \frac53 x^{2/3}-\frac{10}{3}x^{-1/3} = \frac53 x^{-1/3}(x-2),
    \]
    we have $f'(x)=0$ when $x=2$. Note that $x^{-1/3}\cdot x = x^{1-1/3}=x^{2/3}$, which is how we factored out the $x^{-1/3}$ above. Don't like this factoring approach? You could also proceed as follows:
    \begin{align*}
    f'(x) &= \frac53\left(x^{2/3}-\frac{2}{x^{1/3}}\right) \tag{since $x^{-1/3}=1/x^{1/3}$}\\
    & = \frac53\left(\frac{x^{2/3}\cdot x^{1/3}-2}{x^{1/3}}\right) \tag{common denominator}\\
    & = \frac53\left(\frac{x-2}{x^{1/3}}\right) \tag{simplifying exponents}.
    \end{align*}
    
    
    \item $f(x) = \dfrac{1-x^2}{1+x^2}$
    
    Using the quotient rule and simplifying,
    \[
    f'(x) = \frac{-2x(1+x^2)-2x(1-x^2)}{(1+x^2)^2}=\frac{-4x}{(1+x^2)^2},
    \]
    so $f'(x)=0$ when $x=0$.
    \end{enumerate}
    
    
  \end{enumerate}
\end{document}