\documentclass[12pt]{article}
\usepackage{amsmath}
\usepackage{amssymb}
\usepackage[letterpaper,top=1.25in,bottom=1in,left=0.75in,right=0.75in,centering]{geometry}
%\usepackage{fancyhdr}
\usepackage{enumerate}
%\usepackage{lastpage}
\usepackage{multicol}
\usepackage{graphicx}

\reversemarginpar

%\pagestyle{fancy}
%\cfoot{}
%\lhead{Math 1560}\chead{Test \# 1}\rhead{May 18th, 2017}
%\rfoot{Total: 10 points}
%\chead{{\bf Name:}}
\newcommand{\points}[1]{\marginpar{\hspace{24pt}[#1]}}
\newcommand{\skipline}{\vspace{12pt}}
%\renewcommand{\headrulewidth}{0in}
\headheight 30pt

\newcommand{\di}{\displaystyle}
\newcommand{\abs}[1]{\lvert #1\rvert}
\newcommand{\len}[1]{\lVert #1\rVert}
\renewcommand{\i}{\mathbf{i}}
\renewcommand{\j}{\mathbf{j}}
\renewcommand{\k}{\mathbf{k}}
\newcommand{\R}{\mathbb{R}}
\newcommand{\aaa}{\mathbf{a}}
\newcommand{\bbb}{\mathbf{b}}
\newcommand{\ccc}{\mathbf{c}}
\newcommand{\dotp}{\boldsymbol{\cdot}}
\newcommand{\bbm}{\begin{bmatrix}}
\newcommand{\ebm}{\end{bmatrix}}                   
                  
\begin{document}


\author{Instructor: Sean Fitzpatrick}
\thispagestyle{empty}
\vglue1cm
\begin{center}
\emph{University of Lethbridge}\\
Department of Mathematics and Computer Science\\
{\bf MATH 1560 - Tutorial \#11}\\
Monday, March 26
\end{center}
\skipline \ \noindent \skipline

\skipline
Name:\underline{\hspace{348pt}}\\
\skipline

\vspace{4cm}

\textbf{Note:} You may do this assignment as a group, if you wish, by listing additional names under the space above, up to a maximum of 3 students per group.

\newpage
%\thispagestyle{empty}


\begin{enumerate}
\item Evaluate the definite integral:
\begin{enumerate}
\item $\di \int_0^{\pi/2}\cos(x)\,dx$

\vspace{1in}

\item $\di \int_0^2(x^3-2x+3)\,dx$

\vspace{2in}

\item $\di \int_0^3 x\sqrt{1+x}\,dx$

\vspace{2in}

\item $\di \int_0^1 x^2\sin(x^3)\,dx$
\end{enumerate}

\newpage

\item Evaluate the integral $\di \int_0^2 \abs{2x-2}\,dx$.
 
\noindent Suggestion: either use properties of integrals to simplify, or sketch the graph and evaluate by interpreting the result as an area.

\vspace{7cm}

 \item Find the area between the curves $y= 2-x^2$ and $y=x^2$.

\newpage

\item Calculate the indicated Taylor polynomial:
\begin{enumerate}
\item Degree 5, for $f(x)=\cos(x)$, about $x=\pi/3$.

\vspace{2.5in}

\item Degree 2, for $f(x)=\sec(x)$, about $x=0$.
\end{enumerate}

\vspace{2.25in}

\item Use the degree 3 Maclaruin polynomial for $f(x)=\sin(x)$ to approximate the value of $\sin(1)$.
\end{enumerate}

\end{document}