\documentclass[12pt]{article}
\usepackage{amsmath}
\usepackage{amssymb}
\usepackage[letterpaper,margin=0.85in,centering]{geometry}
\usepackage{fancyhdr}
\usepackage{enumerate}
\usepackage{lastpage}
\usepackage{multicol}
\usepackage{graphicx}

\reversemarginpar

\pagestyle{fancy}
\cfoot{}
\lhead{Math 1560}\chead{Tutorial Assignment \# 6}\rhead{June 1st, 2017}
%\rfoot{Total: 10 points}
%\chead{{\bf Name:}}
\newcommand{\points}[1]{\marginpar{\hspace{24pt}[#1]}}
\newcommand{\skipline}{\vspace{12pt}}
%\renewcommand{\headrulewidth}{0in}
\headheight 30pt

\newcommand{\di}{\displaystyle}
\newcommand{\abs}[1]{\lvert #1\rvert}
\newcommand{\len}[1]{\lVert #1\rVert}
\renewcommand{\i}{\mathbf{i}}
\renewcommand{\j}{\mathbf{j}}
\renewcommand{\k}{\mathbf{k}}
\newcommand{\R}{\mathbb{R}}
\newcommand{\aaa}{\mathbf{a}}
\newcommand{\bbb}{\mathbf{b}}
\newcommand{\ccc}{\mathbf{c}}
\newcommand{\dotp}{\boldsymbol{\cdot}}
\newcommand{\bbm}{\begin{bmatrix}}
\newcommand{\ebm}{\end{bmatrix}}                   
                  
\begin{document}
{\bf \large Name:} \hspace{2.5in} {\bf Tutorial time:}

\bigskip

\bigskip

%\author{Instructor: Sean Fitzpatrick}
\thispagestyle{fancy}
%\noindent{{\bf Name and student number:}}

 \begin{enumerate}
  \item For each of the functions below, use the sign diagram of its derivative to find and classify any critical points.
\begin{enumerate}
 \item $f(x) = 2x^4-4x^2+6$

\vspace{2.5in}

 \item $g(x) = \dfrac{x^2}{1+x}$

\vspace{2.5in}

 \item $h(x) = x^{7/3}-7x^{1/3}$
\end{enumerate}

\newpage

 \item For each function below, determine the intervals on which it is increasing/decreasing and concave up/concave down.\\
(Note: these are the same functions as in the previous problem.)

\begin{enumerate}
 \item $f(x) = 2x^4-4x^2+6$

\vspace{2.5in}

 \item $g(x) = \dfrac{x^2}{1+x}$

\vspace{2.5in}

 \item $h(x) = x^{7/3}-7x^{1/3}$
\end{enumerate}


\newpage

\textbf{Bonus challenge fun problem:}

 \item Assume all functions below are differentiable on the interval $(-c,c)$ Using the Mean Value Theorem, show that:
\begin{enumerate}
 \item If $f(0)\geq 0$ and $f'(x)>0$ for all $x\in (0,c)$, then $f(x)>0$ for all $x\in (0,c)$.

\vspace{3in}

 \item If $f(0)>g(0)$ and $f'(x)>g'(x)$ for all $x\in (0,c)$, then $f(x)>g(x)$ for all $x\in (0,c)$.\\
(Hint: apply part (a) to the function $h=f-g$.)

\vspace{2in}

 \item Let $n>1$ be an integer. Prove that $(1+x)^n>1+nx$ for all $x\in(0,\infty)$.
\end{enumerate}

\vspace*{\fill}

(Note: the bonus is the fun you're having with this challenge!)
\end{enumerate}

\end{document}