\documentclass[12pt]{article}
\usepackage{amsmath}
\usepackage{amssymb}
\usepackage[letterpaper,margin=0.75in,centering]{geometry}
\usepackage{fancyhdr}
\usepackage{enumerate}
\usepackage{lastpage}
\usepackage{multicol}
\usepackage{graphicx}

\reversemarginpar

\pagestyle{fancy}
\cfoot{}
\lhead{Math 1410}\chead{Worksheet \# 12}\rhead{Tuesday, 12\textsuperscript{th} April, 2016}
%\rfoot{Total: 10 points}
%\chead{{\bf Name:}}
\newcommand{\points}[1]{\marginpar{\hspace{24pt}[#1]}}
\newcommand{\skipline}{\vspace{12pt}}
%\renewcommand{\headrulewidth}{0in}
\headheight 30pt

\newenvironment{amatrix}[1]{%
  \left[\begin{array}{@{}*{#1}{c}|c@{}}
}{%
  \end{array}\right]
}
\newenvironment{aamatrix}[1]{%
  \left[\begin{array}{@{}*{#1}{c}|*{#1}{c}@{}}
}{%
  \end{array}\right]
}

\newcommand{\di}{\displaystyle}
\newcommand{\abs}[1]{\lvert #1\rvert}
\newcommand{\len}[1]{\lVert #1\rVert}
\renewcommand{\i}{\mathbf{i}}
\renewcommand{\j}{\mathbf{j}}
\renewcommand{\k}{\mathbf{k}}
\newcommand{\R}{\mathbb{R}}
\newcommand{\aaa}{\mathbf{a}}
\newcommand{\bbb}{\mathbf{b}}
\newcommand{\ccc}{\mathbf{c}}
\newcommand{\dotp}{\boldsymbol{\cdot}}
\newcommand{\bbm}{\begin{bmatrix}}
\newcommand{\ebm}{\end{bmatrix}}       
\DeclareMathOperator{\proj}{proj}   
\newcommand{\bam}{\begin{amatrix}}
\newcommand{\eam}{\end{amatrix}}
         
                  
\begin{document}

%\author{Instructor: Sean Fitzpatrick}
\thispagestyle{fancy}
%\noindent{{\bf Name and student number:}}

 \begin{enumerate}
  \item If $z=3-2i$ and $w = -5+4i$, compute:
  \begin{enumerate}
  \item $3z$
  \item $z-2w$
  \item $2w-3z$
  \item $zw$
  \item $\overline{z}$ (The complex conjugate is defined by $\overline{x+iy}=x-iy$.)
  \item $\abs{w}$ (The complex modulus (norm) is defined by $\abs{w} = \sqrt{w\overline{w}}$.)
  \item $\dfrac{z^2}{w}$
  \end{enumerate}
 
  
 \item Solve for $z$ in the following equations:
\begin{enumerate}
\item $z+(2-3i)=-5+4i$
\item $3z-2i = (2-i)(3+4i)$
\item $2iz = 1+i$
\item $(3+2i)z -1+3i = 4+i$
\end{enumerate} 
\item Find the eigenvalues of the following matrices:
\[
A = \bbm 2&4\\-4&2\ebm \quad\quad B = \bbm 3&2+i\\2-i&7\ebm
\]
\item Verify that $\bbm 1\\i\ebm$ and $\bbm i\\1\ebm$ are eigenvectors for the matrix $A$ in the previous problem, and that $\bbm 2+i\\-1\ebm$ and $\bbm 1\\2-i\ebm$ are eigenvectors for the matrix $B$ in the previous problem.
\item (Bonus superfun challenge problem) Let $Z=\bbm 0&1\\-1&0\ebm$.
\begin{enumerate}
\item Verify that $Z$ has eigenvalues $\pm i$ and eigenvectors $\vec{v}=\bbm i\\-1\ebm$ and $\vec{w}=\bbm -1\\i\ebm$.
\item Show that $\langle \vec{v},\vec{w}\rangle = 0$, where $\langle \vec{v},\vec{w}\rangle = \vec{v}\dotp \overline{\vec{w}}$ is the complex version of the dot product. (The notation $\overline{\vec{w}}$ means take the complex conjugate of each entry in $\vec{w}$.) 
\item A matrix $U$ is called \textbf{unitary} if $U^*U=I$, where $U^*=(\overline{U})^T$ is the \textit{Hermitian conjugate} of $U$, formed by taking the transpose of the complex conjugate of $U$.

Let $U = \dfrac{1}{\sqrt{2}}\bbm i&-1\\-1&i\ebm$. (Note that the columns of $U$ are eigenvectors of $Z$.) Show that $U$ is unitary and that $U^*ZU = \bbm i&0\\0&-1\ebm$.
\item Compute $Z^{423}$.
\end{enumerate}
 \end{enumerate}
\newpage

{\large \bf Name:}

\vspace{24pt}

{\large \bf Tutorial time:}

\vspace{24pt}

Please submit \textbf{one} \textit{completed} solution from the worksheet for feedback. 

\textbf{Note:} Your solution needs to contain enough detail for it to be clear what problem you're trying to solve!

\end{document}