\documentclass[12pt]{article}
\usepackage{amsmath}
\usepackage{amssymb}
\usepackage[letterpaper,margin=0.85in,centering]{geometry}
\usepackage{fancyhdr}
\usepackage{enumerate}
\usepackage{lastpage}
\usepackage{multicol}
\usepackage{graphicx}

\reversemarginpar

\pagestyle{fancy}
\cfoot{}
\lhead{Math 1410}\chead{Worksheet \# 6}\rhead{Tuesday, 23\textsuperscript{rd} February, 2016}
%\rfoot{Total: 10 points}
%\chead{{\bf Name:}}
\newcommand{\points}[1]{\marginpar{\hspace{24pt}[#1]}}
\newcommand{\skipline}{\vspace{12pt}}
%\renewcommand{\headrulewidth}{0in}
\headheight 30pt

\newenvironment{amatrix}[1]{%
  \left[\begin{array}{@{}*{#1}{c}|c@{}}
}{%
  \end{array}\right]
}
\newenvironment{aamatrix}[1]{%
  \left[\begin{array}{@{}*{#1}{c}|cc}
}{%
  \end{array}\right]
}

\newcommand{\di}{\displaystyle}
\newcommand{\abs}[1]{\lvert #1\rvert}
\newcommand{\len}[1]{\lVert #1\rVert}
\renewcommand{\i}{\mathbf{i}}
\renewcommand{\j}{\mathbf{j}}
\renewcommand{\k}{\mathbf{k}}
\newcommand{\R}{\mathbb{R}}
\newcommand{\aaa}{\mathbf{a}}
\newcommand{\bbb}{\mathbf{b}}
\newcommand{\ccc}{\mathbf{c}}
\newcommand{\dotp}{\boldsymbol{\cdot}}
\newcommand{\bbm}{\begin{bmatrix}}
\newcommand{\ebm}{\end{bmatrix}}       
\DeclareMathOperator{\proj}{proj}   
\newcommand{\bam}{\begin{amatrix}}
\newcommand{\eam}{\end{amatrix}}
         
                  
\begin{document}
{\bf \large Name:} \hspace{2.5in} {\bf Tutorial time:}

\bigskip

{\bf Problem you want feedback on:}

\bigskip

%\author{Instructor: Sean Fitzpatrick}
\thispagestyle{fancy}
%\noindent{{\bf Name and student number:}}
Please complete all problems below.
 \begin{enumerate}
 \item Let $A=\di \bbm 1&-2&3\\0&4&-2\ebm$, $B=\di \bbm 3&5\\-1&2\\0&-2\ebm$, and $C=\di \bbm 2&-4\\-3&6\ebm$.\\ Compute each of the following, or explain why they're not defined:
\begin{enumerate}
 \item $2A-3B^T$. ($B^T$ denotes the transpose of $B$. Ask if you don't know what that is.)

\vspace{2in}

 \item $2A-3C$.

\vspace{1.25in}

 \item $AB$

\newpage

 \item $BA$

\vspace{3.5in}

 \item $AB+C$

\vspace{2.5in}

 \item $BA+C$


\end{enumerate}
\newpage

\item Compute the inverses of the following matrices, if possible:
\begin{enumerate}
 \item $A=\di \bbm 1& 3\\-4&-2\ebm$

\vspace{3in}

 \item $B = \di \bbm 1&0&4\\0&-3&2\\2&0&9\ebm$

\end{enumerate}
\newpage

\item Solve the following systems. (Hint: use your answer from 2(a))

\begin{enumerate}
 \item \[
        \begin{array}{ccccc}
         x&+&3y&=&3\\
         -4x&-&2y&=&-7
        \end{array}
       \]


\vspace{2.5in}

\item \[
       \begin{array}{ccccc}
        x&+&3y&=&2a\\
       -4x&-&2y&=&-3b
       \end{array}
      \]

\end{enumerate}
\vspace{2.5in}

\textbf{Possibly useful note}: the question ``Can the vector $V$ be written as a linear combination of the vectors $A,B,C$?'' is equivalent to the question ``Do there exist scalars $x,y,z$ such that $xA+yB+zC=V$?'' This latter question can be turned into a system of equations in the variables $x,y,z$.

Similarly, the question ``Given the vectors $A,B,C,D$, can any one of these vectors be written as a linear combination of the others?'' is equivalent to the question, ``Do there exist scalars $w,x,y,z$, {\em not all equal to zero}, such that $wA+xB+yC+zD=0$?'' (See if you can figure out why these two questions are the same.) This latter question can be turned into a homogeneous system of equations, and the answer is ``yes'' if this system has a non-trivial solution.

 \end{enumerate}

\end{document}