\documentclass[12pt]{article}
\usepackage{amsmath}
\usepackage{amssymb}
\usepackage[letterpaper,margin=0.85in,centering]{geometry}
\usepackage{fancyhdr}
\usepackage{enumerate}
\usepackage{lastpage}
\usepackage{multicol}
\usepackage{graphicx}

\reversemarginpar

\pagestyle{fancy}
\cfoot{}
\lhead{Math 1410}\chead{Worksheet \# 1}\rhead{Tuesday, 12\textsuperscript{th} January, 2016}
%\rfoot{Total: 10 points}
%\chead{{\bf Name:}}
\newcommand{\points}[1]{\marginpar{\hspace{24pt}[#1]}}
\newcommand{\skipline}{\vspace{12pt}}
%\renewcommand{\headrulewidth}{0in}
\headheight 30pt

\newcommand{\di}{\displaystyle}
\newcommand{\abs}[1]{\lvert #1\rvert}
\newcommand{\len}[1]{\lVert #1\rVert}
\renewcommand{\i}{\mathbf{i}}
\renewcommand{\j}{\mathbf{j}}
\renewcommand{\k}{\mathbf{k}}
\newcommand{\R}{\mathbb{R}}
\newcommand{\aaa}{\mathbf{a}}
\newcommand{\bbb}{\mathbf{b}}
\newcommand{\ccc}{\mathbf{c}}
\newcommand{\dotp}{\boldsymbol{\cdot}}
\newcommand{\bbm}{\begin{bmatrix}}
\newcommand{\ebm}{\end{bmatrix}}                   
                  
\begin{document}
{\bf \Large Name:}

\bigskip

{\bf \large Tutorial time:}

\bigskip

%\author{Instructor: Sean Fitzpatrick}
\thispagestyle{fancy}
%\noindent{{\bf Name and student number:}}
Please complete all problems below, and indicate which {\bf one} problem you want feedback on.
 \begin{enumerate}
 \item  Let $P=(1,0,-2)$, $Q=(-3,2,4)$, and $R=(0,5,-1)$ be points in $\R^3$.
\begin{enumerate}
 \item Calculate the vectors $\vec{u}=\overrightarrow{PQ}$, $\vec{v}=\overrightarrow{QR}$, and $\vec{w}=\overrightarrow{PR}$.

\vspace{2in}

 \item Show that $\vec{u}+\vec{v} = \vec{w}$.

\vspace{2in}

 \item Explain, with a diagram, why your result in part (b) makes sense. (You do not have to accurately plot the points $P,Q,R$.)


\end{enumerate}
\newpage

\item Let $\vec{a} = \begin{bmatrix}1\\4\\-7\end{bmatrix}$ and $\vec{b} = \begin{bmatrix}-3\\5\\2\end{bmatrix}$.

\medskip

Find the vector $\vec{c}$ given by the linear combination $\vec{c} = 4\vec{a}-3\vec{b}$.

\vspace{3.5in}

\item Let $\vec{u} = \bbm 1\\2\ebm$ and let $\vec{v} = \bbm -2\\1\ebm$ be vectors in $\R^2$. Sketch the vectors $\vec{u}, \vec{v}$, and $3\vec{u}-2\vec{v}$.


\newpage

\item Recall that the {\em absolute value} function $\abs{x}$ is defined by
\[
 \abs{x} = \begin{cases} x, & \text{ if } x\geq 0\\ -x, & \text{ if } x<0\end{cases}.
\]
\begin{enumerate}
 \item Calculate $\abs{2}$, $\abs{3.5}$, $\abs{0}$, $\abs{-5}$, and $\abs{-7/4}$.

\vspace{1.5in}

 \item Explain in your own words what the effect of $\abs{x}$ is on a real number $x$.

\vspace{2in}

 \item Calculate $\sqrt{(2^2)}$, $\sqrt{(0)^2}$, $\sqrt{(-1)^2}$ and $\sqrt{(-2)^2}$. 

\vspace{1.5in}

 \item Explain why it's true that $\sqrt{x^2} = \abs{x}$ for any real number $x$.

\newpage

 \item Let $\vec{v} = \begin{bmatrix}x\\y\\z\end{bmatrix}$ be a vector in $\R^3$, and let $c\in \R$ be any scalar. Recall that $\len{\vec{v}}$ is defined by
\[
 \len{\vec{v}} = \sqrt{x^2+y^2+z^2}.
\]
Show that $\len{c\vec{v}} = \abs{c}\len{\vec{v}}$. How is this related to the geometric interpretation of scalar multiplication?
\end{enumerate}

 \end{enumerate}
\end{document}