\documentclass[12pt]{article}
\usepackage{amsmath}
\usepackage{amssymb}
\usepackage[letterpaper,margin=0.75in,centering]{geometry}
\usepackage{fancyhdr}
\usepackage{enumerate}
\usepackage{lastpage}
\usepackage{multicol}
\usepackage{graphicx}

\reversemarginpar

\pagestyle{fancy}
\cfoot{}
\lhead{Math 1410}\chead{Worksheet \# 7 Solutions}\rhead{Tuesday, 1\textsuperscript{st} March, 2016}
%\rfoot{Total: 10 points}
%\chead{{\bf Name:}}
\newcommand{\points}[1]{\marginpar{\hspace{24pt}[#1]}}
\newcommand{\skipline}{\vspace{12pt}}
%\renewcommand{\headrulewidth}{0in}
\headheight 30pt

\newenvironment{amatrix}[1]{%
  \left[\begin{array}{@{}*{#1}{c}|c@{}}
}{%
  \end{array}\right]
}
\newenvironment{aamatrix}[1]{%
  \left[\begin{array}{@{}*{#1}{c}|*{#1}{c}@{}}
}{%
  \end{array}\right]
}

\newcommand{\di}{\displaystyle}
\newcommand{\abs}[1]{\lvert #1\rvert}
\newcommand{\len}[1]{\lVert #1\rVert}
\renewcommand{\i}{\mathbf{i}}
\renewcommand{\j}{\mathbf{j}}
\renewcommand{\k}{\mathbf{k}}
\newcommand{\R}{\mathbb{R}}
\newcommand{\aaa}{\mathbf{a}}
\newcommand{\bbb}{\mathbf{b}}
\newcommand{\ccc}{\mathbf{c}}
\newcommand{\dotp}{\boldsymbol{\cdot}}
\newcommand{\bbm}{\begin{bmatrix}}
\newcommand{\ebm}{\end{bmatrix}}       
\DeclareMathOperator{\proj}{proj}   
\newcommand{\bam}{\begin{amatrix}}
\newcommand{\eam}{\end{amatrix}}
         
                  
\begin{document}

%\author{Instructor: Sean Fitzpatrick}
\thispagestyle{fancy}
%\noindent{{\bf Name and student number:}}

 \begin{enumerate}
 \item In each case, give an elementary matrix $E$ such that $EA=B$:
\begin{enumerate}
 \item $A=\bbm 2&1\\3&-1\ebm$, $B = \bbm 2&1\\1&-2\ebm$

$B$ is obtained from $A$ using the row operation $R_2\to R_2-R_1$, so $E = \bbm 1&0\\-1&1\ebm$.

 \item $A=\bbm -1&2\\0&1\ebm$, $B = \bbm 1&-2\\0&1\ebm$

$B$ is obtained from $A$ using the row operation $R_1\to -1R_1$, so $E=\bbm -1&0\\0&1\ebm$.

 \item $A=\bbm 1&1\\-2&2\ebm$, $B = \bbm -2&2\\1&1\ebm$

$B$ is obtained from $A$ using the row operation $R_1\leftrightarrow R_2$, so $E = \bbm 0&1\\1&0\ebm$.

 \item $A=\bbm 4&1\\3&2\ebm$, $B = \bbm 1&-1\\ 3&2\ebm$

$B$ is obtained from $A$ using the row operation $R_1\to R_1-R_2$, so $E = \bbm 1&-1\\0&1\ebm$.

 \item $A=\bbm -1&1\\1&-1\ebm$, $B=\bbm -1&1\\-1&1\ebm$

$B$ is obtained from $A$ using the row operation $R_2\to -1R_2$, so $E = \bbm 1&0\\0&-1\ebm$.
\end{enumerate}
 
\item Find an invertible matrix $U$ such that $UA$ is in row-echelon form, where $A = \bbm 3&5&0\\3&7&1\\1&2&1\ebm$. How do you know $U$ is invertible?

\medskip

We reduce $A$ to row-echelon form using elementary row operations, while simultaneously performing the same operations on the identity:
\begin{align*}
 \begin{aamatrix}{3}
  3&5&0&1&0&0\\3&7&1&0&1&0\\1&2&1&0&0&1
 \end{aamatrix} &\xrightarrow{R_1\leftrightarrow R_3}\begin{aamatrix}{3}
  1&2&1&0&0&1\\3&7&1&0&1&0\\3&5&0&1&0&0
 \end{aamatrix}\\
&\xrightarrow{R_2\to R_2-3R_1} \begin{aamatrix}{3}
                                1&2&1&0&0&1\\0&1&-2&0&1&-3\\3&5&0&1&0&0
                               \end{aamatrix}\\
&\xrightarrow{R_3\to R_3-3R_1} \begin{aamatrix}{3}
                                1&2&1&0&0&1\\0&1&-2&0&1&-3\\0&-1&-3&1&0&-3
                               \end{aamatrix}\\
&\xrightarrow{R_3\to R_3+R_2} \begin{aamatrix}{3}
                               1&2&1&0&0&1\\0&1&-2&0&1&-3\\0&0&-5&1&1&-6
                              \end{aamatrix}\\
&\xrightarrow{R_3\to -\frac{1}{5}R_3} \begin{aamatrix}{3}
                                       1&2&1&0&0&1\\0&1&-2&0&1&-3\\0&0&1&-\frac{1}{5}&-\frac{1}{5}&\frac{6}{5}
                                      \end{aamatrix}.
\end{align*}
If we represent the five row operations above by $E_1,E_2,E_3,E_4,E_5$, then the last augmented matrix above can be written as
\[
 [\begin{array}{c|c}
   E_5E_4E_3E_2E_1A & E_5E_4E_3E_2E_1
  \end{array}] = [\begin{array}{c|c}UA&U\end{array}],
\]
where $U=E_5E_4E_3E_2E_1$, and $UA$ is in row-echelon form. \\Thus, we may take $U = \bbm 0&0&1\\0&1&-3\\-\frac{1}{5}&-\frac{1}{5}&\frac{6}{5}\ebm$. We know that $U$ is invertible since it can be expressed as a product of elementary matrices, and every elementary matrix is invertible.

\medskip

\textbf{Note:} Since you were only asked to get $A$ into row-echelon form and not reduced-row echelon form, your answer may have been different from the one above. (Only the \textit{reduced} row-echelon form is unique.)




\item Write the matrix $A = \bbm 1&0&2\\0&1&1\\2&1&6\ebm$ as a product of elementary matrices.

\medskip

We begin by reducing $A$ to reduced row-echelon form, noting that it will only be possible to write $A$ as a product of elementary matrices if $A$ is invertible. (This is why it's a good idea to verify that $A^{-1}$ exists while doing this problem.) We have:
\begin{align*}
 \bbm 1&0&2\\0&1&1\\2&1&6\ebm &\xrightarrow{R_3\to R_3-2R_1} \bbm 1&0&2\\0&1&1\\0&1&2\ebm \xrightarrow{R_3\to R_3-R_2} \bbm 1&0&2\\0&1&1\\0&0&1\ebm\\
&\xrightarrow{R_1\to R_1-2R_3}\bbm 1&0&0\\0&1&1\\0&0&1\ebm \xrightarrow{R_2\to R_2-R_3} \bbm 1&0&0\\0&1&0\\0&0&1\ebm. 
\end{align*}
Since the reduced row-echelon form of $A$ is the identity matrix, we know that $A$ is invertible. Moreover, if we write
\[
 E_1 = \bbm 1&0&0\\0&1&0\\-2&0&1\ebm, E_2 = \bbm 1&0&0\\0&1&0\\0&-1&1\ebm, E_3 = \bbm 1&0&-2\\0&1&0\\0&0&1\ebm, E_4 = \bbm 1&0&0\\0&1&-1\\0&0&1\ebm
\]
for the four elementary matrices corresponding to the four row operations above, then we have
\[
 E_4(E_3(E_2(E_1A))) = (E_4E_3E_2E_1)A = I,
\]
which implies that $E_4E_3E_2E_1 = A^{-1}$, since we know $A$ is invertible, and thus $A^{-1}$ is the unique matrix such that $A^{-1}A=I$. This gives us
\[
 A = (A^{-1})^{-1} = (E_4E_3E_2E_1)^{-1} = E_1^{-1}E_2^{-1}E_3^{-1}E_4^{-1},
\]
so to write $A$ as a product of elementary matrices, we need to find the inverses of the elementary matrices above. We have
\[
 E_1^{-1} = \bbm 1&0&0\\0&1&0\\2&0&1\ebm, E_2^{-1} = \bbm 1&0&0\\0&1&0\\0&1&1\ebm, E_3^{-1} = \bbm 1&0&2\\0&1&0\\0&0&1\ebm, E_4^{-1} = \bbm 1&0&0\\0&1&1\\0&0&1\ebm.
\]
Thus,
\[
 A = \bbm 1&0&2\\0&1&1\\2&1&6\ebm = \bbm 1&0&0\\0&1&0\\2&0&1\ebm \bbm 1&0&0\\0&1&0\\0&1&1\ebm \bbm 1&0&2\\0&1&0\\0&0&1\ebm \bbm 1&0&0\\0&1&1\\0&0&1\ebm.
\]

\textbf{Note:}  Since different people may choose to do different row operations, or do them in a different order, the answer above is not unique: it depends on your choices of row operations.


\item Prove that if $A$ is invertible, so is $A^T$, and $(A^T)^{-1} = (A^{-1})^T$.

\medskip

Suppose that $A$ is invertible. Then $A^{-1}$ exists, and we know that $AA^{-1}=A^{-1}A=I$. To show that $A^T$ is invertible, we need to show that there exists some matrix $B$ such that $A^TB = I$ and $BA^T=I$. It then follows from the uniqueness of the inverse that $B=(A^T)^{-1}$. We will show that $B=(A^{-1})^T$. (Note that this choice of $B$ is defined: we are assuming that $A^{-1}$ exists, and it's always possible to take the transpose of a matrix.) Using properties of the transpose, we have
\[
 A^T(A^{-1})^T = (A^{-1}A)^T = I^T=I, \text{ and } (A^{-1})^TA^T = (AA^{-1})^T = I^T = I.
\]
Thus, it must be the case that $(A^{-1})^T = (A^T)^{-1}$.


\item Prove that if $A$ is invertible and $a\in\R$ is a scalar, with $a\neq 0$, then $aA$ is invertible.

\medskip

The argument is the same as the one above: to show that $aA$ is invertible, we need to find a matrix $B$ such that $(aA)B = I = B(aA)$. We'll show that $B = \frac{1}{a}A^{-1}$ does the job.

\medskip

Suppose that $A$ is an invertible matrix and that $a$ is a nonzero scalar. Using properties of matrix multiplication, we have
\[
 (aA)\left(\frac{1}{a}A^{-1}\right) = \left(a\cdot \frac{1}{a}\right)(AA^{-1}) = 1(I)=I,
\]
and
\[
 \left(\frac{1}{a}A^{-1}\right)(aA) = \left(\frac{1}{a}\cdot a\right)(A^{-1}A) = 1(I)=I.
\]
Thus, it must be the case that $aA$ is invertible, and $(aA)^{-1} = \frac{1}{a}A^{-1}$.

 \end{enumerate}

\end{document}