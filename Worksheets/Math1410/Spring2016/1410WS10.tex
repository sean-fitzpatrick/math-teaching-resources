\documentclass[12pt]{article}
\usepackage{amsmath}
\usepackage{amssymb}
\usepackage[letterpaper,margin=0.75in,centering]{geometry}
\usepackage{fancyhdr}
\usepackage{enumerate}
\usepackage{lastpage}
\usepackage{multicol}
\usepackage{graphicx}

\reversemarginpar

\pagestyle{fancy}
\cfoot{}
\lhead{Math 1410}\chead{Worksheet \# 10}\rhead{Tuesday, 22\textsuperscript{nd} March, 2016}
%\rfoot{Total: 10 points}
%\chead{{\bf Name:}}
\newcommand{\points}[1]{\marginpar{\hspace{24pt}[#1]}}
\newcommand{\skipline}{\vspace{12pt}}
%\renewcommand{\headrulewidth}{0in}
\headheight 30pt

\newenvironment{amatrix}[1]{%
  \left[\begin{array}{@{}*{#1}{c}|c@{}}
}{%
  \end{array}\right]
}
\newenvironment{aamatrix}[1]{%
  \left[\begin{array}{@{}*{#1}{c}|*{#1}{c}@{}}
}{%
  \end{array}\right]
}

\newcommand{\di}{\displaystyle}
\newcommand{\abs}[1]{\lvert #1\rvert}
\newcommand{\len}[1]{\lVert #1\rVert}
\renewcommand{\i}{\mathbf{i}}
\renewcommand{\j}{\mathbf{j}}
\renewcommand{\k}{\mathbf{k}}
\newcommand{\R}{\mathbb{R}}
\newcommand{\aaa}{\mathbf{a}}
\newcommand{\bbb}{\mathbf{b}}
\newcommand{\ccc}{\mathbf{c}}
\newcommand{\dotp}{\boldsymbol{\cdot}}
\newcommand{\bbm}{\begin{bmatrix}}
\newcommand{\ebm}{\end{bmatrix}}       
\DeclareMathOperator{\proj}{proj}   
\newcommand{\bam}{\begin{amatrix}}
\newcommand{\eam}{\end{amatrix}}
         
                  
\begin{document}

%\author{Instructor: Sean Fitzpatrick}
\thispagestyle{fancy}
%\noindent{{\bf Name and student number:}}
Please complete all problems below.
 \begin{enumerate}
  \item Let $S$ be the subspace of $\R^3$ spanned by the vectors
\[
 \vec{v}_1 = \bbm 1\\2\\0\ebm, \quad \vec{v}_2 = \bbm 3\\1\\-2\ebm, \quad \vec{v}_3 = \bbm -3\\4\\4\ebm.
\]
\begin{enumerate}
 \item Determine if the vectors $\vec{a} = \bbm 1\\-8\\-4\ebm$ and $\vec{b}= \bbm 4\\3\\6\ebm$ belong to $S$. If either vector does belong to $S$, write it as a linear combination of the vectors $\vec{v}_1, \vec{v}_2, \vec{v}_3$.

 \bigskip

 \item Are the vectors $\vec{v}_1, \vec{v}_2, \vec{v}_3$ linearly independent? If not, determine a basis for $S$.
\end{enumerate}

\bigskip

% \item Let $S = \operatorname{span}\{\vec{v}_1, \vec{v}_2, \ldots, \vec{v}_k\}$, where the vectors $\vec{v}_1, \vec{v}_2, \ldots, \vec{v}_k$ are linearly independent. Show that any vector $\vec{w}\in S$ can be written \textit{uniquely} as a linear combination of the $\vec{v}_j$.

%\medskip

%\textit{Hint:} Suppose that $\vec{v} = a_1\vec{v}_1+a_2\vec{v}_2+\cdots +a_k\vec{v}_k$ \textbf{and} $\vec{v} = b_1\vec{v}_1+b_2\vec{v}_2+\cdots +b_k\vec{v}_k$ for scalars $a_1,\ldots, a_k$ and $b_1,\ldots, b_k$. Use the definition of linear independence to show that $a_i=b_i$ for each $i=1,2,\ldots, k$.

 \item Let $\vec{b}_1= \bbm 1\\2\\1\\0\ebm, \vec{b}_2 =  \bbm 1\\-1\\1\\3\ebm, \vec{b}_3 =  \bbm 2\\-1\\0\\-1\ebm$, and let $\vec{u} = \bbm 5\\2\\3\\2\ebm, \vec{v} = \bbm 3\\-2\\5\\1\ebm, \vec{w} = \bbm 1\\-1\\4\\3\ebm$.
\begin{enumerate}
 \item Show that $B=\{\vec{b}_1,\vec{b}_2,\vec{b}_3\}$ is an orthogonal set of vectors.
 
\bigskip

 \item Write each the vectors $\vec{u}$, $\vec{v}$, $\vec{w}$ above as a linear combination of the vectors in $B$, or explain why it is impossible to do so.

\textit{Hint:} For any vector $\vec{a} \in S = \operatorname{span}B$, we have the ``Fourier decomposition''
\[
 \vec{a} = \left(\frac{\vec{a}\dotp \vec{b}_1}{\len{\vec{b}_1}^2}\right)\vec{b}_1+\left(\frac{\vec{a}\dotp \vec{b}_2}{\len{\vec{b}_2}^2}\right)\vec{b}_2+\left(\frac{\vec{a}\dotp \vec{b}_3}{\len{\vec{b}_3}^2}\right)\vec{b}_3.
\]
Compute the right-hand side of the above equation for each of the given vector. If the result is equal to the original vector, then you've got you're linear combination. If it isn't then that vector isn't in the span of $B$. (What you've found instead is the \textit{orthogonal projection} of that vector onto the subspace $S$.)
\end{enumerate}
\item Let $L_1, L_2$ be the lines through the origin with direction vectors $\vec{v} = \bbm 1\\-3\\1\ebm$ and $\vec{w} = \bbm 2\\1\\1\ebm$, respectively. 
\begin{enumerate}
 \item Verify that $L_1$ and $L_2$ are perpendicular.
 \item Let $S$ be the plane through the origin containing the lines $L_1$ and $L_2$. (Note that planes through the origin are subspaces.) Find the point in $S$ that is closest to the point $P=(3,3,1)$ using orthogonal projection. 
\end{enumerate}
Question for reflection: your other option for solving 3(b) is to use the cross product to find the normal vector, and then use the method for finding the shortest distance you learned in ealier in the course. Which do you prefer?

Challenge problem: what if the plane isn't through the origin? Can you still use orthogonal projection? If not, how can you modify the method?
 \end{enumerate}

\newpage

{\large \bf Name:}

\vspace{24pt}

{\large \bf Tutorial time:}

\vspace{24pt}

Please submit \textbf{one} \textit{completed} solution from the worksheet for feedback. 

\textbf{Note:} Your solution needs to contain enough detail for it to be clear what problem you're trying to solve!

\end{document}