\documentclass[12pt]{article}
\usepackage{amsmath}
\usepackage{amssymb}
\usepackage[letterpaper,margin=0.75in,centering]{geometry}
\usepackage{fancyhdr}
\usepackage{enumerate}
\usepackage{lastpage}
\usepackage{multicol}
\usepackage{graphicx}

\reversemarginpar

\pagestyle{fancy}
\cfoot{}
\lhead{Math 1410}\chead{Worksheet \# 8 Solutions}\rhead{Tuesday, 8\textsuperscript{th} March, 2016}
%\rfoot{Total: 10 points}
%\chead{{\bf Name:}}
\newcommand{\points}[1]{\marginpar{\hspace{24pt}[#1]}}
\newcommand{\skipline}{\vspace{12pt}}
%\renewcommand{\headrulewidth}{0in}
\headheight 30pt

\newenvironment{amatrix}[1]{%
  \left[\begin{array}{@{}*{#1}{c}|c@{}}
}{%
  \end{array}\right]
}
\newenvironment{aamatrix}[1]{%
  \left[\begin{array}{@{}*{#1}{c}|*{#1}{c}@{}}
}{%
  \end{array}\right]
}

\newcommand{\di}{\displaystyle}
\newcommand{\abs}[1]{\lvert #1\rvert}
\newcommand{\len}[1]{\lVert #1\rVert}
\renewcommand{\i}{\mathbf{i}}
\renewcommand{\j}{\mathbf{j}}
\renewcommand{\k}{\mathbf{k}}
\newcommand{\R}{\mathbb{R}}
\newcommand{\aaa}{\mathbf{a}}
\newcommand{\bbb}{\mathbf{b}}
\newcommand{\ccc}{\mathbf{c}}
\newcommand{\dotp}{\boldsymbol{\cdot}}
\newcommand{\bbm}{\begin{bmatrix}}
\newcommand{\ebm}{\end{bmatrix}}       
\newcommand{\bvm}{\begin{vmatrix}}
\newcommand{\evm}{\end{vmatrix}}  
\DeclareMathOperator{\proj}{proj}   
\newcommand{\bam}{\begin{amatrix}}
\newcommand{\eam}{\end{amatrix}}
         
                  
\begin{document}

%\author{Instructor: Sean Fitzpatrick}
\thispagestyle{fancy}
%\noindent{{\bf Name and student number:}}
%Please complete all problems below.
 \begin{enumerate}
  \item Let $A = \bbm 1&5&-3\\2&3&4\\-2&-7&3\ebm$
\begin{enumerate}
 \item Compute $\det A$ using cofactor (Laplace) expansion along the row or column of your choice.

\bigskip

Expanding along the first row, we have
\begin{align*}
 \det A &= 1(+1)\bvm 3&4\\-7&3\evm + 5(-1)\bvm 2&4\\-2&3\evm + (-3)(+1)\bvm 2&3\\-2&-7\evm\\
&= 1(9+28)-5(6+8)-3(-14+6) = 37-70+24 = -9.
\end{align*}


 \item Compute $\det A$ by first using row operations to reduce $A$ to triangular form. (Keep in mind that some row operations effect the value of $\det A$.)

\bigskip

We proceed as follows:

\begin{align*}
 \bvm 1&5&-3\\2&3&4\\-2&-7&3 \evm & = \bvm 1&5&-3\\0&-7&10\\0&3&-3\evm & & \text{(Using $R_2\to R_2-2R_1$ and $R_3\to R_3 +2R_1$)}\\
& = \bvm 1&5&-3\\0&-1&4\\0&3&-3\evm & & \text{(Using } R_2\to R_2+2R_3)\\
& = \bvm 1&5&-3\\0&-1&4\\0&0&9\evm & & \text{(Using } R_3\to R_3+3R_2)\\
& = 1(-1)(9)=-9.
\end{align*}

Note that the above used only ``Type 3'' row operations, which do not affect the determinant. One option you might have tried is to get rid of the factor of 3 in the third row after the first equality. You need to remember that the row operation $R_3\to \dfrac{1}{3}R_3$ affects the value of the determinant: the determinant of the resulting matrix is one third the determinant of the original matrix. One way to keep track of this is to think of ``factoring out'' the 3, as follows:
\[
 \bvm 1&5&-3\\0&-7&10\\0&3&-3\evm = 3\bvm 1&5&-3\\0&-7&10\\0&1&-1\evm.
\]

Note also that I did one more row operation than necessary in order to avoid fractions. From the right-hand side of the first line, I could have used the row operation $R_3\to R_3+\frac{3}{7}R_2$ to get (in one step)
\[
 \bvm 1&5&-3\\0&-7&10\\0&3&-3\evm = \bvm 1&5&-3\\0&-7&10\\0&0&\frac{9}{7}\evm = 1(-7)\left(\frac{9}{7}\right) = -9.
\]


 \item Use Cramer's rule to find the value of $x$ in the solution to the following system of equations:
\[
 \begin{array}{ccccccc}
  x&+&5y&-&3z&=&2\\
 2x&+&3y&+&4z&=&-1\\
-2x&-&7y&+&3z&=&0
 \end{array}
\]
\end{enumerate}

Cramer's rule states that $x=\dfrac{\det A_x}{\det A}$, where $A_x = \bbm 2&5&-3\\-1&3&4\\0&-7&3\ebm$ is the matrix obtained by replacing the $x$ column in the coefficient matrix $A$ by the column $\bbm 2\\-1\\0\ebm$ of constants from the right-hand sides of the equations. Noting that the coefficient matrix $A$ is the same one from part (a) --- now that I've fixed the typo --- we know that $\det A =-9$. We then compute
\[
 \det A_x = \bvm 2&5&-3\\-1&3&4\\0&-7&3\evm = \bvm 0 & 11&5\\-1&3&4\\0&-7&3\evm = (-1)(-1)\bvm 11&5\\-7&3\evm = 1(33+35)=68,
\]
where we've used the row operation $R_1\to R_1+R_2$ to create a zero in the upper left-hand corner, and then expanded along the third column. We thus have $x=-\dfrac{68}{9}$.

\item Let $A$ be a $3\times 3$ matrix such that $\det A = 4$. Compute the determinant of the following matrices:
\begin{enumerate}
 \item $B=EA$, where $E$ is the elementary matrix $E=\bbm 3&0&0\\0&1&0\\0&0&1\ebm$

\medskip

Since $EA$ is obtained from $A$ by multiplying Row 1 by 3, we know that $\det(EA) = 3\det A = 12$.

Alternatively, $\det(EA) = \det E\cdot \det A = 3\cdot 4 = 12$.

 \item The matrix $C$ obtained by switching rows 2 and 3 of $A$.

\medskip

Since swapping two rows changes the sign of the determinant, $\det C = -\det A = -4$.


 \item The matrix $2A$.

\medskip

Multiplying $A$ by 2 multiplies each of the three rows of $A$ by 2. Since multiplying a single row of $A$ by 2 multiplies the determinant by 2, and we're doing this three times, we have $\det (2A) = 2\cdot 2\cdot 2\cdot \det A = 2^3 \det A = 8\cdot 4 = 32$.

\end{enumerate}

\pagebreak

\item With the help of your classmates, come up with as many answers as possible to fill in the blank below:

\medskip

An $n\times n$ matrix $A$ is invertible if and only if \underline{\hspace{2in}}.

\medskip

Some possible answers include:
\begin{itemize}
 \item $AB = I$ for some $n\times n$ matrix $B$.
 \item $\det A\neq 0$
 \item The system of equations $AX=B$ has a unique solution.
 \item The rank of $A$ is $n$.
 \item There are no rows of zeros in the row-echelon form of $A$.
 \item $A$ is a product of elementary matrices.
\end{itemize}
There are a few more possibilities, but most of them involve language we haven't gotten to yet.

\item In each case, either prove the statement, or give an example showing that it is false:
\begin{enumerate}
 \item $\det(A+B)=\det A+\det B$.

\medskip

If we let $A=\bbm 1&0\\0&1\ebm$ and $B=\bbm -1&0\\0&-1\ebm$, then $\det A = \det B = 1$, so $\det A+\det B = 2$. On the other hand, $A+B = \bbm 0&0\\0&0\ebm$, so $\det(A+B)=0$. Since $2\neq 0$, this statement is false in general.

 \item If $\det A=0$, then $A$ has two equal rows.

\medskip

This is false. For example, the rows of $A=\bbm 1&1\\2&2\ebm$ are not equal, but $\det A = 2-2=0$. (The converse is true however: if $A$ has two equal rows, then $\det A=0$.)




 \item For any $2\times 2$ matrix $A$, $\det (A^T)=\det A$.

\medskip

This is true. A general $2\times 2$ matrix can be written as $A=\bbm a&b\\c&d\ebm$, in which case $A^T = \bbm a&c\\b&d\ebm$, and we can easily verify that $\det A = \det A^T = ad-bc$.

 \item $\det (-A) = -\det A$

\medskip

This is false in general. For example, in our example for part (a), we had $B=-A$, but $\det B = \det A$.

What is true is that $\det (-A) = (-1)^n\det A$, if $A$ is an $n\times n$ matrix, so the result is false when $n$ is even, and true when $n$ is odd.

 \item If $\det A\neq 0$ and $AB=AC$, then $B=C$.

\medskip

This is true. If $\det A\neq 0$, then we know $A$ is invertible, so if $AB=BC$ and $\det A\neq 0$, then we can multiply both sides of this equation on the left by $A^{-1}$ to obtain $B=C$.

\end{enumerate}
\pagebreak
\item What can be said about $\det A$ if:
\begin{enumerate}
 \item $A^2=A$

\medskip

Since $A^2=A$, we know that $\det (A^2) = \det (A)$. Since $\det (AB) = \det A\cdot \det B$ for any $n\times n$ matrices $A$ and $B$, we know that
\[
 \det(A^2) = \det(AA) = \det A\cdot \det A = (\det A)^2.
\]
Thus, if we let $x=\det A$, we must have $x^2=x$, or $x^2-x = x(x-1)=0$, so $\det A=0$ or $\det A=1$.

 \item $A^2 = I$

\medskip

Since $\det I = 1$, we have $\det (A^2) = (\det A)^2 = 1$, which implies that $\det A = \pm 1$.

 \item $PA=P$, where $P$ is invertible.

\medskip

If $PA=P$, then $\det (PA) = \det P$. But $\det (PA) = \det P\cdot \det A$, so we have $\det P \cdot \det A = \det P$. Since $P$ is invertible, we know that $\det P\neq 0$, so we can divide both sides of the last equation by $\det P$, giving us $\det A = 1$.
\end{enumerate}



 \end{enumerate}

\end{document}