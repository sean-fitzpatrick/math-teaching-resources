\documentclass[12pt]{article}
\usepackage{amsmath}
\usepackage{amssymb}
\usepackage[letterpaper,top=1.2in,bottom=1in,left=0.75in,right=0.75in,centering]{geometry}
%\usepackage{fancyhdr}
\usepackage{enumerate}
%\usepackage{lastpage}
\usepackage{multicol}
\usepackage{graphicx}

\reversemarginpar

%\pagestyle{fancy}
%\cfoot{}
%\lhead{Math 1560}\chead{Test \# 1}\rhead{May 18th, 2017}
%\rfoot{Total: 10 points}
%\chead{{\bf Name:}}
\newcommand{\points}[1]{\marginpar{\hspace{24pt}[#1]}}
\newcommand{\skipline}{\vspace{12pt}}
%\renewcommand{\headrulewidth}{0in}
\headheight 30pt

\newenvironment{amatrix}[1]{%
  \left[\begin{array}{@{}*{#1}{c}|c@{}}
}{%
  \end{array}\right]
}
\newcommand{\di}{\displaystyle}
\newcommand{\abs}[1]{\lvert #1\rvert}
\newcommand{\len}[1]{\lVert #1\rVert}
\renewcommand{\i}{\mathbf{i}}
\renewcommand{\j}{\mathbf{j}}
\renewcommand{\k}{\mathbf{k}}
\newcommand{\R}{\mathbb{R}}
\newcommand{\aaa}{\mathbf{a}}
\newcommand{\bbb}{\mathbf{b}}
\newcommand{\ccc}{\mathbf{c}}
\newcommand{\dotp}{\boldsymbol{\cdot}}
\newcommand{\bbm}{\begin{bmatrix}}
\newcommand{\ebm}{\end{bmatrix}}       
\DeclareMathOperator{\proj}{proj}      
\newcommand{\bam}{\begin{amatrix}}
\newcommand{\eam}{\end{amatrix}}  
%\newenvironment{inva}{\left[\begin{array}{cc|cc}}{\end{array}\right]} 
%\newcommand{\binva}{\begin{inva}}
%\newcommand{\einva}{\end{inva}}     
\newcommand{\binv}{\left[\begin{array}{ccc|ccc}}
\newcommand{\einv}{\end{array}\right]}     

\begin{document}


\author{Instructor: Sean Fitzpatrick}
\thispagestyle{empty}
\vglue1cm
\begin{center}
{\bf MATH 1410 - Tutorial \#9 Solutions}
\end{center}




%\thispagestyle{empty}

 \begin{enumerate}
\item Let $A=\bbm 1&-1&4\\-2&3&-5\\1&1&9\ebm$.
\begin{enumerate}
\item Compute $A^{-1}$.

We form the augmented matrix $\bbm A&I\ebm$ and proceed to RREF, as follows:
\begin{align*}
\binv 1&-1&4&1&0&0\\
     -2&3&-5&-&1&0\\
      1&1& 9&0&0&1\einv \xrightarrow[R_3-R_1\to R_3]{R_2+2R_1\to R_2}&
\binv 1&-1&4&1&0&0\\
      0&1&3&2&1&0\\
      0&2&5&-1&0&1\einv\\ \xrightarrow{R_3-2R_2\to R_3}&
\binv 1&-1&4&1&0&0\\
      0&1&3&2&1&0\\
      0&0&-1&-5&-2&1\einv\\ \xrightarrow{-R_3\to R_3}&
\binv 1&-1&4&1&0&0\\
      0&1&3&2&1&0\\
      0&0&1&5&2&-1\einv\\ \xrightarrow[R_2-3R_3\to R_2]{R_1-4R_3\to R_1}&
\binv 1&-1&0&-19&-8&4\\
      0&1&0&-13&-5&3\\
      0&0&1&5&2&-1\einv\\ \xrightarrow{R_1+R_2\to R_1}&
\binv 1&0&0&-32&-13&7\\
      0&1&0&-13&-5&3\\
      0&0&1&5&2&-1\einv      
\end{align*}
Since we now have identity matrix on the left, we can conclude that $A^{-1} = \bbm -32 &-13&7\\-13&-5&3\\5&2&-1\ebm$.

\item Solve for $X$, if $AX+\bbm 2&-3\\-1&-5\\4&2\ebm = \bbm 1&0\\4&-3\\-2&1\ebm$

Since $AX = \bbm 1&0\\4&-3\\-2&1\ebm - \bbm 2&-3\\-1&-5\\4&2\ebm = \bbm -1&3\\5&2\\-6&-1\ebm$, we can multiply by $A^{-1}$ (as determined above) to get
\[
X = \bbm -32 &-13&7\\-13&-5&3\\5&2&-1\ebm \bbm -1&3\\5&2\\-6&-1\ebm = \bbm -75&-129\\-30&-52\\11&20\ebm.
\]

\end{enumerate}
\newpage

\item For each equation given below, explain why $A$ is invertible, and determine an expression (in terms of $A$) for $A^{-1}$.

\textit{Hint:} Recall that $A$ is invertible provided there exists a matrix $B$ such that $AB=I$, in which case $B=A^{-1}$.

\begin{enumerate}
\item $A^5=I$

\bigskip

Since $A(A^4) = A^5 = I$, we can conclude that $A$ is invertible, and $A^{-1}=A^4$.

\bigskip

\item $A^4=9I$

\bigskip

We have $\frac19 A^4 = I$, so $A\left(\frac19 A^3\right) = \frac19 A(A^3)=\frac19 A^4=I$.

Therefore, $A$ is invertible, and $A^{-1} = \frac19 A^3$.

\bigskip



\item $A^2-5A+6I=0$

\bigskip

Notice that $6I=5A-A^2$, so 
\[
I = \frac16 (5A-A^2) = \frac56 A-\frac16 A^2 = (\frac56 I -\frac16 A)A.
\]
Thus, $A$ is invertible, and $A^{-1} = \frac56 I-\frac16 A$.

\textbf{Note:} when factoring an expression such as $5A-A^2$, we must be careful. First instinct suggests that we can write $5A-A^2=A(5-A)$, but the expression $5-A$ is undefined: we cannot subtract a matrix from a number. The correct factorization is $5A-A^2 = A(5I-A)$. Adding the identity matrix ensures that everything is defined, and does not affect the product: $A(5I) = 5(AI)=5A$.
\end{enumerate}

\bigskip

\item Show that the matrix $A=\bbm -3&-3\\10&8\ebm$ satisfies the equation $A^2-5A+6I=0$. Using your result from problem 2(c), determine $A^{-1}$.

We find $A^2 = \bbm -3&-3\\10&8\ebm\bbm -3&-3\\10&8\ebm=\bbm -21&-15\\50&34\ebm$, so
\[
A^2-5A+6I = \bbm -21&-15\\50&34\ebm+\bbm 15&15\\-50&-40\ebm + \bbm 6&0\\0&6\ebm = \bbm 0&0\\0&0\ebm,
\]
as required. It follows from 2(c) that
\[
A^{-1} = \frac56 I-\frac16 A = \bbm 5/6 &0\\0&5/6\ebm+\bbm 1/2&1/2\\-5/3&4/3\ebm = \bbm 4/3&1/2\\-5/3&-1/2\ebm.
\]
\newpage

\item For each problem below, assume $T$ is a linear transformation from $\R^2$ to $\R^3$.
\begin{enumerate}
\item Given vectors $\vec{u},\vec{v}\in\R^2$ such that $T(\vec{u})=\bbm 3\\-1\\4\ebm$ and $T(\vec{v})=\bbm -1\\2\\5\ebm$, determine the value of $T(4\vec{u}-3\vec{v})$.

\[
T(4\vec{u}-3\vec{v})=4T(\vec{u})-3T(\vec{v})=4\bbm 3\\-1\\4\ebm-3\bbm -1\\2\\5\ebm = \bbm 15\\-10\\1\ebm.
\]

\item Determine the matrix of $T$, if $T\left(\bbm 1\\0\ebm\right)=\bbm 1\\-2\\0\ebm$ and $T\left(\bbm 0\\1\ebm\right) = \bbm 4\\-1\\2\ebm$.

\bigskip

It is a general fact (see Theorem 35 on p. 266 of the textbook) that the if $A$ is the matrix of $T$, then the columns of $A$ are given by the values of $T$ on the standard unit vectors. In this case, $T\left(\bbm 1\\0\ebm\right)$ gives the first column, and $T\left(\bbm 0\\1\ebm\right)$
the second, so
\[
A=[T] = \bbm 1&4\\-2&-1\\0&2\ebm.
\]
\item \textit{For fun}: find the matrix of the transformation $T$ in part(a), if $\vec{u}=\bbm 2\\1\ebm$ and $\vec{v}=\bbm -1\\1\ebm$.

\textit{Hint:} First determine how to write $\bbm 1\\0\ebm$ and $\bbm 0\\1\ebm$ in terms of $\vec{u}$ and $\vec{v}$. 

First, notice that $\bbm 1\\0\ebm =\frac 13\vec{u}-\frac13 \vec{v}$, and $\bbm 0\\1\ebm = \frac13 \vec{u}+\frac23\vec{v}$. Thus,
\[
T\left(\bbm 1\\0\ebm\right) = T(\frac13 \vec{u}-\frac13\vec{v}) = \frac13\bbm 3\\-1\\4\ebm-\frac13\bbm -1\\2\\5\ebm = \bbm 4/3\\-1\\-1/3\ebm,
\]
and
\[
T\left(\bbm 0\\1 \ebm\right) T(\frac13\vec{u}+\frac23\vec{v}) = \frac13\bbm 3\\-1\\4\ebm +\frac23 \bbm -1\\2\\5\ebm = \bbm 1/3\\1\\14/3\ebm,
\]
so $[T] = \bbm 4/3&1/3\\-1&1\\-1/3&14/3\ebm$. (Feel free to confirm the values of $T(\vec{u})$ and $T(\vec{v})$ using this matrix.)
\end{enumerate}

 \end{enumerate}
 
\end{document}