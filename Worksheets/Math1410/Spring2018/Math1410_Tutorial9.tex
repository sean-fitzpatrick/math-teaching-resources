\documentclass[12pt]{article}
\usepackage{amsmath}
\usepackage{amssymb}
\usepackage[letterpaper,top=1.2in,bottom=1in,left=0.75in,right=0.75in,centering]{geometry}
%\usepackage{fancyhdr}
\usepackage{enumerate}
%\usepackage{lastpage}
\usepackage{multicol}
\usepackage{graphicx}

\reversemarginpar

%\pagestyle{fancy}
%\cfoot{}
%\lhead{Math 1560}\chead{Test \# 1}\rhead{May 18th, 2017}
%\rfoot{Total: 10 points}
%\chead{{\bf Name:}}
\newcommand{\points}[1]{\marginpar{\hspace{24pt}[#1]}}
\newcommand{\skipline}{\vspace{12pt}}
%\renewcommand{\headrulewidth}{0in}
\headheight 30pt

\newenvironment{amatrix}[1]{%
  \left[\begin{array}{@{}*{#1}{c}|c@{}}
}{%
  \end{array}\right]
}
\newcommand{\di}{\displaystyle}
\newcommand{\abs}[1]{\lvert #1\rvert}
\newcommand{\len}[1]{\lVert #1\rVert}
\renewcommand{\i}{\mathbf{i}}
\renewcommand{\j}{\mathbf{j}}
\renewcommand{\k}{\mathbf{k}}
\newcommand{\R}{\mathbb{R}}
\newcommand{\aaa}{\mathbf{a}}
\newcommand{\bbb}{\mathbf{b}}
\newcommand{\ccc}{\mathbf{c}}
\newcommand{\dotp}{\boldsymbol{\cdot}}
\newcommand{\bbm}{\begin{bmatrix}}
\newcommand{\ebm}{\end{bmatrix}}       
\DeclareMathOperator{\proj}{proj}      
\newcommand{\bam}{\begin{amatrix}}
\newcommand{\eam}{\end{amatrix}}   
                  
\begin{document}


\author{Instructor: Sean Fitzpatrick}
\thispagestyle{empty}
\vglue1cm
\begin{center}
\emph{University of Lethbridge}\\
Department of Mathematics and Computer Science\\
{\bf MATH 1410 - Tutorial \#9}\\
Wednesday, March 7
\end{center}
\skipline \ \noindent \skipline

\vspace*{\fill}







Additional practice: (\textbf{do not submit}).
\begin{enumerate}
\item Determine the inverse of the following matrices, if possible: 
\begin{multicols}{2}
\begin{enumerate}
\item $\bbm 4&-3\\-2&7\ebm$
\item $\bbm 1&0&3\\-2&1&4\\3&2&-2\ebm$
\item $\bbm 2&-6\\-1&3\ebm$
\item $\bbm 2&-1&4\\0&3&-5\\4&1&3\ebm$
\end{enumerate}
\end{multicols}
\end{enumerate}

\newpage
%\thispagestyle{empty}

 \begin{enumerate}
\item Let $A=\bbm 1&-1&4\\-2&3&-5\\1&1&9\ebm$.
\begin{enumerate}
\item Compute $A^{-1}$.

\vspace{4in}

\item Solve for $X$, if $AX+\bbm 2&-3\\-1&-5\\4&2\ebm = \bbm 1&0\\4&-3\\-2&1\ebm$
\end{enumerate}
\newpage

\item For each equation given below, explain why $A$ is invertible, and determine an expression (in terms of $A$) for $A^{-1}$.

\textit{Hint:} Recall that $A$ is invertible provided there exists a matrix $B$ such that $AB=I$, in which case $B=A^{-1}$.

\begin{enumerate}
\item $A^5=I$

\vspace{1.25in}

\item $A^4=9I$

\vspace{1.25in}

\item $A^2-5A+6I=0$

\end{enumerate}
\vspace{1.5in}

\item Show that the matrix $A=\bbm -3&-3\\10&8\ebm$ satisfies the equation $A^2-5A+6I=0$. Using your result from problem 2(c), determine $A^{-1}$.

\newpage

\item For each problem below, assume $T$ is a linear transformation from $\R^2$ to $\R^3$.
\begin{enumerate}
\item Given vectors $\vec{u},\vec{v}\in\R^2$ such that $T(\vec{u})=\bbm 3\\-1\\4\ebm$ and $T(\vec{v})=\bbm -1\\2\\5\ebm$, determine the value of $T(4\vec{u}-3\vec{v})$.

\vspace{2in}

\item Determine the matrix of $T$, if $T\left(\bbm 1\\0\ebm\right)=\bbm 1\\-2\\0\ebm$ and $T\left(\bbm 0\\1\ebm\right) = \bbm 4\\-1\\2\ebm$.

\vspace{2in}

\item \textit{For fun}: find the matrix of the transformation $T$ in part(a), if $\vec{u}=\bbm 2\\1\ebm$ and $\vec{v}=\bbm -1\\1\ebm$.

\textit{Hint:} First determine how to write $\bbm 1\\0\ebm$ and $\bbm 0\\1\ebm$ in terms of $\vec{u}$ and $\vec{v}$. 
\end{enumerate}

 \end{enumerate}
 
\end{document}