\documentclass[12pt]{article}
\usepackage{amsmath}
\usepackage{amssymb}
\usepackage[letterpaper,top=1.2in,bottom=1in,left=0.75in,right=0.75in,centering]{geometry}
%\usepackage{fancyhdr}
\usepackage{enumerate}
%\usepackage{lastpage}
\usepackage{multicol}
\usepackage{graphicx}
\usepackage{extarrows}

\reversemarginpar

%\pagestyle{fancy}
%\cfoot{}
%\lhead{Math 1560}\chead{Test \# 1}\rhead{May 18th, 2017}
%\rfoot{Total: 10 points}
%\chead{{\bf Name:}}
\newcommand{\points}[1]{\marginpar{\hspace{24pt}[#1]}}
\newcommand{\skipline}{\vspace{12pt}}
%\renewcommand{\headrulewidth}{0in}
\headheight 30pt

\newenvironment{amatrix}[1]{%
  \left[\begin{array}{@{}*{#1}{c}|c@{}}
}{%
  \end{array}\right]
}
\newcommand{\di}{\displaystyle}
\newcommand{\abs}[1]{\lvert #1\rvert}
\newcommand{\len}[1]{\lVert #1\rVert}
\renewcommand{\i}{\mathbf{i}}
\renewcommand{\j}{\mathbf{j}}
\renewcommand{\k}{\mathbf{k}}
\newcommand{\R}{\mathbb{R}}
\newcommand{\aaa}{\mathbf{a}}
\newcommand{\bbb}{\mathbf{b}}
\newcommand{\ccc}{\mathbf{c}}
\newcommand{\dotp}{\boldsymbol{\cdot}}
\newcommand{\bbm}{\begin{bmatrix}}
\newcommand{\ebm}{\end{bmatrix}}       
\newcommand{\bvm}{\begin{vmatrix}}
\newcommand{\evm}{\end{vmatrix}}       
\DeclareMathOperator{\proj}{proj}      
\newcommand{\bam}{\begin{amatrix}}
\newcommand{\eam}{\end{amatrix}}   
\DeclareMathOperator{\cof}{cof}
                  
\begin{document}


\author{Instructor: Sean Fitzpatrick}
\thispagestyle{empty}
\vglue1cm
\begin{center}
{\bf MATH 1410 - Tutorial \#10 Solutions}
\end{center}

 \begin{enumerate}
\item Let $A=\bbm 3&0&-2\\0&4&6\\-1&2&0\ebm$. 
\begin{enumerate}
\item Compute $\det(A)$ by doing a cofactor expansion along a row or column (your choice).

We choose to use row 2:
\[
\det(A) = 0+4(-1)^{2+2}\bvm 3&-2\\-1&0\evm +6(-1)^{2+3}\bvm 3&0\\-1&2\evm = 0+4(0-2)-6(6-0)=-44.
\]

For parts (b) and (c), we use the fact that the specified row operations do not affect the value of the determinant, so $\det(A)$ is equal the determinant of the matrix that results:
\item Perform the row operation $R_1+3R_3\to R_1$ and compute the determinant of the resulting matrix by cofactor expansion along the first column.

\[
\det(A) = \bvm 0&6&-2\\0&4&6\\-1&2&0\evm = 0+0+(-1)(-1)^{3+1}\bvm 6&-2\\4&6\evm = -1(36+8)=-44.
\]

\item Perform the row operation $R_2+3R_1\to R_2$ and compute the determinant of the resulting matrix by cofactor expansion along the third column.

\[
\det(A) = \bvm 3&0&-2\\9&4&0\\-1&2&0\evm = -2(-1)^{1+3}\bvm 9&4\\-1&2\evm+0+0=-2(18+4)=-44.
\]

For part (d), we use the fact that doing the same type of operation, on columns instead of rows, also leaves the determinant unaffected.
\item  Perform the \textit{column operation} $C_2+2C_1\to C_2$ and compute the determinant of the resulting matrix by cofactor expansion along the third row.
\[
\det(A) = \bvm 3&6&-2\\0&4&6\\-1&0&0\evm = 0+0+(-1)\bvm 6&-2\\4&6\evm = -44.
\]
\end{enumerate}
\newpage

\item Compute the determinant of the matrix $A=\bbm -1&0&3&4&2\\0&1&4&-1&2\\2&0&-1&3&0\\1&0&-3&-5&-2\\0&2&0&3&1\ebm$.

\textit{Hint:} Row operations of the form $R_i+kR_j\to R_i$ do not change the value of the determinant. Once you have enough zeros in a column, expand.\\ (You might try creating two more zeros in column 1, or one more zero in column 2.)
 
We will demonstrate several different options for computing this determinant.

Option 1:
\begin{align*}
\det(A) &\xlongequal[R_4+R_1\to R_4]{R_3+2R_1\to R_3} \bvm -1&0&3&4&2\\0&1&4&-1&2\\0&0&5&11&4\\0&0&0&-1&0\\0&2&0&3&1\evm \xlongequal{\text{expand }C_1}(-1)(+1)\bvm 1&4&-1&2\\0&5&11&4\\0&0&-1&0\\2&0&3&1\evm\\
&\xlongequal{\text{expand }R_3}(-1)(-1)(+1)\bvm 1&4&2\\0&5&4\\2&0&1\evm\xlongequal{R_3-2R_1\to R_3}\bvm 1&4&2\\0&5&4\\0&-8&-3\evm\\
&\xlongequal{\text{expand }C_1}(1)(+1)\bvm 5&4\\-8&-3\evm = -15+32=17.
\end{align*}

Option 2:
\begin{align*}
\det(A) &\xlongequal{R_1+R_4\to R_1}\bvm 0&0&0&-1&0\\0&1&4&-1&2\\2&0&-1&3&0\\1&0&-3&-5&-2\\-&2&0&3&1\evm\xlongequal{\text{expand }R_1}(-1)(-1)\bvm 0&1&4&2\\2&0&-1&0\\1&0&-3&-2\\0&2&0&1\evm\\
&\xrightarrow{R_2-2R_3\to R_2}\bvm 0&1&4&2\\0&0&5&4\\1&0&-3&-2\\0&2&0&1\evm \xlongequal{\text{expand }C_1}1(+1)\bvm 1&4&2\\0&5&4\\2&0&1\evm\\
&\xlongequal{R_3-2R_1\to R_3}\bvm 1&4&2\\0&5&4\\0&-8&-3\evm \xlongequal{\text{expand }C_1} (1)\bvm 5&4\\-8&-3\evm =17.
\end{align*}

Option 3: Perhaps we won't get too carried away. But yet another option is to start with the row operation $R_5-2R_2\to R_5$ and then expand along column 2.
 \newpage
 
 \item Let $A$ and $B$ be $3\times 3$ matrices, with $\det(A)=2$ and $\det(B)=-3$. What is the value of:
 \begin{enumerate}
 \item $\det(AB^2)=\det(A)(\det(B))^2=2(-3)^2=18.$
 

 
 \item $\det(B^{-1}A^3B)=\det(B^{-1})\det(A^3)\det(B)=\dfrac{1}{\det(B)}(\det(A))^3\det(B) = (\det(A))^3=8.$
 

 
 \item $\det(2A^{-1}B)=2^3\det(A^{-1}B)=8\left(\dfrac{1}{\det(A)}\right)\det(B)=8\cdot \frac12 \cdot (-3) = -12.$
 

 
\end{enumerate}
\item Consider the system
\[\arraycolsep=1pt
\begin{array}{ccccc}
2x&+&ay&=&s\\
3ax&+&6y&=&t
\end{array}
\]
\begin{enumerate}
\item For which values of $a$ will the system have a unique solution? (Hint: use a determinant.)

We know that a system of $n$ equations in $n$ variables has a unique solution if and only if its coefficient matrix is invertible, and this in turn is true if and only if the determinant of that matrix is nonzero.

Our coefficient matrix is $A=\bbm 2&a\\3a&6\ebm$, so we have
\[
\det(A) = \bvm 2&a\\3a&6\evm = 12-3a^2=3(2-a)(2+a),
\]
which tells us that the system has a unique solution for all values of $a$ except $a=2$ and $a=-2$.

\item Use Cramer's rule to solve the system (in terms of $a,s,t$) when possible, as determined by part (a).

Cramer's rule only applies if the system has a unique solution, so we must assume that $a\neq \pm 2$. In this case we have
\[
x=\frac{\abs{A_1}}{\abs{A}} \text{ and } y = \frac{\abs{A_2}}{\abs{A}},
\]
where $\abs{A_i}$ is obtained by replacing column $i$ of $A$ by the column $\bbm s\\t\ebm$. Thus, we have
\[
x = \frac{\bvm s&a\\t&6\evm}{\abs{A}} = \frac{6s-at}{12-3a^2}\text{ and }
y = \frac{\bvm 2&s\\3a&t\evm}{\abs{A}} = \frac{2t-3as}{12-3a^2}.
\]
\end{enumerate}

 \end{enumerate}
\end{document}