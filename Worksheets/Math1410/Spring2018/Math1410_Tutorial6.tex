\documentclass[12pt]{article}
\usepackage{amsmath}
\usepackage{amssymb}
\usepackage[letterpaper,top=1.2in,bottom=1in,left=0.75in,right=0.75in,centering]{geometry}
%\usepackage{fancyhdr}
\usepackage{enumerate}
%\usepackage{lastpage}
\usepackage{multicol}
\usepackage{graphicx}

\reversemarginpar

%\pagestyle{fancy}
%\cfoot{}
%\lhead{Math 1560}\chead{Test \# 1}\rhead{May 18th, 2017}
%\rfoot{Total: 10 points}
%\chead{{\bf Name:}}
\newcommand{\points}[1]{\marginpar{\hspace{24pt}[#1]}}
\newcommand{\skipline}{\vspace{12pt}}
%\renewcommand{\headrulewidth}{0in}
\headheight 30pt

\newenvironment{amatrix}[1]{%
  \left[\begin{array}{@{}*{#1}{c}|c@{}}
}{%
  \end{array}\right]
}
\newcommand{\di}{\displaystyle}
\newcommand{\abs}[1]{\lvert #1\rvert}
\newcommand{\len}[1]{\lVert #1\rVert}
\renewcommand{\i}{\mathbf{i}}
\renewcommand{\j}{\mathbf{j}}
\renewcommand{\k}{\mathbf{k}}
\newcommand{\R}{\mathbb{R}}
\newcommand{\aaa}{\mathbf{a}}
\newcommand{\bbb}{\mathbf{b}}
\newcommand{\ccc}{\mathbf{c}}
\newcommand{\dotp}{\boldsymbol{\cdot}}
\newcommand{\bbm}{\begin{bmatrix}}
\newcommand{\ebm}{\end{bmatrix}}       
\DeclareMathOperator{\proj}{proj}      
\newcommand{\bam}{\begin{amatrix}}
\newcommand{\eam}{\end{amatrix}}   
                  
\begin{document}


\author{Instructor: Sean Fitzpatrick}
\thispagestyle{empty}
\vglue1cm
\begin{center}
\emph{University of Lethbridge}\\
Department of Mathematics and Computer Science\\
{\bf MATH 1410 - Tutorial \#6}\\
Wednesday, February 28
\end{center}
\skipline \ \noindent \skipline

Student \#1 :\underline{\hspace{348pt}}\\

\bigskip

\bigskip

Student \#2 :\underline{\hspace{348pt}}\\

\bigskip

\bigskip

Student \#3 :\underline{\hspace{348pt}}\\

\bigskip

\bigskip

Student \#4 :\underline{\hspace{348pt}}\\





\bigskip

\textbf{(Moodle ID not required.)}

\bigskip

\bigskip

Additional practice: (\textbf{do not submit}).
\begin{enumerate}
\item Use Gaussian elimination to find the reduced row-echelon form of the matrix:
\begin{multicols}{3}
\begin{enumerate}
\item $\bbm 2&3&-1\\1&4&0\ebm$
\item $\bbm 4&8\\-2&-4\ebm$
\item $\bbm 1&3&2&-1\\-2&1&3&4\\-1&4&5&3\ebm$
\end{enumerate}
\end{multicols}
\item Solve the system of equations:
\begin{multicols}{2}
\begin{enumerate}
\item $\arraycolsep2pt\begin{array}{ccccc}
2x&-&3y&=&7\\-x&+&2y&=&2\end{array}$
\item $\arraycolsep2pt\begin{array}{ccccccc}
x&-&2y&+&4z&=&2\\2x&-3y&z&=&-2\\-x&+&2y&-&2z&=&6\end{array}$
\end{enumerate}
\end{multicols}
\end{enumerate}


\newpage
%\thispagestyle{empty}

 \begin{enumerate}
\item For each system of equations below, write down the corresponding augmented matrix.
\begin{multicols}{2}
\begin{enumerate}
 \item $\arraycolsep=1pt\begin{array}{ccccccc}2x&-&3y&+&z&=&2\\[5pt] & &2y&-&5z&=&-3\\[5pt] -3x& & &+&2z&=&7\end{array}$



 \item $\arraycolsep=1pt\begin{array}{ccccccccc}x_1&+&4x_2& & &-&7x_4&=&0\\[5pt]-3x_1&-&x_2&+&4x_3& & &=&2\\[5pt] & &2x_2&-&4x_3&+&x_4&=&-5\end{array}$
 \end{enumerate}
\end{multicols}


\vspace{3.5cm}

%\arraycolsep=4pt

\item For each augmented matrix below, write down a corresponding system of equations using whatever variables you prefer.
\begin{multicols}{2}
\begin{enumerate}
 \item $\bam{3}2&-1&0&4\\-3&4&1&-2\\0&2&3&-7\eam$



 \item $\bam{4}3&2&0&1&-5\\0&4&2&-7&2\eam$
\end{enumerate}
\end{multicols}

\vspace{4cm}

\item State whether or not the given augmented matrix is in reduced row-echelon form (RREF), and if not, why.

\[
 \bam{3}1&0&2&-1\\0&1&2&4\\0&0&0&0\eam \quad \bam{3}1&0&0&2\\0&1&1&0\\0&0&0&4\eam \quad \bam{3}1&2&0&3\\0&1&0&-4\\0&0&1&2\eam \quad \bam{3}1&0&0&7\\0&2&0&3\\0&0&1&0\eam 
\quad \bam{4}0&1&0&2&-3\\0&0&1&-3&4\\0&0&0&1&3\eam
\]

\pagebreak

\item Suppose you want to perform Gaussian elimination on the augmented matrices below. \\For each matrix, what are the first two row operations you would perform, and why?

\begin{multicols}{3}
\begin{enumerate}
\item $\bam{3} 1&-4&2&0\\-2&4&1&6\\3&2&-1&1\eam$
\item $\bam{3} 2&4&-8&10\\-1&2&4&-5\\0&1&5&2\eam$
\item $\bam{3} 3&2&-7&4\\1&2&-4&0\\0&-1&3&2\eam$
\end{enumerate}
\end{multicols}

\vspace{5cm}

\item For each matrix $A$ and $B$ below, write down the row operation that transforms $A$ into $B$.

\begin{enumerate}
\item $A = \bbm 3&-2&5\\2&8&-4\\1&-2&1\ebm$, $B = \bbm 3&-2&5\\1&4&-2\\1&-2&1\ebm$

\bigskip

\item $A = \bbm 2&7&-3\\6&8&1\\1&12&-6\ebm$, $B = \bbm 2&7&-3\\0&-13&10\\1&12&-6\ebm$

\bigskip

\item $A = \bbm 4&-2&3\\1&3&4\\-5&6&0\ebm$, $B = \bbm -5&6&0\\1&3&4\\4&-2&3\ebm$

\end{enumerate}


\pagebreak

\item Write down the augmented matrix of the following system, and then use Gaussian elimination to solve the system.
\[
 \arraycolsep=2pt \begin{array}{ccccccc}
                  x&+&2y&-&z&=&4\\-x&+&y&-&2z&=&-1\\ 2x&+&6y&-&3z&=&5
                  \end{array}
\]

\vspace{4in}

\item A system in variables $x,y,z$ has an augmented matrix with RREF $\di \bam{3} 1&0&-3&4\\0&1&2&6\eam$.

Write down the system of equations corresponding to this matrix. How would you describe the solution to the system?

(Hint: what geometric problem corresponds to a system of two equations in three variables?)

 \end{enumerate}
  
\end{document}