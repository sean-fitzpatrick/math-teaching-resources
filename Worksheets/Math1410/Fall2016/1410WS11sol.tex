\documentclass[12pt]{article}
\usepackage{amsmath}
\usepackage{amssymb}
\usepackage[letterpaper,margin=0.85in,centering]{geometry}
\usepackage{fancyhdr}
\usepackage{enumerate}
\usepackage{lastpage}
\usepackage{multicol}
\usepackage{graphicx}

\reversemarginpar

\pagestyle{fancy}
\cfoot{}
\lhead{Math 1410}\chead{Worksheet \# 11 Solutions}\rhead{November 30/December 1, 2016}
%\rfoot{Total: 10 points}
%\chead{{\bf Name:}}
\newcommand{\points}[1]{\marginpar{\hspace{24pt}[#1]}}
\newcommand{\skipline}{\vspace{12pt}}
%\renewcommand{\headrulewidth}{0in}
\headheight 30pt

\newenvironment{amatrix}[1]{%
  \left[\begin{array}{@{}*{#1}{c}|c@{}}
}{%
  \end{array}\right]
}

\newcommand{\di}{\displaystyle}
\newcommand{\abs}[1]{\lvert #1\rvert}
\newcommand{\len}[1]{\lVert #1\rVert}
\renewcommand{\i}{\mathbf{i}}
\renewcommand{\j}{\mathbf{j}}
\renewcommand{\k}{\mathbf{k}}
\newcommand{\R}{\mathbb{R}}
\newcommand{\aaa}{\mathbf{a}}
\newcommand{\bbb}{\mathbf{b}}
\newcommand{\ccc}{\mathbf{c}}
\newcommand{\dotp}{\boldsymbol{\cdot}}
\newcommand{\bbm}{\begin{bmatrix}}
\newcommand{\ebm}{\end{bmatrix}}                   
\newcommand{\bam}{\begin{amatrix}}
\newcommand{\eam}{\end{amatrix}}   
                 
\begin{document}
{\bf \large Name:} \hspace{2.5in} {\bf Tutorial day and time:}

\bigskip

{\bf Select {\bf one} {\em completed} problem for feedback:}

\bigskip


%\author{Instructor: Sean Fitzpatrick}
\thispagestyle{fancy}
%\noindent{{\bf Name and student number:}}
 \begin{enumerate}
\item Let $A$, $B$, and $C$ be $n\times n$ matrices such that $\det(A) = 2$, $\det(B)=-1$, and $\det(C) = 3$. Evaluate $\det(A^3BC^TB^{-1})$.

\bigskip

\[
 \abs{A^3BC^TB^{-1}} = \abs{A^3}\abs{B}\abs{C^T}\abs{B^{-1}} = \abs{A}^3\abs{B}\abs{C}\left(\frac{1}{\abs{B}}\right) = 2^3(4) = 24.
\]

\bigskip

\item If $A$ and $B$ are $3\times 3$ matrices such that $\det(2A^{-1}) = -4 = \det(A^3(B^{-1})^T)$, what are the values of $\det(A)$ and $\det(B)$?

\bigskip

Since $A$ is $3\times 3$, we have $\det(2A^{-1}) = 2^3\det(A^{-1}) = \frac{8}{\det(A)} = -4$, giving us $\det(A) = -2$. Using this, we find
\[
 \det(A^3(B^{-1})^T) = (\det(A))^3\det(B^{-1}) = \frac{(-2)^3}{\det(B)} = -4,
\]
and solving for $\det(B)$ gives us $\det(B) = 2$.

\bigskip

\item The matrix $A = \bbm 3&2&1\\1&4&1\\1&2&3\ebm$ has eigenvalues $\lambda=2$ and $\lambda=6$. Find the corresponding eigenvectors.

\bigskip

For the eigenvalue $\lambda = 2$, we have
\[
 A-2I = \bbm 3-2&2&1\\1&4-2&1\\1&2&3-2\ebm = \bbm 1&2&1\\1&2&1\\1&2&1\ebm \xrightarrow{\text{RREF}} \bbm 1&2&1\\0&0&0\\0&0&0\ebm.
\]
It follows that if $\vec{x} = \bbm x\\y\\z\ebm$ is a solution to $(A-2I)\vec{x}=\vec{0}$, then $y=s$ and $z=t$ are free variables, while $x+2y+z=0$ gives $x=-2s-t$. Thus,
\[
 \vec{x} = \bbm x\\y\\z\ebm = \bbm -2s-t\\s\\t\ebm = s\bbm -2\\1\\0\ebm + t\bbm -1\\0\\1\ebm.
\]
Since $\vec{u} = \bbm -2\\1\\0\ebm$ and $\vec{v} = \bbm -1\\0\\1\ebm$ are both basic solutions to the homogeneous system $(A-2I)\vec{x}=\vec{0}$, they must both be eigenvectors corresponding to the eigenvalue $\lambda=2$. (You should verify this by computing $A\vec{u}$ and $A\vec{v}$.)

For the eigenvalue $\lambda=6$, we have
\[
 A-6I = \bbm 3-6&2&1\\1&4-6&1\\3&2&1-6\ebm = \bbm -3&2&1\\1&-2&1\\3&2&-5\ebm \xrightarrow{\text{RREF}} \bbm 1&0&-1\\0&1&-1\\0&0&0\ebm.
\]
Thus, any solution $\vec{x}\bbm x\\y\\z\ebm$ to $(A-6I)\vec{x}=\vec{0}$ must satisfy $x=y=z$, where $z=t$ is free. Setting $t=1$ gives us the eigenvector $\vec{w} = \bbm 1\\1\\1\ebm$ corresponding to the eigenvalue $\lambda=6$.


\bigskip

\item Compute the eigenvalues and eigenvectors of the matrix $A = \bbm 1&4\\2&3\ebm$.

\bigskip

We first compute the eigenvalues. We have
\[
 \det(A-xI) = \begin{vmatrix}1-x&4\\2&3-x\end{vmatrix} = (1-x)(3-x)-8 = x^2-4x+3-8 = x^2-4x+5 = (x+1)(x-5).
\]
Thus, $\det(A-xI)=0$ for the eigenvalues $\lambda_1 = -1$ and $\lambda_2=5$.

For the eigenvalue $\lambda_1=-1$, we have
\[
 A-\lambda_1I = A+I = \bbm 2&4\\2&4\ebm \xrightarrow{\text{RREF}} \bbm 1&2\\0&0\ebm,
\]
so any solution $\vec{x} = \bbm x\\y\ebm$ to $(A+\lambda_1I)\vec{x}=\vec{0}$ must satisfy $x=-2y$, where $y$ is free. Setting $y=1$, we get the eigenvector $\vec{x}_1 = \bbm -2\\1\ebm$ corresponding to $\lambda_1=-1$.

For $\lambda_2=5$, we have
\[
 A-\lambda_2I = A-5I = \bbm -4&4\\2&-2\ebm\xrightarrow{\text{RREF}} \bbm 1&-1\\0&0\ebm,
\]
so if $(A-5I)\vec{x}=\vec{0}$, we must have $\vec{x} = \bbm t\\t\ebm$ for some free parameter $t$. Setting $t=1$ gives us $\vec{x}_2 = \bbm 1\\1\ebm$ as the eigenvector corresponding to $\lambda_2=5$.

\newpage


\item Verify that the matrix $Z = \bbm 3&1\\-2&1\ebm$ has eigenvalues $\lambda_\pm = 2\pm i$ with corresponding eigenvectors $\vec{x}_+ = \bbm 1+i\\-2\ebm$, $\vec{x}_- = \bbm 1\\-1-i\ebm$.

\bigskip

We compute 
\[
Z\vec{x}_+ = \bbm 3&1\\-2&1\ebm\bbm 1+i\\-2\ebm = \bbm 3+3i-2\\-2-2i-2\ebm = \bbm 1+3i\\-4-2i\ebm. 
\]
On the other hand,
\[
 \lambda_+\vec{x}_+ = (2+i)\bbm 1+i\\-2\ebm = \bbm (2+i)(1+i)\\(2+i)(-2)\ebm = \bbm 1+3i\\-4-2i\ebm,
\]
so we have verified that $Z\vec{x}_+ = \lambda_+\vec{x}_+$. Similarly,
\[
 Z\vec{x}_- = \bbm 3&1\\-2&1\ebm\bbm 1\\-1-i\ebm = \bbm 3-1-i\\-2-1-i\ebm = \bbm 2-i\\-3-i\ebm = (2-i)\bbm 1\\-1-i\ebm = \lambda_-\vec{x}_-,
\]
as required.
 \end{enumerate}
\end{document}