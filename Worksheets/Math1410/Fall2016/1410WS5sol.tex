\documentclass[12pt]{article}
\usepackage{amsmath}
\usepackage{amssymb}
\usepackage[letterpaper,margin=0.85in,centering]{geometry}
\usepackage{fancyhdr}
\usepackage{enumerate}
\usepackage{lastpage}
\usepackage{multicol}
\usepackage{graphicx}

\reversemarginpar

\pagestyle{fancy}
\cfoot{}
\lhead{Math 1410}\chead{Worksheet \# 5 Solutions}\rhead{October 12/13, 2016}
%\rfoot{Total: 10 points}
%\chead{{\bf Name:}}
\newcommand{\points}[1]{\marginpar{\hspace{24pt}[#1]}}
\newcommand{\skipline}{\vspace{12pt}}
%\renewcommand{\headrulewidth}{0in}
\headheight 30pt

\newcommand{\di}{\displaystyle}
\newcommand{\abs}[1]{\lvert #1\rvert}
\newcommand{\len}[1]{\lVert #1\rVert}
\renewcommand{\i}{\mathbf{i}}
\renewcommand{\j}{\mathbf{j}}
\renewcommand{\k}{\mathbf{k}}
\newcommand{\R}{\mathbb{R}}
\newcommand{\aaa}{\mathbf{a}}
\newcommand{\bbb}{\mathbf{b}}
\newcommand{\ccc}{\mathbf{c}}
\newcommand{\dotp}{\boldsymbol{\cdot}}
\newcommand{\bbm}{\begin{bmatrix}}
\newcommand{\ebm}{\end{bmatrix}}                   
                  
\begin{document}


%\author{Instructor: Sean Fitzpatrick}
\thispagestyle{fancy}
%\noindent{{\bf Name and student number:}}
 \begin{enumerate}
 
\item Consider the vectors $\vec{w} = \bbm 2\\-3\\1\ebm, \vec{v}_1 = \bbm -4\\0\\3\ebm, \vec{v}_2 = \bbm 0\\1\\3\ebm, \vec{v}_3=\bbm 2\\-1\\1\ebm$.\\
Recall that the question ``Does $\vec{w}$ belong to the span of $\{\vec{v}_1,\vec{v}_2,\vec{v}_3\}$?'' is the same as the question ``Are there scalars $x_1,x_2,x_3$ such that $\vec{w} = x_1\vec{v}_1+x_2\vec{v}_2+x_3\vec{v}_3$?''

Show that this question is, in turn, equivalent to the question of whether or not there is a solution to the following system of equations:
\[
 \begin{array}{ccccccc}
  -4x_1 & & &+&2x_3&=&2\\
   & &x_2&-&x_3&=&-3\\
 3x_1&+&3x_2&+&x_3&=&1
 \end{array}
\]
(You do not have to solve the system.)

\bigskip

{\bf Solution:} We begin with the left-hand side of the vector equation $x_1\vec{v}_1+x_2\vec{v}_2+x_3\vec{v}_3=\vec{w}$ and subsitute the vectors $\vec{v}_1,\vec{v}_2,\vec{v}_3$. We then equate the simplified vector to the right-hand side $\vec{w}$, as follows:
\begin{align*}
 x_1\vec{v}_1+x_2\vec{v}_2+x_3\vec{v}_3 &= x_1\bbm -4\\0\\3\ebm+x_2\bbm 0\\1\\3\ebm+x_3\bbm 2\\-1\\1\ebm = \bbm -4x_1\\0\\3x_1\ebm+\bbm 0\\x_2\\3x_2\ebm+\bbm 2x_3\\-x_3\\x_3\ebm\\&= \bbm -4x_1+2x_3\\x_2-x_3\\3x_1+3x_2+x_3\ebm = \bbm 2\\-3\\1\ebm.
\end{align*}
Recalling that two vectors are equal provided that each of their corresponding components are equal, we can equate components, and doing so results in the desired system of equations.

\bigskip

\item Let $U = \left\{\left.\bbm 2x-y\\x+3y\\4y-x\ebm \right|x,y\in\R\right\}$. Find vectors $\vec{a}$ and $\vec{b}$ such that $U=\operatorname{span}\{\vec{a},\vec{b}\}$.

\bigskip

{\bf Solution:} Given a general element $\vec{u}=\bbm 2x-y\\x+3y\\4y-x\ebm\in U$, we have
\[
 \vec{u}=\bbm 2x-y\\x+3y\\4y-x\ebm=\bbm 2x\\x\\-x\ebm+\bbm-y\\3y\\4y\ebm = x\bbm 2\\1\\-1\ebm+y\bbm -1\\3\\4\ebm.
\]
Letting $\vec{a} = \bbm 2\\1\\-1\ebm$ and $\vec{b} = \bbm -1\\3\\4\ebm$, the above equation shows that any element $\vec{u}\in U$ can be written as a linear combination $\vec{u}=x\vec{a}+y\vec{b}$ of the vectors $\vec{a}$ and $\vec{b}$; thus, $\vec{u}$ belongs to $\operatorname{span}\{\vec{a},\vec{b}\}$. Conversely, any element of $\operatorname{span}\{\vec{a},\vec{b}\}$ is a vector of the form $\vec{v}=x\vec{a}+y\vec{b}$. Reversing the above equality, we see that 
\[
 \vec{v}=x\vec{a}+y\vec{b} = \bbm 2x-y\\x+3y\\4y-x\ebm
\]
is an element of $U$. Since $U$ and $\operatorname{span}\{\vec{a},\vec{b}\}$ contain the same vectors, they must be equal.

\bigskip


\item Determine if the following subsets of $\R^2$ are subspaces. Explain your answer.
\begin{enumerate}
 \item $U=\left\{\left.\bbm x\\y\ebm \right| 3x-2y=0\right\}$

\bigskip

{\bf Solution:} The set $U$ is a subspace. We can prove this directly using the subspace test, or by showing that $U$ can be written as a span.

Using the subspace test, we first check that $\vec{0}=\bbm 0\\0\ebm$ belongs to $U$, since $3(0)-2(0)=0$. Now, suppose that $\vec{v}=\bbm a\\b\ebm$ and $\vec{w}=\bbm c\\d\ebm$ are elements of $U$, which means that we must have $3a-2b=0$ and $3c-2d=0$. We then have $\vec{v}+\vec{w} =\bbm  a+c\\b+d\ebm$, and
\[
 3(a+c)-2(b+d) = (3a-2b)+(3c-2d) = 0+0=0,
\]
which shows that $\vec{v}+\vec{w}$ is an element of $U$. Similarly, for any scalar $k$, we have $k\vec{u} = \bbm ka\\kb\ebm$, and
\[
 3(ka)-2(kb) = k(3a-2b) = k(0)=0,
\]
so $k\vec{u}$ belongs to $U$. This shows that $U$ is a subspace.

\medskip

The other approach is to rewrite $U$ as a span. The condition $3x-2y=0$ can be re-written as $y=\dfrac{3}{2}x$; thus, if $\bbm x\\y\ebm$ is an element of $U$, we have
\[
 \bbm x\\y\ebm = \bbm x\\\frac{3}{2}x\ebm = x\bbm 1\\\frac{3}{2}\ebm = x\vec{v},
\]
where $\vec{v} = \bbm 1\\3/2\ebm$. This shows that we can write
\[
 U = \{x\vec{v} | x\in \R\} = \operatorname{span}\{\vec{v}\}.
\]
Since the span of any set of vectors is a subspace, we can conclude that $U$ is a subspace.

 \item $V=\left\{\left.\bbm 2x-1\\x+2\ebm \right| x\in\R\right\}$

\bigskip

{\bf Solution:} The set $V$ is not a subspace, since it does not contain the zero vector $\vec{0} = \bbm 0\\0\ebm$. To see this, suppose that $\bbm 2x-1\\x+2\ebm = \bbm 0\\0\ebm$ for some value of $x$. Looking at the first component, we must have $2x-1=0$, so $x=1/2$. Looking at the second component, we must have $x+2=0$, so $x=-2\neq 1/2$. Thus, there is no value of $x$ that can produce the zero vector.
\end{enumerate}



\item Using only the vector space properties of $\R^n$ (Theorem 19 in Section 4.2), show the following:
\begin{enumerate}
 \item $0\vec{v} = \vec{0}$ for any vector $\vec{v}\in\R^n$. (Hint: use property 10 and the fact that $0+0=0$.)

\bigskip

{\bf Solution:} We proceed as follows:
\begin{align*}
 0\vec{v} & = (0+0)\vec{v} = 0\vec{v}+0\vec{v} \tag*{Distributive property}\\
 -0\vec{v} +0\vec{v} & = -0\vec{v}+(0\vec{v}+0\vec{v}) \tag*{Add $-0\vec{v}$ to both sides}\\
 -0\vec{v}+0\vec{v} & = (-0\vec{v}+0\vec{v})+0\vec{v} \tag*{Associative property}\\
 \vec{0} & = \vec{0} + 0\vec{v} \tag*{Since $-\vec{w}+\vec{w}=\vec{0}$ for any vector $\vec{w}$}\\
 \vec{0} & = 0\vec{v} \tag*{Since $\vec{0}+\vec{w}=\vec{w}$ for any vector $\vec{w}$}
\end{align*}
Thus, we see that $0\vec{v} = \vec{0}$.

\medskip


 \item If $c\vec{v}=\vec{0}$ for some scalar $c$ and vector $\vec{v}$, then either $c=0$ or $\vec{v}=\vec{0}$.\\
(Hint: there are two cases -- either $c$ equals zero, or it doesn't.)

\bigskip

{\bf Solution:} Suppose that $c\vec{v}=\vec{0}$ for some scalar $c$ and vector $\vec{v}$. If $c=0$ then we have our conclusion, so there is nothing to prove. It remains to show that if $c\neq 0$, then we must have $\vec{v}=\vec{0}$, so we suppose that $c\neq 0$. Since $c$ is a nonzero real number, we know that its multiplicative inverse $\dfrac{1}{c}$ is defined. Multiplying both sides of the equation $c\vec{v}=\vec{0}$ by $\dfrac{1}{c}$, we have
\begin{align*}
 \frac{1}{c}(c\vec{v}) &= \frac{1}{c}(\vec{0})\\
 \left(\frac{1}{c}(c)\right)\vec{v} & = \frac{1}{c}(\vec{0}) \tag*{Associativity of scalar multiplication}\\
 (1)\vec{v} & = \frac{1}{c}(\vec{0}) \tag*{Since $c(1/c)=1$ for any real number $c$}\\
 \vec{v} & = \frac{1}{c}(\vec{0}) \tag*{Since $1\vec{v}=\vec{v}$ for any vector $\vec{v}$}
\end{align*}
The last step is to confirm that $\dfrac{1}{c}\vec{0} = \vec{0}$. While true, this isn't actually one of the 10 properties given in Theorem 19. It can, however, be proved using an argument similar to the one in 4(a): for any scalar $k$, $k\vec{0} = k(\vec{0}+\vec{0}) = k\vec{0}+k\vec{0}$, and adding $-k\vec{0}$ to both sides allows us to simplify and deduce that $k\vec{0}=\vec{0}$.
\end{enumerate}

 \end{enumerate}
\end{document}