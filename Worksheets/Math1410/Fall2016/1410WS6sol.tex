\documentclass[12pt]{article}
\usepackage{amsmath}
\usepackage{amssymb}
\usepackage[letterpaper,margin=0.85in,centering]{geometry}
\usepackage{fancyhdr}
\usepackage{enumerate}
\usepackage{lastpage}
\usepackage{multicol}
\usepackage{graphicx}

\reversemarginpar

\pagestyle{fancy}
\cfoot{}
\lhead{Math 1410}\chead{Worksheet \# 6 Solutions}\rhead{October 19/20, 2016}
%\rfoot{Total: 10 points}
%\chead{{\bf Name:}}
\newcommand{\points}[1]{\marginpar{\hspace{24pt}[#1]}}
\newcommand{\skipline}{\vspace{12pt}}
%\renewcommand{\headrulewidth}{0in}
\headheight 30pt

\newcommand{\di}{\displaystyle}
\newcommand{\abs}[1]{\lvert #1\rvert}
\newcommand{\len}[1]{\lVert #1\rVert}
\renewcommand{\i}{\mathbf{i}}
\renewcommand{\j}{\mathbf{j}}
\renewcommand{\k}{\mathbf{k}}
\newcommand{\R}{\mathbb{R}}
\newcommand{\aaa}{\mathbf{a}}
\newcommand{\bbb}{\mathbf{b}}
\newcommand{\ccc}{\mathbf{c}}
\newcommand{\dotp}{\boldsymbol{\cdot}}
\newcommand{\bbm}{\begin{bmatrix}}
\newcommand{\ebm}{\end{bmatrix}}                   
                  
\begin{document}


%\author{Instructor: Sean Fitzpatrick}
\thispagestyle{fancy}
%\noindent{{\bf Name and student number:}}
 \begin{enumerate}
 
\item For the matrices
\[
 A = \bbm 2&-3&3\\1&0&5\ebm, B = \bbm 2&0\\1&-4\ebm, C = \bbm 2&0\\-1&4\\3&2\ebm,
\]
determine which of the products $A^2, AB, AC, BA, B^2, BC, CA, CB, C^2$ are defined.\\
Compute at least {\bf three} of the products that are defined.\\

\bigskip

The following matrix products are defined:
\begin{align*}
 AC&=\bbm 2(2)-3(-1)+3(3)&2(0)-3(4)+3(2)\\1(2)+0(-1)+5(3)&1(0)+0(4)+5(2)\ebm = \bbm 16&-6\\17&10\ebm\\[12pt]
 BA&=\bbm 2(2)+0(1)&2(-3)+0(0)&2(3)+0(5)\\1(2)-4(1)&1(-3)-4(0)&1(3)-4(5)\ebm = \bbm 4&-6&6\\-2&-3&-17\ebm\\[12pt]
 B^2&=\bbm 2(2)+0(1)&2(0)+0(-4)\\1(2)-4(1)&1(0)-4(-4)\ebm = \bbm 4&0\\-2&16\ebm\\[12pt]
 CA&=\bbm 2(2)+0(1)&2(-3)+0(0)&2(3)+0(5)\\-1(2)+4(1)&-1(-3)+4(0)&-1(3)+4(5)\\3(2)+2(1)&3(-3)+2(0)&3(3)+2(5)\ebm = \bbm 4&-6&6\\2&3&17\\8&-9&19\ebm\\[12pt]
 CB&=\bbm 2(2)+0(1)&2(0)+0(-4)\\-1(2)+4(1)&-1(0)+4(-4)\\3(2)+2(1)&3(0)+2(-4)\ebm = \bbm 4&0\\2&-16\\8&-8\ebm
\end{align*}


\bigskip

\bigskip

\item Determine the matrix of the transformation $T:\R^4\to \R^3$ such that
\[
 T\left(\bbm 1\\0\\0\\0\ebm\right) = \bbm 2\\0\\1\ebm, T\left(\bbm 0\\1\\0\\0\ebm\right) = \bbm 0\\-1\\3\ebm, T\left(\bbm 0\\0\\1\\0\ebm\right) = \bbm 1\\7\\5\ebm, \text{ and } T\left(\bbm 0\\0\\0\\1\ebm\right) = \bbm 3\\-1\\4\ebm.
\]


Using the fact that any matrix transformation $T$ maps the standard basis vectors to the corresponding columns of its matrix, we have $T(\vec{x})=A\vec{x}$, where
\[
 A = \bbm 2&0&1&3\\0&-1&7&-1\\1&3&5&4\ebm.
\]
It's straightforward to verify our work by confirming that the values of $T$ on the standard basis vectors are as given above.

\pagebreak


\item Determine the matrix of the transformation $T:\R^2\to\R^2$ that performs the following operations, in order:  First, a horizontal stretch by a factor of 4. Second, a counter-clockwise rotation by $3\pi/4$.  Third, a reflection across the $x$-axis.

\bigskip

The transformation is given as $T(\vec{x}) = A_3(A_2(A_1\vec{x}))$, where
\begin{align*}
 A_1 & = \bbm 4&0\\0&1\ebm  \tag*{performs the horizontal stretch}\\
 A_2 & = \bbm \cos(3\pi/4) & -\sin(3\pi/4)\\\sin(3\pi/4) & \cos(3\pi/4)\ebm = \bbm -1/\sqrt{2} & -1/\sqrt{2}\\1/\sqrt{2} & -1\sqrt{2}\ebm \tag*{perfoms the rotation}\\
 A_3 & = \bbm 1&0\\0&-1\ebm \tag*{perfoms the reflection}
\end{align*}
Thus, we have $T(\vec{x}) = A\vec{x}$, where
\begin{align*}
 A = A_3A_2A_1 & = \bbm 1&0\\0&-1\ebm \bbm -1/\sqrt{2} & -1/\sqrt{2}\\1/\sqrt{2} & -1\sqrt{2}\ebm \bbm 4&0\\0&1\ebm\\
 & = \bbm 1&0\\0&-1\ebm\bbm -2\sqrt{2} & -1/\sqrt{2}\\2\sqrt{2}&-1\sqrt{2}\ebm = \bbm -2\sqrt{2} & -1/\sqrt{2}\\-2\sqrt{2} & 1/\sqrt{2}\ebm.
\end{align*}
Note: we can also determine $T$ by tracking the standard basis vectors. We have
\begin{align*}
 \bbm 1\\0\ebm \xrightarrow{\text{horiz. stretch}} \bbm 4\\0\ebm \xrightarrow{\text{rotation}} \bbm -2\sqrt{2}\\2\sqrt{2}\ebm \xrightarrow{\text{reflection}}\bbm -2\sqrt{2}\\-2\sqrt{2}\ebm\\
 \bbm 0\\1\ebm \xrightarrow{\text{horiz. stretch}} \bbm 0\\1\ebm \xrightarrow{\text{rotation}} \bbm -1/\sqrt{2}\\-1/\sqrt{2}\ebm \xrightarrow{\text{reflection}}\bbm -1/\sqrt{2}\\1/\sqrt{2}\ebm
\end{align*}
The two resulting vectors form the two columns of our matrix $A$, resulting in the same answer as before.

\bigskip

\item For fun: Find a $2\times 2$ matrix $A$ such that $A^{12}$ is the identity matrix, but $A^k$ is not for $1\leq k\leq 11$. (Hint: rotation.)

\bigskip

Consider the rotation matrix $A = \bbm \cos(\pi/6) & -\sin(\pi/6)\\\sin(\pi/6) & \cos(\pi/6)\ebm = \bbm \sqrt{3}/2 & -1/2\\1/2 & \sqrt{3}/2\ebm$.
Multiplying a vector $\bbm x\\y\ebm$ on the left by $A$ corresponds to a rotation through an angle of $\pi/6$. It follows that repeated multiplication corresponds to repeated rotation:

Multiplication by $A^2$ is rotation by $2(\pi/6)=\pi/3$, multiplication by $A^3$ is rotation by $3(\pi/6) = \pi/2$, and so on, up to multiplication by $A^{11}$, which corresponds to rotation by $11\pi/6$, and then $A^{12}$, which represents a rotation by $12\pi/6 = 2\pi$. But a rotation by $2\pi$ returns everything to where we started, and it's easy to check that the rotation matrix for $\theta = 2\pi$ is the identity matrix.
 \end{enumerate}
\end{document}