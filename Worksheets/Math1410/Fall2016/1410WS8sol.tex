\documentclass[12pt]{article}
\usepackage{amsmath}
\usepackage{amssymb}
\usepackage[letterpaper,margin=0.85in,centering]{geometry}
\usepackage{fancyhdr}
\usepackage{enumerate}
\usepackage{lastpage}
\usepackage{multicol}
\usepackage{graphicx}

\reversemarginpar

\pagestyle{fancy}
\cfoot{}
\lhead{Math 1410}\chead{Worksheet \# 8 Solutions}\rhead{November 2/3, 2016}
%\rfoot{Total: 10 points}
%\chead{{\bf Name:}}
\newcommand{\points}[1]{\marginpar{\hspace{24pt}[#1]}}
\newcommand{\skipline}{\vspace{12pt}}
%\renewcommand{\headrulewidth}{0in}
\headheight 30pt

\newenvironment{amatrix}[1]{%
  \left[\begin{array}{@{}*{#1}{c}|c@{}}
}{%
  \end{array}\right]
}

\newcommand{\di}{\displaystyle}
\newcommand{\abs}[1]{\lvert #1\rvert}
\newcommand{\len}[1]{\lVert #1\rVert}
\renewcommand{\i}{\mathbf{i}}
\renewcommand{\j}{\mathbf{j}}
\renewcommand{\k}{\mathbf{k}}
\newcommand{\R}{\mathbb{R}}
\newcommand{\aaa}{\mathbf{a}}
\newcommand{\bbb}{\mathbf{b}}
\newcommand{\ccc}{\mathbf{c}}
\newcommand{\dotp}{\boldsymbol{\cdot}}
\newcommand{\bbm}{\begin{bmatrix}}
\newcommand{\ebm}{\end{bmatrix}}                   
\newcommand{\bam}{\begin{amatrix}}
\newcommand{\eam}{\end{amatrix}}   
\DeclareMathOperator{\rank}{rank}
                
\begin{document}

%\author{Instructor: Sean Fitzpatrick}
\thispagestyle{fancy}
%\noindent{{\bf Name and student number:}}
 \begin{enumerate}
\item Determine the rank of each of the following matrices:
\begin{multicols}{2}
\begin{enumerate}
 \item $A = \bbm 2&-3&1&4\\-1&3&5&-7\\1&0&6&-3\ebm$

\medskip

The reduced row-echelon form of $A$ is
\[
 \bbm 1&0&6&-3\\0&1&11/3&-10/3\\0&0&0&0\ebm.
\]
Since there are two leading ones, we have $\rank(A)=2$.

 \item $B = \bbm 2&6\\5&-3\\3&2\ebm$

\medskip

The reduced row-echelon form of $B$ is
\[
 \bbm 1&0\\0&1\\0&0\ebm.
\]
Since there are two leading ones, we have $\rank(B)=2$.


\end{enumerate}
\end{multicols}

\vspace{0.5in}

\item Determine the basic solutions of the homogeneous system of equations
\[
 \begin{array}{ccccccccc}
  2x_1&-&3x_2& &   &-&4x_4&=&0\\
  -x_1&+&2x_2&-&x_3&+&3x_4&=&0\\
  3x_1&-&4x_2&-&x_3&-&5x_4&=&0
 \end{array}
\]

The system can be written in the form $A\vec{x}=\vec{0}$, where $A = \bbm 2&-3&0&-4\\-1&2&-3&3\\3&-4&-1&-5\ebm$. The reduced row-echelon form of the augmented matrix $[\begin{array}{c|c} A&0\end{array}]$ is
\[
 \bam{4}1&0&-3&1&0\\0&1&-2&2&0\\0&0&0&0&\eam
\]
From the reduced row-echelon form, we can see that $x_3=s$ and $x_4=t$ are free parameters, while $x_1-3x_3+x_4=0$ and $x_2-2x_3+2x_4=0$. Solving for $x_1$ and $x_2$, we have $x_1=3x_3-x_4=3s-t$, and $x_4 = 2x_3-2x_4 = 2s-2t$. Writing our solution in vector form, we have
\[
 \bbm x_1\\x_2\\x_3\\x_4\ebm = \bbm 3s-t\\2s-2t\\s\\t\ebm = s\bbm 3\\2\\1\\0\ebm+t\bbm -1\\-2\\0\\1\ebm,
\]
so our basic solutions are $\vec{v}_1 = \bbm 3\\2\\1\\0\ebm$ and $\vec{v}_2 = \bbm -1\\-2\\0\\1\ebm$.


\newpage

\item Determine whether or not the vectors
\[
 \vec{v}_1 = \bbm 2\\-1\\3\ebm, \vec{v}_2 = \bbm 0\\1\\-4\ebm, \quad \text{ and } \quad \vec{v}_3 = \bbm 3\\-1\\5\ebm
\]
are linearly independent.

\bigskip

We recall that the vectors $\vec{v}_1, \vec{v}_2, \vec{v}_3$ are linearly independent provided that the only solution to the vector equation
\[
 x_1\vec{v}_1+x_2\vec{v}_2+x_3\vec{v}_3 = \vec{0}
\]
is $x_1=0, x_2=0, x_3=0$. Our vector equation can be re-written as a system of linear equations with augmented matrix
\[
 \bam{3}2&0&3&0\\-1&1&-1&0\\3&4&5&0\eam.
\]
The reduced row-echelon form of this matrix is $\bam{3}1&0&0&0\\0&1&0&0\\0&0&1&0\eam$, which shows that we have the unique solution $x_1=0,x_2=0,x_3=0$, and thus, our vectors are linearly independent.

\bigskip


\item Determine whether or not $\vec{w}\in \operatorname{span}\{\vec{v}_1,\vec{v}_2, \vec{v}_3\}$, where
\[
 \vec{v}_1 = \bbm 1\\0\\2\\-1\ebm, \vec{v}_2 = \bbm 2\\1\\0\\-3\ebm, \vec{v}_3 = \bbm -2\\0\\-4\\1\ebm, \quad \text{ and } \quad \vec{w} = \bbm 2\\3\\-8\\6\ebm.
\]

\bigskip

The vector $\vec{w}$ belongs to the span of the vectors $\vec{v}_1,\vec{v}_2,\vec{v}_3$ provided that there exist scalars $x_1,x_2,x_3$ such that
\[
 x_1\vec{v}_1+x_2\vec{v}_2+x_3\vec{v}_3=\vec{w}.
\]
This vector equation leads to a system of linear equations with augmented matrix 
\[
 \bam{3}1&2&-2&2\\0&1&0&3\\2&0&-4&-8\\-1&-3&1&6\eam.
\]
The reduced row-echelon form of this matrix is
\[
 \bam{3}1&0&0&-26\\0&1&0&3\\0&0&1&-11\\0&0&0&0\eam.
\]
Since a solution exists, we can conclude that $\vec{w}\in \operatorname{span}\{\vec{v}_1,\vec{v}_2, \vec{v}_3\}$. Indeed, we have the unique solution $x_1=-26, x_2=3, x_3=-11$, and we can verify that
\[
 -26\vec{v}_1+3\vec{v}_2-11\vec{v}_3 = -26\bbm 1\\0\\2\\-1\ebm +3\bbm 2\\1\\0\\-3\ebm-11\bbm -2\\0\\-4\\1\ebm = \bbm 2\\3\\-8\\6\ebm = \vec{w}.
\]


\vspace{0.5in}

\textbf{Note for Problem 2:} Since the null space of $A$ is defined to be the set of all vectors $\vec{x}$ such that $A\vec{x}=\vec{0}$, we see that
\[
 \operatorname{null}(A) = \left\{\left.  \bbm 3s-t\\2s-2t\\s\\t\ebm \,\right|\, s,t\in\R\right\} = \operatorname{span}\{\vec{v}_1,\vec{v}_2\},
\]
and thus, the set $\{\vec{v}_1,\vec{v}_2\}$ forms a basis for the null space of $A$. (The vectors $\vec{v}_1$ and $\vec{v}_2$ are not multiples of each other, so this set is clearly linearly independent, in addition to spanning the null space.)

According to Theorem 29 in the textbook, a basis for the column space of $A$ is given by the columns of $A$ in which we find leading ones in the RREF of $A$. Since we had leading ones in the first and second columns, it follows that
\[
 \left\{\bbm 2\\-1\\3\ebm, \bbm 3\\2\\4\ebm\right\}
\]
is a basis for the column space of $A$. (Recall that the column space of $A$ is defined to be the span of the columns of $A$. The above result tells us that the first two columns of $A$ are sufficient to generate this span. Consequently, the third and fourth columns must both be linear combinations of the first two. Recall also that if $\vec{y}=A\vec{x}$ for some vector $\vec{x}$, then $\vec{y}$ can be written as a linear combination of the columns of $A$; that is, the column space of $A$ determines the \textit{range} of the matrix transformation $T(\vec{x}) = A\vec{x}$.)
 \end{enumerate}
\end{document}