\documentclass[letterpaper,12pt]{article}

%\usepackage{ucs}
%\usepackage[utf8x]{inputenc}
\usepackage{amsmath}
\usepackage{amsfonts}
\usepackage{amssymb}
%\usepackage[canadian]{babel}
\usepackage[margin=1in]{geometry}
\usepackage{multicol}
%\usepackage[dvips]{hyperref}
\newcommand{\divs}[2]{#1 \, | \, #2}
\renewcommand{\cong}[3]{#1 \equiv #2 \pmod{#3}}

%\author{Sean Fitzpatrick}
\date{October 21, 2015}
\title{Math 2000 Tutorial Worksheet}
\begin{document}
\maketitle

 This week's tutorial will focus on proofs by induction. You should focus on two things with this proofs: the overall structure of the proof, and making sure that your logic is correct in the inductive step (proving that $P(k)\to P(k+1)$). Please discuss the following problems with your classmates:
\begin{enumerate}
 \item (Section 4.1 \# 3) Use mathematical induction to prove the following:
 
 \begin{enumerate}
 \item For each natural number $n$, $2+5+\cdots + (3n-1) = \dfrac{n(3n+1)}{2}$.
 \item For each natural number $n$, $1+5+\cdots + (4n-3) = n(2n-1)$.
 \item For each natural number $n$, $1^3+2^3+\cdots + n^3 = \left[\dfrac{n(n+1)}{2}\right]^2$.
 \end{enumerate}
 
 \item (Section 4.1 \# 8) Use mathematical induction to prove the following:
 \begin{enumerate}
 \item For each natural number $n$, 3 divides $(4^n-1)$.
 \item For each natural number $n$, 6 divides $(n^3-n)$.
 \end{enumerate}
 \item (Section 4.1 \# 12) Let $x$ and $y$ be integers. Prove that for each natural number $n$, $(x-y)$ divides $(x^n-y^n)$.
 
 \item If someone has a copy of the textbook handy, in a group, you should read and discuss problems 18, 19, and 20 in Section 4.1.
 
 \item (Section 4.2 \#1) Use mathematical induction to prove the following:
 \begin{enumerate}
 \item For each natural number $n$ with $n\geq 2$, $3^n>1+2^n$.
 \item For each natural number $n$ with $n\geq 6$, $2^n>(n+1)^2$.
 \item For each natural number $n$ with $n\geq 3$, $\left(1+\dfrac{1}{n}\right)^n<n$.
 \end{enumerate}
 \item (Section 4.2 \#7) For which positive integers $n$ do there exist nonnegative integers $x$ and $y$ such that $n=4x+5y$? Justify your conclusion.
\end{enumerate}

\end{document}
 
