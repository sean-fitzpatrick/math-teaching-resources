\documentclass[letterpaper,12pt]{article}

%\usepackage{ucs}
%\usepackage[utf8x]{inputenc}
\usepackage{amsmath}
\usepackage{amsfonts}
\usepackage{amssymb}
%\usepackage[canadian]{babel}
\usepackage[margin=1in]{geometry}
\usepackage{multicol}
%\usepackage[dvips]{hyperref}

%\author{Sean Fitzpatrick}
%\date{2015-09-15}
\title{Math 2000 Tutorial Worksheet}
\begin{document}
\maketitle

 For this week's tutorial, begin by discussing what it means for a sentence to be a ``statement'' in mathematics, and make sure that you can identify the hypothesis and conclusion in a conditional statement. Under what conditions is a conditional statement true? When is it false?

 If you're confident that you understand the above, then you should try some of the following problems with your classmates:
\begin{enumerate}
 \item (Section 1.1 \#5) Let $P$ be the statement ``Student $X$ passed every assignment in Calculus I'', and let $Q$ be the statement ``Student $X$ received a grade of $C$ or better in Calculus I''.
\begin{enumerate}
 \item What does it mean for $P$ to be true? What does it mean for $Q$ to be true?
\item Suppose that Student $X$ passed every assignment in Calculus I and received a grade of B-, and that the instructor made the statement $P\to Q$. Would you say that the instructor lied, or told the truth?
\item Suppose that Student $X$ passed every assignment in Calculus I and received a grade of C-, and that the instructor made the statement $P\to Q$. Would you say that the instructor lied, or told the truth?
\item Now suppose that Student $X$ failed two assignments in Calculus I and received a grade of D, and that the instructor made the statement $P\to Q$. Would you say that the instructor lied, or told the truth?
\end{enumerate}
\item (Section 1.1 \#7) The following is the statement of a theorem that can be proved using the quadratic formula. (You do not need to know how the theorem is proved.)

\noindent {\bf Theorem:} Let $a$, $b$, and $c$ be real numbers. If $f$ is a quadratic function of the form $f(x)=ax^2+bx+c$ and $ac<0$, then the graph of $f$ has two $x$-intercepts.

Using {\bf only} this theorem, what can you conclude about the graphs of the following functions?
\begin{multicols}{3}
 \begin{enumerate}
  \item $g(x)=-8x^2+5x-2$
  \item $h(x) = -\frac{1}{3}x^2+3x$
  \item $k(x) = 8x^2-5x-7$
  \item $j(x) = -\frac{77}{91}x^2+210$
  \item $f(x) = -4x^2-3x+7$
  \item $F(x) = -x^4+x^3+9$
 \end{enumerate}
\end{multicols}
\pagebreak
\item (Section 1.2 \#2) For each of the following statements, first construct a ``know-show table'' (two-column proof) and then write a formal proof in paragraph form:
\begin{enumerate}
 \item If $x$ is an even integer and $y$ is an even integer, then $x+y$ is an even integer.
 \item If $x$ is an even integer and $y$ is an odd integer, then $x+y$ is an odd integer.
 \item If $x$ is an odd integer and $y$ is an odd integer, then $x+y$ is an odd integer.
\end{enumerate}
\item (Section 1.2 \#8) Is the following statement true or false?
\begin{quotation}
 If $a$ and $b$ are nonnegative real numbers and $a+b=0$, then $a=0$.
\end{quotation}
Either give a counterexample to show that it is false, or give the outline of a proof to argue that it is true.
\end{enumerate}

\end{document}
 
