\documentclass[12pt]{article}
\usepackage{amsmath}
\usepackage{amssymb}
\usepackage[letterpaper,top=1.2in,bottom=1in,left=0.75in,right=0.75in,centering]{geometry}
%\usepackage{fancyhdr}
\usepackage{enumerate}
%\usepackage{lastpage}
\usepackage{multicol}
\usepackage{graphicx}

\reversemarginpar

%\pagestyle{fancy}
%\cfoot{}
%\lhead{Math 1560}\chead{Test \# 1}\rhead{May 18th, 2017}
%\rfoot{Total: 10 points}
%\chead{{\bf Name:}}
\newcommand{\points}[1]{\marginpar{\hspace{24pt}[#1]}}
\newcommand{\skipline}{\vspace{12pt}}
%\renewcommand{\headrulewidth}{0in}
\headheight 30pt

\newcommand{\di}{\displaystyle}
\newcommand{\abs}[1]{\lvert #1\rvert}
\newcommand{\len}[1]{\lVert #1\rVert}
\renewcommand{\i}{\mathbf{i}}
\renewcommand{\j}{\mathbf{j}}
\renewcommand{\k}{\mathbf{k}}
\newcommand{\R}{\mathbb{R}}
\newcommand{\aaa}{\mathbf{a}}
\newcommand{\bbb}{\mathbf{b}}
\newcommand{\ccc}{\mathbf{c}}
\newcommand{\dotp}{\boldsymbol{\cdot}}
\newcommand{\bbm}{\begin{bmatrix}}
\newcommand{\ebm}{\end{bmatrix}}                   
                  
\begin{document}


\author{Instructor: Sean Fitzpatrick}
\thispagestyle{empty}
\vglue1cm
\begin{center}
{\bf MATH 2565 - Tutorial \#8 Solutions}
\end{center}


\textbf{Additional practice}
\begin{enumerate}
\item Solve by separation of variables:
\begin{enumerate}
 \item $x^5y'+y^5=0$

\medskip

First, we note that $y=0$ is a solution. If $y\neq 0$,  $x^5\dfrac{dy}{dx}=-y^5$, so $y^{-5}dy = -x^{-5}dx$. Integrating both sides, we get
 \[
 -\frac14 y^{-4} = \frac14 x^{-4}+C.
 \]
 Multiplying both sides by $-4$, we get $\dfrac{1}{y^4}=k-\dfrac{1}{x^4}$, where $k=-4C$. 
 
 \item $xy'=(1-4x^2)\tan y$
 
 Again, one solution is $y=0$. In fact, $y=k\pi$ is a solution for each integer $k$. If $y\neq k\pi$ (in fact, we need to avoid $y=k\pi/2$ since $\tan y$ is undefined at odd multiples of $\pi/2$) we have
 \[
 x\dfrac{dy}{dx}=(1-4x^2)\tan y \quad\Rightarrow\quad \cot y\,dy = \left(\frac1x -4x\right)\,dx,
 \]
 so $\ln\abs{\sin y} = \ln\abs{x}-2x^2+C$.
 
 \item $xy^2-y'x^2=0$

 Once more, $y=0$ is a solution. If $y\neq 0$, we have $x^2\dfrac{dy}{dx}=xy^2$, so $y^{-2}\,dy = x^{-1}\,dx$, giving $-y^{-1} = \ln\abs{x}+C$, or
 \[
 y = \frac{-1}{\ln\abs{x}+C}.
 \]
 
 \item $(1+x^2)\,dy+(1+y^2)\,dx=0$

 Separation of variables gives $\dfrac{1}{1+y^2}\,dy = -\dfrac{1}{1+x^2}\,dx$, so $\arctan(y) = -\arctan(x)+C$. If desired, we can solve for $y$:
 \[
 y = \tan(C-\arctan(x)) = \frac{\tan(C)-x}{1+\tan{C}x} = \frac{k-x}{1+kx},
 \]
 where $k=\tan(C)$ is a constant.
\end{enumerate}
\newpage


\item Solve the linear differential equation:

\begin{enumerate}
 \item $y'+y=\dfrac{1}{1+e^{2x}}$

Using the integrating factor $P(x)=e^x$, we get
\[
e^xy'+e^xy = \frac{d}{dx}(e^xy) = \dfrac{e^x}{1+e^{2x}}.
\]
Integrating both sides, we find $e^xy = \arctan(e^x)+C$, so
\[
y = e^{-x}(C+\arctan(e^x)).
\]

 \item $xy'-3y=x^4$

We have $y'-\frac{3}{x}y = x^3$, for $x\neq 0$. Our integrating factor is
\[
P(x) = e^{\int (-3/x)\,dx} = e^{-3\ln x} = x^{-3}.
\]
Multiplying by $x^{-3}$ gives us
\[
x^{-3}y'-3x^{-4}y = \frac{d}{dx}(x^{-3}y) = 1,
\]
so $x^{-3}y = x+C$, giving $y = x^4+Cx^3$.

 \item $y'+y\cot x = 2x\csc x$

The integrating factor is
\[
P(x) = e^{\int \cot x\,dx} = e^{\ln(\sin(x))} = \sin(x).
\]
Multiplying by $P(x)$, we get
\[
\sin(x)y'+\cos(x) y = \frac{d}{dx}(\sin(x)y) = 2x,
\]
so $\sin(x)y=x^2+C$, and $y=(x^2+C)\csc(x)$.

 \item $(1+x)y'+4y=3x$

Dividing by $1+x$, with the restriction $x\neq -1$, we have $y'+\frac{4}{1+x}y = \frac{3x}{1+x}$.

This gives us $P(x) = e^{\int 4/(1+x)\,dx} = e^{4\ln(x+1)} = (x+1)^4$, and multiplying by $P(x)$, we get
\[
(1+x)^4y'+4(1+x)^3y \frac{d}{dx}((1+x)^4y)= 3x(1+x)^4.
\]
Noting that
\[
\int x(x+1)^4\,dx = \int (u-1)u^4\,du = \frac{u^6}{6}-\frac{u^5}{5} = \frac{(x+1)^6}{6}-\frac{(x+1)^5}{5},
\]
we get
\[
(1+x)^4y = \frac12 (1+x)^6-\frac35 (1+x)^5+C,
\]
so
\[
y=\frac12 (1+x)^2-\frac35 (1+x) + \frac{C}{(1+x)^4}.
\]

\end{enumerate}


\item Determine if the series converges or diverges:

\begin{enumerate}
 \item $\di \sum_{n=1}^\infty \frac{7^n}{6^n}$ diverges. (Geometric with $r=7/6>1$)
 
 \item $\di \sum_{n=1}^\infty \frac{10}{n!} = 10 \sum_{n=1}^\infty\frac{1}{n!} = 10e$ converges. 
 \item $\di \sum_{n=1}^\infty\left(\frac{1}{n!}+\frac{1}{n}\right)$ diverges since $\di \sum_{n=1}^\infty\frac1n$ diverges.
\end{enumerate}

\end{enumerate}

\newpage
%\thispagestyle{empty}


\textbf{Assigned problems}


 \begin{enumerate}
\item Solve the initial value problem:
\begin{enumerate}
\item $\dfrac{y'}{1+x^2}=\dfrac{x}{y}$, $y(1)=3$.

Using separation of variables, we get $2y\,dy = 2x(1+x^2)=x+x^3\,dx$, (I multiplied by 2 to simplify the antiderivatives in the next step) so
\[
y^2 = x^2+\frac12 x^4+C.
\]
When $x=1$, $3^2 = 1 +\frac12 +C$, giving $C= 9-\frac32=\frac{15}{2}$, so
\[
y = \sqrt{x^2+\frac12 x^4+\frac{15}{2}}.
\]
(Since $y(1)=3>0$, we take the positive square root.)


\item $y'-2xy = 6xe^{x^2}$, $y(0)=1$.

This is linear, with integrating factor $P(x) = e^{\int(-2x)\,dx} = e^{-x^2}$. We find
\[
\frac{d}{dx}(e^{-x^2}y) = e^{-x^2}y' -2xe^{-x^2}y = 6x,
\]
so $e^{-x^2}y=3x^2+C$. When $x=0$, $y=1$, so $e^{0}(1)=0+C$, giving $C=1$ and $y=(3x^2+1)e^{x^2}$.

\item $x^2y'+xy=2x$, $y(1)=4$.

Dividing by $x^2$, $x\neq 0$ (note that our initial condition forces $x>0$ for the domain of solution) we get
\[
y' + \frac1x y = \frac2x.
\]
Our integrating factor is $P(x) = e^{\int 1/x\,dx} = x$, so 
\[
xy'+y = \frac{d}{dx}(xy) = 2.
\]
Integrating gives $xy = 2x+C$, and when $x=1$, $y=4$, so $(1)(4)=2(1)+C$, giving $C=2$, and $y=2+\dfrac{2}{x}$.
\end{enumerate}
 \newpage
 
\item Initially 5 grams of salt are dissolved in 20 litres of water.
Brine with concentration of salt 2 grams of salt per litre is
added at a rate of 3 litres a minute. The tank is mixed well
and is drained at 3 litres a minute. How long does the process have to continue until there are 20 grams of salt in the
tank?

\bigskip

Let $x(t)$ denote the amount of salt at time $t$. We have $x(0)=0$, and since the volume is a constant 20 l (the flow rate in equals the flow rate out), the concentration of salt in the tank is $C_x = x/20$. 

Salt is flowing in at a rate of $(2\text{ g/l})\times (3\text{ l/min})=6\text{ g/min}$ and flowing out at a rate of $(C_x\text{ g/l})\times (3\text{ l/min}) = 3x/20\text{ g/min}$. Thus,
\[
x'(t) = 6-\frac{3x(t)}{20}, \text{ or } x'(t)+\frac{3}{20}x(t)=6.
\]
Multiplying both sides by the integrating factor $e^{3t/20}$, we get
\[
\frac{d}{dt}(e^{3t/20}x(t))=6e^{3t/20},
\]
so $e^{3t/20}x(t) = 40e^{3t/30}+C$. Since $x(0)=5$, we have $5=40+C$, so $C=-35$. Thus,
\[
x(t) = 40-35e^{-3t/30}.
\]

\medskip

When there are 20 grams of salt in the tank, we have 
\[
x(t)=20=40-35e^{-3t/20},
\]
so $35e^{-3t/20}=20$, giving $e^{-3t/20}=\dfrac47$, so $t=\dfrac{20}{3}\ln\left(\frac{7}{4}\right)=\approx 3.73$ minutes.
\newpage

\item Find the limit of the sequence, or explain why it does not converge:\begin{enumerate}
 \item $a_n = \dfrac{n-1}{n}-\dfrac{n}{n-1}$

\[
\lim_{n\to\infty}a_n = \lim_{n\to \infty}\left(\frac{n-1}{n}\right)-\lim_{n\to\infty}\left(\frac{n}{n-1}\right) = 1-1=0.
\]

 \item The sequence $\{a_n\}$ defined by $a_1 = \sqrt{2}$ and $a_{n+1} = \sqrt{2a_n}$ for all $n\geq 1$. (You may assume that the sequence converges.)
 
 Assuming the limit exists, let $a=\lim_{n\to\infty}a_n$. Since the square root function is continuous,
 \[
 a = \lim_{n\to\infty}a_{n+1} = \lim_{n\to\infty}\sqrt{2a_n} = \sqrt{2\lim_{n\to\infty}a_n} = \sqrt{2a}.
 \]
 Squaring both sides, we get $a^2=2a$, so there are two possibilities: $a=0$, or $a=2$. We know that $a_1=\sqrt{2}$ is between these two values, so we might ask: is the sequence increasing or decreasing?
 
Computing the first few terms suggests that the sequence should be increasing. To confirm, note that $a_2 = \sqrt{2\sqrt{2}}=\sqrt[4]{2}\sqrt{2} = \sqrt[4]{2}a_1>a_1$, and if we assume that $a_{n+1}>a_n$ for some $n\geq 1$, then
\[
a_{n+2} = \sqrt{2a_{n+1}}>\sqrt{2a_n} = a_{n+1}.
\]
It follows by induction that the sequence is increasing. A similar argument shows that the sequence is bounded above by 2. This guarantees that the limit exists, and the calculation above shows that the limit must be equal to 2.
 
\end{enumerate}
\item Determine if the series converges or diverges:
\begin{enumerate}

 \item $\di \sum_{n=1}^\infty \frac{\sqrt{n}+1}{n^2} = \sum_{n=1}^\infty \frac{1}{n^{3/2}} +\sum_{n=1}^\infty \frac{1}{n^2}$ converges since it can be written as the sum of two convergent series. (Both are $p$-series with $p>1$.)



 \item $\di \sum_{n=1}^\infty \frac{3^n}{5^n} = \sum_{n=1}^\infty \left(\frac35\right)^n$ converges, since it's a geometric series with $r=3/5<1$.
 
 (The limit is $\frac{3/5}{1-3/5} = \frac{3}{2}$.)


 \item $\di \sum_{n=1}^\infty e^{-n} = \sum_{n=1}^\infty (e^{-1})^n$ is a geometric series with $r=e^{-1}<1$, so it converges.
 
 (The limit is $\dfrac{1/e}{1-1/e} = \dfrac{1}{e-1}$.) 
 
 
 
 \item $\di \sum_{n=1}^\infty \ln\left(\frac{n}{n+1}\right) = \sum_{n=1}^\infty(\ln n - \ln(n+1))$.
 
 This is a telescoping series; however, the $N^{\text{th}}$ partial sum is
 \[
 S_N = (\ln(1)-\ln(2))+(\ln(2)-\ln(3))+\cdots + (\ln(N)-\ln(N+1)) = -\ln(N+1) 
 \]
 so the series diverges, since $\lim_{N\to\infty}S_N = -\infty$.
\end{enumerate}
\end{enumerate}
\end{document}