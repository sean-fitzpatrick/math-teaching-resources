\documentclass[12pt]{article}
\usepackage{amsmath}
\usepackage{amssymb}
\usepackage[letterpaper,margin=0.85in,centering]{geometry}
\usepackage{fancyhdr}
\usepackage{enumerate}
\usepackage{lastpage}
\usepackage{multicol}
\usepackage{graphicx}

\reversemarginpar

\pagestyle{fancy}
\cfoot{}
\lhead{Math 2560}\chead{Worksheet \# 11 Solutions}\rhead{Thursday 7\textsuperscript{th} April, 2016}
%\rfoot{Total: 10 points}
%\chead{{\bf Name:}}
\newcommand{\points}[1]{\marginpar{\hspace{24pt}[#1]}}
\newcommand{\skipline}{\vspace{12pt}}
%\renewcommand{\headrulewidth}{0in}
\headheight 30pt

\newcommand{\di}{\displaystyle}
\newcommand{\abs}[1]{\left\lvert #1\right\rvert}
\newcommand{\len}[1]{\lVert #1\rVert}
\renewcommand{\i}{\mathbf{i}}
\renewcommand{\j}{\mathbf{j}}
\renewcommand{\k}{\mathbf{k}}
\newcommand{\R}{\mathbb{R}}
\newcommand{\aaa}{\mathbf{a}}
\newcommand{\bbb}{\mathbf{b}}
\newcommand{\ccc}{\mathbf{c}}
\newcommand{\dotp}{\boldsymbol{\cdot}}
\newcommand{\bbm}{\begin{bmatrix}}
\newcommand{\ebm}{\end{bmatrix}}                   
                  
\begin{document}


%\author{Instructor: Sean Fitzpatrick}
\thispagestyle{fancy}
%\noindent{{\bf Name and student number:}}

Here are solutions to some of the problems from the textbook that I recommended for yesterday's tutorial.

\begin{itemize}
 \item Section 3.5, \#26: We're given the alternating series $\di \sum_{i=0}^n\frac{(-1)^n}{n!} = \frac{1}{e}$ (this is the Maclaurin series for $f(x)=e^x$ evaluated at $x=-1$), and want to approximate the value of $1/e$ with an error less than $\varepsilon=0.0001$. The \textbf{Alternating Series Approximation Theorem} (Theorem 26 in the textbook) tells us that for an alternating series $\di\sum_{n=k}^\infty (-1)^na_n$, where $\{a_n\}$ is a positive, decreasing sequence with $\lim a_n = 0$ (that is, one for which the Alternating Series Test guarantees convergence of the series), the truncation error (that is, the error introduced by approximating the series by a partial sum) satisfies
\[
 \abs{\sum_{n=N+1}^\infty (-1)^na_n} = \abs{\sum_{n=k}^\infty(-1)^na_n - \sum_{n=k}^N(-1)^na_n} < a_{N+1}.
\]
So to ensure our approximation for $1/e$ has an error less than $0.0001$, we need to find the value of $N$ such that $a_{N+1}<0.0001$. We have the following values:
\[
 \begin{array}{|c|ccccccccc|}
  \hline
  n&0&1&2&3&4&5&6&7&8\\
 n!&1&1&2&6&24&120&720&5040&40320\\
\hline
 \end{array}
\]
Seeing that $a_7 = \dfrac{1}{5760}\approx 0.000198>\varepsilon$, and $a_8 = \dfrac{1}{40320} \approx 0.0000248<\varepsilon$, we have the approximation
\[
 \frac{1}{e}\approx 1-1+\frac{1}{2}-\frac{1}{6}+\frac{1}{24}-\frac{1}{120}+\frac{1}{720}-\frac{1}{5040} \approx 0.3679
\]
to four decimal places, with an error of $\pm 0.0001$.
\item Section 3.5, \#28: We take the same approach as above, for the series $\di \sum_{n=0}^\infty \frac{(-1)^n}{(2n)!} = \cos 1$, with a desired error of $\varepsilon=10^{-8}$. We again compute the values of $a_n$, beginning at $n=0$. We have:
\[
 \begin{array}{|c|ccccccc|}
  \hline
  n&0&1&2&3&4&5&6\\
  (2n)!& 1&2&24&720&40320&3628800&479001600\\
\hline
\end{array}
\]
A quick check on the calculator tells us that $\dfrac{1}{10!} \approx 2.76 \times 10^{-7} >\varepsilon$, but $\dfrac{1}{12!} \approx 2.09\times 10^{-9}<\varepsilon$. Thus, we have the approximation
\[
 \cos 1 \approx 1-\frac{1}{2}+\frac{1}{24}-\frac{1}{720}+\frac{1}{40320}-\frac{1}{3628800} \approx 0.54030230
\]
to eight decimal places, with an error of $\pm 0.00000001$.

\item Section 3.6, \#10: We're given the power series $\di \sum_{n=1}^\infty nx^n$. By the ratio test, this series converges if
\[
 \lim_{n\to \infty}\abs{\frac{(n+1)x^{n+1}}{nx^n}} = \lim_{n\to\infty}\frac{n+1}{n}\abs{x} = \abs{x}<1,
\]
so the radius of convergence is $r=1$. We are thus guaranteed convergence for $-1<x<1$, and to find the interval of convergence, we need to check the endpoints. In this case, we can see that the series $\sum n(-1)^n$ and $\sum n(1)^n$ both diverge, by the test for divergence, so the interval of convergence is $(-1,1)$.

\item Section 3.6, \# 14: Our power series this time is $\di \sum_{n=0}^\infty\frac{(-1)^n(x-5)^n}{10^n}$. We have
\[
 \frac{a_{n+1}}{a_n} = \frac{(-1)^{n+1}(x-5)^{n+1}}{10^{n+1}}\cdot \frac{10^n}{(-1)^n(x-5)^n} = -\frac{x-5}{10},
\]
so $\di\lim_{n\to\infty}\abs{\frac{a_{n+1}}{a_n}} = \frac{1}{10}\abs{x-5}$, and since we need this limit to be less than 1, we must have $\abs{x-5}<10$, so our radius of convergence is $r=10$.

The inequality $\abs{x-5}<10$ is equivalent to $-10<x-5<10$, so $-5<x<15$. When $x=-5$ we have the divergent series $\sum 1$, and when $x=15$ we have the divergent series $\sum (-1)^n$, so the interval of convergence is $(-5,15)$.

\item Section 3.6, \#26: We define $\di f(x) = \sum_{n=1}^\infty \frac{x^n}{n}$, and wish to compute $f'(x)$ and $\int f(x)\,dx$. First, we check that by the ratio test, $f(x)$ has radius of convergence $r=1$, since
\[
 \lim_{n\to\infty}\abs{\frac{x^{n+1}}{n+1}\cdot\frac{n}{x^n}} = \abs{x},
\]
so we need $\abs{x}<1$. At the endpoints of the interval $(-1,1)$ we have the series $\di \sum \frac{(-1)^n}{n}$ at $x=-1$, which converges, by the alternating series test, and the harmonic series $\di \sum \frac{1}{n}$ at $x=1$, which diverges. The interval of convergence (i.e., domain) for $f(x)$ is therefore $[-1,1)$. To compute $f'(x)$, we have
\[
 f'(x) = \frac{d}{dx}\sum_{n=1}^\infty \frac{x^n}{n} = \sum_{n=1}^\infty \frac{1}{n}\frac{d}{dx}(x^n) = \sum_{n=1}^\infty x^{n-1} = \sum_{n=0}^\infty x^n.
\]
Notice that $f'(x)$ has the same radius of convergence as $f(x)$, but $f'(x)$ is not defined at $x=-1$: the interval of convergence here is $(-1,1)$, since $f'(x)$ is given by a geometric series, and therefore only converges for $\abs{x}<1$. Indeed, note further that we can use the geometric series formula to conclude that
\[
 f'(x) = \sum_{n=0}^\infty x^n = \frac{1}{1-x}.
\]
We can then work backwards to identify $f(x)$ by computing antiderivatives: we must have $f(x) = -\ln(1-x)+C$, and since $f(0) = 0$, $C=0$, so $f(x) = -\ln(1-x)$.

Now, to compute $\int f(x)\,dx$, we have
\[
 \int f(x)\,dx = \int\left(\sum_{n=1}^\infty\frac{x^n}{n}\right)\,dx = \sum_{n=1}^\infty \frac{1}{n}\int x^n\,dx = \sum_{n=1}^\infty\frac{x^{n+1}}{n(n+1)}+C.
\]
You can check that once again the radius of convergence is $r=1$, but this time the interval of convergence is $[-1,1]$, since the series $\di\sum_{n=1}^\infty \frac{1^{n+1}}{n(n+1)}$ converges by comparison with the $p$-series $\di\sum\frac{1}{n^2}$.

\item Section 3.5, \#34: Just for fun, let's use power series to solve a differential equation. (This particular equation is linear, so we can also solve it by other means.)

Problem 34 asks us to solve the differential equation $y'=y+1$ with initial condition $y(0)=1$. To do so, we suppose that we have a power series representation $y = f(x) = \sum_{n=0}^\infty c_nx^n$ for the solutions. It follows that $y' = f'(x) = \sum_{n=1}^\infty nc_n x^{n-1}$, and plugging our series for $y$ and $y'$ into the differential equation, we have
\[
 \sum_{n=1}^\infty nc_n x^{n-1} = \sum_{n=0}^\infty c_n x^n + 1 .
\]
If we shift the index for the series on the left, we have
\[
 y' = \sum_{n=0}^\infty(n+1)c_{n+1}x^n = 1+\sum_{n=0}\infty c_nx^n.
\]
(We added one to the index on the left, but begin the sum at $n=0$ instead of $n=1$. You can write out the first few terms to verify that doing this doesn't change anything.)

Now, if we bring the two series to one side and combine them, we have
\[
 \sum_{n=0}^\infty[(n+1)c_{n+1}-c_n]x^n = (c_1-c_0) + \sum_{n=1}^\infty[(n+1)c_{n+1}-c_n]x^n = 1,
\]
where we separated the constant ($n=0$) term from the sum on the left. Comparing left and right-hand sides, we must have
\[
 c_1-c_0 =1 \text{ and } (n+1)c_{n+1}-c_n = 0 \text{ for all } n\geq 1.
\]
Our initial condition is that $c_0 = y(0) = 1$, and the equation $c_1-c_0=1$ then tells us that $c_1=2$. From here, we determine our power series using the recursion formula
\[
 c_{n+1}=\frac{1}{n+1}c_n
\]
for $n\geq 1$, with initial value $c_1=2$. Thus, $c_2 = \frac{1}{2}c_1 = 1$, $c_3 = \frac{1}{3}c_2 = \frac{1}{3}$, $c_4 = \frac{1}{4}c_3 = \frac{1}{12}$, etc.

\textbf{Exercise:} Solve the differential equation as a linear differential equation and show that the exact solution is $y = -1+2e^x$. Then verify that the Taylor series for this function agrees with the values of $c_n$ determined above.

\item Section 3.8, \#8. We're asked for the Taylor series for $f(x)=1/x$, expanded about $c=1$. We first compute that
\[
 f'(x) = \frac{-1}{x^2}, \quad f''(x) = \frac{2}{x^3}, \quad f'''(x) = \frac{-6}{x^4}, \ldots
\]
and make the educated guess that $f^{(n)}(x) = \dfrac{(-1)^nn!}{x^{n+1}}$ for each natural number $n$. (A rigorous proof that this guess is correct requires mathematical induction.) We then have $f^{(n)}(1) = (-1)^nn!$, so the desired Taylor series is
\[
 \frac{1}{x} = \sum_{n=0}^\infty \frac{f^{(n)}(1)}{n!}(x-1)^n = \sum_{n=0}^\infty (-1)^n(x-1)^n = \sum_{n=0}^\infty (1-x)^n.
\]
You can check that this is valid for $\abs{x-1}<1$ (so $x\in (0,2)$) using the formula for the sum of a geometric series:
\[
 \sum_{n=0}^\infty (1-x)^n = \frac{1}{1-(1-x)} = \frac{1}{x}.
\]

\item Section 3.8, \# 28: We're asked to find a Taylor series expansion for $f(x) = \arctan(x/2)$. From Key Idea 18 in the textbook, we know that
\[
 \arctan x = \sum_{n=0}^\infty (-1)^n\frac{x^{2n+1}}{2n+1}
\]
for $x\in [-1,1]$, so if we substitute $x/2$ for $x$ on both sides of this equation, we find that for $x\in [-2,2]$, we have
\[
 \arctan(x/2) = \sum_{n=0}^\infty (-1)^n\frac{(x/2)^{2n+1}}{2n+1} = \sum_{n=0}^\infty (-1)^n\frac{x^{2n+1}}{(2n+1)2^{2n+1}}.
\]

\end{itemize}






\end{document}