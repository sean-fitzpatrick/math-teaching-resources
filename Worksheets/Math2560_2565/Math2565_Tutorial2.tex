\documentclass[12pt]{article}
\usepackage{amsmath}
\usepackage{amssymb}
\usepackage[letterpaper,top=0.85in,bottom=1in,left=0.75in,right=0.75in,centering]{geometry}
%\usepackage{fancyhdr}
\usepackage{enumerate}
%\usepackage{lastpage}
\usepackage{multicol}
\usepackage{graphicx}

\reversemarginpar

%\pagestyle{fancy}
%\cfoot{}
%\lhead{Math 1560}\chead{Test \# 1}\rhead{May 18th, 2017}
%\rfoot{Total: 10 points}
%\chead{{\bf Name:}}
\newcommand{\points}[1]{\marginpar{\hspace{24pt}[#1]}}
\newcommand{\skipline}{\vspace{12pt}}
%\renewcommand{\headrulewidth}{0in}
\headheight 30pt

\newcommand{\di}{\displaystyle}
\newcommand{\abs}[1]{\lvert #1\rvert}
\newcommand{\len}[1]{\lVert #1\rVert}
\renewcommand{\i}{\mathbf{i}}
\renewcommand{\j}{\mathbf{j}}
\renewcommand{\k}{\mathbf{k}}
\newcommand{\R}{\mathbb{R}}
\newcommand{\aaa}{\mathbf{a}}
\newcommand{\bbb}{\mathbf{b}}
\newcommand{\ccc}{\mathbf{c}}
\newcommand{\dotp}{\boldsymbol{\cdot}}
\newcommand{\bbm}{\begin{bmatrix}}
\newcommand{\ebm}{\end{bmatrix}}                   
                  
\begin{document}


\author{Instructor: Sean Fitzpatrick}
\thispagestyle{empty}
\vglue1cm
\begin{center}
\emph{University of Lethbridge}\\
Department of Mathematics and Computer Science\\
{\bf MATH 2565 - Tutorial \#2}\\
Thursday, January 18
\end{center}
\skipline \skipline \skipline \noindent \skipline

\skipline
First Name:\underline{\hspace{348pt}}\\
\skipline

\vspace{1cm}

Last Name:\underline{\hspace{351pt}}
%Student Number:\underline{\hspace{322pt}}\\
%\skipline



\vspace{1cm}

\begin{quote}
Print your name clearly in the space above. 

\medskip

Complete the problems on the back of this page to the best of your ability. If there is a problem you especially desire feedback on, please indicate this. 

\medskip

It is recommended that you work out the details on scrap paper before writing your solutions on the worksheet.
\end{quote}

\vspace{2cm}


Additional practice (don't include your solutions here):
\begin{multicols}{2}
\begin{enumerate}
  \item $\di \int \sin^5(x)\cos^6(x)\,dx$

 \item $\di \int \sin(x)\sin(2x)\,dx$

  \item $\di \int \sqrt{9-x^2}\,dx$

 \item $\di \int \frac{8}{\sqrt{x^2+2}}\,dx$

 \item $\di \int \frac{1-\tan^2(x)}{\sec^2(x)}\,dx$
 
 \item $\di \int \frac{dx}{\cos(x)-1}\,dx$
\end{enumerate}  
\end{multicols}


\newpage
%\thispagestyle{empty}
Evaluate the following integrals.
  \begin{enumerate}
  \item $\di \int \tan^4(x)\sec^6(x)\,dx$
 
 \vspace{1.5in}
 
 \item  $\di \int \tan^3(x)\sec^5(x)\,dx$
 
 \vspace{1.75in}
 
 \item $\di \int \sin(8x)\cos(5x)\,dx$
 
 \vspace{1.75in}
 
 \item $\di \int_0^{\pi/6}\sqrt{1+\cos(2x)}\,dx$
 
 
 
 \pagebreak
 
 \item $\di \int \frac{5x^2}{\sqrt{x^2-10}}\,dx$, using a secant substitution.
 
 \vspace{1.75in}
 
\item  $\di \int \frac{5x^2}{\sqrt{x^2-10}}\,dx$, using a hyperbolic substitution.

\vspace{1.75in}

  \end{enumerate}

Discussion problem (no submission required):

Prove the following formulas, where $m$ and $n$ are integers:
\begin{align*}
\frac{1}{\pi}\int_{-\pi}^\pi\sin(mx)\cos(nx)\,dx & = 0\\
\frac{1}{\pi}\int_{-\pi}^\pi\sin(mx)\sin(nx)\,dx & = \begin{cases}0, & \text{ if } m\neq n\\1, & \text{ if } m=n\end{cases}\\
\frac{1}{\pi}\int_{-\pi}^\pi\cos(mx)\cos(nx)\,dx & = \begin{cases}0, & \text{ if } m\neq n\\1, & \text{ if } m=n\end{cases}
\end{align*}

\medskip

Suppose a function $f$ can be written as a \textit{finite Fourier series}
\[
f(x) = \frac{1}{2}a_0+\sum_{n=1}^N a_n\sin(nx)+\sum_{m=1}^M b_m\cos(mx).
\]

Show that the coefficients $a_n$ ($n=0,\ldots, N$) and $b_m$ ($m=1,\ldots, M$) are given by
\[
a_0 = \frac{1}{\pi}\int_{-\pi}^\pi f(x)\,dx, a_n = \frac{1}{\pi}\int_{-\pi}^\pi f(x)\sin(nx)\,dx, (i=1,\ldots, N), b_m = \frac{1}{\pi}\int_{-\pi}^\pi f(x)\cos(mx)\,dx.
\]
\end{document}