\documentclass[12pt]{article}
\usepackage{amsmath}
\usepackage{amssymb}
\usepackage[letterpaper,top=1.2in,bottom=1in,left=0.75in,right=0.75in,centering]{geometry}
%\usepackage{fancyhdr}
\usepackage{enumerate}
%\usepackage{lastpage}
\usepackage{multicol}
\usepackage{graphicx}

\reversemarginpar

%\pagestyle{fancy}
%\cfoot{}
%\lhead{Math 1560}\chead{Test \# 1}\rhead{May 18th, 2017}
%\rfoot{Total: 10 points}
%\chead{{\bf Name:}}
\newcommand{\points}[1]{\marginpar{\hspace{24pt}[#1]}}
\newcommand{\skipline}{\vspace{12pt}}
%\renewcommand{\headrulewidth}{0in}
\headheight 30pt

\newcommand{\di}{\displaystyle}
\newcommand{\abs}[1]{\lvert #1\rvert}
\newcommand{\len}[1]{\lVert #1\rVert}
\renewcommand{\i}{\mathbf{i}}
\renewcommand{\j}{\mathbf{j}}
\renewcommand{\k}{\mathbf{k}}
\newcommand{\R}{\mathbb{R}}
\newcommand{\aaa}{\mathbf{a}}
\newcommand{\bbb}{\mathbf{b}}
\newcommand{\ccc}{\mathbf{c}}
\newcommand{\dotp}{\boldsymbol{\cdot}}
\newcommand{\bbm}{\begin{bmatrix}}
\newcommand{\ebm}{\end{bmatrix}}                   
                  
\begin{document}


\author{Instructor: Sean Fitzpatrick}
\thispagestyle{empty}
\vglue1cm
\begin{center}
\emph{University of Lethbridge}\\
Department of Mathematics and Computer Science\\
{\bf MATH 2565 - Tutorial \#12}\\
Thursday, April 5
\end{center}
\skipline \skipline \skipline \noindent \skipline

\skipline
Name:\underline{\hspace{348pt}}\\
\skipline

\vspace{2cm}

\textbf{Note:} You may do this assignment as a group, if you wish, by listing additional names under the space above, up to a maximum of 3 students per group.

\vspace*{\fill}

Extra practice:
\begin{enumerate}
\item Plot the polar function:
\begin{enumerate}
\item $r=2+\cos\theta$, $\theta\in [0,2\pi]$
\item $r^2=\cos(2\theta)$, $\theta\in [-\pi/4,\pi/4]\cup [3\pi/4,5\pi/4]$
\end{enumerate}
\item Find the points of intersection of the polar curves. (Note that the point $(0,0)$ requires special care: you might have $r=0$ for \textit{different} values of $\theta$ for the two curves.)
\begin{enumerate}
\item $r=\cos(2\theta)$ and $r=\cos\theta$, on $[0,2\pi]$
\item $r=\sin\theta$ and $r=\sqrt{3}+3\sin\theta$, on $[0,2\pi]$
\end{enumerate}
\item Compute the area:
\begin{enumerate}
\item One loop of the three-leaf rose $r=\sin(3\theta)$.
\item The outer loop of the lima{\c c}on $r=1+2\cos\theta$.
\end{enumerate}
\end{enumerate}


\newpage
%\thispagestyle{empty}





 \begin{enumerate}
\item Plot the given polar function:

\begin{enumerate}
\item $r=\cos(2\theta)$, $\theta\in [0,2\pi]$.

\vspace{4in}

\item $r=2\cos(\theta)$, $\theta\in [-\pi/2,\pi/2]$
\end{enumerate}
\newpage

\item Find the given area:
\begin{enumerate}
\item Inside the circle $r=2\cos\theta$, but outside the circle $r=2\sin\theta$.

\vspace{4in}

\item The area common to the inside of the curves $r=\cos\theta$ and $r=\sin(2\theta)$, in the first quadrant.
\end{enumerate}
\newpage

\item Show that the indicated limit does not exist:
\begin{enumerate}
\item $\di\lim_{(x,y)\to (0,0)}\frac{3x+4y}{x-2y}$

\vspace{1.5in}

\item $\di\lim_{(x,y)\to (0,0)}\frac{xy^4}{x^2+y^8}$

\vspace{1.5in}

\end{enumerate}

\item What geometric object is obtained as the graph of $f(x,y)=2x-3y$?

\vspace{1in}

\item \textit{Challenge:} Describe the level surfaces of $f(x,y,z)=k$ $f(x,y,z)=x^2+y^2-z^2$, for $k=-2,-1,0,1,2$.
\end{enumerate}
\end{document}