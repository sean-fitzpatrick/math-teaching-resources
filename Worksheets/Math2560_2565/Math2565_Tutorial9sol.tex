\documentclass[12pt]{article}
\usepackage{amsmath}
\usepackage{amssymb}
\usepackage[letterpaper,top=1.2in,bottom=1in,left=0.75in,right=0.75in,centering]{geometry}
%\usepackage{fancyhdr}
\usepackage{enumerate}
%\usepackage{lastpage}
\usepackage{multicol}
\usepackage{graphicx}
\usepackage{hyperref}

\reversemarginpar

%\pagestyle{fancy}
%\cfoot{}
%\lhead{Math 1560}\chead{Test \# 1}\rhead{May 18th, 2017}
%\rfoot{Total: 10 points}
%\chead{{\bf Name:}}
\newcommand{\points}[1]{\marginpar{\hspace{24pt}[#1]}}
\newcommand{\skipline}{\vspace{12pt}}
%\renewcommand{\headrulewidth}{0in}
\headheight 30pt

\newcommand{\di}{\displaystyle}
\newcommand{\abs}[1]{\left\lvert #1\right\rvert}
\newcommand{\len}[1]{\lVert #1\rVert}
\renewcommand{\i}{\mathbf{i}}
\renewcommand{\j}{\mathbf{j}}
\renewcommand{\k}{\mathbf{k}}
\newcommand{\R}{\mathbb{R}}
\newcommand{\aaa}{\mathbf{a}}
\newcommand{\bbb}{\mathbf{b}}
\newcommand{\ccc}{\mathbf{c}}
\newcommand{\dotp}{\boldsymbol{\cdot}}
\newcommand{\bbm}{\begin{bmatrix}}
\newcommand{\ebm}{\end{bmatrix}}                   
                  
\begin{document}


\author{Instructor: Sean Fitzpatrick}
\thispagestyle{empty}
\vglue1cm
\begin{center}
{\bf MATH 2565 - Tutorial \#9 Solutions}
\end{center}



 \begin{enumerate}
\item Use the ratio or root test to determine whether the series is convergent or divergent. (If the test is inconclusive, or impractical, determine converge with another test.)
\begin{enumerate}
\item $\di\sum_{n=1}^\infty n\left(\frac{3}{5}\right)^n$

The ratio test gives us
\[
\lim_{n\to\infty}\abs{\frac{a_{n+1}}{a_n}}=\lim_{n\to\infty}\abs{\frac{(n+1)(3/5)^{n+1}}{n(3/5)^n}}=\frac35\lim_{n\to\infty}\frac{n+1}{n}=\frac35 <1,
\]
so the series converges. 

\item $\di\sum_{n=1}^\infty \frac{(2n)!}{(n!)^2}$

Using the ratio test again (note that $2(n+1)=2n+2$),
\[
\lim_{n\to\infty}\abs{\frac{(2n+2)!}{((n+1)!)^2}\cdot\frac{(n!)^2}{(2n)!}}=\lim_{n\to\infty}\frac{(2n+2)(2n+1)(2n)!}{((n+1)n!)^2}\cdot\frac{(n!)^2}{(2n)!}=\lim_{n\to\infty}\frac{(2n+2)(2n+1)}{(n+1)^2}=4>1,
\]
so the series diverges.

\item $\di\sum_{n=2}^\infty\left(\frac{1}{n}-\frac{1}{n^2}\right)^n$

Using the root test, we have
\[
\lim_{n\to\infty}\sqrt[n]{\abs{\left(\frac{1}{n}-\frac{1}{n^2}\right)^2n}} = \lim_{n\to\infty}\left(\frac1n-\frac{1}{n^2}\right)=0<1,
\]
so the series converges.

\item $\di\sum_{n=1}^\infty\frac{5^n+n^4}{7^n+n^2}$

Here, we first split the series into two parts:
\[
\sum_{n=1}^\infty\frac{5^n+n^4}{7^n+n^2} = \sum_{n=1}^\infty\frac{5^n}{7^n+n^2}+\sum_{n=1}^\infty\frac{n^4}{7^n+n^2}.
\]
Since $\dfrac{5^n}{7^n+n^2}<\frac{5^n}{7^n}$ for all $n\geq 1$ and $\di\sum_{n=1}^\infty\frac{5^n}{7^n}$ is a convergent geometric series, the first part of the series converges by comparison.

Now, we note that $\frac{n^4}{7^n+n^2}<\frac{n^4}{7^n}$ for all $n\geq 1$. For the series $\di\sum_{n=1}^\infty \frac{n^4}{7^n}$, the ratio test gives us
\[
\lim_{n\to\infty}\abs{\frac{(n+1)^4}{7^{n+1}}\cdot \frac{7^n}{n^4}}=\frac17 \lim_{n\to\infty}\frac{(n+1)^4}{n^4} = \frac17 < 1,
\]
so this series converges, and thus, the second part of the series converges by comparison.

Since our series is the sum of two convergent series, it converges.

\item $\di\sum_{n=1}^\infty\left(1+\frac{1}{n}\right)^{n^2}$

Here, the root test gives us
\[
\lim_{n\to\infty}\sqrt[n]{\abs{a_n}} =\lim_{n\to \infty} \left(\left(1+\frac1n\right)^{n^2}\right)^{1/n} = \lim_{n\to \infty}\left(1+\frac{1}{n}\right)^n = e>1,
\]
so the series diverges.

\textit{Note:} If you didn't remember this last limit as a result from Math 1565, note that
\[
\lim_{x\to \infty}(1+1/x)^x=\lim_{x\to\infty}e^{x\ln(1+1/x)}= e^{\lim_{x\to\infty}\frac{\ln(1+1/x)}{1/x}} = e^{\lim_{x\to\infty}\frac{1}{1+1/x}}=e^1=e,
\]
using L'Hospital's Rule.
\end{enumerate}
%\newpage

\item Determine if the series converges conditionally, or absolutely, or not at all:
\begin{enumerate}
\item $\di\sum_{n=1}^\infty\frac{\sin(n\pi/3)}{1+n\sqrt{n}}$

We note that for $a_n = \dfrac{\sin(n\pi/3)}{1+n\sqrt{n}}$ we have
\[
\abs{a_n} = \frac{\abs{\sin(n\pi/3)}}{1+n\sqrt{n}}\leq \frac{1}{1+n^{3/2}}<\frac{1}{n^{3/2}},
\]
so the series converges absolutely by comparison with the convergent $p$-series $\sum_{n=1}^\infty 1/n^{3/2}$.

\item $\di\sum_{n=1}^\infty\frac{(-1)^n}{\sqrt{n}}$

The series converges by the Alternating Series Test, since the terms are of the form $(-1)^na_n$, where $a_n=1/\sqrt{n}$, and $\{a_n\}$ is a positive, decreasing sequence with $\lim_{n\to\infty}a_n=0$.

However, the series does not converge absolutely, since $\di\abs{\frac{(-1)^n}{\sqrt{n}}}=\frac{1}{\sqrt{n}}$ is a $p$-series with $p=1/2<1$, which diverges.

\item $\di\sum_{n=1}^\infty n\cos(\pi n)\sin(1/n)$

This series does not converge. To see this, note that $\cos(\pi n)=(-1)^n$, while
\[
\lim_{n\to\infty}n\sin(1/n)=\lim_{n\to\infty}\frac{\sin(1/n)}{1/n}=1.
\]
The series thus fails the divergence test, since $\lim_{n\to\infty}n\cos(\pi n)\sin(1/n)$ does not exist (for large $n$ the terms alternate between values close to 1 and close to -1).
\end{enumerate}


\item One can show that $\di \pi = \sum_{n=0}^\infty \frac{4(-1)^n}{2n+1}$. What is the least value of $N$ such that the partial sum $\di S_N=\sum_{n=1}^N \frac{4(-1)^n}{2n+1}$ approximates the value of $\pi$, correct to 3 decimal places?

According to the Alternating Series Approximation Theorem, if a series $\sum_{n=1}^\infty (-1)^na_n$ converges to some limit $L$, then for each $N$, $\di\abs{\sum_{n=1}^N-L}<a_{N+1}$. Thus, it suffices to determine the least $N$ such that $a_N = \frac{4}{2N+1} < 0.0005$. For this, we must have
\[
N>\frac12\left(\frac{4}{0.0005}-1\right)= \frac{7999}{2}.
\]
Rounding to the nearest integer, we see that we need $N=4000$ terms. 

From this, we can conclude that this is not a very effective method for computing $\pi$. (A method referred to at least once as \href{https://mathwithbaddrawings.com/2016/03/14/the-pi-day-recipe-book/}{Gottfried's slow jam}.)


\item For each series below, indicate whether it converges or diverges. Also indicate which convergence test you used, and why.
\begin{enumerate}
\item $\di \sum_{m=1}^\infty me^{-m}$

We apply the ratio test:
\[
\lim_{n\to\infty}\abs{\frac{a_{n+1}}{a_n}}=\lim_{n\to\infty}\frac{n+1}{e^{n+1}}\cdot \frac{e^n}{n} = \frac{1}{e}<1,
\]
so the series converges.

\item $\di \sum_{n=1}^\infty \frac{2(n^2+2)^{2018}}{3(n^3+n+3)^{2018}}$

Noticing that for large $n$ our terms are approximately $\dfrac{2(n^2)^{2018}}{3(n^3)^{2018}}=\dfrac{2}{3n^{2018}}$, we use limit comparison with the convergent $p$-series $\di\sum_{n=1}^\infty\frac{1}{n^{2018}}$. We have
\[
\lim_{n\to\infty}\frac{\dfrac{2(n^2+2)^{2018}}{3(n^3+n+3)^{2018}}}{\dfrac{1}{n^{2018}}}=\frac23 \lim_{n\to\infty}\frac{n^{2018}(n^2+2)^{2018}}{(n^3+n+3)^{2018}} =\frac23\lim_{n\to\infty}\left(\frac{n^3+2n}{n^3+n+2}\right)^{2018}=\frac23\cdot 1^{2018}=\frac23.
\]
Since this limit is positive and finite, we can conclude that our series converges by the limit comparison test.


\item $\di \sum_{k=1}^\infty \frac{(k!)^k}{k^{4k}}$

Here we try the root test, due to the powers in both the numerator and denominator. (It's less messy than the ratio test in this case.) We have
\[
\lim_{k\to\infty}\sqrt[k]{\abs{a_k}} = \lim_{k\to\infty}\frac{k!}{k^4}=\lim_{k\to\infty}\frac{k(k-1)(k-2)(k-3)}{k^4}\cdot (k-4)! = \infty,
\]
so the series diverges. (One could also have verified that the series fails the divergence test.)
\end{enumerate}
\end{enumerate}
\end{document}