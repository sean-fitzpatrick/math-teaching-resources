\documentclass[12pt]{article}
\usepackage{amsmath}
\usepackage{amssymb}
\usepackage[letterpaper,top=1.2in,bottom=1in,left=0.75in,right=0.75in,centering]{geometry}
%\usepackage{fancyhdr}
\usepackage{enumerate}
%\usepackage{lastpage}
\usepackage{multicol}
\usepackage{graphicx}

\reversemarginpar

%\pagestyle{fancy}
%\cfoot{}
%\lhead{Math 1560}\chead{Test \# 1}\rhead{May 18th, 2017}
%\rfoot{Total: 10 points}
%\chead{{\bf Name:}}
\newcommand{\points}[1]{\marginpar{\hspace{24pt}[#1]}}
\newcommand{\skipline}{\vspace{12pt}}
%\renewcommand{\headrulewidth}{0in}
\headheight 30pt

\newcommand{\di}{\displaystyle}
\newcommand{\abs}[1]{\lvert #1\rvert}
\newcommand{\len}[1]{\lVert #1\rVert}
\renewcommand{\i}{\mathbf{i}}
\renewcommand{\j}{\mathbf{j}}
\renewcommand{\k}{\mathbf{k}}
\newcommand{\R}{\mathbb{R}}
\newcommand{\aaa}{\mathbf{a}}
\newcommand{\bbb}{\mathbf{b}}
\newcommand{\ccc}{\mathbf{c}}
\newcommand{\dotp}{\boldsymbol{\cdot}}
\newcommand{\bbm}{\begin{bmatrix}}
\newcommand{\ebm}{\end{bmatrix}}                   
                  
\begin{document}


\author{Instructor: Sean Fitzpatrick}
\thispagestyle{empty}
\vglue1cm
\begin{center}
\emph{University of Lethbridge}\\
Department of Mathematics and Computer Science\\
{\bf MATH 2565 - Tutorial \#5}\\
Thursday, February 8
\end{center}
\skipline \skipline \skipline \noindent \skipline

\skipline
First Name:\underline{\hspace{348pt}}\\
\skipline

\vspace{1cm}

Last Name:\underline{\hspace{351pt}}
%Student Number:\underline{\hspace{322pt}}\\
%\skipline



\vspace{2cm}


%\vspace{2cm}


\textbf{Additional practice} (don't include your solutions here):
\begin{enumerate}
 \item Find the volume of a right circular cone whose height is 12 and base radius is 4.
 \item Find the volume of the solid generated by revolving the region bounded by $y=x^2$, $x=1$, and $y=0$ about the $x$-axis.
 \item Find the volume of the solid generated by revolving the region bounded by $y=\sqrt{x}$, $y=1$, and $x=0$ about
 \begin{multicols}{2}
 \begin{enumerate}
 \item the $x$-axis
 \item the $y$-axis
\end{enumerate} 
\end{multicols}
 \item Find the volume of the solid generated by the region bounded by $y=4-x^2$ and $y=0$, when rotated about
 \begin{multicols}{3}
\begin{enumerate}
 \item the $x$-axis.
 \item the line $y=-1$.
 \item the line $x=2$.
\end{enumerate}
\end{multicols}
 \item Find the volume of the solid generated by revolving the region bounded by $x=y-y^2$ and $x=0$ about the $y$-axis.
 \item Use the shell method to find the volume of the solid generated by revolving the region bounded by $y=6x-2x^2$ and $y=0$, about the $y$-axis.


\end{enumerate}

\newpage
%\thispagestyle{empty}





 \begin{enumerate}



 \item Find the volume of the solid $S$ whose base is the region of the $xy$-plane bounded by $y=x^2$ and $y=2-x^2$, and whose cross-sections parallel to the $y$-axis are squares.
 
 \vspace{4in}
 
 \item Find the volume of the solid $S$ generated when the region bounded by the curves $y^2=x$ and $x=2y$ is revolved around the $y$-axis.

\newpage

 \item Use the shell method to find the volume of the solid generated by revolving the region bounded by $y=x$ and $y=\sqrt{x}$ about the $x$-axis.
 
 \vspace{4in} 
 
 \item Find the length of the curve $y=\dfrac{1}{12}x^3+\dfrac{1}{x}$, for $x\in [1,4]$.

\end{enumerate}
\newpage

Some integration formulas you may need:

\begin{itemize}
 \item Volume of a solid of revolution, washer method, rotation about a horizontal axis: 
\[
 V = \pi\int_a^b (r_{out}(x)^2-r_{in}(x)^2)\,dx,
\]
where $r_{out}$ gives the outer radius (distance from the far side of the region being rotated to the axis) and $r_{in}$ gives the inner radius. If the axis of rotation is vertical, reverse the roles of $x$ and $y$.



 \item Volume of a solid of revolution, shell method, rotation about a vertical axis:
\[
 V = 2\pi\int_a^b r(x)h(x)\,dx,
\]
where $r(x)$ is the radius of the shell (distance to axis of rotation) and $h(x)$ is the height of the shell. If the region lies between $y=g(x)$ (above) and $y=f(x)$ (below), and the axis of rotation is the $y$-axis we get the formula
\[
 V = 2\pi \int_a^b x(g(x)-f(x))\,dx
\]
as a special case. If the axis of rotation is horizontal, reverse the roles of $x$ and $y$.

\item Arc length of a curve $y=f(x)$, for $a\leq x\leq b$:
\[
 L = \int_a^b\sqrt{1+f'(x)^2}\,dx.
\]
\item Surface area generated by revolving $y=f(x)$, $a\leq x\leq b$ about the $x$-axis:
\[
 S = 2\pi\int_a^b f(x)\sqrt{1+f'(x)^2}\,dx.
\]
\item Surface area generated by revolving $y=f(x)$, $a\leq x\leq b$ about the $y$-axis:
\[
 S = 2\pi\int_a^b x\sqrt{1+f'(x)^2}\,dx.
\]

\end{itemize}
\end{document}