\documentclass[12pt]{article}
\usepackage{amsmath}
\usepackage{amssymb}
\usepackage[letterpaper,top=0.85in,bottom=1in,left=0.75in,right=0.75in,centering]{geometry}
%\usepackage{fancyhdr}
\usepackage{enumerate}
%\usepackage{lastpage}
\usepackage{multicol}
\usepackage{graphicx}

\reversemarginpar

%\pagestyle{fancy}
%\cfoot{}
%\lhead{Math 1560}\chead{Test \# 1}\rhead{May 18th, 2017}
%\rfoot{Total: 10 points}
%\chead{{\bf Name:}}
\newcommand{\points}[1]{\marginpar{\hspace{24pt}[#1]}}
\newcommand{\skipline}{\vspace{12pt}}
%\renewcommand{\headrulewidth}{0in}
\headheight 30pt

\newcommand{\di}{\displaystyle}
\newcommand{\abs}[1]{\lvert #1\rvert}
\newcommand{\len}[1]{\lVert #1\rVert}
\renewcommand{\i}{\mathbf{i}}
\renewcommand{\j}{\mathbf{j}}
\renewcommand{\k}{\mathbf{k}}
\newcommand{\R}{\mathbb{R}}
\newcommand{\aaa}{\mathbf{a}}
\newcommand{\bbb}{\mathbf{b}}
\newcommand{\ccc}{\mathbf{c}}
\newcommand{\dotp}{\boldsymbol{\cdot}}
\newcommand{\bbm}{\begin{bmatrix}}
\newcommand{\ebm}{\end{bmatrix}}                   
                  
\begin{document}


\author{Instructor: Sean Fitzpatrick}
\thispagestyle{empty}
\vglue1cm
\begin{center}
{\bf MATH 2565 - Tutorial \#2 Solutions}
\end{center}

Additional practice problems:

\begin{enumerate}
  \item  \begin{align*}
\int \sin^5(x)\cos^6(x)\,dx &= \int \sin(x)(1-\cos^2(x))^2\cos^6(x)\,dx\\& = \int \sin(x)(\cos^6(x)-2\cos^8(x)+\cos^{10}(x))\,dx, 
       \end{align*}
so letting $u=\cos(x)$, we have $du =\sin (x)\,dx$        and the integral becomes
\[
 \int(-u^6+u^8-u^{10})\,dx = -\frac{u^7}{7}+\frac{u^9}{9}-\frac{u^{11}}{11} +c = -\frac{1}{7}\cos^7(x)+\frac{1}{9}\cos^9(x)-\frac{1}{11}\cos^{11}(x)+C.
\]

 \item $\di \int \sin(x)\sin(2x)\,dx = \int \sin(x)(2\sin(x)\cos(x))\,dx = 2\int \sin^2(x)\cos(x)\,dx = \frac{2}{3}\sin^{3}(x)+C$, using the $u$-substitution $u=\sin(x)$, $du = \sin(x)\,dx$.
 
 (Could you do it by parts? Maybe. But why would you subject yourself to that when a quick application of a trig identity gives you a much easier integral?)

  \item $\di \int \sqrt{9-x^2}\,dx$ 
  
  Here we use the trig substitution $x=3\sin\theta$, which gives us $dx = 3\cos\theta\,d\theta$, and $9-x^2 = 9-9\sin^2\theta = 9\cos^2\theta$, so $\sqrt{9-x^2} = 3\cos\theta$. Thus,
 \begin{align*}
 \int \sqrt{9-x^2}\,dx & = \int 9\cos^2\theta\,d\theta\\
 & = \int \frac{9}{2}(1+\cos(2\theta))\,d\theta\\
 & = \frac{9}{2}\theta + \frac{9}{4}\sin(2\theta)+C\\
 & = \frac{9}{2}\theta + \frac{9}{2}\sin\theta\cos\theta.
 \end{align*}
 Now, we use the fact that $3\sin\theta = x$, so $\theta = \sin^{-1}(x/3)$, and $3\cos\theta = \sqrt{9-x^2}$ to substitute back in terms of $x$, giving us
 \[
 \int \sqrt{9-x^2}\,dx = \frac{9}{2}\sin^{-1}\left(\frac{\theta}{3}\right)+\frac{1}{2}x\sqrt{9-x^2}+C.
 \]
\pagebreak

 \item $\di \int \frac{8}{\sqrt{x^2+2}}\,dx$
 
 There are two options for this integral. We can either let $x=\sqrt{2}\tan\theta$, or $x=\sqrt{2}\sinh(t)$.
 
 Taking the first option, we get $dx = \sqrt{2}\sec^2\theta\,d\theta$ and 
 \[
 \sqrt{x^2+2} = \sqrt{2\tan^2\theta+2} = \sqrt{2\sec^2\theta} = \sqrt{2}\sec\theta,
 \]
 so
 \[
 \int\frac{8}{\sqrt{x^2+2}}\,dx = \int \frac{8\sqrt{2}\sec^2\theta}{\sqrt{2}\sec\theta}\,d\theta = \int 8\sec\theta\,d\theta = 8\ln\abs{\sec\theta+\tan\theta}+C.
 \]
 Now, $\tan\theta = x/\sqrt{2}$, and $\sec\theta = \sqrt{x^2+2}/\sqrt{2}$, so this becomes
 \[
 \int\frac{8}{\sqrt{x^2+2}}\,dx = 8\ln\left\lvert \frac{\sqrt{x^2+2}+x}{\sqrt{2}}\right\rvert+C.
 \]
 (Note: using log laws, you can get rid of the $\sqrt{2}$ in the denominator --  you'll get a $-\ln\sqrt{2}$ term, which is a constant that can be absorbed into the constant of integration.)
 
 \bigskip
 
 If we take the second option, $x=\sqrt{2}\sinh(t)$, so $dx = \sqrt{2}\cosh(t)$, and 
 \[
 \sqrt{x^2+2} = \sqrt{2\sinh^2(t)+2} = \sqrt{2\cosh^2(t)} = \sqrt{2}\cosh(t).
 \]
 Thus,
 \[
 \int \frac{8}{\sqrt{x^2+2}}\,dx = \int \frac{8\sqrt{2}\cosh(t)}{\sqrt{2}}\cosh(t)\,dt = 8t+C = 8\sinh^{-1}(x/\sqrt{2})+C.
 \]
 
{\bf Exercise:} Can you show that the answers given by the two methods are equivalent? In other words, is it true that
\[
 \ln\left\lvert \frac{\sqrt{x^2+2}+x}{\sqrt{2}}\right\rvert=\sinh^{-1}(x/\sqrt{2})+C
\]
for some constant $C$? (If it is, then the derivative of either side should be equal.)

 \item $\di \int \frac{1-\tan^2(x)}{\sec^2(x)}\,dx$
 
 Since
 \[
 \frac{1-\tan^2(x)}{\sec^2(x)} = \cos^2(x)(1-\tan^2(x))=\cos^2(x)-\sin^2(x)=\cos(2x),
 \]
 we have
 \[
 \int \frac{1-\tan^2(x)}{\sec^2(x)}\,dx = \int \cos(2x)\,dx = \frac{1}{2}\sin(2x)+C.
 \]
 
 \item $\di \int \frac{dx}{\cos(x)-1}\,dx$
 
 First we manipulate the integrand:
 \[
 \frac{1}{\cos(x)-1} = \frac{cos(x)+1}{\cos^2(x)-1} = -\frac{\cos(x)+1}{\sin^2(x)} = -\frac{\cos(x)}{\sin^2(x)}-\csc^2(x).
 \]
 Having written the integrand as a sum of two terms, we split up the integral into two pieces:
 \[
 \int \frac{1-\tan^2(x)}{\sec^2(x)}\,dx = -\int\frac{\cos(x)}{\sin^2(x)}\,dx - \int\csc^2(x)\,dx.
 \]
 The first integral can be evaluated using the substitution $u=\sin(x)$, $du = \cos(x)\,dx$, resulting in $\int\frac{-1}{u^2}\,du = \frac{1}{u}+C$; the second integral is immediate. We get
 \[
 \int \frac{dx}{\cos(x)-1}\,dx = \frac{1}{\sin(x)}+\cot(x)+C.
 \]
\end{enumerate}  




%\thispagestyle{empty}
\textbf{Assigned problems}: Evaluate the following integrals.
  \begin{enumerate}
  \item $\di \int \tan^4(x)\sec^6(x)\,dx$
 
\medskip

 We employ the identity 
 \[
 \sec^4(x) = (1+\tan^2(x))^2 =1+2\tan^2(x)+\tan^4(x)
 \]
 to obtain
 \begin{align*}
 \int\tan^4(x)\sec^6(x)\,dx &= \int(\tan^4(x)+2\tan^6(x)+\tan^8(x))\sec^2(x)\,dx\\
 \frac{1}{5}\tan^5(x)+\frac{2}{7}\tan^7(x)+\frac{1}{9}\tan^9(x)+C.
 \end{align*}
 
 
 \item  $\di \int \tan^3(x)\sec^5(x)\,dx$
 
 \medskip
 
 This time we use $\tan^2(x) = \sec^2(x)-1$ to get
 \[
 \tan^3(x)\sec^5(x) = (\tan^2(x)\sec^4(x))\sec(x)\tan(x) = (\sec^6(x)-\sec^4(x))\sec(x)\tan(x),
 \]
 so
 \[
 \int \tan^3(x)\sec^5(x)\,dx = \frac{1}{7}\sec^7(x)-\frac{1}{5}\sec^5(x)+C.
 \]
 
 \item $\di \int \sin(8x)\cos(5x)\,dx$
 
 \medskip
 
 Here we need the product-to-sum identity
 \[
 \sin(8x)\cos(5x) = \frac{1}{2}[\sin(8x-5x)+\sin(8x+5x)] = \frac{1}{2}(\sin(3x)+\sin(13x)),
 \]
 giving us
 \[
 \int \sin(8x)\cos(5x)\,dx = -\frac{1}{6}\cos(6x)-\frac{1}{26}\cos(13x)+C.
 \]
 
 \item $\di \int_0^{\pi/6}\sqrt{1+\cos(2x)}\,dx$
 
 \medskip
 
 Recalling that $\cos(2x) = 2\cos^2(x)-1$, we get
 \[
 \int_0^{\pi/6}\sqrt{1+\cos(2x)}\,dx = \int_0^{\pi/6}\sqrt{2\cos^2(x)}\,dx = \sqrt{2}\int_0^{\pi/6}\cos(x)\,dx = \sqrt{2}/2.
 \]
 (Note that $\cos(x)\geq 0$ on $[0,\pi/6]$, so $\sqrt{\cos^2(x)} = \cos(x)$ -- there is no need to worry about the absolute value.)

 \item $\di \int \frac{5x^2}{\sqrt{x^2-10}}\,dx$, using a secant substitution.
 
As suggested, we let $x=\sqrt{10}\sec\theta$, so that $dx = \sqrt{10}\sec\theta\tan\theta\,d\theta$, and
 \[
 \sqrt{x^2-10} = \sqrt{10(\sec^2\theta-1)} = \sqrt{10\tan^2\theta} = \sqrt{10}\tan\theta.
 \]
 Thus,
 \[
 \int \frac{5x^2}{\sqrt{x^2-10}}\,dx = \int\frac{50\sec^2\theta}{\sqrt{10}\tan\theta}(\sqrt{10}\sec\theta\tan\theta)\,d\theta = \int 50\sec^3\theta\,d\theta.
 \]
 Uh oh... the dreaded $\sec^3\theta$ integral. Luckily we have that answer in our notes, so we can plug it in, giving us
 \[
 \int \frac{5x^2}{\sqrt{x^2-10}}\,dx = 25\sec\theta\tan\theta+25\ln\abs{\sec\theta+\tan\theta}+C,
 \]
 and we note that $\sec\theta = x/\sqrt{10}$ and $\tan\theta = \sqrt{x^2-10}/\sqrt{10}$, so
 \[
 \int \frac{5x^2}{\sqrt{x^2-10}}\,dx = \frac{5}{2}x\sqrt{x^2-10}+25\ln\abs{(x+\sqrt{x^2-10})/\sqrt{10}}+C.
 \]
 
 \bigskip
 

 
\item  $\di \int \frac{5x^2}{\sqrt{x^2-10}}\,dx$, using a hyperbolic substitution.

 If we use a hyperbolic substitution instead, we take $x=\sqrt{10}\cosh(t)$, so $dx = \sqrt{10}\sinh(t)$, and
 \[
 \sqrt{x^2-10} = \sqrt{10(\cosh^2(t)-1)} = \sqrt{10\sinh^2(t)} = \sqrt{10}\sinh(t).
 \]
 Thus,
 \[
 \int \frac{5x^2}{\sqrt{x^2-10}}\,dx = \int \frac{50\cosh^2(t)}{\sqrt{10}\sinh(t)}(\sqrt{10})\sinh(t)\,dt = \int 50\cosh^2(t)\,dt.
 \]
 Now we have to know how to integrate $\cosh^2(t)$. If we recall how $\cosh(t)$ is defined, we have
 \[
 \cosh^2(t) = \left(\frac{e^t+e^{-t}}{2}\right)^2 = \frac{e^{2t}+e^{-2t}+2}{4}.
 \]
 So you could simply write $\cosh^2(t)$ in terms of exponentials as above, and integrate term-by-term. The other option is to notice that there's an identity sitting there: $\dfrac{e^{2t}+e^{-2t}}{4} = \frac{1}{2}\cosh(2t)$, so
 \[
 \int \cosh^2(t)\,dt = \int\left( \frac{1}{2}\cosh(2t)+\frac{1}{2}\right)\,dt = \frac{1}{4}\sinh(2t)+\frac{t}{2}+C.
 \]
 Finally, we have to substitute back in terms of $x$. Would it surprise you to learn that $\sinh(2t)=2\sinh(t)\cosh(t)$? Well, that turns out to be true. Since $\sinh(t) = \sqrt{x^2-10}/\sqrt{10}$ and $\cosh(t) = x/\sqrt{10}$, we get
 \[
 \int\cosh^2(t)\,dt = \frac{5}{2}x\sqrt{x^2-10}+25\cosh^{-1}(x/\sqrt{10})+C.
 \]
 The last thing you might be wondering is whether the two answers are the same. They certainly look different. It's a good exercise to see if you can show that
 \[
 \ln(x+\sqrt{x^2-10}) = \cosh^{-1}(x/\sqrt{10}) \text{ (up to a constant)}
 \]
 The easiest way to do that is to show that their derivatives are the same.

  \end{enumerate}

Discussion problem (no submission required):

Prove the following formulas, where $m$ and $n$ are integers:
\begin{align*}
\frac{1}{\pi}\int_{-\pi}^\pi\sin(mx)\cos(nx)\,dx & = 0\\
\frac{1}{\pi}\int_{-\pi}^\pi\sin(mx)\sin(nx)\,dx & = \begin{cases}0, & \text{ if } m\neq n\\1, & \text{ if } m=n\end{cases}\\
\frac{1}{\pi}\int_{-\pi}^\pi\cos(mx)\cos(nx)\,dx & = \begin{cases}0, & \text{ if } m\neq n\\1, & \text{ if } m=n\end{cases}
\end{align*}

\medskip

Suppose a function $f$ can be written as a \textit{finite Fourier series}
\[
f(x) = \frac{1}{2}a_0+\sum_{n=1}^N a_n\sin(nx)+\sum_{m=1}^M b_m\cos(mx).
\]

Show that the coefficients $a_n$ ($n=0,\ldots, N$) and $b_m$ ($m=1,\ldots, M$) are given by
\[
a_0 = \frac{1}{\pi}\int_{-\pi}^\pi f(x)\,dx, a_n = \frac{1}{\pi}\int_{-\pi}^\pi f(x)\sin(nx)\,dx, (i=1,\ldots, N), b_m = \frac{1}{\pi}\int_{-\pi}^\pi f(x)\cos(mx)\,dx.
\]

\bigskip

If you want to know how to do this problem, you can come ``discuss'' it with me during office hours.
\end{document}