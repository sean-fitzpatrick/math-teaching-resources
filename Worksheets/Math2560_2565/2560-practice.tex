\documentclass[12pt]{article}
\usepackage{amsmath}
\usepackage{amssymb}
\usepackage[letterpaper,margin=0.85in,centering]{geometry}
\usepackage{fancyhdr}
\usepackage{enumerate}
\usepackage{lastpage}
\usepackage{multicol}
\usepackage{graphicx}

\reversemarginpar

\pagestyle{fancy}
\cfoot{Page \thepage \ of \pageref{LastPage}}\rfoot{{\bf Total Points: 100}}
\chead{MATH 2560A}\lhead{Practice Exam}\rhead{Saturday, 15\textsuperscript{th} April, 2016}

\newcommand{\points}[1]{\marginpar{\hspace{24pt}[#1]}}
\newcommand{\skipline}{\vspace{12pt}}
%\renewcommand{\headrulewidth}{0in}
\headheight 30pt

\newcommand{\di}{\displaystyle}
\newcommand{\R}{\mathbb{R}}
\newcommand{\aaa}{\mathbf{a}}
\newcommand{\bbb}{\mathbf{b}}
\newcommand{\ccc}{\mathbf{c}}
\newcommand{\dotp}{\boldsymbol{\cdot}}
\newcommand{\abs}[1]{\lvert #1\rvert}
\newcommand{\len}[1]{\lVert #1\rVert}
\newcommand{\ivec}{\,\boldsymbol{\hat{\imath}}}
\newcommand{\jvec}{\,\boldsymbol{\hat{\jmath}}}
\newcommand{\kvec}{\,\boldsymbol{\hat{k}}}
\DeclareMathOperator{\comp}{comp}

\begin{document}

\author{Instructor: Sean Fitzpatrick}
\thispagestyle{plain}
\begin{center}
\emph{University of Lethbridge}\\
Department of Mathematics and Computer Science\\
15\textsuperscript{th} April, 2016\\
{\bf MATH 2560A - Practice Exam}\\
\end{center}
\skipline \skipline \skipline \noindent \skipline
Last Name:\underline{\hspace{353pt}}\\
\skipline
First Name:\underline{\hspace{350pt}}\\
\skipline
Student Number:\underline{\hspace{323pt}}\\
\skipline
Tutorial Section: \underline{\hspace{320pt}}\\


\vspace{0.5in}


\begin{quote}
 {\bf All problems on this exam are taken from the course text, with textbook references provided for each problem. (A text reference of the form (2.3.15) means Problem 15 from Section 2.3.) I've chosen odd-numbered problems so that you can look up the answers in the back.  If you want help with a solution, you can use the online forum or drop by the help session on Sunday.
 
 Keep in mind that this is my version of what a Math 2560 exam might look like based on the topics provided by Dr. Connolloy: it may bear no resemblance to the real thing. (This is also what a Math 2560 exam might look like if assembled in 30 minutes by someone with far less time than they'd like to have.)}
\end{quote}


\vspace{0.5in}


\skipline

\skipline

\skipline


\newpage


\begin{enumerate}
\item Evaluate the following immediate integrals:
\begin{enumerate}
\item (1.1.5) $\di \int x(x^2+1)^8\,dx = $

\bigskip

\bigskip

\item (1.1.7) $\di \int \frac{1}{2x+7}\,dx = $

\bigskip

\bigskip

\item (1.1.11) $\di \int \frac{e^{\sqrt{x}}}{\sqrt{x}}\,dx = $

\bigskip

\bigskip

\item (1.1.15) $\di \int \sin^2(x)\cos(x)\,dx = $

\bigskip

\bigskip

\item (1.1.19) $\di \int \tan^2(x)\sec^2(x)\,dx = $

\bigskip

\bigskip

\item (1.1.25) $\di \int e^{x^3}x^2\,dx = $

\bigskip

\bigskip

\item (1.1.27) $\di \int\frac{e^x+1}{e^x}\,dx = $

\bigskip

\bigskip

\item (1.1.33) $\di \int\frac{\ln(x^3)}{x}\,dx = $

\bigskip

\bigskip

\item (1.1.77) $\di \int_2^6x\sqrt{x-2}\,dx = $

\bigskip

\bigskip

\item (1.1.81) $\di \int_{-1}^1\frac{1}{1+x^2}\,dx = $
\end{enumerate}
\newpage

\item (2.2. 13) Compute the volume of the solid of revolution obtained by revolving the region bounded by  $y=4-x^2$ and $y=0$ about:
\begin{enumerate}
\item The $x$-axis.

\vspace{2.5in}


\item The line $x=2$.

\vspace{2.5in}
\end{enumerate}

\item (2.4.31) Find the area of the surface generated by revolving $y=x^3$, for $0\leq x\leq 1$, about the $x$-axis.

\newpage

\item State whether the given series converges or diverges. If possible, find its sum.
\begin{enumerate}
\item (3.2.21) $\di \sum_{n=0}^\infty \frac{1}{5^n}$

\vspace{1.25in}

\item (3.2.23) $\di \sum_{n=1}^\infty n^{-4}$.

\vspace{1.25in}

\item (3.2.25) $\di \sum_{n=1}^\infty\frac{10}{n!}$

\vspace{1.25in}

\item (3.2.29) $\di \sum_{n=1}^\infty \frac{1}{2n-1}$.

\vspace{1.25in}

\item (3.2.37) $\di \sum_{n=1}^\infty \frac{3}{n(n+2)}$.
\end{enumerate}
\newpage

\item Determine whether the series converges or diverges:
\begin{enumerate}
\item (3.3.17) $\di \sum_{n=2}^\infty \frac{1}{\sqrt{n^2-1}}$

\vspace{1.7in}

\item (3.3.7) $\di \sum_{n=1}^\infty \frac{n}{n^2+1}$

\vspace{1.7in}

\item (3.3.31) $\di \sum_{n=1}^\infty\frac{\sqrt{n}+3}{n^2+17}$

\vspace{1.7in}

\item (3.3.37) $\di \sum_{n=1}^\infty \frac{1}{3^n+n}$.
\end{enumerate}
\newpage


\item Determine the convergence of the following series:
\begin{enumerate}
\item (3.4.7) $\di \sum_{n=1}^\infty \frac{n!10^n}{(2n)!}$

\vspace{2.5in}

\item (3.4.11) $\di \sum_{n=1}^\infty \frac{10\cdot 5^n}{7^n-3}$.

\vspace{2.5in}

\item (3.5.19) $\di \sum_{n=1}^\infty \frac{(-1)^n}{\sqrt{n}}$.
\end{enumerate}

\newpage

\item Determine the radius and interval of convergence for the following power series:
\begin{enumerate}
\item (3.6.11) $\di \sum_{n=1}^\infty \frac{(-1)^n(x-3)^n}{n}$

\vspace{2.5in}

\item (3.6.19) $\di \sum_{n=0}^\infty \frac{3^n}{n!}(x-5)^n$

\vspace{2.5in}

\item (3.6.31) $\di \sum_{n=1}^\infty \frac{x^n}{n^2}$
\end{enumerate}
\newpage

\item Find a formula for the requested Taylor series:
\begin{enumerate}
\item (3.8.7) $f(x)=\cos x$, expanded about $a=\pi/2$.

\vspace{1.6in}

\item (3.8.11) $f(x)=\dfrac{x}{x+1}$, expanded about $a=1$.

\vspace{1.6in}

\end{enumerate}
\item (3.8.25) Using known Taylor series, determine the Taylor series of the function $f(x)=\cos(x^2)$.

\vspace{1.5in}

\item (3.8.31) Use the first 4 terms of a Taylor series to approximate the value of the integral $\di \int_0^{\sqrt{\pi}}\sin(x^2)\,dx$. 

\newpage

\item Solve the following differential equations:
\begin{enumerate}
\item (4.4.7) $\dfrac{dy}{dx} = \dfrac{y^2+1}{x^2+1}$, where $y(0)=1$.

\vspace{2.25in}

\item (4.4.9) $y' = xe^{-y}$, where $y(0)=1$.

\vspace{2.25in}

\item (4.5.3) $y'+3x^2y=\sin(x)e^{-x^2}$, where $y(0)=1$.
\end{enumerate}

\end{enumerate}
\end{document}