\documentclass[letterpaper,12pt]{article}

\usepackage{ucs}
\usepackage[utf8x]{inputenc}
\usepackage{amsmath}
\usepackage{amsfonts}
\usepackage{amssymb}
\usepackage{amsthm}
\usepackage[margin=1in]{geometry}

\newtheorem{theorem}{Theorem}
\newtheorem{problem}{Problem}
\newcommand{\abs}[1]{\lvert #1\rvert}
\newcommand{\R}{\mathbb{R}}
\newcommand{\C}{\mathbb{C}}
\title{Math 2000 Writing Assignment \#2:\\ The One Where You Choose The Topic\\University of Lethbridge, Spring 2015}
\author{Sean Fitzpatrick}
\begin{document}
 \maketitle

{\bf Due date:} Friday, April 3rd, by 5 pm.

\bigskip

The guidelines for preparing the second assignment are the same as for the first; in particular, you should submit a good copy (hand-written is fine), and cite all sources used. As with every other university class and life in general, plagiarism is not advised. Your finished paper should be around 4 pages long, give or take. (This is for something typed or reasonably hand-written. If you print double-spaced in inch-high letters, you'll need more pages.)

For the second assignment, I am cruelly giving you the freedom to choose your own topic. Take a look at the options below, and either email me or drop by my office to let me know what you want to do. You can also choose a topic that's not on the list. The only restrictions are that it has to be related to the course material, and you can't re-use work that you did for another class (this falls under the advisory against plagiarism above).

\bigskip

Guided activities: I have a few guided activities similar to the last assignment you can choose from, taken from a couple of textbooks. Typing out each project in full isn't practical, so I'll list the topics, and if there's one that you like, you can drop by my office to get the details:
\begin{itemize}
\item Construction of the natural numbers. According to Leopold Kronecker, God gave us the integers, and all else is the work of man. But it turns out that you can construct the natural numbers, too. This project will show how to build the natural numbers using set operations (the von Neumann construction). Your job will be to show that the {\em Peano axioms} for the natural numbers are satisfied by this construction.

Prerequisites: basics of set theory, including indexed families of sets, and familiarity with induction.

\item When does $f^{-1}=1/f$? A common error made by (pre)calculus students is to confuse the inverse of a function with its reciprocal. (It's somewhat understandable, given that the reciprocal {\em is} the inverse for the operation of multiplication in the real number system.) This project will work though figuring out exactly when this error is actually correct. (It has the potential to be slightly open-ended.)

Prerequisites: understanding functions and their inverses (which we'll be covering in Chapter 6), and some comfort with functions of a real variable as encountered in Calculus I.

\item Pascal's Triangle. An exploration of the binomial coefficients (the numbers $\binom{n}{k}$ appearing in the binomial theorem) and their properties. You'll be asked to prove several identities involving binomial coefficients, and for a proof of the binomial theorem itself. (It's a proof by induction.)

Prerequisites: proof by induction, and a willingness to fiddle around algebraically with fractions.

\item RSA cryptography. This project will take you through the basics of the RSA encryption scheme (which is widely used for internet security). This is one of the more interesting projects, but it requires material that we won't have time to cover in class (namely, greatest common divisors and Euclid's theorem). It could be interesting if you're willing to learn a bit on your own -- the necessary material is covered in Chapter 9 of our textbook.

Prerequisites: modular arithmetic, GCDs, and the Euclidean algorithm. Parts of the project are also easier if you know how to work with software like Mathematica or Maple. (It is possible to change the details of the project however.)\\
A smaller version of this project would involve looking at Fermat's Little Theorem and its applications.
\end{itemize}
There are few other projects I can suggest if you have some background in Calculus, including a project on the irrationality of $\pi$ and $e$, and another on complex numbers. Both require some familiarity with sequences and series, however. If you're considering taking Math 3500 (Analysis I) in the future, there are options I can suggest that would be relevant to that course as well.

\bigskip

Another option is to look at the ``Explorations and Activities'' in the textbook. However, you'll have to be careful here to choose one that gives you enough material to work with. Many of them can be answered in a page or less, and you need to produce at least three pages.

\bigskip

Finally, if none of the above appeal to you, you can venture out on your own. In the last few weeks of the course, we'll be studying equivalence relations and the basics of countable and uncountable sets. There's lots of room to go beyond the course on both of these topics. There are other applications to consider, to things like error-correcting codes, and other areas of mathematics that are approachable at the Math 2000 level, like combinatorics (permutations, combinations, and their friends) or graph theory. (To get an idea of the possibilities, you could start with the Wikipedia page on Discrete Mathematics.)
\end{document}
 
