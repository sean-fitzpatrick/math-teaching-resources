\documentclass[letterpaper,12pt]{article}

\usepackage{ucs}
\usepackage[utf8x]{inputenc}
\usepackage{amsmath}
\usepackage{amsfonts}
\usepackage{amssymb}
\usepackage{amsthm}
\usepackage[margin=1in]{geometry}

%\theoremstyle{theorem}
\newtheorem{theorem}{Theorem}
\newtheorem{lemma}{Lemma}
\newtheorem{problem}{Problem}
\theoremstyle{definition}
\newtheorem{definition}{Definition}
\newcommand{\abs}[1]{\lvert #1\rvert}
\newcommand{\R}{\mathbb{R}}
\newcommand{\C}{\mathbb{C}}
\newcommand{\N}{\mathbb{N}}

\title{Math 2000 Writing Assignment \#1: Solutions}
\author{Sean Fitzpatrick}
\begin{document}
 \maketitle

The purpose of this assignment is to investigate the following question: Which positive integers are equal to both the sum {\em and} the product of their proper divisors? We begin with a couple of definitions:
\begin{definition}
 We say that an integer $k\in\N$ is a {\bf proper divisor} of an integer $n\in\N$ if $k|n$ and $k\neq n$.
\end{definition}
\begin{definition}
 We say that a positive integer $n\in\N$ is {\bf perfect} if it is the sum of its proper divisors.
\end{definition}
For example, 6 is perfect, since its proper divisors are 1, 2, and 3, and $1+2+3=6$. It turns out that 6 is the only single-digit perfect number. Indeed, 1 cannot be perfect, since it has no proper divisors. The primes 2, 3, 5, and 7 cannot be perfect, since their only proper divisor is 1. It remains to check 4, 8, and 9. The proper divisors of 4 are 1 and 2, and $1+2=3\neq 4$. The proper divisors of 8 are 1, 2, and 4, and $1+2+4=7\neq 8$, and the proper divisors of 9 are 1 and 3, and $1+4=4\neq 9$.

One can check that there are no other perfect numbers until we reach $n=28$; we see that 28 is indeed perfect since the proper divisors of 28 are 1, 2, 4, 7, and 14, and
\[
 1+2+4+7+14 = 28.
\]
In fact, 28 is the only perfect two-digit number. To see that perfect numbers indeed are rare, we note that we do not see another perfect number until we reach $n=496$. To see that 496 is perfect, we note that $496 = 2^4\cdot 31$, and that 31 is prime. The proper divisors of 496 are thus 1, 2, 4, 8, 16, 31, 62, 124, and 248, and
\[
 1+2+4+8+16+31+62+124+248 = 496.
\]

Another possibility is to look for numbers that are the {\em product} of their proper divisors. These are perhaps less interesting than perfect numbers, since they are more common, and, as we will see below, they can be completely characterized.\footnote{It is possible to prove that all {\em even} perfect numbers are of the form $n=2^{p-1}(2^p-1)$, where $p$ is a prime; however, it is unknown whether or not any odd perfect numbers exist.} Examples of positive integers that are the product of their proper divisors include $6=1\cdot 2\cdot 3$, $8=1\cdot 2\cdot 4$, $10 = 1\cdot 2\cdot 5$, $14 = 1\cdot 2\cdot 7$, and $15=1\cdot 3\cdot 5$.

We note that all of the numbers with this property that we listed above can either be written as the cube of a prime ($8=2^3$), or as the product of two distinct primes ($6=2\cdot 3$, $10=2\cdot 5$, $14=2\cdot 7$, $15=3\cdot 5$). Indeed, all numbers of either form are such that they equal the product of their proper divisors, and they are the only numbers of this form, as we state in the following:
\begin{lemma}\label{b}
 Let $n$ be a positive integer such that $n$ is equal to the product of its proper divisors. Then either $n=p^3$, where $p$ is a prime, or $n=pq$, where $p$ and $q$ are distinct primes. Moreover, all numbers of either form are equal to the product of their proper divisors.
\end{lemma}
\begin{proof}
 We first show that all numbers of either form have the desired property. If $n=p^3$, then the proper divisors of $n$ are $1, p, \text{ and } p^2$, and $1\cdot p\cdot p^2 = p^3$. If $n=pq$ with $p\neq q$, then the proper divisors of $n$ are $1, p, \text{ and } q$, and $1\cdot p\cdot q = pq$.

 We now show that these are the only possiblities. If $n=p$ is itself prime, then its only proper divisor is 1, and $1\neq p$. If $n=p^2$, then the proper divisors of $n$ are 1 and $p$, and $p\neq p^2$ for any prime $p$.

 In all other cases, $n$ is the product of three or more primes, at least two of which are distinct. If $n=p^2q$ with $p\neq q$, then the proper divisors of $n$ are $1,p,p^2,q$, and $pq$, and the product of these is $p^4q^2 = n^2> n$. If $n=pqr$ with $p,q,r$ all distinct, then the divisors of $n$ are $1,p,q,r,pq,pr$, and $qr$, and their product is $p^3q^3r^3=n^3> n$. If $n=p^2qm$ or $n=pqrm$ for some $m>1$, it is clear that the product of the proper divisors of $n$ will again exceed $n$.
\end{proof}
We now note that 6 was both a perfect number, and the product of its proper divisors. In fact, 6 is the only such number, as well will now proceed to prove.
\begin{theorem}\label{a}
 If $n\in \N$ is a perfect number that is equal to the product of its proper divisors, then $n=6$.
\end{theorem}
Before proving the theorem, we will first prove the following:
\begin{lemma}\label{c}
 If $n=p^3$, where $p$ is prime, then $n$ is not perfect.
\end{lemma}
\begin{proof}
 The proper divisors of $p^3$ are $1, p$, and $p^2$. Thus, $p^3$ is perfect if and only if
\[
 1+p+p^2=p^3.
\]
However, it is readily checked (using either software or the formula for the roots of a cubic polynomial) that there are no integer solutions to the equation $x^3-x^2-x-1=0$, and thus in particular, no prime integer solutions.
\end{proof}
It thus remains to show that the only perfect number of the form $n=pq$ that is perfect is 6. We note that if $n=2q$ for some prime $q$, then $n$ is perfect provided that $1+2+q=2q$, and solving for $q$ we see that we must have $q=3$ and thus $n=6$. Similarly, if $n=3q$ for some prime $q$, then we must have $1+3+q=3q$, which implies that $q=2$, and again, $n=6$. 

Now, if we let $p=5$, we would have to have $1+5+q=5q$, implying that $4q=6$, but this is not possible if $q$ is an integer. Similarly, if $p=7$, we end up with $6q=8$, which also does not have an integer solution. In fact, this is true for all primes $p\geq 5$, and once we show this, our theorem follows. Thus, we have:
\begin{proof}[Proof of Theorem \ref{a}]
 Suppose that $n$ is a perfect number, and that $n$ is equal to its proper divisors. By Lemma \ref{b}, we know that either $n=p^3$ for some prime $p$, or $n=pq$, for distinct primes $p$ and $q$. Since Lemma \ref{c} rules out the possibility that $n=p^3$, we must have $n=pq$ for primes $p$ and $q$. 

 Notice that if $n=pq$ is perfect, then we must have $pq=1+p+q$, which implies that
\[
 q+1 = p(q-1).
\]
If $p\geq 5$, then we would have
\[
 q+1 = p(q-1)\geq 5(q-1) = 5q-5,
\]
and thus we have $4q\leq 6$, or $q\leq 3/2<2$. But this is impossible, since there are no primes less than 2. This leaves us with the possiblities of $p=2$ or $p=3$, but as we saw above, both cases lead to the conclusion that $n=6$.
\end{proof}




\end{document}
 
