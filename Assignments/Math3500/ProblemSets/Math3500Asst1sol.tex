\documentclass[12pt,letterpaper]{article}
\usepackage[utf8]{inputenc}
\usepackage{amsmath}
\usepackage{amsfonts}
\usepackage{amssymb}
\usepackage[margin=1in]{geometry}
\author{Sean Fitzpatrick}
\title{Math 3500 Assignment \#1 Solutions}

\newcommand{\abs}[1]{\lvert #1\rvert}

\begin{document}
\maketitle

Most assignments were done reasonably well. A few basic comments: first, make sure you explain your work as completely and clearly as possible. If you're not sure about how to write a good solution, you can use these solutions not just for the answers, but also as a guideline for how to present your work. Second, on working together: a reminder that collaboration is OK, but copying is not. I'll give you a couple of attempts to figure out the difference between them. A good approach is to discuss the problems with your friends, but write out your assignment on your own. 

(If you {\em have} figured something out and are explaining it to others, make certain that your explanation is good, or those you give it to probably won't do well enough for full marks! On the other hand, if you're asking for help, don't accept a solution until you're sure you completely understand it -- it's pretty obvious when you don't.)

Finally -- and most of you were pretty good with this -- a good rule of thumb is that every solution, even to problems that are computational, should involve at least some English words. If you look at your solution and don't see any complete sentences, you probably haven't written enough.
\begin{enumerate}
\item \begin{enumerate}
\item By definition, $\displaystyle \binom{n}{r} = \frac{n!}{r!(n-r)!}$. Thus, using the fact that $k! = k(k-1)!$ for any $k\in \mathbb{N}$, we have
\begin{align*}
\binom{n}{r}+\binom{n}{r-1} & = \frac{n!}{r!(n-r)!}+\frac{n!}{(r-1)!(n-r+1)!}\\
& = n!\left[\frac{n-r+1}{r!(n-r+1)!}+\frac{r}{r!(n-r+1)!}\right]\\
& = \frac{n!}{r!(n-r+1)!}(n-r+1+r)\\
& = \frac{(n+1)!}{r!(n+1-r)!}\\
& = \binom{n+1}{r},
\end{align*}
which is what we needed to show.

\item Let $P(n)$ be the statement that $(a+b)^n = \sum_{r=0}^n\binom{n}{r}a^{n-r}b^r$. We see that $P(1)$ is true, since
\[
(a+b)^1 = a+b = \binom{1}{0}a^1b^0+\binom{1}{1}a^0b^1.
\]
(One could also start with the even easier statement of $P(0)$, which is simply that $1=1$.)

We suppose that $P(k)$ is true for some $k\geq 1$; thus, we assume $(a+b)^k = \sum_{r=0}^k\binom{k}{r}a^{k-r}b^k$. We then have that
\begin{align*}
(a+b)^{k+1} & = (a+b)(a+b)^k\\
& = (a+b)\sum_{r=0}^k\binom{k}{r}a^{k-r}b^r \text{ (since we assume } P(k) \text{ is true)}\\
& = a\left(a^k+\sum_{r=1}^k\binom{k}{r}a^{k-r}b^r\right)+b\left(\sum_{r=0}^{k-1}\binom{k}{r}a^{k-r}b^r+b^k\right)\\
& = a^{k+1}+\sum_{r=1}^k\binom{k}{r}a^{k+1-r}b^r+\sum_{r=0}^{k-1}\binom{k}{r}a^{r-k}b^{r+1} + b^{k+1}\\
& = a^{k+1}+\sum_{r=1}^k\binom{k}{r}a^{k+1-r}b^r+\sum_{r=1}^{k}\binom{k}{r-1}a^{k+1-r}b^{r} + b^{k+1}\\
&\hspace{2in} \text{ (shifting the index by 1 in the second sum)}\\
& = a^{k+1}+\sum_{r=1}^k\left[\binom{k}{r}+\binom{k}{r-1}\right]a^{k+1-r}b^r+b^{k+1}\\
& = a^{k+1}+\sum_{r=1}^k\binom{k+1}{r}a^{k+1-r}b^r + b^{k+1} \text{ (using part (a))}\\
& = \sum_{r=0}^{k+1}\binom{k+1}{r}a^{k+1-r}b^r,
\end{align*}
which shows that $P(k+1)$ is true whenever $P(k)$ is true. Thus, $P(n)$ is true for all $n\in\mathbb{N}$ by the principle of mathematical induction.
\end{enumerate}
\item \begin{enumerate}
\item We can begin the induction with $n=2$, where the result is given as the usual de Morgan's law. (Note that the case $n=1$ doesn't really make sense here. You can start induction at a natural number other than $n=1$ if necessary.)

If we assume that $(A_1\cup \cdots \cup A_k)^c = A_1^c\cap \cdots \cap A_k^c$ for some $k\geq 2$, then, since the union and intersection of three or more sets are associative, we have
\begin{align*}
(A_1\cup\cdots A_k\cup A_{k+1})^c & = \left[(A_1\cup\cdots\cup A_k)\cup A_{k+1}\right]^c\\
& = (A_1\cup\cdots\cup A_k)^c\cap A_{k+1}^c \text{ (using the result for $n=2$)}\\
& = (A_1^c\cap\cdots\cap A_k^c)\cap A_{k+1}^c \text{ (by the induction hypothesis)}\\
& = A_1^c\cap\cdots \cap A_k^c\cap A_{k+1}^c \text{ (by associativity)}.
\end{align*}
Thus, the result holds for $n=k+1$ whenever it holds for $n=k$, and thus it is valid for all $n\geq 2$ by induction.

(Note: A complete solution really should reference the fact that associativity is needed to make the proof work; however, since everyone is so used to this fact, I didn't penalize you for using it without saying so.)
\item The intersection does not need to be infinite, even if all the sets $A_n$ are infinite. For example, if we let $A_n=[0,1/n]$, for $n=1,2,3,\ldots, $ then each $A_n$ contains infinitely many elements, since it's an interval, and $A_{n+1}\subseteq A_n$ for all $n$. However, we have
\[
\bigcap_{n=1}^\infty [0,1/n] = \{0\},
\]
which is a finite set, containing only the element zero. To see that this is true, notice that $0\in A_n$ for all $n\in\mathbb{N}$, so that $0\in \bigcap A_n$ by definition. Moreover, if $x>0$, then we know that $x<1/n$ for some sufficiently large $n$, by the Archimedean property of $\mathbb{R}$, and thus $x\notin A_n$, so $x\notin \bigcap A_n$. 

Remark: some people may have been unsure as to what was meant by ``infinite set'' or by the intersection of an infinite family of sets. If you're not sure how something in a problem is defined, this is precisely the sort of thing you should be asking about at office hours or on our Piazza forum before proceeding. It's really very little effort to ask for clarification and it can save you a lot of trouble.

\item The method of induction allows one to prove that a statement $P(n)$ is true for all values of $n\in\mathbb{N}$. Even though the result then holds for (countably) infinitely many $n$, each $n$ itself is {\em finite}: one cannot conclude from $P(n)$ for all $n$ that ``$P(\infty)''$ is true. For example, $P(n)$ could be the statement that $\{1,2,\ldots, n\}$ is a finite set. This is true for all $n$, but clearly is false for the set $\{1,2,3,\ldots\} = \mathbb{N}$.

\item The result from part (c) is valid, although one cannot prove it using induction. (Note that this means we can {\bf not} simply infer the result from part (a)!) To prove the equality of sets, we must show that each one is contained in the other. To do so is simply an exercise in following the definitions. For those who didn't take the time to ask (or look it up), we define
\begin{align*}
\bigcap_{n=1}^\infty S_n & = \{x : x\in S_n \text{ for all } n\in\mathbb{N}\}, \text{ and}\\
\bigcup_{n=1}^\infty S_n & = \{x : x\in S_n \text{ for some } n\in\mathbb{N}\}.
\end{align*}
We first show that $\bigcap A_n^c\subseteq \left(\bigcup A_n\right)^c$. If $x\in \bigcap A_n^c$, then $x\in A_n^c$ for all $n$, which means that $x\notin A_n$ for all $n$, by definition of the complement. But this means that $x\notin \bigcup A_n$, so $x\in \left(\bigcup A_n\right)^c$.

Similarly, if $x\in\left(\bigcup A_n\right)^c$, then $x\notin \bigcup A_n$, so $x\notin A_n$ for all $n$, which means that $x\in A_n^c$ for all $n$, and therefore $x\in \bigcap A_n^c$. 
\end{enumerate}
\item \begin{enumerate}
\item We know that for all $x,y\in\mathbb{R}$, we have $\abs{x+y}\leq\abs{x}+\abs{y}$ (the triangle inequality). Thus, we have
\[
\abs{x} = \abs{x-y+y} \leq \abs{x-y}+\abs{y},
\]
and rearranging, we have $\abs{x}-\abs{y}\leq \abs{x-y}$. Reversing the roles of $x$ and $y$, we get
\[
\abs{y}-\abs{x} \leq \abs{y-x}=\abs{x-y},
\]
which is equivalent to $-\abs{x-y}\leq \abs{x}-\abs{y}$. Since $\abs{b}\leq a$ if and only if $-a\leq b\leq a$ for any $a,b\in\mathbb{R}$, the result follows.
\item Suppose that $\abs{x-y}<c$. Then from part (a) we have that
\[
\abs{x}-\abs{y}\leq \abs{\abs{x}-\abs{y}}\leq \abs{x-y}<c.
\]
Rearranging, we have $\abs{x}<\abs{y}+c$, as required.
\item We prove the contrapositive: suppose that $x\neq y$. Then $\abs{x-y}>0$, so we can take $\epsilon=\abs{x-y}/2>0$, and it is not the case that $\abs{x-y}<\epsilon$.
\end{enumerate}
\item For all $x,y\geq 0\in \mathbb{R}$, we have (since $a^2\geq 0$ for any real number $a$)
\[
0\leq (\sqrt{x}-\sqrt{y})^2 = x-2\sqrt{xy}+y.
\]
We can rearrange this to give $\sqrt{xy}\leq (x+y)/2$, as was to be shown.

Note: there are two pitfalls here. The most common one is to start with the inequality you were supposed to prove, and deduce some true statement. This is not a valid logical argument: showing that a statement $P$ implies a true statement does not mean that $P$ itself has to be true. (We know that $P\Rightarrow Q$ is valid when $P$ is false and $Q$ is true.) The other thing to be careful of is that squaring (or taking square roots) in an inequality is not always valid, unless you've noted in advance that all terms involved are non-negative.

\item Suppose that $S$ and $T$ are nonempty subsets of $\mathbb{R}$ (note that these do {\bf not} have to be intervals!). If $a=\inf T$, then $a\leq s$ for all $s\in S$, since if $s\in S$, then $s\in T$ and $a$ is a lower bound for $T$. But this means that $\inf S\geq \inf T=a$, since $\inf S$ is the {\em greatest} lower bound. Similarly, the supremum of $T$ is an upper bound for $S$, since $S\subseteq T$, so $\sup S\leq \sup T$, since $\sup S$ is the least upper bound of $S$. Finally, since $S$ is nonempty, we can take any $s\in S$, and then (as we saw in class), we have $\inf S\leq s\leq \sup S$, by definition of the infimum and supremum. The result now follows by the transitivity of the order relation on $\mathbb{R}$.
\end{enumerate}
\end{document}