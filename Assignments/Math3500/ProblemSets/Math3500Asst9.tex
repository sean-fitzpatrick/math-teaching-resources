\documentclass[letterpaper,12pt]{article}

\usepackage{ucs}
\usepackage[utf8x]{inputenc}
\usepackage{amsmath}
\usepackage{amsfonts}
\usepackage{amssymb}
\usepackage[margin=1in]{geometry}
%\usepackage{enumerate}

\newcommand{\R}{\mathbb{R}}
\newcommand{\N}{\mathbb{N}}
\newcommand{\Z}{\mathbb{Z}}
\newcommand{\Q}{\mathbb{Q}}
\renewcommand{\ss}{\subseteq}

\newcommand{\abs}[1]{\lvert #1\rvert}
\newcommand{\Abs}[1]{\left| #1\right|}

\title{Math 3500 Assignment \#9\\University of Lethbridge, Fall 2014}
\author{Sean Fitzpatrick}
\begin{document}
 \maketitle


{\bf Due date}: Friday, November 28th, by 6 pm.

\bigskip

This is the last regular assignment for the course. But don't forget that you have an essay assignment to submit by the last lecture!
\begin{enumerate}
 \item Let $f$ be a bounded function on $[a,b]$, let $\mathcal{P}$ denote the set of all partitions of $[a,b]$, and let $P\in \mathcal{P}$ be an arbitrary partition of $[a,b]$.
\begin{enumerate}
 \item Prove that $U(f)\geq L(f,P)$, where $U(f) = \inf\{U(f,P) | P\in\mathcal{P}\}$.
 \item Prove that $U(f)\geq L(f)$, where $L(f)=\sup\{L(f,P) | P\in\mathcal{P}\}$.
\end{enumerate}
 \item Let $f$ be a bounded function on $[a,b]$.
\begin{enumerate}
 \item Prove that $f$ is integrable on $[a,b]$ if and only if there exists a sequence of partitions $(P_n)_{n=1}^\infty$ satisfying
\[
 \lim_{n\to\infty}[U(f,P_n)-L(f,P_n)]=0.
\]
 \item For each $n$, let $P_n$ denote the uniform partition of $[0,1]$ into $n$ equal subintervals of length $1/n$, and let $f(x)=x$. Find formulas for $U(f,P_n)$ and $L(f,P_n)$ in terms of $n$.

Hint: recall the summation formula $1+2+\cdots+n = n(n+1)/2$.
 \item Use the results from (a) and (b) to prove that $f(x)=x$ is integrable on $[0,1]$.
\end{enumerate}
 \item Let $f:[a,b]\to\R$ be bounded and increasing. Show that $f$ is integrable on $[a,b]$.

Hint: Use a uniform partition and the results of the previous problem. You should find that for an increasing function, the sum $U(f,P_n)-L(f,P_n)$ simplifies considerably.
\newpage
 \item Define the function $\displaystyle H(x) = \int_1^x\frac{1}{t}\,dt$, where $x>0$.
\begin{enumerate}
 \item What is the value of $H(1)$? What is $H'(x)$ for any $x>0$?
 \item Show that if $0<x<y$, then $H(x)<H(y)$; that is, that $H$ is strictly increasing on $(0,\infty)$.
 \item Show that $H(cx)=H(c)+H(x)$ for any $c>0$.

Hint: it's possible to prove this using $u$-substitution, but a more efficient/clever approach is to treat $c$ as a constant and consider the derivative of $g(x)=H(cx)$ with respect to $x$ using the Chain Rule. (Of course via the FTC the Chain Rule and $u$-substitution are really just two sides of the same coin.) Keep in mind that two functions with the same derivative must differ by a constant.
 \item Use a similar argument to show that $H(x^a)=aH(x)$.
\end{enumerate}
Note: One often writes the function $H(x)$ as $\ln(x)$, and refers to this function as the natural logarithm. Parts (c) and (d) then tell us that $\ln(xy)=\ln x+\ln y$ and $\ln(x^y)=y\ln x$. Since $H$ is strictly increasing on $(0,\infty)$, it is one-to-one and therefore has a well-defined inverse function, which is usually denoted by $H^{-1}(x)=e^x$.

\item ({\bf Bonus}) Define a bounded function $f$ on $[0,1]$ by $\displaystyle f(x)=\begin{cases} 1, & \text{ if } x=1/n\\ 0, & \text{ otherwise}\end{cases}$. \\Prove that $f$ is integrable on $[0,1]$.





\end{enumerate}

\end{document}
 
