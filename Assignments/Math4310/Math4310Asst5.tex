\documentclass[letterpaper,12pt]{article}

\usepackage{ucs}
\usepackage[utf8x]{inputenc}
\usepackage{amsmath}
\usepackage{amsfonts}
\usepackage{amssymb}
\usepackage[margin=1in]{geometry}
\usepackage{enumerate}

\newcommand{\abs}[1]{\lvert #1\rvert}
\newcommand{\len}[1]{\lVert #1\rVert}
\newcommand{\R}{\mathbb{R}}
\newcommand{\x}{\mathbf{x}}
\newcommand{\y}{\mathbf{y}}
\newcommand{\inter}[1]{\overset{\,\,\circ}{#1}}
\newcommand{\T}{\mathcal{T}}
\DeclareMathOperator{\Int}{Int}

\title{Math 4310 Assignment \#5\\University of Lethbridge, Fall 2014}
\author{Sean Fitzpatrick}
\begin{document}
 \maketitle

{\bf Due date:} Friday, October 10th, by 5 pm.

\bigskip

Please submit solutions to the following problems (except where indicated). As usual, all the textbook exercises are recommended as practice problems if they don't appear as an assigned problem below.

\begin{enumerate}
\item If $X\times Y$ has the product topology, and $A\subseteq X$, $B\subseteq Y$, show that $\overline{A\times B} = \overline{A}\times \overline{B}$, $(A\times B)^\circ = \inter{A}\times\inter{B}$, and $\partial (A\times B) = \partial A\times \partial B$.
\item Given a countable number of spaces $X_1,X_2,\ldots$, the product $X=\prod_i X_i$ is the set of all points of the form $x=(x_1,x_2,\ldots)$, where $x_i\in X_i$ for $i=1,2,\ldots$. The product topology on $X$ is defined to be the coarsest topology for which all of the projections $\pi_i:X\to X_i$ are continuous. (This is sometimes known as the {\em weak} product topology.)
\begin{enumerate}
 \item Construct a basis for this topology in terms of the open subsets of the spaces $X_1,X_2,\ldots$.
 \item (In which a hint for part (a) is given.) The {\em box topology} on $X=\prod_i X_i$ is the topology defined by the basis $\mathcal{B} = \{U_1\times U_2\times \cdots \,|\, U_i \text{ is open in } X_i, i=1,2,\ldots\}$. Show that the box topology contains the product topology in part (a) (this is your hint: your answer for (a) should be different than the basis in (b)!), and that the two sets are equal if and only if $X_i$ has the indiscrete topology for all but finitely many values of $i$.
\end{enumerate}
Note: this problem is a bit tricky. For 2(a), start by asking yourself which sets definitely have to be open for the projections to be continuous. These won't quite form a basis on their own, but they do form a subbasis.
\end{enumerate}
For problems 3 and 4, the diagonal map $\Delta:X\to X\times X$ is given by $\Delta(x) = (x,x)$, and the {\em diagonal} in $X\times X$ is the image $\Delta(X) = \{(x,x)\,|, x\in X\}$ of the diagonal map.

\begin{enumerate}
\setcounter{enumi}{2}
\item Prove that a topological space $X$ is discrete if and only if $\Delta(X)$ is an open subset of $X\times X$ in the product topology.

Hint: if you read the Chapter 10 supplement you'll get about 80\% of the proof.
\item Prove that $X$ is Hausdorff if and only if $\Delta(X)$ is a closed subset of $X\times X$ in the product topology.
\item Suppose that $X$ is Hausdorff.
\begin{enumerate}
 \item Show that $\{x\}$ is closed, for any $x\in X$.
 \item Let $\mathcal{U}$ denote the collection of all open neighbourhoods of some $x\in X$. Prove that $\bigcap_{U\in \mathcal{U}}U = \{x\}$.
\end{enumerate}
 \item Suppose that $X$ is an infinite set with the finite complement (co-finite) topology.
\begin{enumerate}
 \item Show that $\{x\}$ is closed, for any $x\in X$.
 \item Show that, in spite of part (a), $X$ is not Hausdorff.
 \item Let $\mathcal{U}$ denote the collection of all open neighbourhoods of some $x\in X$. Show that it is not true that $\bigcap_{U\in \mathcal{U}}U = \{x\}$.
\end{enumerate}
 \item (Bonus) Show that $[0,1)\times [0,1)$ is homeomorphic to $[0,1]\times [0,1)$.

(If you sketch the two spaces as subsets of $\R^2$ you can see visually how one can be continuously deformed into the other, but to give a rigorous proof you need to come up with a transformation $T:\R^2\to \R^2$ that is maps $[0,1)\times [0,1)$ onto $[0,1]\times [0,1)$. I couldn't think of one off the top of my head so I figured I probably shouldn't assign the problem for credit.)


\end{enumerate}
\end{document}
 
