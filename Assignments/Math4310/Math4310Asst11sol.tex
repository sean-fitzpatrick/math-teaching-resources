\documentclass[letterpaper,12pt]{article}

\usepackage{ucs}
\usepackage[utf8x]{inputenc}
\usepackage{amsmath}
\usepackage{amsfonts}
\usepackage{amssymb}
\usepackage[margin=1in]{geometry}
\usepackage{enumerate}

\newcommand{\abs}[1]{\lvert #1\rvert}
\newcommand{\len}[1]{\lVert #1\rVert}
\newcommand{\R}{\mathbb{R}}
\newcommand{\N}{\mathbb{N}}
\newcommand{\Z}{\mathbb{Z}}
\newcommand{\x}{\mathbf{x}}
\newcommand{\y}{\mathbf{y}}
\newcommand{\inter}[1]{\overset{\,\,\circ}{#1}}
\newcommand{\T}{\mathcal{T}}
\newcommand{\cla}[1]{\left[ #1\right]}
\DeclareMathOperator{\Int}{Int}

\title{Math 4310 Assignment \#11 Solutions\\University of Lethbridge, Fall 2014}
\author{Sean Fitzpatrick}
\begin{document}
 \maketitle


\begin{enumerate}
\item \begin{enumerate}
\item Let $\gamma: [0,1]\to X$ be a path, and let $\rho:[0,1]\to [0,1]$ be any continuous function such that $\rho(0)=0$ and $\rho(1)=1$. Prove that the paths $\gamma$ and $\gamma\circ \rho$ are homotopic.

Hint: $\rho$ is itself a path from 0 to 1 in $[0,1]$, and all such paths are homotopic in $[0,1]$.

\bigskip

Let $I:[0,1]\to [0,1]$ denote the identity map $I(s)=s$. Given any $\rho:[0,1]\to [0,1]$ preserving the endpoints, the map
\[
 F(s,t) = ts+(1-t)\rho(s)
\]
is a homotopy relative to $\{0,1\}$ between the maps $I$ and $\rho$, so $I$ and $\rho$ are homotopic as paths in $[0,1]$. Given any path $\gamma:{0,1}\to X$, the maps $\gamma\circ\rho$ and $\gamma\circ I = \gamma$ are also paths, and since $\rho\underset{F}{\simeq} I$, we have $\gamma\circ\rho \underset{G}{\simeq}\gamma$, where $G:[0,1]\times [0,1]\to X$ is the homotopy
\[
 G(s,t) = \gamma\circ F(s,t) = \gamma(ts+(1-t)\rho(s)).
\]

\bigskip

\item Let $\alpha, \beta$, and $\gamma$ be loops based at a point $x_0\in X$. Write down explicit formulas for $\alpha\ast(\beta\ast\gamma)$ and $(\alpha\ast\beta)\ast\gamma$.

\bigskip

Recalling that for two loops $\delta$ and $\epsilon$ (not intending to cause analysis flashbacks, but the first three Greek letters were already taken), we define
\[
 \delta\ast\epsilon(s) = \begin{cases}\delta(2s), & 0\leq s\leq 1/2\\ \epsilon(2s-1), & 1/2\leq s\leq 1\end{cases},
\]
we have
\begin{align*}
 \alpha\ast(\beta\ast\gamma)(s) &= \begin{cases} 
 				   \alpha(2s), & 0\leq s\leq 1/2\\
                                   (\beta\ast\gamma)(2s-1), & 1/2\leq s\leq 1
                                  \end{cases}\\
& = \begin{cases}
\alpha(2s), & 0\leq s\leq 1/2\\
\beta(2(2s-1)), & 0\leq 2s-1\leq 1/2\\
\gamma(2(2s-1)-1), &  1/2\leq 2s-1\leq 1
\end{cases}\\
& = \begin{cases}
     \alpha(2s), & 0\leq s\leq 1/2\\
     \beta(4s-2), & 1/2\leq s\leq 3/4\\
     \gamma(4s-3), & 3/4\leq s\leq 1
    \end{cases}
\end{align*}
Similarly, we find that
\[
 (\alpha\ast\beta)\ast\gamma = \begin{cases} \alpha(4s), &0\leq s\leq 1/4\\ \beta(4s-1), & 1/4\leq s\leq 1/2\\\gamma(2s-1), & 1/2\leq s\leq 1\end{cases}.
\]

\bigskip

\item Prove that $\cla{\alpha}\ast(\cla{\beta}\ast\cla{\gamma}) = (\cla{\alpha}\ast\cla{\beta})\ast\cla{\gamma}$.

Hint: use (a), and try the map $\displaystyle \rho(s)=\begin{cases} s/2 & \text{ if } 0\leq s\leq 1/2\\ s-1/4 & \text{ if } 1/2\leq s\leq 3/4\\ 2s-1 & \text{ if } 3/4\leq s\leq 1\end{cases}$.

\bigskip

With $\rho(s)$ as given, we note that 
\begin{align*}
 \text{If } 0\leq s\leq 1/2, &\text{ then } 0\leq \rho(s)=s/2\leq 1/4,\\
 \text{if } 1/2\leq s\leq 3/4, &\text{ then } 1/4\leq \rho(s)=s-1/4\leq 1/2,\\
 \text{if } 3/4\leq s\leq 1, &\text{ then } 1/2\leq \rho(s) = 2s-1 \leq 1.
\end{align*}
Thus, $\rho:[0,1]\to [0,1]$ is a path from 0 to 1 in $[0,1]$ that maps $[0,1/2]$ to $[0,1/4]$, $[1/2,3/4]$ to $[1/4,1/2]$, and $[3/4,1]$ to $[1/2,1]$. Now we note that for any $s\in [0,1]$,
\begin{align*}
 (\alpha\ast\beta)\ast\gamma(\rho(s)) & = \begin{cases}
                                           \alpha(4\rho(s)), & 0\leq \rho(s)\leq 1/4\\
					   \beta(4\rho(s)-1), & 1/4\leq \rho(s)\leq 1/2\\
					   \gamma(2\rho(s)-1), & 1/2\leq \rho(s)\leq 1
                                          \end{cases}\\
& = \begin{cases}
     \alpha(2s), & 0\leq s\leq 1/2\\
     \beta(4s-2), & 1/2\leq s\leq 3/4\\
     \gamma(4s-3), & 3/4\leq s\leq 1
    \end{cases}\\
& = \alpha\ast(\beta\ast\gamma)(s).
\end{align*}
Since $(\alpha\ast\beta)\ast\gamma(\rho(s)) = \alpha\ast(\beta\ast\gamma)(s)$ for all $s\in [0,1]$, it follows that $\cla{(\alpha\ast\beta)\ast\gamma} = \cla{\alpha\ast(\beta\ast\gamma)}$ by part (a).
\end{enumerate}

\bigskip

\item Let $X$ be a space and let $\alpha,\beta:[0,1]\to X$ be two paths from $x_0$ to $x_1$, for two points $x_0,x_1\in X$. These paths define isomorphisms $\varphi_\alpha,\varphi_\beta:\pi_1(X,x_0)\to \pi_1(X,x_1)$, but as noted in class, they may be different isomorphisms. Prove that the isomorphism $\varphi_\beta$ is the composition of $\varphi_\alpha$ with the inner automorphism of $\pi_1(X,x_1)$ induced by the element $\cla{\beta^{-1}\ast\alpha}$.

\bigskip

We recall that for any path $\delta$ from $x_0$ to $x_1$, the isomorphism $\varphi_\delta$ is given by
\[
 \varphi_\delta(\cla{\gamma}) = \cla{\delta^{-1}\ast\gamma\ast\delta}.
\]
Note that the product on the right is given by concatenation of paths within the larger path groupoid $G\rightrightarrows X$ and not by the group multiplication in $\pi_1(X,x_1)$, since $\delta$ is a path from $x_0$ to $x_1$ and not a loop. Given two paths $\alpha,\beta:[0,1]\to X$ from $x_0$ to $x_1$, we see that $\beta^{-1}\ast \alpha$ is a loop based at $x_1$, since $\beta^{-1}$ takes us from $x_1$ to $x_0$, and $\alpha$ takes us back to $x_1$. Note that the inverse of $\cla{\beta^{-1}\ast\alpha}$ is given by $\cla{\alpha^{-1}\ast\beta}$. Thus, given a loop $\gamma$ based at $x_0$, we have
\[
 \varphi_\alpha(\cla{\gamma}) = \cla{\alpha^{-1}\ast\gamma\ast\alpha},
\]
and
\begin{align*}
 \cla{\beta^{-1}\ast\alpha}\ast\varphi_\alpha(\cla{\gamma})\ast\cla{\beta^{-1}\ast\alpha}^{-1} & = \cla{\beta^{-1}\ast(\alpha\ast\alpha^{-1})\ast\gamma\ast(\alpha\ast\alpha^{-1})\ast\beta}\\
& = \varphi_{\beta}([\alpha\ast\alpha^{-1}]\ast\cla{\gamma}\ast[\alpha\ast\alpha^{-1}])\\
& = \varphi_\beta(\cla{\gamma}).
\end{align*}



\bigskip

\item Prove that the two isomorphisms in the previous problem are the same if and only if $\pi_1(X,x_0)$ is Abelian.

\bigskip

If $\pi_1(X,x_0)$ is Abelian, then so is $\pi_1(X,x_1)$, since the two groups are isomorphic. With $g=[\beta^{-1}\ast\alpha]$, we have, for any $[\gamma]\in \pi_1(X,x_0)$, that
\[
 \varphi_\beta(\cla{\gamma}) = g\varphi_{\alpha}([\gamma])g^{-1} = gg^{-1}\varphi_{\alpha}(\cla{\gamma}) = \varphi_\alpha(\cla{\gamma}).
\]
Conversely, suppose that all such isomorphisms $\varphi_\alpha,\varphi_\beta$ are equal, and let $g_1= \cla{\gamma_1},g_2=\cla{\gamma_2}\in \pi_1(X,x_0)$. Choose any path $\alpha$ from $x_0$ to $x_1$, and note that $\gamma_1\ast\alpha$ is also a path from $x_0$ to $x_1$. (This is the path that follows $\gamma_1$ from $x_0$ back to $x_0$ and then $\alpha$ from $x_0$ to $x_1$. By assumption, $\varphi_\alpha = \varphi_{\gamma_1\ast\alpha}$, which gives
\[
 \cla{\alpha^{-1}\gamma_2\alpha}=\varphi_\alpha(g_2)=\varphi_{\gamma_1\ast\alpha}(g_2) = \cla{\alpha^{-1}\ast\gamma_1^{-1}\ast\gamma_2\ast\gamma_1\ast\alpha} = \varphi_\alpha(\cla{\gamma_2^{-1}\ast\gamma_1\ast\gamma_2}).
\]
Since $\varphi_\alpha$ is an isomorphism, it is a bijection, so
\[
 \cla{\gamma_2} = \cla{\gamma_1}^{-1}\ast\cla{\gamma_2}\ast\cla{\gamma_1},
\]
from which it follows that $\pi_1(X,x_0)$ is Abelian.

\bigskip

\item Given spaces $X$ and $Y$, let $\cla{X,Y}$ denote the set of homotopy classes of maps $f:X\to Y$.
\begin{enumerate}
\item Let $I=[0,1]$. Show that for any space $X$, $\cla{X,I}$ contains a single element.

\bigskip

Let $X$ be a space and let $f,g:X\to I$ be continuous. Since $I$ is convex, we have the homotopy
\[
 F(x,t) = tg(x)+(1-t)f(x)
\]
between $f$ and $g$. Since $f$ and $g$ were arbitary, $\cla{X,I} = \{\cla{f}\}$ for any $f:X\to I$.

\bigskip

\item Show that if $Y$ is path connected, then the set $\cla{I,Y}$ contains a single element.

\bigskip

Let $Y$ be a space and let $f,g:I\to Y$ be continuous maps. Suppose $f(0)=x_0$ and $g(0)=x_1$. Since $Y$ is path connected, there exists a path $\gamma:[0,1]\to Y$ with $\gamma(0)=x_0$ and $\gamma(1)=x_1$. Consider the map $F:[0,1]\times [0,1]\to Y$ given by
\[
 F(s,t) = \begin{cases}
           f((1-3t)s), & \text{ if } 0\leq t\leq 1/3\\
	   \gamma(3t-1), & \text{ if } 1/3\leq t\leq 2/3\\
	   g((3t-2)s), & \text{ if } 2/3\leq t\leq 1
          \end{cases}.
\]
Then $F(s,0) = f(s)$, $F(s,1) = g(s)$, and $F$ is continuous by the gluing lemma, since $F(s,1/3) = f(0) = x_0 = \gamma(0)$ for all $s$, and $F(s,2/3) = \gamma(1) = x_1 = g(0)$ for all $s$.

Or, to put it another way, $f$ and $g$ are both homotopic to constant maps, and any two constant maps are homotopic, since $Y$ is path-connected. Since homotopy of maps is an equivalence relation, $f$ must be homotopic to $g$.

\bigskip

\end{enumerate}
\item ({\bf Do not submit}) A space $X$ is called {\bf contractible} if the identity map $i_X:X\to X$ is homotopic to a constant map. (If $f$ is homotopic to a constant map, we say $f$ is {\bf nullhomotopic}.)
\begin{enumerate}
\item Show that $I$ and $\R$ are contractible.

\bigskip

With either $X=I$ or $X=\R$, define $F:X\times I\to X$ by $F(x,t) = (1-t)x$. Then $F$ is clearly continuous, $F(x,0)=x$ is the identity map, and $F(x,1)=0$ is a constant map.

\bigskip

\item Show that a contractible space is path-connected.

\bigskip

Suppose that $X$ is contractible, and let $x_1,x_2\in X$. If $I_X:X\to X$ denotes the identity map, let $F(x,t)$ be a homotopy between $I_X$ and a constant map $g(x)=x_0$, for some $x_0$ in $x$, so $F(x,0)=x$ for all $x\in X$, and $F(x,1)=x_0$ for all $x$ in $X$. Now define a path $\gamma:[0,1]\to X$ by
\[
 \gamma(t) = \begin{cases}
              F(x_1,2t), & 0\leq t\leq 1/2\\
	      F(x_2,2-2t), & 1/2\leq t\leq 1
             \end{cases}.
\]
Then $\gamma$ is continuous by the gluing lemma, since $F(x_1,2(1/2))=F(x_1,1)=x_0$ and $F(x_2,2-2(1/2))=F(x_2,1)=x_0$, and $\gamma(0)=F(x_1,0) = x_1$, and $\gamma(1) = F(x_2,0)=x_2$. Thus, $\gamma$ is a path from $x_0$ to $x_1$.

\bigskip

\item Show that if $Y$ is contractible, then for any set $X$, the set $\cla{X,Y}$ has a single element.

\bigskip

Let $f,g:X\to Y$ be any two maps. Since $Y$ is contractible, the identity map $I_Y:Y\to Y$ is nullhomotopic. We now basically repeat the argument from the previous problem: either argue that $f = I_Y\circ f$ must be homotopic to a constant map since $I_Y$ is, and that the same is true of $g$ or let $F:Y\times [0,1]\to Y$ be the homotopy from $I_Y$ to a constant map, and consider the map $G:X\times [0,1]\to Y$ given by
\[
 G(x,t) = \begin{cases}
           F(f(x),2t), & 0\leq t\leq 1/2\\
	   F(g(x), 2-2t), & 1/2\leq t\leq 1
          \end{cases}.
\]


\bigskip

\item Show that if $X$ is contractible and $Y$ is path-connected, then the set $\cla{X,Y}$ has a single element.

\bigskip

The argument is the same as the one given in 4(b): Since $I_X$ is homotopic to a constant map $c(x)=x_0$, $f=f\circ I_X$ is homotopic to the constant map $(f\circ c)(x) = f(x_0)$, and similarly $g$ is homotopic to the constant map with value $g(x_0)$. Since $Y$ is path-connected, a path in $Y$ from $f(x_0)$ to $g(x_0)$ gives a homotopy between the constant maps with values $f(x_0)$ and $g(x_0)$, respectively.

\bigskip

\end{enumerate}
\item Let $A\subseteq X$. Recall that a {\bf retraction} of $X$ onto $A$ is a continuous map $r:X\to A$ such that $r(a)=a$ for all $a\in A$. If $a_0\in A$, show that
\[
r_*:\pi_1(X,a_0)\to \pi_1(A,a_0)
\]
is a surjection.

\bigskip

Suppose $r:X\to A$ is a retraction map, and let $i:A\to X$ denote inclusion. Then $r\circ i:A\to A$ is the identity map on $A$, and thus the composition
\[
 \pi_1(A,a_0)\xrightarrow[]{i_*} \pi_1(X,a_0) \xrightarrow[]{r_*} \pi_1(A,a_0)
\]
is equal to the identity map $I:\pi_1(A,a_0)\to \pi_1(A,a_0)$, since $r_*\circ i_* = (r\circ i)_* = (I_A)_*$. Since the identity map is a surjection, it follows that $r_*$ is a surjection.

(For any functions $f:A\to B$ and $g:B\to C$ between arbitrary sets, if $g\circ f:A\to C$ is a surjection, then so is $g$, since if $c\in C$, there exists some $a\in A$ such that $(g\circ f)(a) = c$, but $(g\circ f)(a) = g(f(a))$, so setting $b=f(a)$ gives an element of $B$ such that $g(b)=c$. Note that a similar argument guarantees that $i_*$ is an injection.)
\end{enumerate}
\end{document}
 
