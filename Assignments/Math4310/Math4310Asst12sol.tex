\documentclass[letterpaper,12pt]{article}

\usepackage{ucs}
\usepackage[utf8x]{inputenc}
\usepackage{amsmath}
\usepackage{amsfonts}
\usepackage{amssymb}
\usepackage[margin=1in]{geometry}
\usepackage{enumerate}
\usepackage{graphicx}
\usepackage{hyperref}

\newcommand{\abs}[1]{\lvert #1\rvert}
\newcommand{\len}[1]{\lVert #1\rVert}
\newcommand{\R}{\mathbb{R}}
\newcommand{\N}{\mathbb{N}}
\newcommand{\Z}{\mathbb{Z}}
\newcommand{\x}{\mathbf{x}}
\newcommand{\y}{\mathbf{y}}
\newcommand{\inter}[1]{\overset{\,\,\circ}{#1}}
\newcommand{\T}{\mathcal{T}}
\DeclareMathOperator{\Int}{Int}

\title{Math 4310 Assignment \#12 Solutions\\University of Lethbridge, Fall 2014}
\author{Sean Fitzpatrick}
\begin{document}
 \maketitle

\begin{enumerate}
\item Prove that the antipodal map $f:S^1\to S^1$ given by $f(z)=-z$ is homotopic to the identity map.

\bigskip

The antipodal map can be written as $f(e^{i\theta}) = e^{i(\theta+\pi)}$, so an explicit homotopy between the two maps is given by
\[
F(e^{i\theta},t) = e^{i(\theta+t\pi)}.
\]
The map $F$ is continuous (it's a composition of continuous functions) and $F(e^{i\theta},0)$ is the identity on $S^1$, while $F(e^{i\theta},1)$ is the antipodal map.

\bigskip

\item Let $f:S^1\to S^1$ be the antipodal map as above. What is the effect of the induced isomorphism
\[
f_*:\pi_1(S^1,1)\to \pi_1(S^1,-1)?
\]

\bigskip

From the above, we see that the effect of the antipodal map is a rotation of the circle through an angle of $\pi$ (rather than, say, a reflection). Let $\gamma(t) = e^{2n\pi i t}$ be a representative of an element of $\pi_1(S^1,1)$ corresponding to $n\in \Z$ under the isomorphism $\pi_1(S^1,1)\to \Z$. Then $f_*[\gamma] = [f\circ \gamma]$, where
\[
f\circ \gamma(t) = f(e^{2n\pi i t}) = e^{i(2n\pi t+\pi)}.
\]
In terms of the isomorphism with $\Z$, the induced isomorphism acts as the identity isomorphism $\Z\to\Z$. One way to see this is to note that to define the isomorphism $\pi_1(S^1,-1)\to\Z$, we simply change the covering map $\R\to S^1$ from $p(t)=e^{2\pi i t}$ to $q(t) = e^{i(2\pi t+\pi)}$. Viewing $\R$ as a `helix'  over the circle, when we rotate the circle, we rotate the helix. 

Since there are choices involved in defining the various isomorphisms, it's hard to be more precise. The main thing to observe is that since the antipodal map is a rotation, each loop based at 1 is simply rotated by $\pi$ to get a loop at $-1$. The only two possibilities were a rotation (which preserves the direction of the loop) or a reflection (which would reverse the direction) and in this case we have a rotation.

\bigskip

\item (Do not hand in) Let $T = S^1\times S^1$ denote the two-dimensional torus. Show that there is a deformation retraction of the punctured torus $T\setminus\{p\}$ onto a pair of circles joined at a point, where $p\in T$ is any point.

Think about this one for a bit. I was going to make it a bonus except it's incredibly easy to Google the solution. So instead, think about it. Once you've thought about it, Google it. You should be able to find a nice YouTube video illustrating exactly how the retraction works.

\bigskip

I hope you found the YouTube video. If not, here's a way to see the retraction: view $T$ as the quotient of the square $[0,1]\times [0,1]$ obtained by identifying opposite edges, and let the point $p$ correspond to four corners of the square. (Note that all four corners are identified to a single point.) Now enlarge the missing point to an open disc (you'll have a quarter disc in each corner), and from here you should be able to see that you can continue to expand the hole until you're left with a $\Large{+}$ in the middle of the square, and when you join the ends of the horizontal and vertical parts, you'll get two circles joined at a point.

\item Prove that the space $X = \R^2\setminus \{p,q\}$, where $p,q\in\R^2$ with $p\neq q$ is homotopy equivalent to the figure-eight space $Y$ given by joining two circles at a point $b$.

\bigskip

As a ``proof'' I accepted any argument by pictures that illustrated how to obtain a deformation retraction of $\R\setminus \{p,q\}$ onto the figure eight. Writing down an explicit homotopy is tricky, but if you wanted to do so, here is a suggestion that might work: look up the family of curves known as Cassini Ovals -- these are curves defined by requiring that the {\em product} of the distances from a point on the curve to two foci are constant. One member of the family is the lemniscate, which is a figure eight curve. Each curve is given as the level set of a function, so you could use the gradient of that function to construct an orthogonal family of curves (this falls under the heading of orthogonal trajectories; you may have encountered these in the calculus sequence). Given any point other than the two foci, it should then be possible to follow a curve orthogonal to the family of Cassini ovals until you land on the lemniscate, and use this to define a homotopy.

There are also simpler maps out there. (They're a little bit hard to track down on Google though.) If you want to argue using a picture, there's a good example on page 2 of the first chapter of Hatcher's book, which you can access via this link: \href{http://www.math.cornell.edu/~hatcher/AT/ATch0.pdf}{http://www.math.cornell.edu/~hatcher/AT/ATch0.pdf}.

Alternatively, note that it's sufficient to convince yourself that a thickened figure eight retracts onto the figure eight curve.
\end{enumerate}
\end{document}
 
