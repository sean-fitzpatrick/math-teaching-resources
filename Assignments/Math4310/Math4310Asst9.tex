\documentclass[letterpaper,12pt]{article}

\usepackage{ucs}
\usepackage[utf8x]{inputenc}
\usepackage{amsmath}
\usepackage{amsfonts}
\usepackage{amssymb}
\usepackage[margin=1in]{geometry}
\usepackage{enumerate}

\newcommand{\abs}[1]{\lvert #1\rvert}
\newcommand{\len}[1]{\lVert #1\rVert}
\newcommand{\R}{\mathbb{R}}
\newcommand{\N}{\mathbb{N}}
\newcommand{\x}{\mathbf{x}}
\newcommand{\y}{\mathbf{y}}
\newcommand{\inter}[1]{\overset{\,\,\circ}{#1}}
\newcommand{\T}{\mathcal{T}}
\DeclareMathOperator{\Int}{Int}

\title{Math 4310 Assignment \#8\\University of Lethbridge, Fall 2014}
\author{Sean Fitzpatrick}
\begin{document}
 \maketitle

{\bf Due date:} Friday, November 7th, by 5 pm.

\bigskip

Please submit solutions to the following problems (except where indicated). As usual, all the textbook exercises are recommended as practice problems if they don't appear as an assigned problem below.

\begin{enumerate}
\item Let $p:X\to Y$ be a quotient map, and let $A\subseteq X$ be a subspace. Show that the restricted map $q=p|_A:A\to p(A)$ need not be a quotient map. (Hint: consider the following example: $X=[0,1]\cup [2,3]$, $A=[0,1)\cup [2,3]$, and $p(x)=x$ for $x\in [0,1]$, and $p(x)=x-1$ for $x\in [2,3]$.)
\item With the same terminology as the previous problem, show that if either $A$ is open in $X$ and $p$ is an open map, or $A$ is closed in $X$ and $p$ is a closed map, then $p_A:A\to p(A)$ is a quotient map.
\item Let $X$ denote the quotient space obtained from $\R$ by identifying all of the integers to a single point.
\begin{enumerate}
 \item Explain why $X$ can be viewed as a countable union of circles that are all joined at a single point.
 \item Let $Y$ be the union of the circles $(x-1/n)^2+y^2=1/n^2$, for $n\in \N$. (The space $Y$ is called the ``Hawaiian Earring''.) Show that $Y$ is {\em not} homeomorphic to $X$. (For a hint, see the first paragraph of the Wikipedia entry on the Hawaiian Earring.)
\end{enumerate}
 \item Let $f:X\to X'$ be a continuous function and suppose that we have partitions $\mathcal{P},\mathcal{P}'$ of $X$ and $X'$, respectively, such that if two points in $X$ lie in the same member of $\mathcal{P}$, then $f(x)$ and $f(x')$ lie in the same member of $\mathcal{P}'$. If $Y$ and $Y'$ are the quotient spaces of $X$ and $X'$ corresponding to the given partitions, show that $f$ induces a map $\tilde{f}:Y\to Y'$ and that if $f$ is a quotient map, then so is $\tilde{f}$.
 \item \begin{enumerate}
        \item Let $p:X\to Y$ be a continuous map. Show that if there is a continuous map $f:Y\to X$ such that $p\circ f$ equals the identity map of $Y$, then $p$ is a quotient map.
        \item If $A\subseteq X$, a {\em retraction} of $X$ onto $A$ is a continuous map $r:X\to A$ such that $r(a)=a$ for all $a\in A$. Show that any retraction map is a quotient map.
       \end{enumerate}

\end{enumerate}
\end{document}
 
