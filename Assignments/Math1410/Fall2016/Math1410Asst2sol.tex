\documentclass[letterpaper,12pt]{amsart}

\usepackage{ucs}
\usepackage[utf8x]{inputenc}
\usepackage{amsmath}
%\usepackage{amsfonts}
%\usepackage{amssymb}
\usepackage[margin=1in]{geometry}
\usepackage{multicol}
\usepackage[bitstream-charter]{mathdesign}
\usepackage[T1]{fontenc}

\newcommand{\len}[1]{\lVert #1\rVert}
\newcommand{\abs}[1]{\lvert #1\rvert}
\newcommand{\R}{\mathbb{R}}
\newcommand{\bbm}{\begin{bmatrix}}
\newcommand{\ebm}{\end{bmatrix}}
\newcommand{\dotp}{\boldsymbol{\cdot}}
                   
\title{Math 1410 Assignment \#2\\University of Lethbridge, Fall 2016}
\author{Sean Fitzpatrick}
\begin{document}
 \maketitle

\begin{enumerate}
\item Recall from class that a key property of the cross product $\vec{u}\times\vec{v}$ is that it is orthogonal to both of the vectors $\vec{u}$ and $\vec{v}$. Prove that this is the case: show that for \textbf{any} vectors $\vec{u}$ and $\vec{v}$ in $\R^3$, the vector $\vec{u}$ is orthogonal to $\vec{u}\times \vec{v}$.

\bigskip

{\bf Solution:} Let $\vec{u}=\langle u_1,u_2,u_3\rangle$ and $\vec{v} = \langle v_1, v_2, v_3\rangle$ be any two vectors in $\R^3$. By definition, we have
\[
 \vec{u}\times\vec{v} = \langle u_2v_3-u_3v_2, u_3v_1-u_1v_3, u_1v_2-u_2v_1\rangle,
\]
and thus
\begin{align*}
 \vec{u}\dotp(\vec{u}\times\vec{v}) &= u_1(u_2v_3-u_3v_2)+u_2(u_3v_1-u_1v_3)+u_3(u_1v_2-u_2v_1)\\
 & = u_1u_2v_3-u_1u_3v_2+u_2u_3v1-u_2u_1v_3+u_3u_1v_2-u_3u_2v_1\\
 & = u_1u_2v_3-u_2u_1v_3 + u_2u_3v_1 - u_3u_2v_1 +u_3u_1v_2-u_1u_2v_2\\
 & = 0,
\end{align*}
as required.

\bigskip

\item Consider the lines $\ell_1$ and $\ell_2$ given by the equations
\begin{align*}
 \ell_1(t) = \langle x,y,z\rangle & = \langle 1,2,1\rangle+s\langle 2,-1,1\rangle\\
 \ell_2(t) = \langle x,y,z\rangle & = \langle 3,3,3\rangle+t\langle 4,2,-1\rangle
\end{align*}

\medskip

\begin{enumerate}
 \item Show that the two lines are \textit{skew} (do not intersect). 

\bigskip

{\bf Solution:} The two lines intersect if there is a point $(x,y,z)$ such that
\[
 \langle x,y,z\rangle = \langle 1+2s,2-s,1+s\rangle = \langle 3+4t, 3+2t, 3-t\rangle.
\]
Equating components of the two vectors, and simplifying, we get the system of equations
\[
 \begin{array}{ccccc}
  2s&-&4t&=&2\\
  -s&-&4t&=&1\\
  s&+&t&=&2.
 \end{array}
\]
Inputting this system into Wolfram Alpha (or solving by hand), we find that there is no solution to the system, and therefore, the two lines do not intersect.

\bigskip

 \item Find the points $P_1$ on $\ell_1$ and $P_2$ on $\ell_2$ such that the distance from $P_1$ to $P_2$ is a minimum, and determine the distance between them.

\bigskip

Let $P_1 = (1+2s,2-s,1+s)$ be a point on $\ell_1$, and let $P_2=(3+4t,3+2t,3-t)$ be a point on $\ell_2$. The points $P_1$ and $P_2$ will be closest together when the vector $\overrightarrow{P_1P_2}$ is orthogonal to both lines. We find that
\[
 \overrightarrow{P_1P_2} = \langle 2+4t-2s, 1+2t+s, 2-t-s\rangle.
\]
This vector must be orthogonal to the direction vectors $\vec{d}_1 = \langle 2, -1, 1\rangle$ and $\vec{d}_2 = \langle 4, 2 ,-1\rangle$; thus, we must have
\[
 \vec{d}_1\dotp \overrightarrow{P_1P_2} = 2(2+4t-2s)-1(1+2t+s)+1(2-t-s) = -6s+5t+5=0,
\]
and
\[
 \vec{d}_2\dotp \overrightarrow{P_1P_2} = 4(2+4t-2s)+2(1+2t+s)-1(2-t-s) = -5s+21t+8=0.
\]
This gives us the system of equations
\[
 \begin{array}{ccccc}
  6s&-&5t&=&5\\
  5s&-&21t&=&8.
 \end{array}
\]
Using Wolfram Alpha (or solving by hand), we find the solution to be\\ $s=\dfrac{65}{101}$ and $t=-\dfrac{23}{101}$. Putting these values back into our points, we find that the closest points are
\begin{align*}
 P_1 & = \left(1+2\cdot\frac{65}{101}, 2-\frac{65}{101}, 1+\frac{65}{101}\right) = \left(\frac{231}{101}, \frac{137}{101}, \frac{166}{101}\right) \text{ and} \\
 P_2 & = \left(3+4\left(-\frac{23}{101}\right), 3+2\left(-\frac{23}{101}\right), 3+\frac{23}{101}\right) = \left(\frac{211}{101}, \frac{257}{101}, \frac{326}{101}\right).
\end{align*}
The vector $\overrightarrow{P_1P_2}$ is then equal to
\begin{align*}
 \overrightarrow{P_1P_2} &= \left\langle 2+4\left(-\frac{23}{101}\right)-2\left(\frac{65}{101}\right), 1+2\left(-\frac{23}{101}\right)+\frac{65}{101}, 2+\frac{23}{101}+\frac{65}{101}\right\rangle\\
 & = \left\langle -\frac{20}{101}, \frac{120}{101}, \frac{160}{101}\right\rangle\\
& = \frac{20}{101}\langle -1,6,8\rangle,
\end{align*}
and the desired distance is
\[
 \len{\overrightarrow{P_1P_2}} = \frac{20}{101}\sqrt{(-1)^2+6^2+8^2} = \frac{20}{101}\sqrt{101} = \frac{20}{\sqrt{101}}.
\]

 \end{enumerate}

\newpage

\item Consider the set of $3\times 1$ column vectors $\{\vec{u},\vec{v}\}\subseteq\R^3$, where
\[
 \vec{u} = \bbm 1\\-2\\3\ebm, \text{ and } \vec{v} = \bbm 4\\2\\-1\ebm.
\]
Determine a $3\times 1$ column vector $\vec{w}\in\R^3$ such that the set $\{\vec{u},\vec{v},\vec{w}\}$ is

\medskip

\begin{enumerate}
 \item Linearly dependent.

\bigskip

{\bf Solution:} One choice is to take $\vec{w}=\vec{0}$, since any set containing the zero vector is linearly independent. Another option is to let $\vec{w}$ be any linear combination of $\vec{u}$ and $\vec{v}$; for example $\vec{w} = 2\vec{u}$ would work.

\bigskip

 \item Linearly independent.

\bigskip

To obtain a linearly independent set it is necessary that $\vec{w}$ cannot be written as a linear combination of $\vec{u}$ and $\vec{v}$ (since otherwise the set $\{\vec{u},\vec{v},\vec{w}\}$ would be linearly dependent by definition). It turns out that this is also a sufficient condition. Since $\vec{u}$ is clearly not a scalar multiple of $\vec{v}$, and vice versa, choosing a vector $\vec{w}$ that is not a linear combination of $\vec{u}$ and $\vec{v}$ will also guarantee that $\vec{u}$ cannot be written as a linear combination of $\vec{v}$ and $\vec{w}$, and that $\vec{v}$ cannot be written in terms of $\vec{u}$ and $\vec{w}$.

To see this, suppose $\vec{w}$ does not belong to the span of $\vec{u}$ and $\vec{v}$, and suppose that
\[
 \vec{u} = a\vec{v}+b\vec{w}
\]
for some choice of scalars $a$ and $b$. If $b\neq 0$, then we can solve for $\vec{w}$, obtaining $\vec{w} = \frac{1}{b}\vec{u}-\frac{a}{b}\vec{v}$, but we are assuming that $\vec{w}$ can \textbf{not} be written in terms of $\vec{u}$ and $\vec{v}$, so we must have $b=0$. But if $b=0$, then we would have $\vec{u}=a\vec{v}$, implying that $\vec{u}$ is a scalar multiple of $\vec{v}$, and this is clearly not the case.

A similar argument shows that $\vec{v}$ cannot be written in terms of $\vec{u}$ and $\vec{w}$.

It remains to determine a vector $\vec{w}$ that does not belong to the span of $\vec{u}$ and $\vec{v}$. We know that the subspace spanned by $\vec{u}$ and $\vec{v}$ is a plane through the origin, so any vector $\vec{w}$ that is not parallel to that plane will do; in particular, this would be the case if we took $\vec{w}$ to be orthogonal to the plane. We know that one such vector is given by
\[
 \vec{w} = \vec{u}\times \vec{v},
\]
and so it follows that the set $\{\vec{u},\vec{v}, \vec{u}\times\vec{v}\}$ is linearly independent.

(A complete explanation such as the above is not necessary for full credit, but it is needed if we want our argument to be air tight.)

\bigskip

We can also show that the set $\{\vec{u},\vec{v}, \vec{u}\times\vec{v}\}$ is linearly independent directly. Let $\vec{w}=\vec{u}\times\vec{v}$, and recall from class that the set $\{\vec{u},\vec{v}, \vec{u}\times\vec{v}\}$ is linearly independent provided that whenever
\begin{equation}\label{ref}
 a\vec{u}+b\vec{v}+c\vec{w} = \vec{0}
\end{equation}
for scalars $a,b,c$, we must have $a=b=c=0$. Thus, let us assume that equation \eqref{ref} holds for some scalars $a$, $b$, and $c$. Since $\vec{w}$ is orthogonal to both $\vec{u}$ and $\vec{v}$, it follows that we must have
\begin{align*}
 0 = \vec{w}\dotp \vec{0} & = \vec{w}\dotp(a\vec{u}+b\vec{v}+c\vec{w})\\
 & = a(\vec{w}\dotp\vec{u})+b(\vec{w}\dotp\vec{v})+c(\vec{w}\dotp\vec{w})\\
 & = a(0)+b(0)+c\len{\vec{w}}^2.
\end{align*}
We can easily compute that $\vec{w}\neq\vec{0}$, which implies that $\len{\vec{w}}\neq 0$, and thus $c\len{\vec{w}}^2=0$ implies that $c=0$. Putting $c=0$ in equation \eqref{ref} gives us the equation
\[
 a\vec{u}+b\vec{v}=\vec{0}.
\]
From this we can deduce that $a=0$ and $b=0$, since if either scalar were non-zero, we would have either $\vec{u} = -\frac{b}{a}\vec{v}$ (if $a\neq 0$) or $\vec{v} = -\frac{a}{b}\vec{u}$ (if $b\neq 0$), and we know that neither $\vec{u}$ nor $\vec{v}$ is a scalar multiple of the other.
\end{enumerate}

\bigskip

 \item Determine whether or not the following subsets of $\R^3$ are subspaces. Support your answer.

\medskip


\begin{enumerate}
 \item $U = \left\{\left.\bbm x+2y\\3x-y\\4x\ebm \,\right|\, x,y\in\R\right\}$

\bigskip

The set $U$ is a subspace. The easiest way to see this is as follows. Let $\vec{u}=\bbm x+2y\\3x-y\\4x\ebm$ be any vector in $U$. Then we have
\[
 \vec{u} = \bbm x+2y\\3x-y\\4x\ebm = \bbm x\\3x\\4x\ebm + \bbm 2y\\-y\\0\ebm = x\bbm 1\\3\\4\ebm+y\bbm 2\\-1\\0\ebm = x\vec{a}+y\vec{b},
\]
where $\vec{a} = \bbm 1\\3\\4\ebm$ and $\vec{b} = \bbm 2\\-1\\0\ebm$. Thus, every $\vec{u}\in U$ can be written as a linear combination of $\vec{a}$ and $\vec{b}$, which implies that
\[
 U = \operatorname{span}\left\{\bbm 1\\3\\4\ebm, \bbm 2\\-1\\0\ebm\right\},
\]
and since the span of any set of vectors is a subspace, we can conclude that $U$ is a subspace.

\bigskip

If we prefer to prove that $U$ is a subspace directly, using the subspace test, then we proceed as follows:

First, we check to see if $U$ is non-empty by verifying that $\vec{0}$ is in $U$. Setting $x=y=0$ gives us the element
\[
 \bbm 0+2(0)\\3(0)-0\\4(0)\ebm = \bbm 0\\0\\0\ebm = \vec{0},
\]
so $\vec{0}\in U$.

Next, we check to see if $U$ is closed under addition. Let $\vec{u} = \bbm a+2b\\3a-b\\4a\ebm$ and $\vec{v} = \bbm c+2d\\3c-d\\4c\ebm$ be any two elements of $U$. Then
\[
 \vec{u}+\vec{v} = \bbm a+2b\\3a-b\\4a\ebm+\bbm c+2d\\3c-d\\4c\ebm = \bbm (a+c) + 2(b+d)\\3(a+c)-(b+d)\\4(a+c)\ebm = \bbm x+2y\\3x-y\\4x\ebm,
\]
where $x=a+c$ and $y=b+d$, which shows that $\vec{u}+\vec{v}\in U$, and thus $U$ is closed under addition.

Finally, we check to see if $U$ is closed under scalar multiplication. Given $\vec{u} = \bbm a+2b\\3a-b\\4a\ebm$ and any scalar $k\in\R$, we have
\[
 k\vec{u} = k\bbm a+2b\\3a-b\\4a\ebm = \bbm k(a+2b)\\k(3a-b)\\k(4a)\ebm = \bbm ka+2(kb)\\3(ka)-(kb)\\4(ka)\ebm = \bbm x+2y\\3x-y\\4x\ebm,
\]
where $x=ka$ and $y=kb$, showing that $k\vec{u}\in U$, so $U$ is closed under scalar multiplication.

\bigskip

 \item $V = \left\{\left.\bbm x\\y\\z\ebm\,\right|\, x+y+z=1\right\}$

\bigskip

{\bf Solution:} We note that $\vec{0}$ does not belong to $V$, since $0+0+0=0\neq 1$. Since $\vec{0}\notin V$, $V$ is not a subspace.
\end{enumerate}


\bigskip


\end{enumerate}



\end{document}
 
