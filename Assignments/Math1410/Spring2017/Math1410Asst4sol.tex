\documentclass[letterpaper,12pt]{article}

\usepackage{ucs}
\usepackage[utf8x]{inputenc}
\usepackage{amsmath}
%\usepackage{amsfonts}
%\usepackage{amssymb}
\usepackage[margin=1in]{geometry}

\usepackage[bitstream-charter]{mathdesign}
\usepackage[T1]{fontenc}

\newcommand{\len}[1]{\lVert #1\rVert}
\newcommand{\abs}[1]{\lvert #1\rvert}
\newcommand{\R}{\mathbb{R}}
\newcommand{\dotp}{\boldsymbol{\cdot}}
\newcommand{\bbm}{\begin{bmatrix}}
\newcommand{\ebm}{\end{bmatrix}}
\DeclareMathOperator{\proj}{proj}
\newenvironment{amatrix}[1]{%
  \left[\begin{array}{@{}*{#1}{c}|c@{}}
}{%
  \end{array}\right]
}
\DeclareMathOperator{\comp}{comp}
\newcommand{\bam}{\begin{amatrix}}
\newcommand{\eam}{\end{amatrix}}


\title{Math 1410 Assignment \#4 Solutions\\University of Lethbridge, Spring 2017}
\author{Sean Fitzpatrick}
\begin{document}
 \maketitle


\begin{enumerate}
\item Determine the null space and column space of the matrix $A = \bbm 2&-3&1&4\\-1&2&2&-3\\1&0&8&-1\ebm$.

\bigskip

We begin by computing the reduced row-echelon form of $A$. We have:
\begin{align*}
 \bbm 2&-3&1&4\\-1&2&2&-3\\1&0&8&-1\ebm\xrightarrow{R_1\leftrightarrow R_3} & \bbm 1&0&8&-1\\-1&2&2&-3\\2&-3&1&4\ebm\\
\xrightarrow[R_3-2R_1\to R_3]{R_2+R_1\to R_2} & \bbm 1&0&8&-1\\0&2&10&-4\\0&-3&-15&6\ebm\\
\xrightarrow{\frac{1}{2}R_2\to R_2} & \bbm 1&0&8&-1\\0&1&5&-2\\0&-3&-15&6\ebm\\
\xrightarrow{R_3+3R_2\to R_3} & \bbm 1&0&8&-1\\0&1&5&-2\\0&0&0&0\ebm.
\end{align*}
This last matrix is in reduced row-echelon form. The null space is the set of all vectors $\vec{x}=\bbm x_1\\x_2\\x_3\\x_4\ebm$ such that $A\vec{x}=\vec{0}$; this matrix equation corresponds to a system of equations with augmented matrix $\bam{4} 2&-3&1&4&0\\-1&2&2&-3&0\\1&0&8&-1&0\eam$ which, by the above, reduces to
\[
 \bam{4} 1&0&8&-1&0\\0&1&5&-2&0\\0&0&0&0&0\eam.
\]
This tells us that $x_3$ and $x_4$ are free variables, while $x_1+8x_3-x_4=0$ and $x_2+5x_3-2x_4=0$, so $x_1=-8x_3+x_4$ and $x_2=-5x_3+2x_4$. Thus, any solution $\vec{x}$ to $A\vec{x}=\vec{0}$ satisfies
\[
 \vec{x}=\bbm x_1\\x_2\\x_3\\x_4\ebm = \bbm -8x_3+x_4\\-5x_3+2x_4\\x_3\\x_4\ebm = x_3\bbm -8\\-5\\1\\0\ebm + x_4\bbm 1\\2\\0\\1\ebm.
\]
It follows that $\operatorname{null}(A) = \operatorname{span}\left\{\bbm -8\\-5\\1\\0\ebm, \bbm 1\\2\\0\\1\ebm\right\}$.

\medskip

According to Theorem 30 in the textbook, the column space is generated by the columns of $A$ that correspond to columns with leading ones in the reduced row-echelon form of $A$. Since there are leading ones in columns 1 and 2, we have that
\[
 \operatorname{col}(A) = \operatorname{span}\left\{\bbm 2\\-1\\1\ebm, \bbm -3\\2\\0\ebm\right\}.
\]


\item Factor the matrix $A = \bbm 1&-2&3\\2&-3&1\\-1&2&4\ebm$ as a product of elementary matrices.

\bigskip

We reduce the matrix $A$ to reduced row-echelon form, keeping track of the row operations. For each row operation on the left, we write down the corresponding elementary matrix on the right, along with its inverse. We have
\begin{align*}
 \bbm 1&-2&3\\2&-3&1\\-1&2&4\ebm \xrightarrow{R_2-2R_1 \to R_2} & \bbm 1&-2&3\\0&1&-5\\-1&2&4\ebm & E_1 & = \bbm 1&0&0\\-2&1&0\\0&0&1\ebm & E_1^{-1} & = \bbm 1&0&0\\2&1&0\\0&0&1\ebm\\
 \xrightarrow{R_3+R_1\to R_3} & \bbm 1&-2&3\\0&1&-5\\0&0&7\ebm & E_2 & = \bbm 1&0&0\\0&1&0\\1&0&1\ebm & E_2^{-1} & = \bbm 1&0&0\\0&1&0\\-1&0&1\ebm\\
 \xrightarrow{\frac{1}{7}R_3\to R_3} & \bbm 1&-2&3\\0&1&-5\\0&0&1\ebm & E_3 & = \bbm 1&0&0\\0&1&0\\0&0&\frac{1}{7}\ebm & E_3^{-1} & = \bbm 1&0&0\\0&1&0\\0&0&7\ebm\\
 \xrightarrow{R_2+5R_3\to R_2} & \bbm 1&-2&3\\0&1&0\\0&0&1\ebm & E_4 & = \bbm 1&0&0\\0&1&5\\0&0&1\ebm & E_4^{-1} & = \bbm 1&0&0\\0&1&-5\\0&0&1\ebm\\
 \xrightarrow{R_1-3R_3\to R_1} & \bbm 1&-2&0\\0&1&0\\0&0&1\ebm &  E_5 & = \bbm 1&0&-3\\0&1&0\\0&0&1\ebm & E_5^{-1} & = \bbm 1&0&3\\0&1&0\\0&0&1\ebm \\
 \xrightarrow{R_1+2R_2\to R_1} & \bbm 1&0&0\\0&1&0\\0&0&1\ebm & E_6 & = \bbm 1&2&0\\0&1&0\\0&0&1\ebm & E_6^{-1} & = \bbm 1&-2&1\\0&1&0\\0&0&1\ebm.
\end{align*}
Since performing a row operation is the same as multiplying on the left by the corresponding elementary matrix, we have
\[
 I = E_6(E_5(E_4(E_3(E_2(E_1A))))) = (E_6E_5E_4E_2E_2E_1)A.
\]
It follows that $A^{-1} = E_6E_5E_4E_3E_2E_1$, and thus
\begin{align*}
 A & = (A^{-1})^{-1} = (E_6E_5E_4E_2E_2E_1)^{-1} = E_1^{-1}E_2^{-1}E_3^{-1}E_4^{-1}E_5^{-1}E_6^{-1}\\
   & = \bbm 1&0&0\\2&1&0\\0&0&1\ebm \bbm 1&0&0\\0&1&0\\-1&0&1\ebm \bbm 1&0&0\\0&1&0\\0&0&7\ebm \bbm 1&0&0\\0&1&-5\\0&0&1\ebm \bbm 1&0&3\\0&1&0\\0&0&1\ebm \bbm 1&-2&1\\0&1&0\\0&0&1\ebm.
\end{align*}

\bigskip


\item For each statement below, either prove the statement or give a counterexample showing that it is false.
\begin{enumerate}
 \item If $A$ and $B$ are both invertible, then $A+B$ is invertible.

\medskip

The statement is false. For example the matrix $A=\bbm 1&-2\\0&1\ebm$ is invertible, with $A^{-1} = \bbm 1&2\\0&1\ebm$, and the matrix $B = \bbm -1&2\\0&-1\ebm$ is invertible, with inverse $B^{-1} = \bbm -1&-2\\0&-1\ebm$. However, $A+B = \bbm 0&0\\0&0\ebm$, which is not invertible.

\medskip

 \item If $AB=I$, then $AB=BA$.

\medskip

This is false, since there is no assumption made on the sizes of $A$ and $B$. If $A = \bbm 1&0&0\\0&1&0\ebm$ and $B = \bbm 1&0\\0&1\\0&0\ebm$, then $AB = \bbm 1&0\\0&1\ebm$; however, $BA = \bbm 1&0&0\\0&1&0\\0&0&0\ebm \neq AB$.

If, on the other hand, we make the additional assumption that $A$ and $B$ are both of size $n\times n$, then the statement is true. If $A$ and $B$ are square matrices of the same size and $AB=I$, then by the Invertible Matrix Theorem, $A$ is invertible, and by the uniqueness of the inverse, we must have $B=A^{-1}$. It then follows from the definition of the inverse that $BA = A^{-1}A = I$.

\medskip

 \item If $AB=B$ for some matrix $B\neq 0$, then $A$ is invertible. 

\medskip

This is false. For example, the matrix $A=\bbm 0&0\\0&1\ebm$ is not invertible, and the matrix $B=\bbm 0\\1\ebm$ is non-zero, but
\[
 AB = \bbm 0&0\\0&1\ebm\bbm 0\\1\ebm = \bbm 0\\1\ebm = B.
\]

\medskip

 \item If $A^3$ is invertible, then $A$ is invertible.

\medskip

This is true. Suppose that $A^3$ is invertible. Then there exists a matrix $B$ (the inverse of $A^3$) such that $A^3B = I$. But since $A^3=A(A^2)$, we have
\[
 A(A^2B) = (A(A^2))B = A^3B = I.
\]
It follows from the Invertible Matrix Theorem that $A$ is invertible and from the uniqueness of the inverse that $A^{-1} = A^2B$.
\end{enumerate}

\bigskip


\item Let $A$ be a non-zero $n\times n$ matrix, and let $I$ be the $n\times n$ identity matrix.
\begin{enumerate}
 \item Show that if $A^2=0$, then $(I-A)^{-1} = I+A$.

\medskip

By the uniqueness of the inverse and the Invertible Matrix Theorem, it suffices to show that $(I-A)(I+A) = I$. We have
\[
 (I-A)(I+A) = I(I)+I(A)-A(I)-A(A) = I+A-A-A^2 = I,
\]
since $A^2=0$ by assumption.

\medskip

 \item Show that if $A^3=0$, then $(I-A)^{-1} = I+A+A^2$.

\medskip

Similarly, it suffices to show that $(I-A)(I+A+A^2) = I$. We have
\[
 (I-A)(I+A+A^2) = I(I)+I(A)+I(A^2)-A(I)-A(A)-A(A^2) = I+A+A^2-A-A^2-A^3 = 0,
\]
since we're assuming that $A^3=0$.

\medskip

 \item Find the inverse of $B = \bbm 1&3&-2\\0&1&4\\0&0&1\ebm$.

\medskip

We notice that $B = I-A$, where $A =\bbm 0&-3&2\\0&0&-4\\0&0&0\ebm$. We compute 
\[
 A^2 = \bbm 0&-3&2\\0&0&-4\\0&0&0\ebm\bbm 0&-3&2\\0&0&-4\\0&0&0\ebm = \bbm 0&0&12\\0&0&0\\0&0&0\ebm,
\]
and 
\[
 A^3 = A^2(A) = \bbm 0&0&12\\0&0&0\\0&0&0\ebm\bbm 0&-3&2\\0&0&-4\\0&0&0\ebm=\bbm 0&0&0\\0&0&0\\0&0&0\ebm = 0.
\]
Since $B=I-A$ where $A^3=0$, it follows from part (b) that
\[
 B = I+A+A^2 = \bbm 1&0&0\\0&1&0\\0&0&1\ebm + \bbm 0&-3&2\\0&0&-4\\0&0&0\ebm + \bbm 0&0&12\\0&0&0\\0&0&0\ebm = \bbm 1&-3&14\\0&1&-4\\0&0&1\ebm.
\]

 \item Given that $A\neq 0, A^2\neq 0, \ldots, A^{n-1}\neq 0$ but $A^n=0$, determine a formula for $(I-A)^{-1}$, and show that your answer is correct.

\bigskip

Following the pattern above, we conjecture that $(I-A)^{-1} = I+A+A^2+\cdots + A^{n-1}$. To confirm that this is correct, we compute
\begin{align*}
 (I-A)(I+A+A^2+\cdots +A^{n-1}) &= I+A+A^2+\cdots + A^{n-1} - (A+A^2+A^3+\cdots +A^n)\\& = I+(A-A)+(A^2-A^2)+\cdots+(A^{n-1}-A^{n-1}) + A^n\\& = I,
\end{align*}
since we're assuming $A^n = 0$.
\end{enumerate}


\end{enumerate}

\end{document}
 
