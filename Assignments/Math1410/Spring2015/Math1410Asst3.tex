\documentclass[letterpaper,12pt]{article}

\usepackage{ucs}
\usepackage[utf8x]{inputenc}
\usepackage{amsmath}
\usepackage{amsfonts}
\usepackage{amssymb}
\usepackage[margin=1in]{geometry}

\usepackage[bitstream-charter]{mathdesign}
\usepackage[T1]{fontenc}

\newcommand{\len}[1]{\lVert #1\rVert}
\newcommand{\abs}[1]{\lvert #1\rvert}

\title{Math 1410 Assignment \#3\\University of Lethbridge, Spring 2015}
\author{Sean Fitzpatrick}
\begin{document}
 \maketitle

{\bf Due date:} {\bf Wednesday}, March 11th, by 5 pm.

\bigskip

For instructions on completing this assignment, please see Assignment \#1, but please keep the following additional points in mind:
\begin{itemize}
 \item You don't have to type your work. (If you think your handwriting is too illegible to read, you can still type, but it's generally more work than it's worth to type out your math homework.)
 \item If you do your work in a spiral notebook and tear out the pages, cut off the tattered edges before submitting. (This is pretty much the most annoying thing for a grader to deal with. If you get a bunch of assignments like this in a pile, the tattered bits get all tangled up.)
 \item There's a stapler right next to the assignment drop box. Please use it.
 \item On the topic of staples, using industrial-sized staples of death that result in sharp pointy bits poking out of your assignment is a bad idea.
 \item Plastic covers aren't really necessary. (They mostly just get in the way.)
\end{itemize}

\subsection*{Assigned problems}
\begin{enumerate}

\item Given a polynomial $p(x)=a+bx+cx^2+dx^3+x^4$, the matrix
\[
 C = \begin{bmatrix}
      0&1&0&0\\
0&0&1&0\\
0&0&0&1\\
-a&-b&-c&-d
     \end{bmatrix}
\]
is called the {\em companion matrix} of $p(x)$. Show that $\det(xI_4-C)=p(x)$.
\newpage
\item If $p(x)=a_0+a_1x+a_2x^2+\cdots + a_kx^k$ is a polynomial of degree $k$ (the degree of $p(x)$ is the highest power of $x$, so we're assuming that $a_k\neq 0$). Given any such polynomial $p(x)$ and any $n\times n$ (square) matrix $A$, it's possible to plug $A$ into the polynomial to obtain a {\em new} matrix, denoted $p(A)$, given by
\[
 p(A)=a_0I_n+a_1A+a_2A^2+\cdots + a_kA^k.
\]
For example, if $p(x)=2-3x+x^2$, then $p(A) = 2I_n-3A+A^2$.
\begin{enumerate}
 \item If $p(x) = 3-4x+2x^2$ and $A=\begin{bmatrix}1&-3\\0&2\end{bmatrix}$, compute $p(A)$.
 \item The {\em characteristic polynomial} of an $n\times n$ matrix $A$ is defined by
\[
 c_A(x) = \det(xI_n-A).
\]
 The {\em Cayley-Hamilton Theorem} is a famous theorem in linear algebra which states that for any $n\times n$ matrix $A$, $c_A(A) = 0$ (where the zero on the right is the zero matrix).

Verify that the Cayley-Hamilton Theorem is true for $A=\begin{bmatrix}3&2\\1&-1\end{bmatrix}$.
\end{enumerate}
{\bf Bonus opportunity}: Prove the Cayley-Hamilton Theorem for the $n=2$ case. That is, show that the theorem holds for a general $2\times 2$ matrix $A=\begin{bmatrix}a&b\\c&d\end{bmatrix}$.

\item In each case, either explain why the statement is true (in general), or give an example showing that it is false:
\begin{enumerate}
 \item If $\len{\vec{v}-\vec{w}}=0$, then $\vec{v}=\vec{w}$.
 \item If $\vec{v}=-\vec{v}$, then $\vec{v}=\vec{0}$.
 \item If $\len{\vec{v}}=\len{\vec{w}}$, then $\vec{v}=\vec{w}$.
 \item If $\len{\vec{v}}=\len{\vec{w}}$, then $\vec{v}=\pm\vec{w}$.
 \item $\len{\vec{v}+\vec{w}} = \len{\vec{v}}+\len{\vec{w}}$.
\end{enumerate}
\item Let $\vec{u} = \begin{bmatrix}3&-1&0\end{bmatrix}^T, \vec{v} = \begin{bmatrix}4&0&1\end{bmatrix}^T$, and $\vec{w} = \begin{bmatrix}1&1&1\end{bmatrix}^T$. In each case, either find scalars $a,b,c$ such that $\vec{x} = a\vec{u}+b\vec{v}+c\vec{w}$, or explain why no such scalars exist:
\begin{enumerate}
 \item $\vec{x} = \begin{bmatrix}5&1&2\end{bmatrix}^T$
 \item $\vec{x} = \begin{bmatrix}1&2&1\end{bmatrix}^T$.
\end{enumerate}

 \end{enumerate}
{\bf Note}: for problems 1 and 2 above, $xI_n$ denotes the identity matrix multiplied by the (variable) scalar $x$. For example, if $A = \begin{bmatrix}a&b\\c&d\end{bmatrix}$, then
\[
 xI_2-A = x\begin{bmatrix}1&0\\0&1\end{bmatrix}-\begin{bmatrix}a&b\\c&d\end{bmatrix} = \begin{bmatrix}x-a&-b\\-c&x-d\end{bmatrix}.
\]

\end{document}
 
