\documentclass[letterpaper,12pt]{article}

\usepackage{ucs}
\usepackage[utf8x]{inputenc}
\usepackage{amsmath}
\usepackage{amsfonts}
\usepackage{amssymb}
\usepackage[margin=1in]{geometry}

\usepackage[bitstream-charter]{mathdesign}
\usepackage[T1]{fontenc}

\newcommand{\len}[1]{\lVert #1\rVert}
\newcommand{\abs}[1]{\lvert #1\rvert}
\newenvironment{amatrix}[1]{%
  \left[\begin{array}{@{}*{#1}{c}|c@{}}
}{%
  \end{array}\right]
}
\title{Math 1410 Assignment \#3 Solutions\\University of Lethbridge, Spring 2015}
\author{Sean Fitzpatrick}
\begin{document}
 \maketitle

\begin{enumerate}

\item Given a polynomial $p(x)=a+bx+cx^2+dx^3+x^4$, the matrix
\[
 C = \begin{bmatrix}
      0&1&0&0\\
0&0&1&0\\
0&0&0&1\\
-a&-b&-c&-d
     \end{bmatrix}
\]
is called the {\em companion matrix} of $p(x)$. Show that $\det(xI_4-C)=p(x)$.

\bigskip

{\bf Solution:} We have that
\[
 xI_4-C = \begin{bmatrix}x&0&0&0\\
0&x&0&0\\
0&0&x&0\\
0&0&0&x           
          \end{bmatrix} - \begin{bmatrix}
      0&1&0&0\\
0&0&1&0\\
0&0&0&1\\
-a&-b&-c&-d
     \end{bmatrix} = \begin{bmatrix}x&-1&0&0\\0&x&-1&0\\0&0&x&-1\\a&b&c&x+d\end{bmatrix}.
\]
Thus, using a cofactor expansion along the first row, we have
\begin{align*}
 \det(xI_4-C) & = x\begin{vmatrix}x&-1&0\\0&x&-1\\b&c&x+d\end{vmatrix} - (-1)\begin{vmatrix}0&-1&0\\0&x&-1\\a&c&x+d\end{vmatrix}\\
& = x^2\begin{vmatrix}x&-1\\c&x+d\end{vmatrix}+x\begin{vmatrix}0&-1\\b&x+d\end{vmatrix} +a\begin{vmatrix}-1&0\\x&-1\end{vmatrix}\\
& = x^2(x^2+dx+c)+xb+a\\
& = a+bx+cx^2+dx^3+x^4 = p(x).
\end{align*}
(You could also simplify the determinant using row/column operations if you prefer.)

\bigskip

\item If $p(x)=a_0+a_1x+a_2x^2+\cdots + a_kx^k$ is a polynomial of degree $k$ (the degree of $p(x)$ is the highest power of $x$, so we're assuming that $a_k\neq 0$). Given any such polynomial $p(x)$ and any $n\times n$ (square) matrix $A$, it's possible to plug $A$ into the polynomial to obtain a {\em new} matrix, denoted $p(A)$, given by
\[
 p(A)=a_0I_n+a_1A+a_2A^2+\cdots + a_kA^k.
\]
For example, if $p(x)=2-3x+x^2$, then $p(A) = 2I_n-3A+A^2$.
\begin{enumerate}
 \item If $p(x) = 3-4x+2x^2$ and $A=\begin{bmatrix}1&-3\\0&2\end{bmatrix}$, compute $p(A)$.

\bigskip

{\bf Solution:} Since $A^2 = \begin{bmatrix}1&-3\\0&2\end{bmatrix}\begin{bmatrix}1&-3\\0&2\end{bmatrix}=\begin{bmatrix}1&-9\\0&4\end{bmatrix}$, we have
\[
 p(A) = 3I-4A+2A^2 = 3\begin{bmatrix}1&0\\0&1\end{bmatrix}-4\begin{bmatrix}1&-3\\0&2\end{bmatrix}+2\begin{bmatrix}1&-9\\0&4\end{bmatrix} = \begin{bmatrix}1&-6\\0&3\end{bmatrix}.
\]

\bigskip

 \item The {\em characteristic polynomial} of an $n\times n$ matrix $A$ is defined by
\[
 c_A(x) = \det(xI_n-A).
\]
 The {\em Cayley-Hamilton Theorem} is a famous theorem in linear algebra which states that for any $n\times n$ matrix $A$, $c_A(A) = 0$ (where the zero on the right is the zero matrix).

Verify that the Cayley-Hamilton Theorem is true for $A=\begin{bmatrix}3&2\\1&-1\end{bmatrix}$.

\bigskip

{\bf Solution:} For $A=\begin{bmatrix}3&2\\1&-1\end{bmatrix}$ we have
\[
 xI_2-A = \begin{bmatrix}x-3&-2\\-1&x+1\end{bmatrix},
\]
so 
\[
 c_A(x) = \det(xI_2-A) = (x-3)(x+1)-2 = x^2-2x-5.
\]
We check that $A^2 = \begin{bmatrix}11&4\\2&3\end{bmatrix}$; thus,
\[
 c_A(A) = A^2-2A-5I_2 = \begin{bmatrix}11&4\\2&3\end{bmatrix}-2\begin{bmatrix}3&2\\1&-1\end{bmatrix}-5\begin{bmatrix}1&0\\0&1\end{bmatrix} = \begin{bmatrix}0&0\\0&0\end{bmatrix},
\]
as required.
\end{enumerate}

\bigskip

{\bf Bonus opportunity}: Prove the Cayley-Hamilton Theorem for the $n=2$ case. That is, show that the theorem holds for a general $2\times 2$ matrix $A=\begin{bmatrix}a&b\\c&d\end{bmatrix}$.

\bigskip

Given $A=\begin{bmatrix}a&b\\c&d\end{bmatrix}$, we have
\[
c_A(x) = \begin{vmatrix}
x-a&-b\\-c&x-d
\end{vmatrix}=(x-a)(x-d)-bc = x^2-(a+d)x+(ad-bc).
\]
Note that $A^2=\begin{bmatrix}
a^2+bc&ab+bd\\ac+dc&bc+d^2
\end{bmatrix}$, so we have
\begin{align*}
c_A(x) &= A^2-(a+d)A+(ad-bc)I_2\\
& = \begin{bmatrix}
a^2+bc&ab+bd\\ac+dc&bc+d^2
\end{bmatrix} - \begin{bmatrix}
a^2+ad&ab+bd\\ac+dc&ad+d^2
\end{bmatrix}+\begin{bmatrix}
ad-bc&0\\0&ad-bc
\end{bmatrix}\\
&=\begin{bmatrix}
0&0\\0&0
\end{bmatrix},
\end{align*}
which shows that the theorem is true for any $2\times 2$ matrix.

\bigskip

\item In each case, either explain why the statement is true (in general), or give an example showing that it is false:
\begin{enumerate}
 \item If $\len{\vec{v}-\vec{w}}=0$, then $\vec{v}=\vec{w}$.
 
 \bigskip
 
 {\bf Solution:} This is true. Let $\vec{v}=\langle v_1,\ldots, v_n\rangle$ and $\vec{w} = \langle w_1,\ldots, w_n\rangle$. Then
 \[
 \vec{v}-\vec{w} = \langle v_1-w_1,v_2-w_2,\ldots, v_n-w_n\rangle,
 \]
so if
\[
\len{\vec{v}-\vec{w}} = \sqrt{(v_1-w_1)^2+(v_2-w_2)^2+\cdots + (v_n-w_n)^2} = 0,
\]
then we must have $v_1-w_1=0, v_2-w_2=0,\ldots, v_n-w_n=0$, since the only value of $x$ for which $\sqrt{x}=0$ is $x=0$, and a sum of squares is zero if and only if each one of the squares is zero (since the square of a real number can't be negative). Thus, $v_i=w_i$ for $i=1,2,\ldots, n$, which implies that $\vec{v}=\vec{w}$.

\bigskip


 \item If $\vec{v}=-\vec{v}$, then $\vec{v}=\vec{0}$.

\bigskip

{\bf Solution:} This is true. Given $\vec{v}=-\vec{v}$, we can add $\vec{v}$ to both sides of the equation, giving us $2\vec{v}=\vec{0}$. If we now multiply both sides by $\frac{1}{2}$, we're left with $\vec{v}=\vec{0}$.

\bigskip


 \item If $\len{\vec{v}}=\len{\vec{w}}$, then $\vec{v}=\vec{w}$.
 
 \bigskip
 
 {\bf Solution:} This is false. For example if $\vec{v} = \langle 1,0\rangle$ and $\vec{w}=\langle 0,1\rangle$, then $\vec{v}\neq \vec{w}$, but $\len{\vec{v}}=\len{\vec{w}}=1$.
 
 \bigskip
 
 \item If $\len{\vec{v}}=\len{\vec{w}}$, then $\vec{v}=\pm\vec{w}$.
 
 \bigskip
 
 {\bf Solution:} This is also false, and the previous counterexample can be applied here as well.
 
 \bigskip
 
 \item $\len{\vec{v}+\vec{w}} = \len{\vec{v}}+\len{\vec{w}}$.
 
 \bigskip
 
 {\bf Solution:} This is false. If we take our two vectors in the solution to part (c), then we see that $\vec{v}+\vec{w} = \langle 1,1\rangle$, so $\len{\vec{v}+\vec{w}} = \sqrt{2}$, while $\len{\vec{v}}+\len{\vec{w}}=1+1=2$.
\end{enumerate}
\item Let $\vec{u} = \begin{bmatrix}3&-1&0\end{bmatrix}^T, \vec{v} = \begin{bmatrix}4&0&1\end{bmatrix}^T$, and $\vec{w} = \begin{bmatrix}1&1&1\end{bmatrix}^T$. In each case, either find scalars $a,b,c$ such that $\vec{x} = a\vec{u}+b\vec{v}+c\vec{w}$, or explain why no such scalars exist:
\begin{enumerate}
 \item $\vec{x} = \begin{bmatrix}5&1&2\end{bmatrix}^T$
 
 \bigskip
 
 If $\vec{x} = a\vec{u}+b\vec{v}+c\vec{w}$, we obtain the vector equation
 \[
 a\begin{bmatrix}
 3\\-1\\0\\
 \end{bmatrix}+b\begin{bmatrix}
 4\\0\\1
 \end{bmatrix}+c\begin{bmatrix}
 1\\1\\1
 \end{bmatrix} = \begin{bmatrix}
 5\\1\\2
 \end{bmatrix}
 \]
 which leads to the system of equations
 \[
 \begin{array}{ccccccc}
 3a&+&4b&+&c&=&5\\
 -a& & &+&c&=&1\\
   &+&b&+&c&=&2
 \end{array}
 \]
The general solution to this system (found, as usual, by setting up and reducing the augmented matrix of the system) is given by
\begin{align*}
a&=-1+t\\
b&=2-t\\
c&=t,
\end{align*} 
where $t$ can be any real number. In particular, we see that for $t=0$, we have $a=-1$,  $b=2$, and $c=0$ and we verify that 
\[
-\vec{u}+2\vec{v}+0\vec{w} = (-1)\begin{bmatrix}
 3\\-1\\0\\
 \end{bmatrix}+2\begin{bmatrix}
 4\\0\\1
 \end{bmatrix}=\begin{bmatrix}
 5\\1\\2
 \end{bmatrix}=\vec{x},
 \]
 as required.

\bigskip
 
 \item $\vec{x} = \begin{bmatrix}1&2&1\end{bmatrix}^T$.
 
 \bigskip
 
 The setup here is the same as in part (a), and yields the system of equations
  \[
 \begin{array}{ccccccc}
 3a&+&4b&+&c&=&1\\
 -a& & &+&c&=&2\\
   &+&b&+&c&=&1
 \end{array}.
 \]
 (Note that the only change is to the constants on the right-hand side given by the vector $\vec{x}$.) This time, if we set up and reduce our augmented matrix, we end up with the row-echelon form
 \[
 \begin{amatrix}{3}
 1&0&1&-2\\
 0&1&1&7/4\\
 0&0&0&-3/4
 \end{amatrix},
 \]
 and the last row tells us that the system must be inconsistent, since $0=-3/4$ is impossible. Thus, in this case there can be no values of $a,b$ and $c$ that satisfy the given vector equation.
\end{enumerate}
 \end{enumerate}

\end{document}
 
