\documentclass[letterpaper,12pt]{article}

\usepackage{ucs}
\usepackage[utf8x]{inputenc}
\usepackage{amsmath}
%\usepackage{amsfonts}
%\usepackage{amssymb}
\usepackage[margin=1in]{geometry}
\usepackage{graphicx}
\usepackage[bitstream-charter]{mathdesign}
\usepackage[T1]{fontenc}

\newcommand{\len}[1]{\lVert #1\rVert}
\newcommand{\abs}[1]{\lvert #1\rvert}
\newcommand{\R}{\mathbb{R}}
\newcommand{\di}{\displaystyle}
\title{Math 1560 Assignment \#1 Solutions\\University of Lethbridge, Fall 2017}
\author{Sean Fitzpatrick}
\begin{document}
 \maketitle


\begin{enumerate}
\item Evaluate the following limits:
\begin{enumerate}
\item $\di \lim_{x\to \infty}(\sqrt{9x^2+x}-3x)$

This limit has the indeterminate form $\infty-\infty$, so we investigate algebraically:
\begin{align*}
\lim_{x\to \infty}(\sqrt{9x^2+x}-3x) & = \lim_{x\to \infty}(\sqrt{9x^2+x}-3x)\cdot\frac{(\sqrt{9x^2+x}+3x)}{(\sqrt{9x^2+x}+3x)}\\
& = \lim_{x\to \infty}\frac{(9x^2+x)-9x^2}{\sqrt{9x^2+x}+3x}\\
& = \lim_{x\to \infty}\frac{x}{\sqrt{x^2(9+1/x)}+3x}\\
\intertext{At this point, we note that since $x\to \infty$, $x>0$, and thus $\sqrt{x^2}=x$, allowing us to factor out an $x$ from the denominator. Proceeding, we find:}
\lim_{x\to \infty}(\sqrt{9x^2+x}-3x) & = \lim_{x\to\infty}\frac{x\cdot 1}{x(\sqrt{9+1/x}+3)}\\
& = \lim_{x\to\infty}\frac{1}{\sqrt{9+1/x}+3}\\
& = \frac{1}{\sqrt{9}+3}=\frac{1}{6},
\end{align*}
since $1/x\to 0$ as $x\to \infty$.

\item $\di \lim_{x\to -\infty}(\sqrt{9x^2+x}-3x)$

In this case, $x\to -\infty$, so in particular, $x$ is negative. If we look at the two parts of our function, the square root is always positive, and for $x<0$, $-3x>0$, so when $x$ is negative, both parts of the function are positive, and both approach $\infty$ as $x\to \infty$. 

Thus, there is no indeterminate form in this case: both halves of the function are approaching $+\infty$, so
\[
\lim_{x\to -\infty}(\sqrt{9x^2+x}-3x)=\infty.
\]
\end{enumerate}

\bigskip

\item 
\begin{enumerate}
\item Show that the function $g(x)=\abs{x}$ is continuous on $\R$.

We recall that the absolute value function is defined by
\[
\abs{x} = \begin{cases} x, & \text{ if } x\geq 0\\-x, & \text{ if } x<0.\end{cases}
\]
At any point $a>0$, we see that $g(x)$ is continuous at $a$, since $g(x)=x$ for $x>a$, and any polynomial function is continuous. Similarly, $g$ is continuous at $a$ for any $a<0$.

It remains to be shown that $g$ is continuous at 0. From the definition we see that $g(0)=\abs{0}=0$; we need to show that the limit of $g$ as $x\to 0$ is also 0. We consider left- and right-hand limits:
\begin{align*}
\lim_{x\to 0^-}\abs{x} & = \lim_{x\to 0^-}(-x)\tag{Since $x<0$}\\
& = 0 \tag{By direct subsitution}\\
\lim_{x\to 0^+}\abs{x} & = \lim_{x\to 0^+}(x) \tag{Since $x>0$}\\
& = 0 \tag{By direct subsitution}
\end{align*} 
Since the left- and right-hand limits both equal zero, we have that
\[
\lim_{x\to 0}\abs{x} = 0 = \abs{0},
\]
and thus our function is continuous at 0, and therefore on all of $\R$.

\item Show that if $f$ is a continuous function, then so is $\abs{f}$.

Let $f$ be a continuous function, and let $g(x)=\abs{x}$. Then $g\circ f = \abs{f}$, since $g(f(x)) = \abs{f(x)}$ for any $x$ in the domain of $f$. It follows that $\abs{f}$ is continuous, since it is the composition of two continuous functions.

\item Give a counterexample showing that the converse to part (b) is false. That is, find a function $f$ such that $\abs{f}$ is continuous, but $f$ is not.

Consider the function
\[
f(x) = \begin{cases} 1, & \text{ if } x\geq 0\\-1, & \text{ if } x<0.\end{cases}
\]
We immediately see that $f$ has a jump discontinuity at 0, since $\di \lim_{x\to 0^-}f(x) = -1$, but $\di \lim_{x\to 0^+}f(x)=1$.

However, since $\abs{-1}=1$, we see that for any $x\in \R$, $\abs{f(x)} = 1$, so $\abs{f}$ is a constant function, and we know that every constant function is continuous.
\end{enumerate}

\newpage

 \item Show that $\di \lim_{x\to 0}f(x)=0$, where
 \[
 f(x) = \begin{cases}x^2, &\text{ if } x \text{ is rational}\\ 0, & \text{ if } x \text{ is irrational}.\end{cases}
 \]
 
 Let $x$ be any real number. We know that $f(x)=0$ or $f(x)=x^2$, depending on whether $x$ is rational or irrational. Since $x^2\geq 0$ for all $x\in \R$, we see that
 \[
 0\leq f(x)\leq x^2
 \]
 for any real number $x$. Since
 \[
 \lim_{x\to 0}(0) = 0 = \lim_{x\to 0}x^2,
 \]
 it follows from the Squeeze Theorem that $\di \lim_{x\to 0}f(x)=0$.
 
 \end{enumerate}

\end{document}
 
