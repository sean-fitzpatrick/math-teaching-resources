\documentclass[12pt]{article}
\usepackage{amsmath}
\usepackage{amssymb}
\usepackage[letterpaper,margin=0.85in,centering]{geometry}
\usepackage{fancyhdr}
\usepackage{enumerate}
\usepackage{lastpage}
\usepackage{multicol}
\usepackage{graphicx}

\reversemarginpar

\pagestyle{fancy}
\cfoot{Page \thepage \ of \pageref{LastPage}}\rfoot{{\bf Total Points: 100}}
\chead{MATH 1410}\lhead{DEFERRED FINAL EXAM}\rhead{21st December, 2016}

\newcommand{\points}[1]{\marginpar{\hspace{24pt}[#1]}}
\newcommand{\skipline}{\vspace{12pt}}
%\renewcommand{\headrulewidth}{0in}
\headheight 30pt

\newcommand{\di}{\displaystyle}
\newcommand{\abs}[1]{\lvert #1\rvert}
\newcommand{\R}{\mathbb{R}}
\newcommand{\C}{\mathbb{C}}
\renewcommand{\P}{\mathcal{P}}
\DeclareMathOperator{\nul}{null}
\DeclareMathOperator{\range}{range}
\DeclareMathOperator{\spn}{span}
\newcommand{\len}[1]{\lVert #1\rVert}
\newcommand{\Q}{\mathbb{Q}}
\newcommand{\N}{\mathbb{N}}
\renewcommand{\L}{\mathcal{L}}
\newcommand{\dotp}{\boldsymbol{\cdot}}
\newenvironment{amatrix}[1]{%
  \left[\begin{array}{@{}*{#1}{c}|c@{}}
}{%
  \end{array}\right]
}
\newcommand{\bam}{\begin{amatrix}}
\newcommand{\eam}{\end{amatrix}}
\newcommand{\bbm}{\begin{bmatrix}}
\newcommand{\ebm}{\end{bmatrix}}

\begin{document}


\author{Instructor: Sean Fitzpatrick}
\thispagestyle{plain}
\begin{center}
\emph{University of Lethbridge}\\
Department of Mathematics and Computer Science\\
21st December, 2016, 9:00 am - 12:00 pm\\
{\bf MATH 1410 - FINAL EXAM (DEFERRED)}\\
\end{center}
\skipline \skipline \skipline \noindent \skipline
Last Name:\underline{\hspace{350pt}}\\
\skipline
First Name:\underline{\hspace{348pt}}\\
\skipline
Student Number:\underline{\hspace{322pt}}\\


\vspace{0.1in}


\begin{quote}

 
 Record your answers below each question in the space provided.    Left-hand pages may be used as scrap paper for rough work.  If you want any work on the left-hand pages to be graded, please indicate so on the right-hand page. Additional scrap paper may be requested if needed.
 
 \bigskip
 
 Partial credit will be awarded for partially correct work, so be sure to show your work, and include all necessary justifications needed to support your arguments. 

\bigskip

 \textbf{No external aids are permitted}, with the exception of a basic (non-scientific and non-graphing) calculator.
\end{quote}


\vspace{0.2in}

For grader's use only:

\begin{table}[hbt]
\begin{center}
\begin{tabular}{|l|r|} \hline
Page&Grade\\
\hline \hline
\cline{1-2} 2 & \enspace\enspace\enspace\enspace\enspace\enspace/10\\
\cline{1-2} 3 & \enspace\enspace\enspace\enspace\enspace\enspace/10\\
\cline{1-2} 4 & \enspace\enspace\enspace\enspace\enspace\enspace/12\\
\cline{1-2} 5 & \enspace\enspace\enspace\enspace\enspace\enspace/11\\
\cline{1-2} 6 & \enspace\enspace\enspace\enspace\enspace\enspace/9\\
\cline{1-2} 7 & \enspace\enspace\enspace\enspace\enspace\enspace/8\\
\cline{1-2} 8 & \enspace\enspace\enspace\enspace\enspace\enspace/9\\
\cline{1-2} 9 & \enspace\enspace\enspace\enspace\enspace\enspace/9\\
\cline{1-2} 10 & \enspace\enspace\enspace\enspace\enspace\enspace/8\\
\cline{1-2} 11 & \enspace\enspace\enspace\enspace\enspace\enspace/7\\
\cline{1-2} 12 & \enspace\enspace\enspace\enspace\enspace\enspace/7\\
\cline{1-2} Total & \enspace\enspace\enspace\enspace\enspace\enspace/100\\
\hline
\end{tabular}

\skipline

\skipline

\skipline

\end{center}
\end{table}
\newpage


\begin{enumerate}
\item DEFINITIONS. (2 points each) Give the definition of the term in bold. Write your answer as a complete sentence. 

\begin{enumerate}
 \item What is the \textbf{modulus} of a complex number?

\vspace{1.3in}

 \item What is a \textbf{unit vector}?

\vspace{1.3in}

 \item What does it mean for an $n\times n$ matrix $A$ to be \textbf{anti-symmetric}?

\vspace{1.3in}

 \item What is the \textbf{null space} of a matrix?

\vspace{1.3in}

 \item What is an \textbf{eigenvalue} of an $n\times n$ matrix $A$?


 
\end{enumerate}
\newpage

\item SHORT ANSWER -- Calculations (2 points each): You do not have to show your work.

\begin{enumerate}
 \item Compute the magnitude (length) of the vector $\vec{v} = \langle -3,4,1\rangle$

\vspace{1.5in}

 \item Compute the product $zw$ of the complex numbers $z=-2+3i$  and $w=5+4i$.

\vspace{1.5in}

 

 \item Given $A = \bbm 3&-5\\-4&2\ebm$, determine an elementary matrix $E$ such that $EA = \bbm 3&-5\\2&-8\ebm$.

\vspace{1.5in}

 \item Compute the trace of the matrix $A = \bbm 2&-3&7\\-4&6&0\\3&-5&-1\ebm$.

\vspace{1.25in}

 \item Compute the determinant of $A = \bbm 4&-3\\2&1\ebm$.

\end{enumerate}
\newpage

\item SHORT ANSWER -- More calculations (3 points each): You don't have to show work, but it can earn you part marks if you do.

\begin{enumerate}
 \item The augmented matrix for a system of equations in the variables $x,y,z$ can be put into the row-echelon form $\bam{3}1&-2&-4&2\\0&1&3&4\eam$. What is the solution of the system?

\vspace{1.75in}
 
 \item Suppose $A$ and $B$ are $2\times 2$ matrices such that $\det(A) = 3$ and $\det(B)=-11$. What is the value of $\det(2A^2B^T(AB)^{-1})$?


\vspace{1.6in}

 \item Calculate the projection of the vector $\vec{w} = \langle 4,2,-1\rangle$ onto the vector $\vec{v} = \langle 2,-1,2\rangle$.

\vspace{1.75in}

 \item What is the reduced row-echelon form of the matrix $A = \bbm 3&-6\\2&-3\ebm$?


\end{enumerate}
\newpage

\item SHORT ANSWER -- Yet more calculations (points as indicated):

\begin{enumerate}
 \item Write the complex number $\dfrac{5-2i}{3+4i}$ in the form $x+iy$. \points{3}

\vspace{1.75in}

 \item Determine the matrix $X$ such that $4X+\bbm 2&3\\-2&7\ebm = 3\bbm 5&0\\1&-2\ebm$. \points{3}

\vspace{1.75in}

 \item Compute the matrix product $\bbm 4&0\\-2&3\ebm \, \bbm -3&1\\0&2\ebm$. \points{3}

\vspace{1.75in}

 \item Compute $T\left(\bbm -3\\2\ebm\right)$, where $T:\R^2\to \R^2$ is a matrix transformation such that $T\left(\bbm 1\\0\ebm\right) = \bbm 1\\-4\ebm$ and $T\left(\bbm 0\\1\ebm\right) = \bbm 7\\-9\ebm$. \points{2}

\end{enumerate}
\newpage

\item Compute the eigenvalues and eigenvectors of the matrix $A = \bbm 1&4\\2&3\ebm$. \points{9}

\newpage

\item Compute the determinant of the matrix \points{8}
\[
 A = \bbm 12 & 0 & 1 & -3\\1&-2&7&-4\\3&0&1&0\\0&1&-4&2\ebm
\]

\newpage

\item Consider the matrix $A = \bbm 1&0&2\\0&1&1\\2&1&6\ebm$.
\begin{enumerate}
 \item Compute the inverse of $A$.\points{6}

\vspace{5in}

 \item Solve the system $A\vec{x} = \bbm 3\\-2\\5\ebm$. \points{3}
\end{enumerate}

\newpage

\item Consider the vectors $\vec{u} = \bbm 2\\-1\\5\ebm$, $\vec{v} = \bbm 6\\3\\-4\ebm$, and $\vec{w} = \bbm -2\\-5\\14\ebm$.
\begin{enumerate}
 \item Determine whether or not the vector $\bbm 3\\-1\\7\ebm$ belongs to the span of $\vec{u}, \vec{v}$, and $\vec{w}$. \points{6}

\vspace{5.5in}

 \item Are the vectors $\vec{u}, \vec{v}, \vec{w}$ linearly independent? Explain. \points{3}
\end{enumerate}

\newpage

\item Consider the lines $\ell_1$ and $\ell_2$ given by
\begin{align*}
 \langle x,y,z\rangle &= \langle 2, -1, 3\rangle + s\langle 3, 0, -4\rangle \quad \text{for $\ell_1$, and}\\
 \langle x,y,z\rangle & = \langle 1, 3, -5\rangle + t\langle 1, 2, -6\rangle \quad \text{for $\ell_2$.}
\end{align*}
\begin{enumerate}
 \item Show that the lines intersect. \points{4}

\vspace{3.5in}

 \item Determine the equation of the plane containing $\ell_1$ and $\ell_2$. \points{4}
\end{enumerate}

\newpage


 \item Let $A$ be an $n\times n$ matrix and let $I$ denote the $n\times n$ identity matrix.
\begin{enumerate}
 \item Show that if $A^2=0$, then $(I-A)^{-1} = I+A$. \points{2} (Hint: $Y=X^{-1}$ if and only if $XY=I$.)

\vspace{2in}

 \item Show that if $A^3=0$, then $(I-A)^{-1} = I+A+A^2$. \points{2}

\vspace{2.25in}

 \item Suppose that $A^n=0$ for some integer $n\geq 3$. Propose a formula for $(I-A)^{-1}$ and show that your formula is correct. \points{3}



\end{enumerate}


\newpage

 \item Verify that $\bbm 1\\i\ebm$ and $\bbm i\\1\ebm$ are eigenvectors of the matrix $Z = \bbm 2&4\\-4&2\ebm$. \points{3}

\vspace{4in}

 \item Prove that if $A$ is an invertible matrix and $\lambda$ is an eigenvalue of $A$, then $\dfrac{1}{\lambda}$ is an eigenvalue of $A^{-1}$. \points{4}
(For a 2 point bonus: why does the invertibility of $A$ guarantee that $\lambda\neq 0$?)






\end{enumerate}

\end{document}