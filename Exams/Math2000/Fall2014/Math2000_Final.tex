\documentclass[12pt]{article}
\usepackage{amsmath}
\usepackage{amssymb}
\usepackage[letterpaper,margin=0.85in,centering]{geometry}
\usepackage{fancyhdr}
\usepackage{enumerate}
\usepackage{lastpage}
\usepackage{multicol}
\usepackage{graphicx}

\reversemarginpar

\pagestyle{fancy}
\cfoot{Page \thepage \ of \pageref{LastPage}}\rfoot{{\bf Total Points: 100}}
\chead{MATH 2000A}\lhead{FINAL EXAM}\rhead{Wednesday, 10\textsuperscript{th} December, 2014}

\newcommand{\points}[1]{\marginpar{\hspace{24pt}[#1]}}
\newcommand{\skipline}{\vspace{12pt}}
%\renewcommand{\headrulewidth}{0in}
\headheight 30pt

\newcommand{\di}{\displaystyle}
\newcommand{\R}{\mathbb{R}}
\newcommand{\N}{\mathbb{N}}
\newcommand{\Z}{\mathbb{Z}}
\newcommand{\Q}{\mathbb{Q}}

\begin{document}

\author{Instructor: Sean Fitzpatrick}
\thispagestyle{plain}
\begin{center}
\emph{University of Lethbridge}\\
Department of Mathematics and Computer Science\\
10\textsuperscript{th} December, 2014, 2:00 - 5:00 pm\\
{\bf Math 2000A - FINAL EXAM}\\
\end{center}
\skipline \skipline \skipline \noindent \skipline
Last Name:\underline{\hspace{350pt}}\\
\skipline
First Name:\underline{\hspace{348pt}}\\
\skipline
Student Number:\underline{\hspace{322pt}}\\


\vspace{0.5in}


\begin{quote}
 Record your answers below each question in the space provided.    {\bf Left-hand pages may be used as scrap paper for rough work.}  If you want any work on the left-hand pages to be graded, please indicate so on the right-hand page.
 
 \bigskip
 
Partial credit will be awarded for partially correct work, so be sure to show your work, and include all necessary justifications needed to support your arguments. 

The value of each problem is indicated in the left-hand margins. The value of a problem does not always indicate the amount of work required to do the problem.

{\bf No external aids are permitted, including calculators, cell phones, cheat sheets, laptops, spy cameras, drones, and secret radio communications devices.}
\end{quote}


\vspace{0.5in}

For grader's use only:

\begin{table}[hbt]
\begin{center}
\begin{tabular}{|l|r|c|l|r|} \hline
Page&Grade & \hspace{24pt} &Page & Grade\\
\hline \hline
 2 & \enspace\enspace\enspace\enspace\enspace\enspace/13 & &  7 & \enspace\enspace\enspace\enspace\enspace\enspace/10 \\
\hline
 3 & \enspace\enspace\enspace\enspace\enspace\enspace/8 & &  8 & \enspace\enspace\enspace\enspace\enspace\enspace/8\\
\hline
 4 & \enspace\enspace\enspace\enspace\enspace\enspace/8 & & 9 & \enspace\enspace\enspace\enspace\enspace\enspace/12\\
\hline
 5 & \enspace\enspace\enspace\enspace\enspace\enspace/12 & &  10 & \enspace\enspace\enspace\enspace\enspace\enspace/10\\
\hline
 6 & \enspace\enspace\enspace\enspace\enspace\enspace/9 & &  11 & \enspace\enspace\enspace\enspace\enspace\enspace/10\\
\hline
\multicolumn{5}{|c|}{}\\
\multicolumn{5}{|l|}{{\bf Total:}\hspace{168 pt} /100}\\
\hline
\end{tabular}

\medskip


\end{center}
\end{table}
\newpage


\begin{enumerate}
\item Establish the following logical equivalence (justify each step):\points{5}
\[
 P\to (Q\vee R)\equiv (P\wedge \neg Q)\to R 
\]

\vspace{2.25in}

\item Given a function $f:A\to B$, we say that $f$ is {\em onto} if for every $b\in B$ there exists some $a\in A$ such that $f(a)=b$.
\begin{enumerate}
 \item Re-write this definition using symbolic logic and the quantifier symbols $\forall$ and $\exists$. \points{2}

\vspace{1.25in}

 \item Using symbolic logic, define what it means to say that $f$ is {\bf not} onto. \points{3}

\vspace{1.25in}
\end{enumerate}
\item What is the contrapositive of the conditional statement ``If $a^2=25$, then $a=5$ or $a=-5$''?\points{3}

\newpage

\item Let $U = \{0,1,2,3,4,5,6,7,8,9\}$, and let $A,B,C\subseteq U$ be given by
\[
 A = \{2,5,7,8\}, \quad B = \{1,3,4,9\}, \quad C = \{0,2,4\}.
\]
\begin{enumerate}
 \item What is the power set of $C$? \points{2}

\vspace{2in}

 \item What is $A\cup (B\cap C)$? \points{2}

\vspace{1.5in}

 \item What is $A\setminus C$? \points{2}

\vspace{1.5in}

 \item What is the complement of $A\cup B\cup C$? \points{2}
\end{enumerate}
\newpage

\item Determine if the statements below are true or false, and support your answer with a proof or counterexample.
\begin{enumerate}
 \item For any subsets $A,B,C$ of some universal set $U$,  \points{4}
\[
 A\setminus (B\cap C) = (A\setminus B)\cup (A\setminus C)
\]

\vspace{4in}

 \item For any subsets $A,B,C$ of some universal set $U$,  \points{4}
\[
\text{if } A\cap C\subseteq B\cap C, \text{ then } A\subseteq B.
\]

\end{enumerate}
\newpage

\item Determine if the statements below are true or false, and support your answer with a proof or counterexample.
\begin{enumerate}
       \item For all $a,b,c\in \Z$, if $a|b$ and $a|c$, then $a|(b-c)$.\points{4}

\vspace{2.5in}

       \item For all $a,b\in\Z$, if $ab\equiv 0\pmod{6}$, then $a\equiv 0\pmod{6}$ or $b\equiv 0\pmod{6}.$\points{4}

\vspace{2in}

       \item For each $a\in \Z$, if 3 does not divide $a$, then 3 divides $2a^2+1$.\points{4}
      \end{enumerate}
\newpage

\item Let $f:\Z\to \Z$ be defined by $f(x) = 5x+3$.
\begin{enumerate}
 \item Is $f$ one-to-one? \points{3}

\vspace{2.5in}

 \item Is $f$ onto? \points{3}

\vspace{2.5in}

 \item Does your answer to either (a) or (b) change if $f$ is a function from $\Q$ to $\Q$? Explain. \points{3}
\end{enumerate}

\newpage

\item \begin{enumerate}
       \item If $f:\R\to\R$ is given by $f(x) = 2x+1$ and $g:\R\to\R$ is given by $g(x) = \dfrac{1}{x^2+1}$, \points{4} determine formulas for the compositions $f\circ g$ and $g\circ f$.

\vspace{2.5in}

\item Construct an example of functions $f:A\to B$ and $g:B\to C$ such that $g\circ f:A\to C$ is onto, but $f$ is not. \points{2}

\vspace{2.5in}

\item Given $f:A\to B$ and $g:B\to C$, prove that if $g\circ f:A\to C$ is onto, then so is $g$. \points{4}
      \end{enumerate}
\newpage

\item Let $A = \{1,2,3,4,5,6\}$ and let $B = \{r,s,t,u,v\}$. Define $f:A\to B$ by
\[
 f(1) = s,\, f(2) = t,\, f(3) = r,\, f(4) = t,\, f(5) = u,\, f(6) = s.
\]
\begin{enumerate}
 \item If $C = \{2,4,6\}$, what is $f(C)$? \points{2}

\vspace{1.5in}

 \item If $D = \{r,s,v\}$, what is $f^{-1}(D)$? \points{2}

\vspace{1.5in}


\end{enumerate}
\item Prove that for any function $f:A\to B$ and any subsets $C,D\subseteq B$, we have \points{5}
\[
 f^{-1}(C\cup D) = f^{-1}(C)\cup f^{-1}(D).
\]


\newpage

\item \begin{enumerate}
       \item If $k\geq l$, show that there exists an onto function $f:\{1,2,\ldots, k\}\to\{1,2,\ldots, l\}$. \points{4}

\vspace{2.5in}

       \item Using part (a), show that if $A$ and $B$ are finite sets with $|A|\geq |B|$, then there exists an onto function $g:A\to B$. \points{4}
      \end{enumerate}

\vspace{2.5in}

\item Give four examples of countably infinite sets. \points{4}

\newpage

\item Use mathematical induction to prove the following statement: \points{10}
\[
\forall n\in\N,\quad 2+5+8+\cdots + (3n-1) = \frac{n(3n+1)}{2}.
\]


\newpage

\item Let $A=\{0,1,2,3,4,5,6,7,8\}$, and define a relation $\sim$ on $A$ by 
\[
 a\sim b \, \text{ if and only if } \, a^2\equiv b^2 \pmod{9}.
\]
\begin{enumerate}
 \item Prove that $\sim$ is an equivalence relation on $A$. \points{5}

\vspace{3in}

 \item Determine all of the distinct equivalence classes determined by $\sim$.\points{3}

\vspace{2in}
\end{enumerate}
\item For which $[x]\in\Z_4 = \{[0],[1],[2],[3]\}$ is it true that $([2]\odot[x])\oplus [1] = [3]$?\points{2}\\
(Hint: there are only four values of $[x]$ to check.)

\end{enumerate}
\newpage

This page is left intentionally blank for rough work. Please do not remove this page.


\end{document}