\documentclass[12pt]{article}
\usepackage{amsmath}
\usepackage{amssymb}
\usepackage[letterpaper,margin=0.85in,centering]{geometry}
\usepackage{fancyhdr}
\usepackage{enumerate}
\usepackage{lastpage}
\usepackage{multicol}
\usepackage{graphicx}

\reversemarginpar

\pagestyle{fancy}
\cfoot{Page \thepage \ of \pageref{LastPage}}\rfoot{{\bf Total Points: 100}}
\chead{MATH 2000 A\&B}\lhead{Final Exam}\rhead{Friday, 18\textsuperscript{th} December, 2015}

\newcommand{\points}[1]{\marginpar{\hspace{24pt}[#1]}}
\newcommand{\skipline}{\vspace{12pt}}
%\renewcommand{\headrulewidth}{0in}
\headheight 30pt

\newcommand{\di}{\displaystyle}
\newcommand{\R}{\mathbb{R}}
\newcommand{\N}{\mathbb{N}}
\newcommand{\Z}{\mathbb{Z}}

\begin{document}

\author{Instructor: Sean Fitzpatrick}
\thispagestyle{plain}
\begin{center}
\emph{University of Lethbridge}\\
Department of Mathematics and Computer Science\\
18\textsuperscript{th} December, 2015, 9:00 am - 12:00 pm\\
{\bf Math 2000 A\&B - FINAL EXAM}\\
\end{center}
\skipline \skipline \skipline \noindent \skipline
Last Name:\underline{\hspace{350pt}}\\
\skipline
First Name:\underline{\hspace{348pt}}\\
\skipline
Student Number:\underline{\hspace{322pt}}\\


\vspace{0.5in}


\begin{quote}
 Record your answers below each question in the space provided.    {\bf Left-hand pages may be used as scrap paper for rough work.}  If you want any work on the left-hand pages to be graded, please indicate so on the right-hand page.
 
 \bigskip
 
Partial credit will be awarded for partially correct work, so be sure to show your work, and include all necessary justifications needed to support your arguments.

No external aids are permitted.
\end{quote}


\vspace{0.5in}

For grader's use only:

\begin{table}[hbt]
\begin{center}
\begin{tabular}{|l|r|} \hline
Page&Grade\\
\hline \hline
\cline{1-2} 2 & \enspace\enspace\enspace\enspace\enspace\enspace/10\\
\cline{1-2} 3 & \enspace\enspace\enspace\enspace\enspace\enspace/9\\
\cline{1-2} 4 & \enspace\enspace\enspace\enspace\enspace\enspace/9\\
\cline{1-2} 5 & \enspace\enspace\enspace\enspace\enspace\enspace/11\\
\cline{1-2} 6 & \enspace\enspace\enspace\enspace\enspace\enspace/10\\
\cline{1-2} 7 & \enspace\enspace\enspace\enspace\enspace\enspace/10\\
\cline{1-2} 8 & \enspace\enspace\enspace\enspace\enspace\enspace/10\\
\cline{1-2} 9 & \enspace\enspace\enspace\enspace\enspace\enspace/10\\
\cline{1-2} 10 & \enspace\enspace\enspace\enspace\enspace\enspace/10\\
\cline{1-2} 11 & \enspace\enspace\enspace\enspace\enspace\enspace/11\\
\cline{1-2} Total & \enspace\enspace\enspace\enspace\enspace\enspace/100\\
\hline
\end{tabular}

\skipline

\skipline

\skipline


\end{center}
\end{table}
\newpage


\begin{enumerate}
 \item Give a precise definition of each of the terms written in {\bf bold} below.
\begin{enumerate}
 \item The {\bf Cartesian product} of two sets $A$ and $B$.\points{2}

\vspace{1.5in}

 \item The {\bf range} of a function $f:A\to B$. \points{2}

\vspace{1.5in}

 \item The {\bf pre-image} of a subset $Y\subseteq B$ with respect to a function $f:A\to B$. \points{2}

\vspace{1.5in}

 \item The {\bf equivalence class} of an object $a\in A$, with respect to an equivalence relation $\sim$ on $A$. \points{2}

\vspace{1.5in}

 \item What it means for a set $A$ to be {\bf finite}. \points{2}
\end{enumerate}


\newpage

\item Let $A=\{1,3,4,6,8,9\}$, $B = \{2,3,5,6,7,9\}$, and $C=\{2,3,5,8\}$ be subsets of\\ $U = \{0,1,2,3,4,5,6,7,8,9\}$.
\begin{enumerate}
 \item Calculate $A\cup B$. \points{1}

\vspace{1in}
 
 \item Calculate $A\cap B$. \points{1}

\vspace{1in}

 \item Calculate $A\setminus B$. \points{1}

\vspace{1in}

 \item Verify that $(A\cap B)^c = A^c\cup B^c$. \points{3}

\vspace{2in}

 \item Verify that $A\cap (B\cup C) = (A\cap B)\cup (A\cap C)$. \points{3}

\end{enumerate}
\newpage

\item Let $A, B$, and $C$ be subsets of some universal set $U$. Prove or disprove each of the following:
\begin{enumerate}
 \item $A\cap B\subseteq A$. \points{3}

 \vspace{2.5in}

 \item If $A\cup C\subseteq B\cup C$, then $A\subseteq B$. \points{3}

\vspace{2.5in}

 \item If $A\subseteq B$, then $A\cup C\subseteq B\cup C$. \points{3}
\end{enumerate}
\newpage

\item Let $A$, $B$, and $C$ be sets. 
\begin{enumerate}
 \item Show that $(A\cup B)\setminus C = (A\setminus C)\cup (B\setminus C)$. \points{3}

\vspace{2.5in}

 \item Suppose that $A\cup C=B\cup C$ {\bf and} $A\cap C=B\cap C$. Prove that $A=B$. \points{4}\\
{\bf Hint:} It's enough to prove that $A\subseteq B$. The proof that $B\subseteq A$ is identical.

\vspace{3.5in}

 \item Prove that if $X\subseteq A$, then $X\times B\subseteq A\times B$. \points{4}


\end{enumerate}
\newpage

\item Let $A=\{1,2,3,4,5\}$ and let $B = \{s,t,u,v\}$, and define $f:A\to B$ by
\[
 f(1) = u, \, f(2) = s,\, f(3) = u,\, f(4) = t,\, f(5) = v.
\]
\begin{enumerate}
 \item Draw an arrow diagram that represents the function $f$. \points{2}

\vspace{1.75in}

 \item Is the function $f$ an injection? Why or why not? \points{2}

\vspace{1in}

 \item Is the function $f$ a surjection? Why or why not?\points{2}

\vspace{1in}

 \end{enumerate}
\item Let $\Z_5 = \{0,1,2,3,4\}$, and define $g:\Z_5\to \Z_5$ by $g(x) = 3x+2 \pmod{5}$.\\
Verify that $g$ is a bijection, and determine the function $g^{-1}$. \points{4}\\
{\bf Note:} You don't need to find a formula for $g^{-1}$ but it'll be extra nifty if you do.

\newpage

\item Let $f:A\to B$ and $g:B\to C$ be functions such that $g\circ f:A\to C$ is a surjection.
\begin{enumerate}
 \item Either prove that $g$ must be a surjection, or give an example where it is not. \points{3}

\vspace{2.5in}

 \item Either prove that $f$ must be a surjection, or give an example where it is not. \points{3}

\vspace{2.5in}

\end{enumerate}
\item Prove that if $f:A\to B$ and $g:B\to C$ are both injections, then $g\circ f:A\to C$ is an injection.\points{4}

\newpage

\item Let $A=\{1,2,3,4\}$, and consider the following relations on $A$:
\begin{align*}
 R &=\{(1,1), (2,4), (3,3), (4,1)\}\\
 S &=\{(1,1), (1,2), (2,1), (2,2), (3,3), (3,4), (4,3), (4,4)\}\\
 T &=\{(1,2), (1,3), (2,2), (2,3), (3,2), (3,3), (4,4)\}
\end{align*}
\begin{enumerate}
 \item Which of the above relations define a function? Which do not? Justify your answer. \points{3}

\vspace{2in}

 \item Which of the above relations are reflexive? Which are not? Justify your answer.\points{3}

\vspace{2in}

 \item Which of the above relations are symmetric? Which are not? Justify your answer. \points{3}

\vspace{2in}

 \item Which one is an equivalence relation? (No justification needed.) \points{1}
\end{enumerate}
\newpage

 \item Let $A = \{1,2,3,4,5\}$. Draw a directed graph that represents an equivalence relation on $A$ with two distinct equivalence classes.\points{4}

\vspace{3in}

 \item  For any $m\in\mathbb{Z}$, let $[m]=\{n\in\mathbb{Z} \,|\, n\equiv m\pmod{3}\}$ denote the congruence class of $m$ modulo 3. Use mathematical induction to prove that $[10^n]=[1]$ for all natural numbers $n$.\points{6}
 
\newpage

\item Give an example of two sets $A$ and $B$ such that $A\subseteq B$, and $A\approx B$.\points{4}\\
Demonstrate the equivalence by giving an explicit bijection $f:A\to B$.

\vspace{3in}

\item Suppose $A$ and $B$ are disjoint finite sets, with cardinalities $|A|=3$ and $|B|=7$. \\
What are the cardinalities of $A\cup B$ and $A\times B$? \points{3}

\vspace{3in}

\item Give three examples of countably infinite sets, and three examples of uncountable sets.\points{3}

\newpage

\item Let $B$ be a countable set. Prove that any subset $A\subseteq B$ is countable. \points{4}\\
{\bf Hint:} You need to show there is an injection $f:A\to\mathbb{N}$.

\vspace{2.5in}

\item Prove that if $A$ is uncountable and $A\subseteq B$, then $B$ is uncountable. \points{3}
 
\vspace{2in}

\item Let $g:\mathbb{N}\to A$ be a surjection. For each $a\in A$, let $n_a\in\mathbb{N}$ be the least element of the set $g^{-1}(a)$, and define $f:A\to \mathbb{N}$ by $f(a) = n_a$. Show that $g\circ f = I_A$, and explain why this implies that $f$ is an injection.\points{4}

\vspace{3in}

{\bf Bonus (3 points)}: On the next page, explain why the above definition of $f$ makes sense.
\end{enumerate}
\newpage

{\bf Extra space for rough work. Do not remove unless there is nothing on this page you want to be graded.}
\end{document}