\documentclass[12pt]{article}
\usepackage{amsmath}
\usepackage{amssymb}
\usepackage[letterpaper,margin=0.85in,centering]{geometry}
\usepackage{fancyhdr}
\usepackage{enumerate}
\usepackage{lastpage}
\usepackage{multicol}
\usepackage{graphicx}

\reversemarginpar

\pagestyle{fancy}
\cfoot{Page \thepage \ of \pageref{LastPage}}\rfoot{{\bf Total Points: 100}}
\chead{MATH 2000B}\lhead{Final Exam}\rhead{Wednesday, 29\textsuperscript{th} April, 2015}

\newcommand{\points}[1]{\marginpar{\hspace{24pt}[#1]}}
\newcommand{\skipline}{\vspace{12pt}}
%\renewcommand{\headrulewidth}{0in}
\headheight 30pt

\newcommand{\di}{\displaystyle}
\newcommand{\R}{\mathbb{R}}
\newcommand{\N}{\mathbb{N}}
\newcommand{\Z}{\mathbb{Z}}

\begin{document}

\author{Instructor: Sean Fitzpatrick}
\thispagestyle{plain}
\begin{center}
\emph{University of Lethbridge}\\
Department of Mathematics and Computer Science\\
29\textsuperscript{th} April, 2015, 2:00-5:00 pm\\
{\bf Math 2000B - FINAL EXAM}\\
\end{center}
\skipline \skipline \skipline \noindent \skipline
Last Name:\underline{\hspace{350pt}}\\
\skipline
First Name:\underline{\hspace{348pt}}\\
\skipline
Student Number:\underline{\hspace{322pt}}\\


\vspace{0.5in}


\begin{quote}
 Record your answers below each question in the space provided.    {\bf Left-hand pages may be used as scrap paper for rough work.}  If you want any work on the left-hand pages to be graded, please indicate so on the right-hand page.
 
 \bigskip
 
Partial credit will be awarded for partially correct work, so be sure to show your work, and include all necessary justifications needed to support your arguments.

No external aids are permitted.
\end{quote}


\vspace{0.5in}

For grader's use only:

\begin{table}[hbt]
\begin{center}
\begin{tabular}{|l|r|} \hline
Page&Grade\\
\hline \hline
\cline{1-2} 2 & \enspace\enspace\enspace\enspace\enspace\enspace/10\\
\cline{1-2} 3 & \enspace\enspace\enspace\enspace\enspace\enspace/10\\
\cline{1-2} 4 & \enspace\enspace\enspace\enspace\enspace\enspace/10\\
\cline{1-2} 5 & \enspace\enspace\enspace\enspace\enspace\enspace/10\\
\cline{1-2} 6 & \enspace\enspace\enspace\enspace\enspace\enspace/10\\
\cline{1-2} 7 & \enspace\enspace\enspace\enspace\enspace\enspace/10\\
\cline{1-2} 8 & \enspace\enspace\enspace\enspace\enspace\enspace/10\\
\cline{1-2} 9 & \enspace\enspace\enspace\enspace\enspace\enspace/10\\
\cline{1-2} 10 & \enspace\enspace\enspace\enspace\enspace\enspace/10\\
\cline{1-2} 11 & \enspace\enspace\enspace\enspace\enspace\enspace/10\\
\cline{1-2} Total & \enspace\enspace\enspace\enspace\enspace\enspace/100\\
\hline
\end{tabular}

\skipline

\skipline

\skipline


\end{center}
\end{table}
\newpage


\begin{enumerate}
\item Write the converse and the contrapositive of each of the following statements:
\begin{enumerate}

 \item If $a$ is an odd integer, then $3a$ is an odd integer. \points{2}

\vspace{1.75in}

 \item If $ab=0$, then $a=0$ or $b=0$. \points{3}

\vspace{1.75in}
\end{enumerate}
\item Write the negation of each of the following statements. Use of symbolic notation is allowed but not required.
\begin{enumerate}
 \item For all integers $m$, $m^2$ is even.\points{2}

\vspace{1.25in}

 \item For all integers $m$, there exists an integer $n$ such that $2m+3n=5$.\points{3}
\end{enumerate}
\newpage

\item Let $A=\{1,3,6,7\}$, $B=\{2,3,4,8,9\}$, and $C = \{1,5,7,8\}$ be subsets of the universal set $U=\{1,2,3,4,5,6,7,8,9\}$. Verify the following by explicitly computing the sets involved:

\begin{enumerate}

 \item $A^c\cup B^c = (A\cap B)^c$. \points{3}

\vspace{2.5in}

 \item $(A\cup B)\setminus B = A\setminus (A\cap B)$ \points{3}

\vspace{2.5in}

 \item $A\cap (B\cup B) = (A\cap B)\cup (A\cap C)$. \points{4}
\end{enumerate}
\newpage

\item For each $n\in\N$, let $A_n = \{k\in \Z : -n\leq k\leq n\}$.
\begin{enumerate}
 \item Compute $\di \bigcup_{n=1}^4 A_n$. \points{2}

\vspace{1.25in}

 \item Compute $\di \bigcap_{n=3}^6 A_n$. \points{2}

\vspace{1.25in}

 \item What are $\di \bigcup_{n=1}^\infty A_n$ and $\di \bigcap_{n=1}^\infty A_n$? \points{2}

\vspace{1.5in}

 \item Let $\mathcal{A} = \{A_\alpha : \alpha\in I\}$ be any indexed collection of sets (where the index set $I$ is  non-empty). Prove that if $B\subseteq A_\alpha$ for all $\alpha\in I$, then $B\subseteq \bigcap_{\alpha\in I}A_\alpha$. \points{4}
\end{enumerate}
\newpage

\item Let $f:A\to B$ be a function.
\begin{enumerate}
 \item Given an element $b\in B$, define the preimage $f^{-1}(b)$. \points{2}

\vspace{1in}

 \item Given a subset $C\subseteq B$, define the preimage $f^{-1}(C)$. \points{1}

\vspace{1in}

\end{enumerate}
\item Let $f:A\to B$ and $g:B\to C$ be functions.
\begin{enumerate}
 \item Prove that if $g\circ f:A\to C$ is an injection, then $f$ is an injection.  \points{5}

\vspace{3in}

 \item Give an example to show that $g$ need not be an injection.\points{2}
\end{enumerate}
\newpage

\item Determine if each of the following statements is true. If it is true, give a direct proof of the statement. If it is false, give a counterexample.
\begin{enumerate}
 \item For all integers $a,b,c$, with $a\neq 0$, if $a|b$ then $a|(bc)$. \points{3}

\vspace{2.5in}

 \item For all integers $a$ and $b$ with $a\neq 0$, if $6|(ab)$, then $6|a$ or $6|b$. \points{2}

\vspace{2in}

 \item The function $f:\Z_5\to \Z_5$ given by $f(x) = x^2+3 \pmod{5}$ is a bijection, where $\Z_5=\{0,1,2,3,4\}$. \points{5}
\end{enumerate}
\newpage

\item Prove the following biconditional statement: For any integer $n$, $n$ is odd if and only if $n^2$ is odd. \points{4}

\vspace{2.5in}

\item Use mathematical induction to prove that for all natural numbers $n$,\points{6}
\[
 1+5+9+\cdots + (4n-3) = n(2n-1).
\]
\newpage

\item Prove that for any real numbers $x$ and $y$, if $x$ is rational and $y$ is irrational, then $x+y$ is irrational. (Hint: proof by contradiction.) \points{5}




\vspace{3.5in}

\item Prove that $8|(n^2-1)$ for all {\bf odd} natural numbers $n$.\points{5}

({\em Hint:} a proof by induction is possible, but a proof by cases is easier.)

\newpage

\item Let $A=\{1,2,3\}$. For each of the relations $R\subseteq A\times A$ on $A$ below, determine if $R$ is a function, an equivalence relation, or neither.
\begin{enumerate}
 \item $R=\{(1,1),(2,1), (3,1), (2,3), (3,3)\}$ \points{3}

\vspace{1.75in}

 \item $R=\{(1,1), (1,2), (2,1), (2,2), (3,3)\}$ \points{3}

\vspace{1.75in}

 \item $R=\{(1,1), (2,3), (3,3)\}$ \points{3}
\end{enumerate}
\vspace{1.5in}

 \item Give an example of a relation $R$ on a nonempty set $A$ such that $R$ is both a function and an equivalence relation, or explain why such a relation cannot exist. \points{1}


\newpage

\item Determine the addition and multiplication tables for modular arithmetic on $\Z_3 = \{[0],[1],[2]\}$. \points{4}

\vspace{4in}

\item \begin{enumerate}
       \item Prove the following: For all $[a],[b]\in \Z_3$, if $[a]^2\oplus [b]^2=[0]$, then $[a]=[0]$ and $[b]=[0]$. \points{4}

\vspace{2in}

 \item Explain why it follows from Part (a) that for all $a,b\in\Z$, if 3 divides $a^2+b^2$, then 3 divides $a$ and 3 divides $b$. \points{2}
      \end{enumerate}

\newpage

\item \begin{enumerate}
       \item Define what it means for two sets to be {\bf equivalent}. \points{1}

\vspace{1.25in}

\item Define what it means for a set to be {\bf finite}. \points{1}

\vspace{1.25in}

\item Suppose $f:A\to B$ is an injection. Prove that there exists a subset $C\subseteq B$ such that $A$ is equivalent to $C$. \points{4}

\vspace{3in}

\item For each subset of $\R$ below, indicate whether the set is finite, countably infinite, or uncountable: \points{4}
\begin{itemize}
 \item $A = \{x\in\R : x^2 = 1\}$.

\bigskip

 \item $B = \{x\in\R : x^2\leq 1\}$.

\bigskip

 \item $C = \{1, 1/2, 1/3, 1/4, 1/5, \ldots\}$.

\bigskip

 \item $\mathbb{Q}$ (the set of rational numbers).
\end{itemize}

      \end{enumerate}

\end{enumerate}
\newpage

{\bf Extra space for rough work. Do not remove unless there is nothing on this page you want to be graded.}
\end{document}