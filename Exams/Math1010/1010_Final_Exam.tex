\documentclass[12pt]{article}
\usepackage{amsmath}
\usepackage{amssymb}
\usepackage[letterpaper,margin=0.85in,centering]{geometry}
\usepackage{fancyhdr}
\usepackage{enumerate}
\usepackage{lastpage}
\usepackage{multicol}
\usepackage{graphicx}

\reversemarginpar

\pagestyle{fancy}
\cfoot{Page \thepage \ of \pageref{LastPage}}\rfoot{{\bf Total Points: 100}}
\chead{MATH 1010}\lhead{FINAL EXAM}\rhead{December 18, 2015}

\newcommand{\points}[1]{\marginpar{\hspace{24pt}[#1]}}
\newcommand{\skipline}{\vspace{12pt}}
%\renewcommand{\headrulewidth}{0in}
\headheight 30pt

\newcommand{\di}{\displaystyle}
\newcommand{\abs}[1]{\lvert #1\rvert}
\newcommand{\R}{\mathbb{R}}
\newcommand{\C}{\mathbb{C}}
\renewcommand{\P}{\mathcal{P}}
\DeclareMathOperator{\nul}{null}
\DeclareMathOperator{\range}{range}
\DeclareMathOperator{\spn}{span}
\newcommand{\len}[1]{\lVert #1\rVert}
\newcommand{\Q}{\mathbb{Q}}
\newcommand{\N}{\mathbb{N}}
\renewcommand{\L}{\mathcal{L}}
\newcommand{\dotp}{\boldsymbol{\cdot}}
\newcommand{\ds}{\displaystyle}

\begin{document}

\author{Instructor: Sean Fitzpatrick}
\thispagestyle{plain}
\begin{center}
\emph{University of Lethbridge}\\
Department of Mathematics and Computer Science\\
December 18th, 2015, 6:00 - 9:00 pm\\
{\bf MATH 1010 - FINAL EXAM}\\
\end{center}
\skipline \skipline \skipline \noindent \skipline
Last Name:\underline{\hspace{350pt}}\\
\skipline
First Name:\underline{\hspace{348pt}}\\
\skipline
Student Number:\underline{\hspace{322pt}}\\
\skipline

\vspace{0.5in}


\begin{quote}

 
 Record your answers below each question in the space provided.    Left-hand pages may be used as scrap paper for rough work.  If you want any work on the left-hand pages to be graded, please indicate so on the right-hand page.
 
 \bigskip
 
Partial credit will be awarded for partially correct work, so be sure to show your work, and include all necessary justifications needed to support your arguments. 

\bigskip

\end{quote}


\vspace{0.25in}

For grader's use only:

\begin{table}[hbt]
\begin{center}
\begin{tabular}{|l|r|} \hline
Page&Grade\\
\hline \hline
\cline{1-2} 2 & \enspace\enspace\enspace\enspace\enspace\enspace/10\\
\cline{1-2} 3 & \enspace\enspace\enspace\enspace\enspace\enspace/10\\
\cline{1-2} 4 & \enspace\enspace\enspace\enspace\enspace\enspace/10\\
\cline{1-2} 5 & \enspace\enspace\enspace\enspace\enspace\enspace/10\\
\cline{1-2} 6 & \enspace\enspace\enspace\enspace\enspace\enspace/10\\
\cline{1-2} 7 & \enspace\enspace\enspace\enspace\enspace\enspace/10\\
\cline{1-2} 8 & \enspace\enspace\enspace\enspace\enspace\enspace/10\\
\cline{1-2} 9-10 & \enspace\enspace\enspace\enspace\enspace\enspace/15\\
\cline{1-2} 11-12 & \enspace\enspace\enspace\enspace\enspace\enspace/15\\
\cline{1-2} Total & \enspace\enspace\enspace\enspace\enspace\enspace/100\\
\hline
\end{tabular}


\end{center}
\end{table}
\newpage


\begin{enumerate}
 \item Evaluate the following limits:
\begin{enumerate}
\item $\ds \lim_{x\to 0}\frac{99x^{11}-23x^6+1010}{5x^4-7x^2+141x+1}$ \points{2}

\vspace{1.25in} 

\item $\ds \lim_{x\to 2}\frac{x^2-2x}{x^2-5x+6}$ \points{2}

\vspace{1.75in}

 \item $\ds \lim_{x\to 0}\frac{\sin 3x}{x}$ \points{3}

\vspace{2in}

 \item $\ds \lim_{x\to -\infty}\frac{\sqrt{3x^2+4}}{x}$ \points{3}
\end{enumerate}
\newpage

\item Determine the set of all $x$ such that the function \points{5}
\[
 f(x) = \begin{cases}
         2\sin x, & \text{ if } x\leq 0\\
         x^2-5, & \text{ if } 0<x\leq 2\\
         3x-7, & \text{ if } x>2
        \end{cases}
\]
is continuous. Express your answer in interval notation.

\vspace{3.5in}

\item Using only the {\bf definition of the derivative}, compute $f'(2)$ if $f(x) = \dfrac{1}{x-1}$. \points{5}


\newpage

\item Compute the derivatives of the following functions:
\begin{enumerate}
 \item $f(x) = 2x^7-3x^3+\dfrac{4}{x}+\sqrt{2}$ \points{2}

\vspace{1in}

 \item $g(x) = \tan (x)\cos(x)$ \points{2}

\vspace{1in}

 \item $h(x) = \dfrac{2x^4-3x}{x^2}$ \points{3}

\vspace{1.75in}

\end{enumerate}

 \item Suppose you know that $f(3)=4$, $g(3)=-2$, $f'(3) = \frac{4}{3}$, and $g'(3) = \frac{1}{2}$ for two functions $f$ and $g$.\\
 If $h(x) = f(x)g(x)$, what is the equation of the tangent line to the curve $y=h(x)$ when $x=3$?\points{3}

\newpage

\item Find the global (absolute) maximum and minimum values of $f(x) = x^3-3x$ on the interval $[0,2]$. \points{5}

\vspace{4in}

\item Find and classify the critical points of the function $g(x) = x^{5/3}-5x^{2/3}$. \points{5}

\newpage

\item Solve the following inequalities. Express your answers in interval notation.
\begin{enumerate}
 \item $\lvert 3x-5\rvert \leq 7$. \points{3}

\vspace{2in}

 \item $x^2<12-x$. \points{3}
 
 \vspace{2.5in}
 
 \item $1\leq \dfrac{6}{x^2+x}$. \points{4}
\end{enumerate}
\newpage



\item Find the exact values of the trigonometric functions below at the given angles:
\begin{enumerate}
 \item $\cos (11\pi/3)$ \points{2}

\vspace{1.5in}

 \item $\sin (43\pi/4)$ \points{2}

\vspace{1.5in}

 \item $\cot (-23\pi/6)$ \points{3}

\vspace{2in}

 \item $\cos(\theta)$, if the angle $\theta$ lies in Quadrant II and $\sin(\theta) = 4/5$. \points{3}
\end{enumerate}
\newpage

\item \begin{enumerate}
\item Show that $\dfrac{d}{dx}(\tan (x)) = \sec^2 (x)$, or that $\dfrac{d}{dx}(\sec(x)) = \sec(x)\tan(x)$.\\
\hskip-60pt(Do {\bf only one} of the two. The derivatives of $\sin (x)$ and $\cos (x)$ are given on the last page.)\points{3}

\vspace{2.5in}

\item Show that $\dfrac{1}{1-\sin (x)} = \sec^2(x)+\sec(x)\tan(x)$.\points{4}

\vspace{2.5in}

\item Determine the function $f(x)$ such that $f'(x) = \dfrac{1}{1-\sin(x)}$ and $f(\pi) = 3$. \points{3}
\end{enumerate}
\newpage

\item Consider the function $f(x) = x^3-12x+16$.
\begin{enumerate}
 \item Given that $a=2$ is a zero of multiplicity 2, find the remaining real zero of $f$. \points{3}

\vspace{3in}

 \item Construct the sign diagram for $f$. \points{1}

\vspace{1.25in}

 \item Calculate $f'(x)$. \points{2}
 
\vspace{1.5in}

 \item Construct the sign diagram for $f'$. \points{2} 
\newpage

\item Calculate $f''(x)$, and construct the sign diagram for $f''$. \points{2}

\vspace{2in}

 \item Sketch the graph of $f(x)$. Your sketch should be clearly labelled with the following information:\\
 $x$ and $y$-intercepts, any local maxima or minima, and any inflection points.\points{5}\\
{\bf Note:} Your vertical scale does not need to match your horizontal scale.
\end{enumerate}
\newpage

\item Consider the function $f(x) = \dfrac{x}{x^2-2x+1}$.
\begin{enumerate}
 \item What is the domain of $f$? \points{1}
\vspace{2in} 

 \item Construct a sign diagram for $f$. \points{1}

\vspace{2in}

 \item Determine any $x$ or $y$-intercepts for the graph of $f$. \points{1}

\vspace{1.5in}

 \item Determine any horizontal or vertical asymptotes for the graph of $f$. \points{2}

\newpage

 \item Calculate $f'(x)$. \points{3}

\vspace{2in}

 \item Construct the sign diagram for $f'$, and determine any local maxima or minima.\points{2}

\vspace{2in}


 \item Sketch the graph of $f$. \points{5}
\end{enumerate}




\end{enumerate}
\newpage

\begin{center}
 \textbf{Some true stuff from the course that you possibly didn't remember}
\end{center}
\begin{itemize}
 \item We say $x=a$ is a zero of {\bf multiplicity} $k$ for a polynomial $p(x)$ if $(x-a)^k$ is a factor of $p(x)$, but $(x-a)^{k+1}$ is not.
 \item The {\bf Factor Theorem} states that for a polynomial function $p(x)$, $p(a)=0$ if and only if $(x-a)$ is a factor of $p$.
 \item Values of $\sin\theta$ and $\cos\theta$ in the first quadrant:
\begin{align*}
 \sin 0 &= 0  \quad& \quad \cos 0 &= 1\\
 \sin \pi/6 &= 1/2   \quad& \quad\cos \pi/6 &= \sqrt{3}/2\\
 \sin \pi/4 &= \sqrt{2}/2  \quad &\quad \cos \pi/4& = \sqrt{2}/2\\
 \sin \pi/3 &= \sqrt{3}/2   \quad &\quad\cos \pi/3& = 1/2\\
 \sin \pi/2 &= 1  \quad &\quad \cos \pi/2 &= 0
\end{align*}

\item Fundamental identities:
\begin{enumerate}
 \item $\tan\theta = \dfrac{\sin\theta}{\cos\theta}$, $\cot\theta = \dfrac{\cos\theta}{\sin\theta}$, $\sec\theta = \dfrac{1}{\cos\theta}$, $\csc\theta = \dfrac{1}{\sin\theta}$
 \item $\cos^2\theta + \sin^2\theta =1$
 \item $\cos(\alpha + \beta) = \cos\alpha\cos\beta - \sin\alpha\sin\beta$
 \item $\cos(\alpha - \beta) = \cos\alpha\cos\beta + \sin\alpha\sin\beta$
 \item $\sin(\alpha + \beta) = \sin\alpha\cos\beta + \cos\alpha\sin\beta$
 \item $\sin(\alpha - \beta) = \sin\alpha\cos\beta - \cos\alpha\sin\beta$
\end{enumerate}
\item Obvious but occasionally forgotten facts that are sometimes useful in conjunction with some of the identities above:
\begin{enumerate}
 \item $2\theta = \theta + \theta$
 \item $\theta = \dfrac{\theta}{2}+\dfrac{\theta}{2}$
\end{enumerate}
 \item $\lim_{\theta\to 0}\dfrac{\sin\theta}{\theta} = 1$.
 \item $\ds f'(a) = \lim_{h\to 0}\frac{f(a+h)-f(a)}{h}$.
 \item $\dfrac{d}{dx}(x^n) = nx^{n-1}$, $\dfrac{d}{dx}(\sin x) = \cos x$, and $\dfrac{d}{dx}(\cos x) = -\sin x$.
 \item $(fg)' = f'g+fg'$ and $\left(\dfrac{f}{g}\right)' = \dfrac{f'g-fg'}{g^2}$.
 \item A {\bf critical point} for a function $f$ is a number $c$ such that $f'(c)=0$ (or doesn't exist); $f(c)$ is the corresponding critical value.
 \item If you spend all your time reading this page, you won't have time to complete the exam.
\end{itemize}




\end{document}