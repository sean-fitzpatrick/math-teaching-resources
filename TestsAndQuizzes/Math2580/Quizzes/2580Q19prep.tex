\documentclass[letterpaper,12pt]{article}

%\usepackage{ucs}
%\usepackage[utf8x]{inputenc}
\usepackage{amsmath}
\usepackage{amsfonts}
\usepackage{amssymb}
\usepackage{graphicx}
%\usepackage[canadian]{babel}
\usepackage[margin=1in]{geometry}
\usepackage{multicol}
\newcommand{\R}{\mathbb{R}}
\renewcommand{\i}{\mathbf{i}}
\renewcommand{\j}{\mathbf{j}}
\renewcommand{\k}{\mathbf{k}}
\newcommand{\pd}[2]{\dfrac{\partial #1}{\partial #2}}
\newcommand{\di}{\displaystyle}
\renewcommand{\r}{\mathbf{r}}
\newcommand{\len}[1]{\left\lVert #1\right\rVert}
\newcommand{\dotp}{\boldsymbol{\cdot}}
\newcommand{\F}{\mathbf{F}}

\title{Practice for Quiz 19\\Math 2580\\Spring 2016}
\author{Sean Fitzpatrick}
\date{March 29th, 2016}

\begin{document}
 \maketitle

If you can answer the following problems, you should be well-prepared for Quiz 19:



\begin{enumerate}
 \item Find a function $f$ such that $\nabla f = \F$, and use this to evaluate $\int_C \F\dotp d\r$ for the given curve:
\begin{enumerate}
 \item $\F(x,y) = \dfrac{y^2}{1+x^2}\i + 2y\arctan x\j$, $C$ parameterized by $\r(t) = t^2\i+2t\j$, $0\leq t\leq 1$.
 \item $\F(x,y,z) = (2xz+y^2)\i+2xy\j + (x^2+3z^2)\k$, $C$ parameterized by $x=t^2, y=t+1, z=2t-1$, $0\leq t\leq 2$.
 \item $\F(x,y,z) = \langle e^y, xe^y, (z+1)e^z\rangle$, $C$ parameterized by $\r(t) = \langle t, t^2, t^3\rangle$, $t\in [0,1]$.
\end{enumerate}
 \item Calculate the curl of the given vector field:
\begin{enumerate}
 \item $\F(x,y,z) = \langle 2xy, xz, y^2z\rangle$
 \item $\F(x,y,z) = \langle y\cos xy, x\cos xy, -\sin z\rangle$
\end{enumerate}
 \item Verify that Green's Theorem holds for the following line integrals in the plane:
\begin{enumerate}
 \item $\int_C xy^2\,dx+x^3\,dy$, where $C$ is the rectangle with corners at $(0,0), (2,0), (2,3)$, and $(0,3)$.
 \item $\int_C y\,dx-x\,dy$, where $C$ is the unit circle.
 \item $\int_C x\,dx+y\,dy$, where $C$ consists of the line segments from $(0,1)$ to $(0,0)$, and from $(0,0)$ to $(1,0)$, and the portion of the parabola $y=1-x^2$ from $(1,0)$ to $(0,1)$.
\end{enumerate}

\end{enumerate}




\end{document}
 
