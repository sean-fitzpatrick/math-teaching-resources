\documentclass[letterpaper,12pt]{article}

%\usepackage{ucs}
%\usepackage[utf8x]{inputenc}
\usepackage{amsmath}
\usepackage{amsfonts}
\usepackage{amssymb}
%\usepackage[canadian]{babel}
\usepackage[margin=1in]{geometry}
\newcommand{\R}{\mathbb{R}}
\renewcommand{\i}{\mathbf{i}}
\renewcommand{\j}{\mathbf{j}}
\renewcommand{\k}{\mathbf{k}}
\newcommand{\pd}[2]{\dfrac{\partial #1}{\partial #2}}

\title{Solutions to Quiz 7 Practice Problems\\Math 2580\\Spring 2016}
\author{Sean Fitzpatrick}
\date{February 2nd, 2016}

\begin{document}
 \maketitle

\begin{enumerate}
 \item Find the equation of the tangent plane to the surface $x^2+2y^2+3xz=10$ at the point $(1,2,\frac{1}{3})$.

\bigskip

Since the surface is given as a level surface $f(x,y,z)=10$, where $f(x,y,z) = x^2+2y^2+3xz$, we evalate the gradient of $f$ at the given point to obtain the normal vector:
\[
 \nabla f(x,y,z) = \langle 2x+3z, 4y, 3x\rangle, \quad \text{ so } \quad \nabla f(1,2,\frac{1}{3}) = \langle 3, 8, 3\rangle.
\]
The equation of the tangent plane is therefore $3(x-1)+8(y-2)+3(z-\frac{1}{3})=0$.

\bigskip

 \item On Assignment 3 you're asked to derive the following formula: if $y=g(x)$ is a function satisfying the relation $F(x,y)=C$ for some constant $C$, then
\[
 \frac{dy}{dx} = g'(x) = -\frac{F_x(x,g(x))}{F_y(x,g(x))}.
\]
Use this result to find the slope of the tangent line to the curve $x^2+y^4=5$ at the point $(2,1)$.

\bigskip

We first confirm that $2^1+1^4=5$, so the point $(2,1)$ is indeed on the curve. With $F(x,y)=x^2+y^4$, we have $F_x(x,y) = 2x$ and $F_y(x,y)=4y^3$, so $F_x(2,1)=4$ and $F_y(2,1)=4$. The slope of the tangent line is therefore given by $m=g'(1) = -\dfrac{4}{4} = -1$, and the equation of the line is $y-1=-(x-2)$.

\bigskip

 \item Let $F(x,y,z) = xy^2-x^2z+2yz^2$, and suppose $z=g(x,y)$ satisfies the relation $F(x,y,z)=1$. Use implicit differentiation to compute $g_x(1,1)$ and $g_y(1,1)$.

\bigskip

Suppose that $xy^2-x^2z+2yz^2=1$ defines $z$ as a function of $x$ and $y$. If $x=1$ and $y=1$, the equation of our surface gives us $1-z+2z^2=1$, so $2z^2-z=z(2z-1)=0$. Thus, there are two points on the surface with $x=1$ and $y=1$: either $(1,1,0)$ or $(1,1,1/2)$. Let's write $z=g(x,y)$ for the implicit function satisfying $g(1,1)=0$, and $z=h(x,y)$ for the implicit function satisfying $h(1,1)=1/2$. Taking the derivative of both sides of the equation of the surface with respect to $x$, we have
\[
 y^2-2xz-x^2\pd{z}{x}+4yz\pd{z}{x}=0.
\]
Solving for $\pd{z}{x}$, we have $\pd{z}{x} = \dfrac{2xz-y^2}{4yz-x^2}$. (Notice that we have $\pd{z}{x} = -\dfrac{F_x(x,y,z)}{F_z(x,y,z)}$.) For the point $(1,1,0)$ we have
\[
 g_x(1,1) = \dfrac{2(1)(0)-1^2}{4(1)(0)-1^2} = \dfrac{-1}{-1} = 1,
\]
and for the point $(1,1,1/2)$ we have
\[
 h_x(1,1) = \dfrac{2(1)(1/2)-1^2}{4(1)(1/2)-1^2} = \dfrac{1-1}{1} = 0.
\]
If we take the derivative of both sides of the equation of the surface with respect to $y$, we have
\[
 2xy-x^2\pd{z}{y}+2z^2+4yz\pd{z}{y} = 0.
\]
Solving for $\pd{z}{y}$, we have $\pd{z}{y} = \dfrac{-2xy-2z^2}{4yz-x^2} = -\dfrac{F_y(x,y,z)}{F_z(x,y,z)}$. For the point $(1,1,0)$ we have
\[
 g_y(1,1) = \dfrac{-2-0}{0-1} = \dfrac{-2}{-1} = 2,
\]
and for the point $(1,1,1/2)$ we have
\[
 h_y(1,1) = \dfrac{-2-2(1/4)}{4(1)(1/2)-1^2} = \dfrac{-5/2}{1} = -\dfrac{5}{2}.
\]


 \item Suppose $\vec{n} = \langle a,b,c\rangle$ is the normal vector for the tangent plane at a point on surface in $\R^3$. What can you say about the values of $a$, $b$, and $c$ if the plane is
\begin{enumerate}
 \item Horizontal?

\medskip

If the tangent plane is horizontal, the normal vector must be vertical, so $\vec{n}$ must be a scalar multiple of $\k = \langle 0,0,1\rangle$. Thus we must have $a=b=0$.

\medskip

 \item Vertical?

\medskip

If the tangent plane is vertical, the normal vector must be horizontal. This amounts to saying that $\vec{n}$ must be parallel to the $xy$-plane, which means we must have $c=0$.

\medskip

\end{enumerate}
 \item Find any points $(a,b)$ at which the tangent plane to the surface $z=x^2-2x+y^2$ is horizontal.

\bigskip

Letting $f(x,y)=x^2-2x+y^2$, we know that the normal vector to the surface $z=f(x,y)$ when $(x,y)=(a,b)$ is given by $\vec{n} = \langle f_x(a,b), f_y(a,b), -1\rangle$. By the previous problem, the tangent plane will therefore be horizontal if $f_x(a,b)=f_y(a,b)=0$. We have
\[
 f_x(x,y) = 2x-2=2(x-1) \quad \text{ and } \quad f_y(x,y) = 2y,
\]
from which we see that we must have $x=1$ and $y=0$. Thus, the tangent plane is horizontal at the point $(1,0)$. (Technically we should say it's the point $(1,0,-1)$, giving the $z$-coordinate, since we're talking about a point on a surface in $\R^3$ but often it's convenient to just give the values of the independent variables.)
\end{enumerate}

\end{document}
 
