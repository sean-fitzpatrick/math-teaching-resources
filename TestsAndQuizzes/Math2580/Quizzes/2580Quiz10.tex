\documentclass[12pt]{article}
\usepackage{amsmath}
\usepackage{amssymb}
\usepackage[letterpaper,margin=0.85in,centering]{geometry}
\usepackage{fancyhdr}
\usepackage{enumerate}
\usepackage{lastpage}
\usepackage{multicol}
\usepackage{graphicx}

\reversemarginpar

\pagestyle{fancy}
\cfoot{}
\lhead{Math 2580A}\chead{Quiz \# 10}\rhead{Tuesday, 23\textsuperscript{rd} January, 2016}
%\rfoot{Total: 10 points}
%\chead{{\bf Name:}}
\newcommand{\points}[1]{\marginpar{\hspace{24pt}[#1]}}
\newcommand{\skipline}{\vspace{12pt}}
%\renewcommand{\headrulewidth}{0in}
\headheight 30pt

\newcommand{\di}{\displaystyle}

\renewcommand{\i}{\mathbf{i}}
\renewcommand{\j}{\mathbf{j}}
\renewcommand{\k}{\mathbf{k}}
\newcommand{\R}{\mathbb{R}}
\newcommand{\aaa}{\mathbf{a}}
\newcommand{\bbb}{\mathbf{b}}
\newcommand{\ccc}{\mathbf{c}}
\newcommand{\dotp}{\boldsymbol{\cdot}}
\begin{document}
{\bf \Large Name:}
%\author{Instructor: Sean Fitzpatrick}
\thispagestyle{fancy}
%\noindent{{\bf Name and student number:}}

\bigskip

\begin{enumerate}

 \item Express the region bounded by the curves $y=\sqrt{x}$, $y=0$, and $x=4$ in two ways (you may find it helpful to sketch the region):
\begin{enumerate}
 \item Via inequalities of the form $a\leq x\leq b$, $f_1(x)\leq y\leq f_2(x)$.

\vspace{4in}

 \item Via inequalities of the form $c\leq y\leq d$, $g_1(y)\leq x\leq g_2(y)$.
\end{enumerate}


\end{enumerate}
\end{document}