\documentclass[letterpaper,12pt]{article}

%\usepackage{ucs}
%\usepackage[utf8x]{inputenc}
\usepackage{amsmath}
\usepackage{amsfonts}
\usepackage{amssymb}
\usepackage{graphicx}
%\usepackage[canadian]{babel}
\usepackage[margin=1in]{geometry}
\usepackage{multicol}
\newcommand{\R}{\mathbb{R}}
\renewcommand{\i}{\mathbf{i}}
\renewcommand{\j}{\mathbf{j}}
\renewcommand{\k}{\mathbf{k}}
\newcommand{\pd}[2]{\dfrac{\partial #1}{\partial #2}}
\newcommand{\di}{\displaystyle}
\renewcommand{\r}{\mathbf{r}}
\newcommand{\len}[1]{\left\lVert #1\right\rVert}
\newcommand{\dotp}{\boldsymbol{\cdot}}
\newcommand{\F}{\mathbf{F}}
\renewcommand{\S}{\mathbf{S}}
\newcommand{\N}{\mathbf{N}}

\title{Solutions to Quiz 24 Practice Problems\\Math 2580\\Spring 2016}
\author{Sean Fitzpatrick}
\date{April 12th, 2016}

\begin{document}
 \maketitle

\begin{enumerate}
 \item Use Stokes' theorem to evaluate $\iint_S(\nabla\times \F)\dotp d\mathbf{S}$, where $\F(x,y,z)=\langle yz, xz, xy\rangle$, and $S$ is the part of the paraboloid $z=9-x^2-y^2$ that lies above the plane $z=5$, oriented upward.

\bigskip

The boundary $C$ of $S$ is the circle $x^2+y^2=4$ in the plane $z=5$, oriented in the counter-clockwise direction, as seen from above. To use Stokes' theorem directly, we paramterize the circle using $\r(t) = \langle 2\cos t, 2\sin t, 5\rangle$, with $t\in [0,2\pi]$. We then have
\begin{align*}
 \iint_S(\nabla \times \F)\dotp d\S & = \int_C \F\dotp \,d\r\\
& = \int_0^{2\pi} \langle 10\sin t, 10\cos t, 4\sin t\cos t\rangle\dotp \langle -2\sin t, 2\cos t, 0\rangle \,dt\\
& = \int_0^{2\pi} -20\sin^2 t+20\cos^2t\rangle\,dt\\
& = 20\int_0^{2\pi}\cos 2t\,dt = 0.
\end{align*}

Of course, if we actually compute the curl, we immediately have $\nabla\times\F = \mathbf{0}$, so the result above isn't particularly surprising. (If the curl had been nonzero, another option would be to integrate $\nabla\times\F$ over the disc $D$ in the plane $z=5$ given by $x^2+y^2\leq 4$.)



 \item Evaluate $\iint_S(\nabla\times \F)\dotp d\mathbf{S}$, directly, where $\F(x,y,z)=\langle yz, xz, xy\rangle$, and $S$ is the disk $x^2+y^2\leq 4$ in the plane $z=5$.

\bigskip

As noted above, $\nabla \times \F=\mathbf{0}$, so the answer is immediately 0. 

 \item How are Problems 1 and 2 related?

\bigskip

This ended up being a bit boring, due to the curl being zero, but the point was that the both surfaces share the same boundary, so the integral of $\nabla\times \F$ over either surface should be the same.

 \item Use Stokes' theorem to evaluate $\int_C\F\dotp \,d\r$, where $\F(x,y,z) = \langle xy, 2x, 3y\rangle$, and $C$ is the curve of intersection of the plane $x+z=5$ and the cylinder $x^2+y^2=9$.

\bigskip

Using Stokes' theorem, we need to compute the integral $\iint_S (\nabla\times \F)\dotp d\S$, where $S$ is any surface with boundary curve $C$. We first compute the curl:
\[
 \nabla \times \F = \begin{vmatrix}\i&\j&\k\\ \pd{}{x}&\pd{}{y}&\pd{}{z}\\xy& 2x& 3y\end{vmatrix} = \langle 3, 0, 2-x\rangle.
\]
The simplest surface $S$ to use is the portion of the plane $x+z=1$ that lies within the cylinder $x^2+y^2=9$. Treating the plane as the graph $z=1-x$, we can use the parameterization
\[
 \r(x,y) = \langle x, y, 1-x\rangle, \quad x^2+y^2\leq 9.
\]
The normal vector in this case is found to be $\N(x,y) = \langle 1, 0, 1\rangle$, so we have
\[
 \int_C\F\dotp\,d\r = \iint_S(\nabla\times\F)\dotp d\S = \iint_D\langle 3, 0, 2-x\rangle\dotp \langle 1, 0, 1\rangle\,dA = \iint_D(5-x)\,dA,
\]
where $D$ is the disc $x^2+y^2\leq 9$. By symmetry, the $-x$ term does not contribute to the integral, so the result is simply 5 times the area of the disc, or $45\pi$.

 \item Use Stokes' theorem to show that if $\F$ is $C^1$ vector field defined on all of $\R^3$ such that $\nabla\times \F = \mathbf{0}$, then $\F$ is conservative.

\bigskip

If $\F$ is $C^1$ on $\R^3$, then $\nabla\times \F$ is defined and continuous on all of $\R^3$. If $C$ is any simple, closed curve, we can choose a surface $S$ with $\partial S=C$, and then by Stokes' theorem,
\[
 \int_C \F\dotp\,d\r = \iint_S(\nabla\times \F)\dotp\,d\S = 0.
\]
By a previous result, since the integral of $\F$ around any closed curve is 0, we can conclude that $\F$ is conservative.

 \item Use the Divergence theorem to evaluate $\iint_S\F\dotp d\mathbf{S}$ for the vector field $\F(x,y,z) = \langle x^4,-x^3z^2, 4xy^2z\rangle$, where $S$ is the boundary of the region bounded by the cylinder $x^2+y^2=1$ and the planes $z=0$ and $z=x+2$.

\bigskip

We compute $\nabla\dotp\F(x,y,z) = \pd{}{x}(x^4)+\pd{}{y}(-x^3z^2)+\pd{}{z}(4xy^2z) = 4x^3+4xy^2 = 4x(x^2+y^2)$. The region $E$ bounded by $S$ is given in cylindrical coordinates by $0\leq z\leq 2+r\cos\theta$, where $0\leq r\leq 1$ and $0\leq \theta \leq 2\pi$. The Divergence theorem thus gives us
\[
 \iint_S \F\dotp d\S = \iiint_E (\nabla\dotp\F)dV = \int_0^{2\pi}\int_0^1\int_0^{2+r\cos\theta}4r\cos\theta(r^2)r\,dz\,dr\,d\theta = \frac{16\pi}{5}.
\]
 
 \item Use the Divergence theorem to evaluate $\iint_S(2x+2y+z^2)\,dS$, where $S$ is the sphere $x^2+y^2+z^2=1$.

\bigskip

On the unit sphere, we know that $\mathbf{n} = \langle x, y, z\rangle$ is the unit normal vector. Note also that
\[
 2x+2y+z^2 = \langle 2, 2, z\rangle\dotp \langle x, y, z\rangle = \F\dotp \mathbf{n},
\]
where $\F(x,y,z) = \langle 2, 2, z\rangle$. The Divergence theorem thus gives us
\[
 \iint_S(2x+2y+z^2)\,dS = \iiint_E (\nabla\dotp\F)\,dV = \iiint_E (1) \,dV = \frac{4\pi}{3},
\]
where $E$ is the ball $x^2+y^2+z^2\leq 1$, which has volume $\frac{4}{3}\pi(1)^3$.
\end{enumerate}




\end{document}
 
