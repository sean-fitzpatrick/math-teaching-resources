\documentclass[letterpaper,12pt]{article}

%\usepackage{ucs}
%\usepackage[utf8x]{inputenc}
\usepackage{amsmath}
\usepackage{amsfonts}
\usepackage{amssymb}
\usepackage{graphicx}
%\usepackage[canadian]{babel}
\usepackage[margin=1in]{geometry}
\usepackage{multicol}
\newcommand{\R}{\mathbb{R}}
\renewcommand{\i}{\mathbf{i}}
\renewcommand{\j}{\mathbf{j}}
\renewcommand{\k}{\mathbf{k}}
\newcommand{\pd}[2]{\dfrac{\partial #1}{\partial #2}}
\newcommand{\di}{\displaystyle}

\title{Practice for Quiz 13\\Math 2580\\Spring 2016}
\author{Sean Fitzpatrick}
\date{March 3rd, 2016}

\begin{document}
 \maketitle

If you can answer the following problems, you should be well-prepared for Quiz 13:



\begin{enumerate}
 \item Evaluate $\di \iint_D(x^2+y^2)^{3/2}\,dA$, where $D$ is the disc $x^2+y^2\leq 4$, using polar coordinates.

\bigskip

In polar coordinates we have $(x^2+y^2)^{3/2} = (r^2)^{3/2}=r^3$, and the disc is given by $0\leq r\leq 2$, and $0\leq \theta\leq 2\pi$. Therefore,
\[
 \iint_D(x^2+y^2)^{3/2}\,dA = \int_0^{2\pi}\int_0^2 r^3\cdot r \,dr\,d\theta = 2\pi\left(\frac{2^4}{4}\right) = 8\pi.
\]


 \item Evaluate $\di \int_{-1}^1\int_{-\sqrt{1-x^2}}^{\sqrt{1-x^2}}\sin(x^2+y^2)\,dy\,dx$ by converting to polar coordinates.

\bigskip

The bounds $-1\leq x\leq 1$, $-\sqrt{1-x^2}\leq y\leq \sqrt{1-x^2}$ describe the unit disc, so we have $0\leq r\leq 1$ and $0\leq \theta\leq 2\pi$, and thus
\[
 \int_{-1}^1\int_{-\sqrt{1-x^2}}^{\sqrt{1-x^2}}\sin(x^2+y^2)\,dy\,dx = \int_0^{2\pi}\int_0^1 \sin(r^2)r\,dr\,d\theta = \pi(1-\cos(1)).
\]

 \item Evaluate $\di \iiint_W (x^2+y^2+z^2)^{5/2}\,dV$, where $W$ is the ball $x^2+y^2+z^2\leq 1$.

\bigskip

This integral is most easily done in spherical coordinates: the unit ball is given by $0\leq \rho\leq 1$, with $0\leq \phi\leq \pi$ and $0\leq \theta\leq 2\pi$. Since $\rho^2=x^2+y^2+z^2$, we have
\[
 \iiint_W (x^2+y^2+z^2)^{5/2}\,dV = \int_0^{2\pi}\int_0^\pi\int_0^1\rho^5\cdot\rho^2\sin\phi\,d\rho\,d\phi\,d\theta = 4\pi\left[\frac{\rho^8}{8}\right]_0^1 = \frac{\pi}{2}.
\]
Note: the integral $\int_0^{2\pi}\int_0^\pi \sin\phi \,d\phi\,d\theta = 4\pi$ comes up often enough in spherical coordinates that you just end up remembering the value.

 \item Evaluate $\di \iiint_W \frac{1}{(x^2+y^2+z^2)^{3/2}}\,dV$, where $W$ is the solid bounded by the spheres $x^2+y^2+z^2=a^2$ and $x^2+y^2+z^2=b^2$, where $a,b>0$.

\bigskip

In spherical coordinates we have $a\leq\rho\leq b$, with $0\leq \phi\leq \pi$ and $0\leq \theta\leq 2\pi$, so we get
\[
 \iiint_W \frac{1}{(x^2+y^2+z^2)^{3/2}}\,dV = \int_0^{2\pi}\int_0^\pi\int_a^b \frac{1}{\rho^3}\rho^2\sin\phi \,d\rho\,d\phi\,d\theta  = 4\pi \ln\left(\frac{b}{a}\right).
\]

 \item Find the volume of the region enclosed by the cones $z=\sqrt{x^2+y^2}$ and $z=1-2\sqrt{x^2+y^2}$.

\bigskip

This volume is most easily computed in cylindrical coordinates. The cones $z=r$ and $z=1-2r$ intersect when $3r=1$, or $r=\dfrac{1}{3}$. We thus have $0\leq r\leq \dfrac{1}{3}$ and $0\leq \theta\leq 2\pi$, so the volume is given by
\[
 V = \int_0^{2\pi}\int_0^{1/3}[(1-2r)-r]r\,dr\,d\theta = \frac{\pi}{27}.
\]

 \item Find the average of $f(x,y)=e^{x+y}$ over the triangle $D$ with vertices at $(0,0)$, $(0,1)$, and $(1,0)$.

\bigskip

We treat the region $D$ as a Type 1 region, with $0\leq y\leq 1-x$, and $0\leq x\leq 1$. The area of the region is $A=\dfrac{1}{2}$, since it's a triangle with base 1 and height 1. The average of $f$ is thus
\[
 f_{av} = \frac{1}{A}\iint_D e^{x+y}\,dA = 2\int_0^1\int_0^{1-x}e^xe^y\,dy\,dx = 2\int_0^1 e^x(e^{1-x}-1)\,dx = 2\int_0^1 (e-e^{-x})\,dx = 4e-2.
\]

 \end{enumerate}

\end{document}
 
