\documentclass[12pt]{article}
\usepackage{amsmath}
\usepackage{amssymb}
\usepackage[letterpaper,margin=0.85in,centering]{geometry}
\usepackage{fancyhdr}
\usepackage{enumerate}
\usepackage{lastpage}
\usepackage{multicol}
\usepackage{graphicx}

\reversemarginpar

\pagestyle{fancy}
\cfoot{}
\lhead{Math 2580A}\chead{Quiz \# 13 Solutions}\rhead{Thursday, 3\textsuperscript{rd} March, 2016}
%\rfoot{Total: 10 points}
%\chead{{\bf Name:}}
\newcommand{\points}[1]{\marginpar{\hspace{24pt}[#1]}}
\newcommand{\skipline}{\vspace{12pt}}
%\renewcommand{\headrulewidth}{0in}
\headheight 30pt

\newcommand{\di}{\displaystyle}

\renewcommand{\i}{\mathbf{i}}
\renewcommand{\j}{\mathbf{j}}
\renewcommand{\k}{\mathbf{k}}
\newcommand{\R}{\mathbb{R}}
\newcommand{\aaa}{\mathbf{a}}
\newcommand{\bbb}{\mathbf{b}}
\newcommand{\ccc}{\mathbf{c}}
\newcommand{\dotp}{\boldsymbol{\cdot}}
\begin{document}

 \begin{enumerate}
 \item In terms of the spherical coordinates $\rho, \varphi, \theta$, we have:

\bigskip

$x=\rho\sin\varphi\cos\theta$

$y=\rho\sin\varphi\sin\theta$

$z=\rho\cos\varphi$


 \item Evaluate $\di \int_{-1}^1\int_{-\sqrt{1-x^2}}^{\sqrt{1-x^2}}\sin(x^2+y^2)\,dy\,dx$ by converting to polar coordinates.

\bigskip

The bounds $-1\leq x\leq 1$, $-\sqrt{1-x^2}\leq y\leq \sqrt{1-x^2}$ describe the entire disc $x^2+y^2\leq 1$. In polar coordinates, we have
\[
 \int_0^{2\pi}\int_0^1 \sin(r^2) r\,dr\,d\theta = \int_0^{2\pi}\left(\left. -\frac{1}{2}\cos(r^2)\right|_0^1\right)\,d\theta = \pi(1-\cos(1)).
\]

\item Describe the surface given in spherical coordinates by $\varphi = \pi/4$.

\bigskip

This surface is the upper half ($z\geq 0$) of the cone $x^2+y^2=z^2$. To see this, note that when $\varphi = \pi/4$, we have $x=\rho\cos\theta/\sqrt{2}$ and $y=\rho\sin\theta/\sqrt{2}$, so $x^2+y^2 = \dfrac{\rho^2}{2}$, and $z = \rho/\sqrt{2}$, so $z^2=x^2+y^2$. Since $\varphi$ is measured from the positive $z$-axis, if $\varphi\in [0,\pi/2]$, we must be above the $xy$-plane.

\item Convert the following integral to polar coordinates:
\[
 \int_{1/\sqrt{2}}^1\int_{\sqrt{1-x^2}}^x xy\,dy\,dx + \int_1^{\sqrt{2}}\int_0^x xy\,dy\,dx + \int_{\sqrt{2}}^2\int_0^{\sqrt{4-x^2}}xy\,dy\,dx
\]

\bigskip

The trick here is to notice the three variable limits: $y=\sqrt{1-x^2}$, $y=\sqrt{4-x^2}$, and $y=x$. The first two describe cicles, of radius 1 and 2, respectively. If we plot these circles in the first quadrant, along with the line $y=x$, we see that $y=x$ intersects $y=\sqrt{1-x^2}$ when $x=1/\sqrt{2}$. The first integral thus involves the region above the circle $x^2+y^2=1$, but below the line $y=x$. The line $y=x$ intersects $y=\sqrt{4-x^2}$ when $x=\sqrt{2}$. The second integral involves the region between the $x$-axis and $y=x$, for $1\leq x\leq \sqrt{2}$. The third is the region above the $x$-axis and below the circle $x^2+y^2=4$.

In polar coordinates, this becomes $1\leq r\leq 2$, where $0\leq \theta\leq \pi/4$, so our integral is simply $\int_0^{\pi/4}\int_1^2 (r\cos\theta)(r\sin\theta) r\,dr\,d\theta$. 
\end{enumerate}






\end{document}