\documentclass[12pt]{article}
\usepackage{amsmath}
\usepackage{amssymb}
\usepackage[letterpaper,margin=0.85in,centering]{geometry}
\usepackage{fancyhdr}
\usepackage{enumerate}
\usepackage{lastpage}
\usepackage{multicol}
\usepackage{graphicx}

\reversemarginpar

\pagestyle{fancy}
\cfoot{Page \thepage \ of \pageref{LastPage}}\rfoot{{\bf Total Points: 60}}
\chead{MATH 2580A}\lhead{Test \#2}\rhead{Thursday, 17\textsuperscript{th} March, 2015}

\newcommand{\points}[1]{\marginpar{\hspace{24pt}[#1]}}
\newcommand{\skipline}{\vspace{12pt}}
%\renewcommand{\headrulewidth}{0in}
\headheight 30pt

\newcommand{\di}{\displaystyle}
\newcommand{\R}{\mathbb{R}}
\newcommand{\N}{\mathbb{N}}
\newcommand{\Z}{\mathbb{Z}}
\newcommand{\pd}[2]{\dfrac{\partial #1}{\partial #2}}
\newcommand{\rd}[2]{\dfrac{d #1}{d #2}}
\newcommand{\dotp}{\boldsymbol{\cdot}}

\begin{document}

\author{Instructor: Sean Fitzpatrick}
\thispagestyle{plain}
\begin{center}
\emph{University of Lethbridge}\\
Department of Mathematics and Computer Science\\
17\textsuperscript{th} March, 2016, 1:40 - 2:55 pm\\
{\bf Math 2580A - Term Test \#2}\\
\end{center}
\skipline \skipline \skipline \noindent \skipline
Last Name:\underline{\hspace{350pt}}\\
\skipline
First Name:\underline{\hspace{348pt}}\\
\skipline
Student Number:\underline{\hspace{322pt}}\\


\vspace{0.5in}


\begin{quote}
 Record your answers below each question in the space provided.    {\bf Left-hand pages may be used as scrap paper for rough work.}  If you want any work on the left-hand pages to be graded, please indicate so on the right-hand page.
 
 \bigskip
 
Partial credit will be awarded for partially correct work, so be sure to show your work, and include all necessary justifications needed to support your arguments. 

The value of each problem is indicated in the left-hand margins. The value of a problem does not always indicate the amount of work required to do the problem.

Outside aids, including, but not limited to, cheat sheets, smart phones, laptops, spy cameras, drones, and divine intervention, are not permitted. You can keep a calculator with you if it makes you feel better.
\end{quote}


\vspace{0.5in}

For grader's use only:

\begin{table}[hbt]
\begin{center}
\begin{tabular}{|l|r|} \hline
Page&Grade\\
\hline \hline
\cline{1-2} 2 & \enspace\enspace\enspace\enspace\enspace\enspace/10\\
\cline{1-2} 3 & \enspace\enspace\enspace\enspace\enspace\enspace/10\\
\cline{1-2} 4 & \enspace\enspace\enspace\enspace\enspace\enspace/10\\
\cline{1-2} 5 & \enspace\enspace\enspace\enspace\enspace\enspace/10\\
\cline{1-2} 6 & \enspace\enspace\enspace\enspace\enspace\enspace/10\\
\cline{1-2} Total & \enspace\enspace\enspace\enspace\enspace\enspace/50\\
\hline
\end{tabular}

\skipline

\skipline

\skipline


\end{center}
\end{table}
\newpage


\begin{enumerate}
\item Evaluate the following integrals:
\begin{enumerate}
 \item $\di \int_0^2\int_0^1 (2x+3y^2)\,dx\,dy$.\points{5}

\vspace{3in}

 \item $\di \iint_D x\cos y\,dA$, where $D$ is the region bounded by $y=0$, $y=x^2$, and $x=1$.\points{5}
\end{enumerate}
\newpage

\item Evaluate the following four integrals by either reversing the order of integration, or switching to polar coordinates.\\


\begin{enumerate}
 \item $\di \int_0^1\int_{y}^1 e^{x^2}\,dx\,dy$ \points{5}

\vspace{3.5in}

 \item $\di \iint_D\sqrt{x^2+y^2}\,dA$, where $D$ is the region enclosed by the circle $r=2\cos\theta$ \points{5}\\
 ($D$ is given by $(x-1)^2+y^2\leq 1$ in rectangular coordinates).

\newpage
\hspace{-32pt}{\em Hint:} There is one integral on this page that looks like it should be easiest in polar \\coordinates, but you're better off reversing the order of integration.

 \item $\di \int_{-1}^1\int_0^{\sqrt{1-x^2}}\sqrt{1-y^2}\,dy\,dx$.\points{5}

\vspace{4in}

 \item $\di \int_0^1\int_y^{\sqrt{2-y^2}}xy\,dy\,dx$. \points{5}
\end{enumerate}
\newpage

\item Let $W$ be the solid region in $\R^3$ bounded by the cone $z=\sqrt{x^2+y^2}$ and the sphere $x^2+y^2+z^2=8$. Find the volume of $W$:
\begin{enumerate}
 \item Using cylindrical coordinates. \points{5}

\vspace{3.5in}

 \item Using spherical coordinates. \points{5}
\end{enumerate}

\newpage

\item Evaluate the  triple integral $\di \int_0^1\int_x^{2x}\int_0^y 2xyz\,dz\,dy\,dx$. \points{5}


\vspace{3in}

 \item Evaluate the integral $\di \iint_D (x+y)^2\,dA$, where $D$ is the parallelogram given by the inequalities $0\leq x+y\leq 1, 0\leq 2x-y\leq 3$ by using an appropriate change of variables. \points{5}

{\em Hint:} It's not too hard to see how to define $u$ and $v$ in terms of $x$ and $y$. You can either solve for $x$ and $y$ in terms of $u$ and $v$, or use the fact that $J_T(u,v) = \dfrac{1}{J_{T^{-1}}(x,y)}$.

\end{enumerate}
\newpage

\begin{center}
List of potentially useful facts and formulas
\end{center}
\begin{itemize}
\item Fubini's Theorem: if $f$ is continuous on the rectangle $R=[a,b]\times[c,d]$ then
\[
\iint_R f(x,y)\,dA = \int_c^d\int_a^bf(x,y)\,dx\,dy = \int_a^b\int_c^d f(x,y)\,dy\,dx.
\]
\item For a Type I region $D$ given by $a\leq x\leq b$, $g(x)\leq y\leq h(x)$,
\[
\iint_D f(x,y)\,dA = \int_a^b\int_{g(x)}^{h(x)}f(x,y)\,dy\,dx.
\]
\item For a Type II region $D$ given by $g(y)\leq x\leq h(y)$, $c\leq y\leq d$,
\[
\iint_D f(x,y)\,dA = \int_c^d\int_{g(y)}^{h(y)}f(x,y)\,dx\,dy.
\]
\item Polar coordinates: $x=r\cos\theta$, $y=r\sin\theta$, and
\[
\iint_D f(x,y)\,dA = \int_\alpha^\beta\int_{r_1(\theta)}^{r_2(\theta)}f(r\cos\theta,r\sin\theta)r\,dr\,d\theta.
\]
\item Triple integrals: like double integrals, but with one more variable. (Fubini still applies.)
\item Cylindrical coordinates: $x=r\cos\theta$, $y=r\sin\theta$, $z=z$, $dV = r\,dz\,dr\,d\theta$.
\item Spherical coordinates: $x=\rho\cos\theta\sin\phi$, $y=\rho\sin\theta\sin\phi$, $z=\rho\cos\phi$, $dV = \rho^2\sin\phi\,d\rho\,d\phi\,d\theta$.
\item Jacobian: if $T(u,v) = (x(u,v),y(u,v))$, then $\di J_T(u,v) = \frac{\partial x}{\partial u}\frac{\partial y}{\partial v}-\frac{\partial x}{\partial v}\frac{\partial y}{\partial u}$.
 \item Change of variables: if $T$ is a transformation from $R$ to $D$, then 
\[
 \iint_D f(x,y)\,dA = \iint_R f(T(u,v))\lvert J_T(u,v)\rvert\,du\,dv.
\]
\end{itemize}
\end{document}