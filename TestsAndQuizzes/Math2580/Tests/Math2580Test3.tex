\documentclass[12pt]{article}
\usepackage{amsmath}
\usepackage{amssymb}
\usepackage[letterpaper,margin=0.85in,centering]{geometry}
\usepackage{fancyhdr}
\usepackage{enumerate}
\usepackage{lastpage}
\usepackage{multicol}
\usepackage{graphicx}

\reversemarginpar

\pagestyle{fancy}
\cfoot{Page \thepage \ of \pageref{LastPage}}\rfoot{{\bf Total Points: 60}}
\chead{MATH 2580A}\lhead{Test \#3}\rhead{Thursday, 21\textsuperscript{st} April, 2016}

\newcommand{\points}[1]{\marginpar{\hspace{24pt}[#1]}}
\newcommand{\skipline}{\vspace{12pt}}
%\renewcommand{\headrulewidth}{0in}
\headheight 30pt

\newcommand{\di}{\displaystyle}
\newcommand{\R}{\mathbb{R}}
\newcommand{\N}{\mathbf{N}}
\newcommand{\Z}{\mathbb{Z}}
\newcommand{\pd}[2]{\dfrac{\partial #1}{\partial #2}}
\newcommand{\rd}[2]{\dfrac{d #1}{d #2}}
\newcommand{\dotp}{\boldsymbol{\cdot}}
\newcommand{\F}{\mathbf{F}}
\newcommand{\G}{\mathbf{G}}
\renewcommand{\i}{\mathbf{i}}
\renewcommand{\j}{\mathbf{j}}
\renewcommand{\k}{\mathbf{k}}
\renewcommand{\S}{\mathbf{S}}
\renewcommand{\r}{\mathbf{r}}
\newcommand{\len}[1]{\left\lVert #1\right\rVert}

\begin{document}

\author{Instructor: Sean Fitzpatrick}
\thispagestyle{plain}
\begin{center}
\emph{University of Lethbridge}\\
Department of Mathematics and Computer Science\\
21\textsuperscript{st} April, 2016, 2:00 - 4:00 pm\\
{\bf Math 2580A - Test \#3}\\
\end{center}
\skipline \skipline \skipline \noindent \skipline
Last Name:\underline{\hspace{350pt}}\\
\skipline
First Name:\underline{\hspace{348pt}}\\
\skipline
%Student Number:\underline{\hspace{322pt}}\\


\vspace{0.5in}


\begin{quote}
 Record your answers below each question in the space provided.    {\bf Left-hand pages may be used as scrap paper for rough work.}  If you want any work on the left-hand pages to be graded, please indicate so on the right-hand page.
 
 \bigskip
 
Partial credit will be awarded for partially correct work, so be sure to show your work, and include all necessary justifications needed to support your arguments. 

The value of each problem is indicated in the left-hand margins. The value of a problem does not always indicate the amount of work required to do the problem.

Outside aids, including, but not limited to, cheat sheets, smart phones, in-ear radio communication devices, and telepathy are not permitted. You can keep a calculator with you if it makes you feel better.
\end{quote}


\vspace{0.5in}

For grader's use only:

\begin{table}[hbt]
\begin{center}
\begin{tabular}{|l|r|} \hline
Page&Grade\\
\hline \hline
\cline{1-2} 2 & \enspace\enspace\enspace\enspace\enspace\enspace/10\\
\cline{1-2} 3 & \enspace\enspace\enspace\enspace\enspace\enspace/10\\
\cline{1-2} 4 & \enspace\enspace\enspace\enspace\enspace\enspace/10\\
\cline{1-2} 5 & \enspace\enspace\enspace\enspace\enspace\enspace/10\\
\cline{1-2} 6 & \enspace\enspace\enspace\enspace\enspace\enspace/10\\
\cline{1-2} 7 & \enspace\enspace\enspace\enspace\enspace\enspace/10\\
\cline{1-2} Total & \enspace\enspace\enspace\enspace\enspace\enspace/60\\
\hline
\end{tabular}

\skipline

\skipline

\skipline


\end{center}
\end{table}
\newpage


\begin{enumerate}
\item Let $\F = yz\i+xz\j+(xy+2z)\k$.
\begin{enumerate}
\item Show that $\F$ is conservative, and find a function $f$ such that $\nabla f=\F$. \points{5}

\vspace{5in}

\item Compute $\displaystyle \int_C \F\dotp \,d\r$, where $C$ is given by $\r(t) = \langle t^2,t+1,2t-1\rangle$, with $t\in [0,1]$.\points{5}
\end{enumerate}


\newpage

\item Let $\F(x,y) = \langle xy, x^2\rangle$, and let $C$ be the triangular path from $(0,0)$ to $(1,0)$ to $(1,2)$ and back to $(0,0)$. Evaluate the integral $\int_C \F\dotp \,d\r$:
\begin{enumerate}
\item Directly.\points{6}

\vspace{5in}

\item Using Green's Theorem.\points{4}

\end{enumerate}
\newpage


\item Let $S$ be the oriented surface given by the vector-valued function
\[
\r(u,v) = \langle u^2,v^2,uv\rangle.
\]
\begin{enumerate}
\item Find the equation of tangent plane to $S$ at the point given by $(u,v)=(1,2)$.\points{5}

\vspace{3.5in}

\item Set up, but do not evaluate, the integral that will compute the surface area of the portion of $S$ where $u^2+v^2\leq 9$. \points{3}

\vspace{2.75in}

\item At what (if any) points is the tangent plane to $S$ horizontal? \points{2}
\end{enumerate}

\newpage

\item Let $\F(x,y,z) = \langle yz, -xz, z^3\rangle$, and let $S$ be the surface given by the part of the cone $z = \sqrt{x^2+y^2}$ between the planes $z=1$ and $z=4$, oriented downwards.\points{10}

Use Stokes' theorem to evaluate 
\begin{equation*}
\iint\limits_S\left(\nabla\times\F\right)\dotp\,d\S.
\end{equation*}
\noindent {\bf Hint:} Sketch the surface. The boundary of $S$ consists of two circles with opposite orientations.


\newpage


\item Verify the Divergence Theorem for $\F(x,y,z) = \len{\r}\r$, where $\r = x\i + y\j + z\k$, and $S$ is the spherical surface $x^2+y^2+z^2 = 4$.\points{10}


\newpage


\item \begin{enumerate}
\item Prove that $\nabla\dotp (\nabla\times \F) = 0$ for any $C^2$ vector field $\F$.\points{4}

\vspace{3.5in}

%\item Verify Stokes' Theorem in this case by computing the original surface integral directly. \points{8}

\item Evaluate $\di \iint_S \F\dotp d\S$, where $\F(x,y,z) = \langle x, -y, 3\rangle$, and $S$ is the hemisphere\\ $z=\sqrt{4-x^2-y^2}$. \points{6} \\(Hint: There is an easy way and a hard way -- what is the divergence of $\F$?.)
\end{enumerate}

\end{enumerate}
\newpage

\begin{center}
List of potentially useful facts and formulas
\end{center}
\begin{itemize}
\item Line integrals: for a curve $C$ parameterized by $\r(t) = \langle x(t),y(t),z(t)\rangle$, $t\in [a,b]$, we have
\begin{align*}
\int_C f\,ds &= \int_a^b f(\r(t))\len{\r'(t)}\,dt \quad \text{ for a scalar field } f(x,y,z)\\
\int_C \F\dotp\,d\r & = \int_a^b \F(\r(t))\dotp \r'(t)\,dt \quad \text{ for a vector field } \F(x,y,z).
\end{align*}

\item Surface integrals: For a surface $S$ parameterized by $\r(u,v) = \langle x(u,v), y(u,v), z(u,v)\rangle$, $(u,v)\in D\subseteq \R^2$, the normal vector to the surface at a point $\r(u_0,v_0)$ is given by
\[
\N(u,v) = \r_u(u,v)\times \r_v(u,v),
\]
and 
\begin{align*}
\iint_S f\,dS & = \iint_D f(\r(u,v))\len{\N(u,v)}\,du\,dv \quad \text{ for a scalar field } f(x,y,z)\\
\iint_S \F\dotp\,d\S & = \iint_D \F(\r(u,v))\dotp \N(u,v)\,du\,dv \quad \text{ for a vector field } \F(x,y,z).
\end{align*}

\item $\nabla\times (\nabla f) = \mathbf{0}$ and $\nabla \dotp (\nabla\times \F) = 0$.

\item Fundamental Theorem of Calculus: if $\F = \nabla f$ for some $C^1$ function $f$, and $C$ is a curve from a point $\mathbf{a}$ to a point $\mathbf{b}$, then
\[
\int_C \F\dotp \,d\r = f(\mathbf{b})-f(\mathbf{a}).
\]


\item Green's Theorem: let $D\subseteq \R^2$ be a region whose positively-oriented boundary is a simple, closed piecewise-smooth curve $C$. Then
\[
\int_C P\,dx+Q\,dy = \iint_D \left(\pd{Q}{x}-\pd{P}{y}\right)\,dA
\]
for any $C^1$ functions $P(x,y)$ and $Q(x,y)$ on $D$.

\item Stokes' Theorem: let $S$ be a piecewise-smooth surface in $\R^3$ with positively-oriented boundary curve $C$. If $\F$ is a $C^1$ vector field defined on an open set containing $S$, then
\[
\iint_S (\nabla\times \F)\dotp \,d\S = \int_C \F\dotp \,d\r.
\]

\item Divergence Theorem: let $E\subseteq \R^3$ be a region bounded by a piecewise-smooth surface $S$, and let $\F$ be a $C^1$ vector field defined on an open set containing $E$. Then
\[
\iiint_E (\nabla\dotp \F)\,dV = \iint_S \F\dotp \,d\S.
\]
\end{itemize}
\end{document}