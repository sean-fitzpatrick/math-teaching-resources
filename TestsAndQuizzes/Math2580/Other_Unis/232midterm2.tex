\documentclass[12pt]{article}
\usepackage{amsmath}
\usepackage{amssymb}
\usepackage{fullpage}
\usepackage{fancyhdr}
\usepackage{lastpage}
%\input quizstyle.tex

\reversemarginpar

\pagestyle{fancy}
\addtolength{\headheight}{\baselineskip}

\lhead{{\bf Date:} Friday, June 8th }
\chead{{\bf Time:} 9:10 pm}
\rhead{MAT232HF}
\cfoot{Page \thepage \ of \pageref{LastPage}}
\rfoot{{\bf Total Marks:} 60}

%Redefine the plain page style in fancyhdr package
\fancypagestyle{plain}{%  
\fancyhead{}
\fancyfoot{}
\fancyfoot[C]{Page \thepage \ of \pageref{LastPage}}
\renewcommand{\headrulewidth}{0pt}}

\newcounter{probnum}
\newcounter{subprobnum}

\DeclareMathOperator{\im}{im}
\DeclareMathOperator{\spn}{span}
\DeclareMathOperator{\col}{col}
\DeclareMathOperator{\rank}{rank}
\DeclareMathOperator{\diag}{diag}
\DeclareMathOperator{\adj}{adj}
\DeclareMathOperator{\nll}{null}
\DeclareMathOperator{\tr}{tr}

\newenvironment{problems}{
\begin{list}{\arabic{probnum}.}{\usecounter{probnum}}
}{
\end{list}
}

\newenvironment{subproblem}{ % start for subprob
\begin{list}{ % first arg for list
(\alph{subprobnum})
}{ % second arg for list
\usecounter{subprobnum}
\setlength{\topsep}{0in}
} % end of list def
}{ % end for subprob
\end{list}
}

\newcommand{\skipline}{\vspace{12pt}}
%\input local.tex
\renewcommand{\labelenumi}{(\roman{enumi})}

\begin{document}
\thispagestyle{plain}
%Supresses the headers on the front page

\centerline {\bf University of Toronto at Mississauga}
\medskip
\centerline {\bf Mid-Term Make-up Exam}
\medskip
\centerline {\bf MAT232HF}
\centerline {\bf Calculus of Several Variables}
\medskip
\centerline {Instructor: Sean Fitzpatrick}
\centerline {Duration: 110 minutes}
\bigskip
\bigskip

\noindent {\bf NO AIDS ALLOWED.} \hfill {\bf Total: 60 marks}
\vglue .25truein
\begin{tabular}{ll}
Family Name: &\underbar {\hskip 4.5in} \\
   &{\hskip 2truein } {\footnotesize (Please Print)}\\
[15pt]
Given Name(s): &\underbar {\hskip 4.5in} \\
    &{\hskip 2truein } {\footnotesize (Please Print)}\\
[15pt]
Please sign here: &\underbar {\hskip 4.5in}\\
[25pt]
Student ID Number: &\underbar {\hskip 4.5in}\\
\end{tabular}
\bigskip


%\vspace{1in}
\begin{quote}
{\large \bf You may not use calculators, cell phones, or PDAs during
the exam.  Partial credit will be given for partially correct work.
Please read through the entire test before starting, and take note of
how many points each question is worth.  Please put a box around your
solutions so that the grader may find them easily.  }
\end{quote}

\vspace{.25in}
\begin{center}
\begin{tabular}{|l|r|}
\hline \hline
\multicolumn{2}{|c|}
{\rule[-3mm]{0mm}{8mm}
FOR MARKER'S USE ONLY} \\
\hline
Problem 1: & \hspace{.5in}  /10 \\ [3pt]
\hline
Problem 2: & \hspace{.5in}  /10 \\ [3pt]
\hline
Problem 3: & \hspace{.5in}  /10 \\ [3pt]
\hline
Problem 4: & \hspace{.5in}  /15 \\ [3pt]
\hline
Problem 5: & \hspace{.5in}  /8 \\ [3pt]
\hline
Problem 6: & \hspace{.5in}  /7 \\ [3pt]
\hline
\hline 
{\rule[-3mm]{0mm}{8mm} TOTAL:}  & /60  \\
\hline
\end{tabular}
\end{center}

\newpage
\vglue1pt
%vglue adds space between headers and text

\begin{problems}
\item   \begin{subproblem}
	\item Obtain the equation of the ellipse with foci at $(\pm 3, 0)$ 
and major axis of length 10, and then sketch the graph.
\marginpar{[5]}
\vspace{4in}
	\item Eliminate the parameter to sketch the parametric curve $x(t) 
=\cos 2t$, $y(t) = \sin t$, $t\in [-\pi,\pi]$. Then, 
describe the motion of a particle moving according to these 
equations as $t$ changes. \marginpar{[5]}
	\end{subproblem}
\newpage
\vglue1pt

\item Recall the identity $\vec{u}\times(\vec{v}\times\vec{w}) = 
(\vec{u}\cdot\vec{w})\vec{v} - (\vec{u}\cdot\vec{v})\vec{w}$.
	\begin{subproblem}
	\item Show that \marginpar{[4]}
\begin{equation*}
(\vec{u}\times\vec{v})\times\vec{w} = (\vec{w}\cdot\vec{u})\vec{v} - 
(\vec{v}\cdot\vec{w})\vec{u}
\end{equation*}

\noindent {\bf Hint:} $\vec{a}\times\vec{b} = -\vec{b}\times\vec{a}$.

\vspace{2.8in}
	\item Let $\vec{a}$ be a given non-zero vector, and $\hat{u}$ a 
unit vector.  Show that $\vec{a}$ can be written as $\vec{a} = 
\vec{a}_\parallel + \vec{a}_\bot$, where $\vec{a}_\parallel = 
(\vec{a}\cdot\hat{u})\hat{u}$ is parallel to $\hat{u}$ and $\vec{a}_\bot = 
(\hat{u}\times\vec{a})\times\hat{u}$ is perpendicular to $\vec{a}$. 
\marginpar{[6]}
	\end{subproblem}
\newpage
\vglue1pt
\item In each case, determine whether or not the given line $L$ and plane 
$P$ are 
parallel, or intersect.  If they intersect, find the point of 
intersection.
	\begin{subproblem}
	\item $L:\: x = 7-4t,\: y=3+6t,\: z=9+5t$, and $P:\: 4x+y+2z=17$.
\marginpar{[5]}
\vspace{3.2in}
	\item $L:\: x=3+3t,\: y=6-5t,\: z = 2+3t$, and $P:\: 3x+2y-4z=1$.
\marginpar{[5]}
	\end{subproblem}
\newpage
\vglue1pt
\item \begin{subproblem}
	\item Find all first-order partial derivatives 
of the following functions:\marginpar{[8]}
	\begin{enumerate}
		\item $f(x,y) = {\displaystyle \frac{x\sin y}{y\cos x}}$.
\vspace{2in}
                \item $g(x,y,z) = \ln(\frac{x}{y})-ye^{xz}$.
\vspace{2in}

                \item $h(x,y,z) = xy+yz+zx$.
	\end{enumerate}
\newpage
\vglue1pt
	\item Given $u(x,y) = {\displaystyle \frac{xy}{x+y}}$, show that
\marginpar{[7]}
\begin{equation*}
x^2\frac{\partial^2 u}{\partial x^2} + 2xy\frac{\partial^2 u}{\partial 
x\partial y} + y^2\frac{\partial^2 u}{\partial y^2}=0
\end{equation*}
	\end{subproblem}
\newpage
\vglue1pt

\item Consider the elliptic paraboloid $x^2 + \frac{y^2}{b^2} = z$.
\begin{subproblem}
	\item Describe the trace of this paraboloid in the plane 
$z=1$.\marginpar{[2]}
\vspace{1.5in}
	\item What happens to this trace as 
$b\rightarrow\infty$?\marginpar{[3]}
\vspace{2.5in}
	\item What happens to the paraboloid as 
$b\rightarrow\infty$?\marginpar{[3]}
\end{subproblem}
\newpage
\vglue1pt

\item Show that the limit \marginpar{[7]}
\begin{equation*}
\lim_{(x,y)\rightarrow (1,1)}\frac{x-y^4}{x^3-y^4}
\end{equation*}
does not exist.

\newpage
\vglue1pt
Extra space for rough work. Do {\bf not} tear out this page.
\end{problems}
\end{document}
\newpage

