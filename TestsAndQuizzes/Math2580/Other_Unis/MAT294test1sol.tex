\documentclass[12pt]{article}
\usepackage{amsmath}
\usepackage{amssymb}
\usepackage{fullpage}
\usepackage{fancyhdr}
\usepackage{lastpage}
%\input quizstyle.tex

\reversemarginpar

\pagestyle{fancy}
\addtolength{\headheight}{\baselineskip}

\lhead{{\bf Date:} 9th October, 2008 }
\chead{{\bf Time:} 4:10 pm}
\rhead{MAT294H1Y}
\cfoot{Page \thepage \ of \pageref{LastPage}}
\rfoot{{\bf Total Marks:} 50}

%Redefine the plain page style in fancyhdr package
\fancypagestyle{plain}{%  
\fancyhead{}
\fancyfoot{}
\fancyfoot[C]{Page \thepage \ of \pageref{LastPage}}
\renewcommand{\headrulewidth}{0pt}}

\newcounter{probnum}
\newcounter{subprobnum}

\DeclareMathOperator{\im}{im}
\DeclareMathOperator{\spn}{span}
\DeclareMathOperator{\col}{col}
\DeclareMathOperator{\rank}{rank}
\DeclareMathOperator{\diag}{diag}
\DeclareMathOperator{\adj}{adj}
\DeclareMathOperator{\nll}{null}
\DeclareMathOperator{\tr}{tr}

\newenvironment{problems}{
\begin{list}{\arabic{probnum}.}{\usecounter{probnum}}
}{
\end{list}
}

\newenvironment{subproblem}{ % start for subprob
\begin{list}{ % first arg for list
(\alph{subprobnum})
}{ % second arg for list
\usecounter{subprobnum}
\setlength{\topsep}{0in}
} % end of list def
}{ % end for subprob
\end{list}
}

\newcommand{\skipline}{\vspace{12pt}}
%\input local.tex
\renewcommand{\labelenumi}{(\roman{enumi})}

\begin{document}
\thispagestyle{plain}
%Supresses the headers on the front page

\centerline {\bf FACULTY OF APPLIED SCIENCE AND ENGINEERING}
\centerline {\bf University of Toronto}
\medskip
\centerline {\bf MAT294H1Y}
\centerline {\bf Calculus and Differential Equations}
\medskip
\centerline {Term Test \#1}
\centerline {Duration: 110 minutes}
\bigskip
\bigskip

\noindent {\bf NO AIDS ALLOWED.} \hfill {\bf Total: 50 marks}
\vglue .25truein
\begin{tabular}{ll}
Family Name: &\underbar{SOLUTIONS {\hskip 3.5in}} \\
   &{\hskip 2truein } {\footnotesize (Please Print)}\\
[15pt]
Given Name(s): &\underbar{ THE {\hskip 4.05in}} \\
    &{\hskip 2truein } {\footnotesize (Please Print)}\\
[15pt]
Please sign here: &\underbar {\hskip 4.5in}\\
[25pt]
Student ID Number: &\underbar {\hskip 4.5in}\\
\end{tabular}
\bigskip


%\vspace{1in}
\begin{quote}
{\large \bf You may not use calculators, cell phones, or PDAs during
the test.  Partial credit will be given for partially correct work.
Please read through the entire test before starting, and take note of
how many points each question is worth.}
\end{quote}

\vspace{.25in}
\begin{center}
\begin{tabular}{|l|r|}
\hline \hline
\multicolumn{2}{|c|}
{\rule[-3mm]{0mm}{8mm}
FOR MARKER'S USE ONLY} \\
\hline
Problem 1: & \hspace{.5in}  /10 \\ [3pt]
\hline
Problem 2: & \hspace{.5in}  /15 \\ [3pt]
\hline
Problem 3: & \hspace{.5in}  /8\phantom{1} \\ [3pt]
\hline
Problem 4: & \hspace{.5in}  /9\phantom{1} \\ [3pt]
\hline
Problem 5: & \hspace{.5in}  /8\phantom{1} \\ [3pt]
\hline
\hline 
  {\rule[-3mm]{0mm}{8mm} TOTAL:}  & /50  \\
\hline
\end{tabular}
\end{center}

\newpage
\vglue1pt
%vglue adds space between headers and text

\begin{problems}
\item
\begin{subproblem}
\item Calculate the first-order partial derivatives of the following functions:
\begin{enumerate}
\item $f(x,y,z) = z\sin(x-y)$\marginpar{[2]}
\begin{align*}
f_x(x,y,z) & = z\cos(x-y)\\
f_y(x,y,z) & = -z\cos(x-y)\\
f_z(x,y,z) & = \sin(x-y)
\end{align*}
\bigskip
\item $f(x,y) = x^2y e^{xy}$\marginpar{[2]}
\begin{align*}
f_x(x,y) & = 2xye^{xy} + x^2y^2e^{xy}\\
f_y(x,y) & = x^2e^{xy} + x^3ye^{xy}
\end{align*}
\bigskip 
\item $\displaystyle f(x,y,z) = \frac{\ln z}{xy}$\marginpar{[2]}
\begin{align*}
f_x(x,y,z) & = -\frac{\ln z}{x^2y}\\
f_y(x,y,z) & = -\frac{\ln z}{xy^2}\\
f_z(x,y,z) & = \frac{1}{xyz}
\end{align*}
\end{enumerate}
\bigskip

\item Find all second-order partial derivatives of the function \marginpar{[4]}
\begin{equation*}
f(x,y) = x^2\tan y + y\ln x.
\end{equation*}
We have first-order partial derivatives
\begin{equation*}
f_x(x,y) = 2x\tan y +\frac{y}{x}\quad\text{and}\quad f_y(x,y) = x^2\sec^2 y + \ln x,
\end{equation*}
so
\begin{align*}
f_{xx}(x,y) & = \frac{\partial}{\partial x}(2x\tan y +\frac{y}{x})  = 2\tan y -\frac{y}{x^2},\\
f_{yy}(x,y) & = \frac{\partial}{\partial y}(x^2\sec^2 y + \ln x)  = 2x^2 \sec^2 y \tan y,\\
f_{xy}(x,y) & = \frac{\partial}{\partial y}(2x\tan y + \frac{y}{x})  = 2x\sec^2 y + \frac{1}{x} = f_{yx}(x,y)
\end{align*}
\end{subproblem}

\newpage
\vglue1pt

\item  Let $f(x,y) = x^3-3xy-y^3$. 
\begin{subproblem}
\item Locate all critical points of $f$. \marginpar{[3]}

\medskip

The partial derivatives of $f$ are
\begin{equation*}
f_x(x,y) = 3x^2-3y \quad \text{and} \quad f_y(x,y) = -3x-3y^2,
\end{equation*}
and the critical points of $f$ will be those points where both $f_x$ and $f_y$ are zero.  We find:
\begin{equation*}
f_x(x,y) = 0 \:\:\text{and}\:\: f_y(x,y) = 0\Rightarrow y = x^2 \Rightarrow (x^2)^2+x = 0,
\end{equation*}
which gives $x=0$, (and $y=0$) or $x=-1$ (and $y=(-1)^2 = 1$), so there are two critical points: $(0,0)$ and $(-1,1)$.
\vspace{1in}
  
\item Classify any critcal points found in part (a) as local maxima, local minima, or saddle points. \marginpar{[4]}
\medskip

We have $f_{xx}(x,y) = 6x$, $f_{xy}(x,y) = -3$ and $f_{yy}(x,y) = -6y$.

At $(0,0)$, this gives $A=0$, $B= -3$, $C=0$ and $\Delta = 0-(-3)^2 = -9$.  Since $\Delta<0$, $(0,0)$ is a saddle point.

At $(-1,1)$, this gives $A=-6$, $B=-3$, $C=-6$ and $\Delta = 36-9 = 27$.  Since $\Delta>0$ and $A<0$, $(-1,1)$ is a local maximum.
\newpage
\vglue1pt
\item Find the maximum and minimum of $f(x,y)$ subject to the constraint $x+2y-1 = 0$ (**For $-1\leq x \leq 1$).\marginpar{[5]}
\medskip

We let $g(x,y) = x+2y-1$.  The Lagrange multiplier equations $\nabla f(x,y) = \lambda\nabla g(x,y)$, and $g(x,y)=0$ become:
\begin{enumerate}
\item $f_x = \lambda g_x \Rightarrow \: 3x^2 - 3y = \lambda(1)$
\item $f_y = \lambda g_y \Rightarrow \: -3x-3y^2 = \lambda(2)$ 
\item $g(x,y) =0 \Rightarrow \: x+2y=1$
\end{enumerate}
Comparing (i) and (ii) we see $f_y = 2\lambda = 2f_x$, so
\begin{equation*}
-3x-3y^2 = 2(3x^2-3y).
\end{equation*}
If we use (iii) to solve for $x=1-2y$ and plug this into the above equation, we get
\begin{equation*}
6(1-2y)^2-6y+3(1-2y)+3y^2 = 27y^2-36y+9 = 9(3y-1)(y-1) = 0
\end{equation*}
This gives two points at which a max/min can occur: $y=1/3$ and $x=1-2(1/3) = 1/3$, or $y=1$ and $x=-1$.  $x=-1$ is one of our ``end points''; the other end point is $(1,0)$.

We find $f(1/3,1/3) = -1/3$ is the minimum, and $f(-1,1) = f(1,0)= 1$ is the maximum.

** If you don't apply the restriction $-1\leq x \leq 1$, then there technically is no maximum or minimum - the two points found above are ``local'' max/min points: $f(x,y)$ can be arbitrarily large (positive or negative).
\bigskip
 
\item Find the absolute maximum and minumum of $f(x,y)$ on the region bounded by the coordinate axes and the line $x+2y=1$.\marginpar{[3]}

\noindent {\em Hint:} The only thing you still have to check is the value of $f(x,y)$ along the axes.
\bigskip

At the critical points, we have $f(0,0) = 0$ and $f(-1,1) = 1$, but only $(0,0)$ lies inside of the region.

Along the line $x+2y=1$ we have the minimum $f(1/3,1/3) = -1/3$ found above.

Along the $y$-axis, $f(0,y) = -y^3$ has maximum $f(0,0)=0$ and minimum $f(0,1/2) = -1/8$.

Along the $x$-axis, $f(x,0) = x^3$ has maximum $f(1,0)=1$ and minimum $f(0,0)=0$.

Thus, the absolute maximum is $f(1,0)=1$ and the absolute minimum is $f(1/3, 1/3) = -1/3$.
\end{subproblem}
\newpage
\vglue1pt
\item Let $f(x,y) = \sqrt{x^2+y^2}$.
\begin{subproblem}
\item Find $\nabla f(x,y)$.\marginpar{[2]}
\bigskip
\begin{equation*}
\nabla f(x,y) = \frac{x}{\sqrt{x^2+y^2}}\,\hat{\imath} + \frac{y}{\sqrt{x^2+y^2}}\,\hat{\jmath}.
\end{equation*}
\bigskip

\item Find the equation of the tangent plane to the surface $z=\sqrt{x^2+y^2}$ at the point $(3,4)$.\marginpar{[3]}
\bigskip

The general equation is $z-z_0 = \nabla f(x_0,y_0)\cdot \left<x-x_0,y-y_0\right>$.

Here we have $x_0 = 3$, $y_0 = 4$, and $z_0 = \sqrt{3^2+4^2} = 5$.  Therefore, the equation of the plane is
\begin{align*}
z-5 &= \nabla f(3,4)\cdot\left<x-3,y-4\right>\\
&=\left<3/5,4,5\right>\cdot\left<x-3,y-4\right>\\
&=\frac{3}{5}(x-3) + \frac{4}{5}(y-4).
\end{align*}
\bigskip

\item Use the differential $df$ to approximate the value of $\sqrt{(2.97)^2+(3.04)^2}$.\marginpar{[3]}

\noindent {\em Note:} $3/5 = 0.6$ and $4/5 = 0.8$.

We have the approximation $f(\vec{x}+\Delta\vec{x}) \approx f(\vec{x}) + df(\vec{x})$, where $df(\vec{x}) = \nabla f(\vec{x})\cdot\Delta\vec{x}$.

Here, $\vec{x} = 3\hat{\imath} + 4\hat{\jmath}$, $\Delta\vec{x} = -0.03\hat{\imath} + 0.04\hat{\jmath}$, and $\nabla f(\vec{x}) = \frac{3}{5}\hat{\imath} + \frac{4}{5}\hat{\jmath}$.

Therefore, we get:
\begin{align*}
\sqrt{(2.97)^2 + (3.04)^2} & = f(3-0.03, 4+0.04)\\
& \approx f(3,4) + \nabla f(3,4)\cdot \left<-0.03, 0.04\right>\\
& \approx 5 + 0.6(-0.03) + 0.8(0.04) = 5.014.
\end{align*}
\end{subproblem}
\newpage
\vglue1pt

\item Let $f(x,y,z) = \ln(x+y+z)$ and let $\vec{r}(t) = \cos^2 t\,\hat{\imath} + \sin^2 t\,\hat{\jmath} + t^2\,\hat{k}$.
\begin{subproblem}
\item Write the chain rule formula for the derivative $\displaystyle \frac{d}{dt}(f(\vec{r}(t)))$. \marginpar{[2]}
\bigskip

\begin{equation*}
\frac{d}{dt}(f(\vec{r}(t))) = \nabla f(\vec{r}(t))\cdot \vec{r}\,'(t),
\end{equation*}
or
\begin{multline*}
\frac{d}{dt}(f(x(t),y(t),z(t)) = \\f_x(x(t),y(t),z(t))x\,'(t) + f_y(x(t),y(t),z(t))y\,'(t) + f_z(x(t),y(t),z(t))z\,'(t),
\end{multline*}
or
\begin{equation*}
\frac{df}{dt} = \frac{\partial f}{\partial x}\frac{dx}{dt} + \frac{\partial f}{\partial y}\frac{dy}{dt} + \frac{\partial f}{\partial z}\frac{dz}{dt}.
\end{equation*}
\bigskip

\item Use your formula from (a) to evaluate $\displaystyle \frac{d}{dt}(f(\vec{r}(t)))$ when $t=\pi/4$.\marginpar{[2]}
\bigskip

When $t=\pi/4$, $\vec{r}(t) = \frac{1}{2}\hat{\imath} + \frac{1}{2}\hat{\jmath} + \frac{\pi^2}{16}\hat{k}$.

$\displaystyle \nabla f(x,y,z) = \frac{1}{x+y+z}(\hat{\imath} + \hat{\jmath} + \hat{k})$, so $\displaystyle \nabla f(\vec{r}(\pi/4)) = \frac{\hat{\imath}+\hat{\jmath} + \hat{k}}{1+\frac{\pi^2}{16}}$.

We have $\vec{r}\,'(t) = -2\cos t\sin t \hat{\imath} + 2\cos t\sin t\hat{\jmath} + 2t\hat{k}$, so $\vec{r}\,'(\pi/4) = -\hat{\imath} + \hat{\jmath} +\frac{\pi}{2}\hat{k}$.

Thus,
\begin{equation*}
\frac{d}{dt}f(\vec{r}(t))|_{t=\pi/4} = \frac{\hat{\imath}+\hat{\jmath} + \hat{k}}{1+\frac{\pi^2}{16}}\cdot (-\hat{\imath} + \hat{\jmath} +\frac{\pi}{2}\hat{k}) = \frac{\pi/2}{1+\pi^2/16}.
\end{equation*}
\newpage
\vglue1pt
\item Verify your calculation in part (b) by first making the substitutions $x=\cos^2 t$, $y=\sin^2 t$ and $z=t^2$ in $f$ and then differentiating with respect to $t$. \marginpar{[2]}
\bigskip

If we let $g(t) = f(x(t),y(t),z(t))$, then we get
\begin{equation*}
g(t) = \ln (\cos^2t + \sin^2t + t^2) = \ln(1+t^2)\quad \text{and}\quad g\,'(t) = \frac{2t}{1+t^2}.
\end{equation*}
Thus,
\begin{equation*}
\frac{d}{dt}f(\vec{r}(t))|_{t=\pi/4} = g\,'(\pi/4) = \frac{\pi/2}{1+\pi^2/16}.
\end{equation*}
\bigskip
 
\item Find the derivative of $f$ in the direction of the curve $\vec{r}(t)$ at the point $(1/2, 1/2, \pi^2/16)$.
\marginpar{[3]}

\noindent {\em Hint:} You've done most of the work for this problem already!
\end{subproblem}
\bigskip

Differentiating in the direction of the curve means taking the derivative at the point $\vec{r}(\pi/4)$ in the direction of $\vec{r}\,'(\pi/4)$.

The unit vector is thus
\begin{equation*}
\hat{u} = \frac{\vec{r}\,'(\pi/4)}{||\vec{r}\,'(\pi/4)||},
\end{equation*}
and the directional derivative we want is
\begin{equation*}
D_{\hat{u}}f(1/2,1/2,\pi^2/16) = \nabla f(\vec{r}(\pi/4))\cdot \frac{\vec{r}\,'(\pi/4)}{||\vec{r}\,'(\pi/4)||},
\end{equation*}
which is just $\displaystyle \frac{d}{dt}f(\vec{r}(t))|_{t=\pi/4}$ divided by $||\vec{r}\,'(\pi/4)|| = \sqrt{2+\pi^2/4}$.
\newpage
\vglue1pt
\item
\begin{subproblem}
\item Show that the limit \marginpar{[3]}
\begin{equation*}
\lim_{(x,y)\rightarrow (0,0)}\frac{x^2-y^2}{x^2+y^2}
\end{equation*}
does not exist by making the substitution $y=mx$, where $m$ can be any real number.
\bigskip

If we substitute $y=mx$ into the limit, then we get
\begin{equation*}
\lim_{(x,y)\rightarrow (0,0)}\frac{x^2-y^2}{x^2+y^2} = \lim_{x\rightarrow 0}\frac{x^2-m^2x^2}{x^2+m^2x^2},
\end{equation*}
which simplifies to
\begin{equation*}
\lim_{x\rightarrow 0} \frac{1-m^2}{1+m^2} =\frac{1-m^2}{1+m^2}.
\end{equation*} 
Since this result depends on the choice of $m$, (for example, $m=1$ gives 0, while $m=0$ gives 1) the value of the limit is not the same along all paths to the origin, and therefore the limit does not exist.
\newpage
\vglue1pt
\item Recall that $f(\vec{x})$ is differentiable at $\vec{a}$ if and only if
\begin{equation*}
\lim_{\vec{h}\rightarrow \vec{0}}\frac{f(\vec{a}+\vec{h})-f(\vec{a}) - \nabla f(\vec{a})\cdot \vec{h}}{||\vec{h}||} = 0.
\end{equation*}
Show that the statement, ``$f$ is continuous at any point at which it is differentiable'' holds for functions of more than one variable.

\noindent {\em Hint:} Note that $\lim\limits_{\vec{x}\rightarrow \vec{a}}f(\vec{x}) = \lim\limits_{\vec{h}\rightarrow \vec{0}}f(\vec{a} + \vec{h})$.
\bigskip

We have
\begin{align*}
\lim_{\vec{x}\rightarrow \vec{a}}f(\vec{x}) &= \lim_{\vec{h}\rightarrow \vec{0}} f(\vec{a}+\vec{h})\\
&= \lim_{\vec{h}\rightarrow \vec{0}} \left(f(\vec{a}+\vec{h}) - f(\vec{a}) - \nabla f(\vec{a})\cdot \vec{h} + (f(\vec{a}) + \nabla f(\vec{a})\cdot \vec{h})\right)\\
&= \lim_{\vec{h}\rightarrow \vec{0}} \left(\frac{f(\vec{a}+\vec{h}) - f(\vec{a}) - \nabla f(\vec{a})\cdot \vec{h}}{||\vec{h}||}||\vec{h}|| + (f(\vec{a}) + \nabla f(\vec{a})\cdot \vec{h})\right)\\
&= 0 + f(\vec{a}) + \nabla(\vec{a})\cdot\vec{0} = f(\vec{a}),
\end{align*}
where in the last line we used the fact that $f$ was differentiable at $\vec{a}$.

Since we've shown that $\lim_{\vec{x}\rightarrow \vec{a}}f(\vec{x}) = f(\vec{a})$, $f$ is continuous at $\vec{a}$.
\end{subproblem}
\newpage
\vglue1pt

Extra space for rough work. Do {\bf not} tear out this page.
\end{problems}
\end{document}


