\documentclass[12pt]{article}
\usepackage{amsmath}
\usepackage{amssymb}
\usepackage{fullpage}
\usepackage{fancyhdr}
\usepackage{lastpage}
%\input quizstyle.tex

\reversemarginpar

\pagestyle{fancy}
\addtolength{\headheight}{\baselineskip}

\lhead{{\bf Date:} Friday, June 8th }
\chead{{\bf Time:} 9:10 pm}
\rhead{MAT232HF}
\cfoot{Page \thepage \ of \pageref{LastPage}}
\rfoot{{\bf Total Marks:} 60}

%Redefine the plain page style in fancyhdr package
\fancypagestyle{plain}{%  
\fancyhead{}
\fancyfoot{}
\fancyfoot[C]{Page \thepage \ of \pageref{LastPage}}
\renewcommand{\headrulewidth}{0pt}}

\newcounter{probnum}
\newcounter{subprobnum}

\DeclareMathOperator{\im}{im}
\DeclareMathOperator{\spn}{span}
\DeclareMathOperator{\col}{col}
\DeclareMathOperator{\rank}{rank}
\DeclareMathOperator{\diag}{diag}
\DeclareMathOperator{\adj}{adj}
\DeclareMathOperator{\nll}{null}
\DeclareMathOperator{\tr}{tr}

\newenvironment{problems}{
\begin{list}{\arabic{probnum}.}{\usecounter{probnum}}
}{
\end{list}
}

\newenvironment{subproblem}{ % start for subprob
\begin{list}{ % first arg for list
(\alph{subprobnum})
}{ % second arg for list
\usecounter{subprobnum}
\setlength{\topsep}{0in}
} % end of list def
}{ % end for subprob
\end{list}
}

\newcommand{\skipline}{\vspace{12pt}}
%\input local.tex
\renewcommand{\labelenumi}{(\roman{enumi})}

\begin{document}
\thispagestyle{plain}
%Supresses the headers on the front page

\centerline {\bf University of Toronto at Mississauga}
\medskip
\centerline {\bf Mid-Term Make-up Exam}
\medskip
\centerline {\bf MAT232HF}
\centerline {\bf Calculus of Several Variables}
\medskip
\centerline {Instructor: Sean Fitzpatrick}
\centerline {Duration: 110 minutes}
\bigskip
\bigskip

\noindent {\bf NO AIDS ALLOWED.} \hfill {\bf Total: 60 marks}
\vglue .25truein
\begin{tabular}{ll}
Family Name: &\underbar {\hskip 4.5in} \\
   &{\hskip 2truein } {\footnotesize (Please Print)}\\
[15pt]
Given Name(s): &\underbar {\hskip 4.5in} \\
    &{\hskip 2truein } {\footnotesize (Please Print)}\\
[15pt]
Please sign here: &\underbar {\hskip 4.5in}\\
[25pt]
Student ID Number: &\underbar {\hskip 4.5in}\\
\end{tabular}
\bigskip


%\vspace{1in}
\begin{quote}
{\large \bf You may not use calculators, cell phones, or PDAs during
the exam.  Partial credit will be given for partially correct work.
Please read through the entire test before starting, and take note of
how many points each question is worth.  Please put a box around your
solutions so that the grader may find them easily.  }
\end{quote}

\vspace{.25in}
\begin{center}
\begin{tabular}{|l|r|}
\hline \hline
\multicolumn{2}{|c|}
{\rule[-3mm]{0mm}{8mm}
FOR MARKER'S USE ONLY} \\
\hline
Problem 1: & \hspace{.5in}  /10 \\ [3pt]
\hline
Problem 2: & \hspace{.5in}  /10 \\ [3pt]
\hline
Problem 3: & \hspace{.5in}  /10 \\ [3pt]
\hline
Problem 4: & \hspace{.5in}  /15 \\ [3pt]
\hline
Problem 5: & \hspace{.5in}  /8 \\ [3pt]
\hline
Problem 6: & \hspace{.5in}  /7 \\ [3pt]
\hline
\hline 
{\rule[-3mm]{0mm}{8mm} TOTAL:}  & /60  \\
\hline
\end{tabular}
\end{center}

\newpage
\vglue1pt
%vglue adds space between headers and text

\begin{problems}
\item   \begin{subproblem}
	\item Obtain the equation of the ellipse with foci at $(\pm 3, 0)$ 
and major axis of length 10, and then sketch the graph.
\marginpar{[5]}

\skipline

Since the foci are at $(\pm 3,0)$ the major axis is along the $x$-axis, 
and the vertices along that axis are $(\pm 5, 0)$ since the centre must be 
at $(0,0)$.  The relationship $5^2 = 3^2 + a^2$, where $a$ is the length 
of the semi-minor axis gives $a=4$, so the other two vertices are at 
$(0,\pm 4)$, and the equation of the ellipse is $\frac{x^2}{25} + 
\frac{y^2}{16} = 1$.  The graph then follows.

\vspace{2in}
	\item Eliminate the parameter to sketch the parametric curve $x(t) 
=\cos 2t$, $y(t) = \sin t$, $t\in [-\pi,\pi]$. Then, 
describe the motion of a particle moving according to these 
equations as $t$ changes. \marginpar{[5]}

\skipline

Since $\cos 2t = 1-2\sin^2 t$ we get $x=1-2y^2$, which is a parabola with 
vertex $(1,0)$, opening to the left.

As $t$ varies from $-\pi$ to $\pi$ the particle starts at the vertex 
$(1,0)$, moves along the parabola to the point $(-1,-1)$ (at $t=-\pi/2$), 
then back to the vertex (at $t=0$), out to the point $(-1, 1)$ 
($t=\pi/2$), and then back again to the vertex at $t=\pi$.
	\end{subproblem}
\newpage
\vglue1pt

\item Recall the identity $\vec{u}\times(\vec{v}\times\vec{w}) = 
(\vec{u}\cdot\vec{w})\vec{v} - (\vec{u}\cdot\vec{v})\vec{w}$.
	\begin{subproblem}
	\item Show that \marginpar{[4]}
\begin{equation*}
(\vec{u}\times\vec{v})\times\vec{w} = (\vec{w}\cdot\vec{u})\vec{v} - 
(\vec{v}\cdot\vec{w})\vec{u}
\end{equation*}

\noindent {\bf Hint:} $\vec{a}\times\vec{b} = -\vec{b}\times\vec{a}$.

\skipline

We have
\begin{align*}
(\vec{u}\times\vec{v})\times\vec{w} &= 
-\vec{w}\times(\vec{u}\times\vec{v})\\
&= -(\vec{w}\cdot\vec{v})\vec{u}+ (\vec{w}\cdot\vec{u})\vec{v}\\
&= (\vec{w}\cdot\vec{u})\vec{v}-(\vec{w}\cdot\vec{v})\vec{u}.
\end{align*}

\vspace{1in}

	\item Let $\vec{a}$ be a given non-zero vector, and 
$\hat{u}$ a 
unit vector.  Show that $\vec{a}$ can be written as $\vec{a} = 
\vec{a}_\parallel + \vec{a}_\bot$, where $\vec{a}_\parallel = 
(\vec{a}\cdot\hat{u})\hat{u}$ is parallel to $\hat{u}$ and $\vec{a}_\bot = 
(\hat{u}\times\vec{a})\times\hat{u}$ is perpendicular to $\vec{a}$. 
\marginpar{[6]}

\skipline

We have
\begin{align*}
\vec{a}_\bot &= (\hat{u}\times\vec{a})\times\hat{u}\\
&= (\hat{u}\cdot\hat{u})\vec{a} - (\hat{u}\cdot\vec{a})\hat{u}\\
&= \vec{a} - \vec{a}_\parallel.
\end{align*}

	\end{subproblem}
\newpage
\vglue1pt
\item In each case, determine whether or not the given line $L$ and plane 
$P$ are 
parallel, or intersect.  If they intersect, find the point of 
intersection.
	\begin{subproblem}
	\item $L:\: x = 7-4t,\: y=3+6t,\: z=9+5t$, and $P:\: 4x+y+2z=17$.
\marginpar{[5]}

\skipline

The direction vector for the line is $\vec{v} = <-4, 6, 5>$, while the normal vector for the plane is $\vec{n} = <4,1,2>$.
Since $\vec{v}\cdot\vec{n} = -4\cdot 4 + 6\cdot 1 + 5\cdot 2 = 0$, the vector is parallel to the plane. ($\vec{n}$ is perpendicular to 
all directions parallel to the plane.)

\vspace{1in}
	\item $L:\: x=3+3t,\: y=6-5t,\: z = 2+3t$, and $P:\: 3x+2y-4z=1$.
\marginpar{[5]}

\skipline

In this case we find $\vec{v}\cdot\vec{n} = <3,-5,3>\cdot <3,2,-4> = -13\neq 0$, so the line is not parallel to the plane.

To find the point of intersection, we need to find the value of $t$ such that $x=3+3t$, $y=6-5t$, and $z=2+3t$ satisfy the equation 
$3x+2y-4z=1$ of the plane.  Subsituting, we find
\begin{equation*}
1 = 3(3+3t) +2(6-5t) -4(2+3t) = 9+12-8 +t(9-10-12) = 13-13t,
\end{equation*}
so we must have $t=12/13$, giving the point of intersection $(75/13,18/13,62/13)$.

	\end{subproblem}
\newpage
\vglue1pt
\item \begin{subproblem}
	\item Find all first-order partial derivatives 
of the following functions:\marginpar{[8]}
	\begin{enumerate}
		\item $f(x,y) = {\displaystyle \frac{x\sin y}{y\cos x}}$.

\skipline

\begin{equation*}
f_x(x,y) = \frac{\sin y}{y\cos x} - \frac{x\sin y}{y\cos^2 x}(-\sin x)
\end{equation*}
\begin{equation*}
f_y(x,y) = \frac{x}{\cos y} - \frac{x\sin y}{y^2\cos x}
\end{equation*}

                \item $g(x,y,z) = \ln(\frac{x}{y})-ye^{xz}$.

\skipline

\begin{equation*}
g_x(x,y,z) = \frac{y}{x}-yze^{xz}
\end{equation*}
\begin{equation*}
g_y(x,y,z) = \frac{y}{x}(-\frac{x}{y^2})-e^{xz}
\end{equation*}
\begin{equation*}
g_z(x,y,z) = -xye^{xz}
\end{equation*}

                \item $h(x,y,z) = xy+yz+zx$.

\skipline

\begin{equation*}
h_x(x,y,z) = y+z
\end{equation*}
\begin{equation*}
h_y(x,y,z) = x+z
\end{equation*}
\begin{equation*}
h_z(x,y,z) = y+x
\end{equation*}

	\end{enumerate}
\newpage
\vglue1pt
	\item Given $u(x,y) = {\displaystyle \frac{xy}{x+y}}$, show that
\marginpar{[7]}
\begin{equation*}
x^2\frac{\partial^2 u}{\partial x^2} + 2xy\frac{\partial^2 u}{\partial 
x\partial y} + y^2\frac{\partial^2 u}{\partial y^2}=0
\end{equation*}

\skipline

I think I will skip typing out the details; I think it's a reasonably straight-forward calculation; any correct derivatives should get 
marks 
even if they don't manage to get 0 in the end.

	\end{subproblem}
\newpage
\vglue1pt

\item Consider the elliptic paraboloid $x^2 + \frac{y^2}{b^2} = z$.
\begin{subproblem}
	\item Describe the trace of this paraboloid in the plane 
$z=1$.\marginpar{[2]}

\skipline

The trace in the plane $z=1$ is the ellipse $x^2 + \frac{y^2}{b^2} = 1$.
\vspace{.6in}
	\item What happens to this trace as 
$b\rightarrow\infty$?\marginpar{[3]}

\skipline

As $b$ gets larger and larger, the above equation describes an ellipse with major axis (for $b>1$) along the $y$ axis getting 
increasingly longer. n In the limit, we are left with just a pair of lines: $x=1$ and $x=-1$.

\vspace{1.5in}
	\item What happens to the paraboloid as 
$b\rightarrow\infty$?\marginpar{[3]}

For other positive values of $z$, the trace is still an ellipse, and in the limit, we get the lines $x=\pm \sqrt{z}$.  At $z=0$ we get 
the 
single line $x=0$, and for $z<0$ we get nothing.  Thus the surface becomes the parabolic cylinder $z=x^2$. 
\end{subproblem}
\newpage
\vglue1pt

\item Show that the limit \marginpar{[7]}
\begin{equation*}
\lim_{(x,y)\rightarrow (1,1)}\frac{x-y^4}{x^3-y^4}
\end{equation*}
does not exist.

\skipline

The limit as $y\rightarrow 1$ along the line $x=1$ is equal to 1, while the limit as $x\rightarrow 1$ along the line $y=1$ is 1/3.  
Thus, the limit cannot exist, as we do not get the same result along these two paths.

\newpage
\vglue1pt
Extra space for rough work. Do {\bf not} tear out this page.
\end{problems}
\end{document}
\newpage

