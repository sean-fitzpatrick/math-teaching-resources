\documentclass[12pt]{article}
\usepackage{amsmath}
\usepackage{amssymb}
\usepackage[letterpaper,margin=0.85in,centering]{geometry}
\usepackage{fancyhdr}
\usepackage{enumerate}
\usepackage{lastpage}
\usepackage{multicol}
\usepackage{graphicx}

\reversemarginpar

\pagestyle{fancy}
\cfoot{Page \thepage \ of \pageref{LastPage}}\rfoot{{\bf Total Points: 40}}
\chead{MATH 53}\lhead{Test \# 2}\rhead{Friday, 15\textsuperscript{th} March, 2013}

\newcommand{\points}[1]{\marginpar{\hspace{24pt}[#1]}}
\newcommand{\skipline}{\vspace{12pt}}
%\renewcommand{\headrulewidth}{0in}
\headheight 30pt

\newcommand{\di}{\displaystyle}
\newcommand{\x}{\mathbf{x}}
\newcommand{\R}{\mathbb{R}}
\newcommand{\aaa}{\mathbf{a}}
\newcommand{\bbb}{\mathbf{b}}
\newcommand{\ccc}{\mathbf{c}}
\newcommand{\dotp}{\boldsymbol{\cdot}}
\newcommand{\pd}[2]{\frac{\partial #1}{\partial #2}}
\newcommand{\rd}[2]{\frac{d #1}{d #2}}
\begin{document}

\author{Instructor: Sean Fitzpatrick}
\thispagestyle{plain}
\begin{center}
\emph{University of California, Berkeley}\\
Department of Mathematics\\
15\textsuperscript{th} March, 2013, 12:10-12:55 pm\\
{\bf MATH 53 - Test \#2}\\
\end{center}
\skipline \skipline \skipline \noindent \skipline
Last Name:\underline{\hspace{100pt}Solutions\hspace{200pt}}\\
\skipline
First Name:\underline{\hspace{100pt}The\hspace{225pt}}\\
\skipline
Student Number:\underline{\hspace{322pt}}\\
\skipline
Discussion Section: \underline{\hspace{307pt}}\\
\skipline
Name of GSI: \underline{\hspace{336pt}}\\

\vspace{0.5in}


\begin{quote}
 {\bf Record your answers below each question in the space provided.    Left-hand pages may be used as scrap paper for rough work.  If you want any work on the left-hand pages to be graded, please indicate so on the right-hand page.
 
 \bigskip
 
Partial credit will be awarded for partially correct work, so be sure to show your work, and include all necessary justifications needed to support your arguments. 

There is a list of potentially useful formulas available on the last page of the exam.}
\end{quote}


\vspace{0.5in}

For grader's use only:

\begin{table}[hbt]
\begin{center}
\begin{tabular}{|l|r|} \hline
Page&Grade\\
\hline \hline
\cline{1-2} 1 & \enspace\enspace\enspace\enspace\enspace\enspace/15\\
\cline{1-2} 2 & \enspace\enspace\enspace\enspace\enspace\enspace/15\\
\cline{1-2} 3 & \enspace\enspace\enspace\enspace\enspace\enspace/10\\
\cline{1-2} Total & \enspace\enspace\enspace\enspace\enspace\enspace/40\\
\hline
\end{tabular}

\skipline

\skipline

\skipline

B
\end{center}
\end{table}
\newpage


\begin{enumerate}
\item Let $f(x,y) = y^2e^{xy}$.
\begin{enumerate}
\item Find the linearization of $f$ at the point $(0,1)$. \points{4}


\bigskip

The partial derivatives of $f$ are $f_x(x,y) = y^3e^{xy}$ and $f_y(x,y) = 2ye^{xy}+xy^2e^{xy}$, so the linearization of $f$ at $(0,1)$ is given by
\begin{align*}
L(x,y) & = f(0,1)+f_x(0,1)(x-0)+f_y(0,1)(y-1)\\
& = 1+x+2(y-1) = x+2y-1.
\end{align*}

\bigskip

\bigskip

\item Find the derivative of $f$ in the direction of $\mathbf{v} = \langle 3,-4\rangle$ at the point $(0,1)$. \points{3}

\bigskip

The directional derivative is given by 
\[
D_{\mathbf{v}}f(0,1) = \frac{\nabla f(0,1)\dotp \mathbf{v}}{\lVert \mathbf{v}\rVert} = \frac{\langle 1,2\rangle \dotp \langle 3,-4\rangle}{\sqrt{3^2+(-4)^2}} = -1.
\]

\bigskip


\item If $x(t)=2-2t$ and $y(t) = t^2$, use the chain rule to find the tangent vector to the curve $\mathbf{r}(t) = \langle x(t),y(t),z(t)\rangle$ when $t=1$, where $z(t)=f(x(t),y(t))$. \points{5}

\bigskip

We have $x'(t) = -2$, $y'(t) = 2t$ and
\[
z'(t) = \frac{d}{dt}f(x(t),y(t)) = \pd{f}{x}\rd{x}{t}+\pd{f}{y}\rd{y}{t}.
\]
When $t=1$, $x(1)=0$, $y(1)=1$, $x'(1)=-2$, and $y'(1) = 2$. Thus, $z'(1) = f_x(0,1)x'(1)+f_y(0,1)y'(1) = 1(-2)+2(2)=2$, so $\mathbf{r}'(1) = \langle -2,2,2\rangle$.

\bigskip

\item Verify that the tangent vector found in part (c) is tangent to the surface $z=f(x,y)$ at the point $(0,1,1)$. \points{3}

\bigskip

The tangent plane is given by $z=L(x,y) = x+2y-1$, so a normal vector is $\mathbf{n} = \langle 1,2,-1\rangle$. Since $\mathbf{n}\dotp \mathbf{r}'(1) = \langle 1,2,-1\rangle\dotp \langle -2,2,2\rangle = -2+4-2=0$, $\mathbf{r}'(1)$ must lie in the tangent plane at $(0,1,1)$.
\end{enumerate}
\newpage

\item Let $f(x,y) = 8x^3+12xy-y^3$.
\begin{enumerate}
\item Find and classify the critical points of $f$. \points{8}

The gradient of $f$ is given by
\[
\nabla f(x,y) = \langle 24x^2+12y, 12x-3y^2\rangle,
\]
which vanishes when $y=-2x^2$ and $4x=y^2$. Plugging the first equation into the second gives $4x=4x^4$, so $x=0$ or $x=1$. This gives us the critical points $(0,0)$ and $(1,-2)$.

The second derivatives of $f$ are given by $f_{xx}(x,y) = 48x$, $f_{xy}(x,y) = 12$, and $f_{yy}(x,y) = -6y$. Thus, we have
\begin{itemize}
\item At $(0,0)$, $A=0, \, B= 12,\, C=0$, and $D = AC-B^2=-144<0$, so $(0,0)$ is a saddle point.
\item At $(1,-2)$, $A=48>0,\, B=12,\, C=12$, and $D = AC-B^2 = 4(12)^2-12^2>0$, so $(1,-2)$ is a local minimum.
\end{itemize}

\bigskip

\item Find the absolute maximum and minimum of $f$ on the set $D$ given by the triangular region with vertices at $(0,0)$, $(1,0)$, and $(1,-2)$. \points{7}

\bigskip

From part (a) we have the critical values $f(0,0)=0$ and $f(1,-2) = -8$, which occur at points in $D$, since $(0,0)$ and $(1,-2)$ are two of the three vertices of the triangle. The value of $f$ at the remaining vertex is $f(1,0) = 8$.

Since there are no critical points on the interior of $D$, it remains to check for any boundary extrema. The boundary consists of three lines: $y=0$, $x=1$, and $y=-2x$. Since we've already checked the end points of these lines (the vertices of the triangle), we just have to check for critical points of the restriction of $f$ to each line.

For $y=0$ we get $f(x,0) = 8x^3$, which has its only critical point when $x=0$, which gives the point $(0,0)$, which we've already checked. For $x=1$ we get $g(y) = f(1,y) = 8+12y-y^3$. We have $g'(y) = 12-3y^2$, so $g$ has critical points when $y=\pm 2$, but $(1,2)$ is not in $D$, and $(1,-2)$ we've already checked.

Finally, if $y=-2x$ we have $h(x) = f(x,-2x) = 16x^3-24x^2$, so $h'(x) = 48x-48x^2$, and $h$ has critical points $x=0$ and $x=1$, corresponding to the points $(0,0)$ and $(1,-2)$, which we've already checked. Having exhausted all possibilities, we conclude that the absolute maximum is $f(1,0)=8$, and the absolute minimum is $f(1,-2)=-8$.
\end{enumerate}
\newpage

\item \begin{enumerate}
\item Define what it means for a function $f(x,y,z)$ to be {\em continuous} at a point $(a,b,c)$ in its domain. \points{2}

\bigskip

A function $f:D\subseteq \R^3\to \R$ is continuous at $(a,b,c)\in D$ if 
\[
\lim_{(x,y,z)\to (a,b,c)}f(x,y,z) = f(a,b,c).
\]

\bigskip


\item Define what it means for a function $f(x,y,z)$ to be {\em differentiable} at a point $(a,b,c)$ in its domain. \points{3}

\bigskip

A function $f:D\subseteq \R^3\to \R$ is differentiable at $(a,b,c)\in D$ if
\[
\lim_{(x,y,z)\to (a,b,c)}\frac{f(x,y,z)-f(a,b,c)-\nabla f(a,b,c)\dotp \langle x-a, y-b, z- c\rangle}{\sqrt{(x-a)^2+(y-b)^2+(z-c)^2}}=0.
\]

\bigskip

\item Show that if $f$ is differentiable at a point $(a,b,c)$, then it is continuous at $(a,b,c)$.\points{5}

{\em Hint:} You can show this using only the above two definitions and the limit laws.

\bigskip

Let $\x=\langle x,y,z\rangle$ be the position vector for $(x,y,z)$, and similarly define $\aaa=\langle a,b,c\rangle$. Suppose that $f$ is differentiable at $\aaa$. Then we have
\begin{align*}
\lim_{\x\to\aaa}f(\x) & = \lim_{\x\to\aaa}\left(f(\x)-f(\aaa)-\nabla f(\aaa)\dotp (\x-\aaa) + \nabla f(\aaa)\dotp (\x-\aaa) + f(\aaa)\right)\\
& = \lim_{\x\to\aaa}\left[\left(\frac{f(\x)-f(\aaa)-\nabla f(\aaa)\dotp (\x-\aaa)}{\lVert \x-\aaa\rVert}\right)(\lVert \x-\aaa\rVert)\right]\\
& \hspace{2in} + \lim_{\x\to\aaa}\nabla f(\aaa)\dotp (\x-\aaa) + \lim_{\x\to\aaa}f(\aaa)\\
& = 0(0) + \nabla f(\aaa)\dotp \mathbf{0} +f(\aaa)\\
& = f(\aaa).
\end{align*}
Thus, $f$ is continuous at $\aaa$.
\end{enumerate}
\end{enumerate}


\end{document}