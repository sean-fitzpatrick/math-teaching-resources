\documentclass[12pt]{article}
\usepackage{amsmath}
\usepackage{amssymb}
\usepackage[letterpaper,margin=0.85in,centering]{geometry}
\usepackage{fancyhdr}
\usepackage{enumerate}
\usepackage{lastpage}
\usepackage{multicol}
\usepackage{graphicx}

%\reversemarginpar

\pagestyle{fancy}
\cfoot{Page \thepage \ of \pageref{LastPage}}\rfoot{{\bf Total Points: 40}}
\chead{MATH 53}\lhead{Test \# 3}\rhead{Friday, 12\textsuperscript{th} April, 2013}

\newcommand{\points}[1]{\marginpar{\hspace{24pt}[#1]}}
\newcommand{\skipline}{\vspace{12pt}}
%\renewcommand{\headrulewidth}{0in}
\headheight 30pt

\newcommand{\di}{\displaystyle}
\newcommand{\R}{\mathbb{R}}
\newcommand{\aaa}{\mathbf{a}}
\newcommand{\bbb}{\mathbf{b}}
\newcommand{\ccc}{\mathbf{c}}
\newcommand{\dotp}{\boldsymbol{\cdot}}
\newcommand{\pd}[2]{\frac{\partial #1}{\partial #2}}
\newcommand{\rd}[2]{\frac{d #1}{d #2}}
\begin{document}

\author{Instructor: Sean Fitzpatrick}
\thispagestyle{plain}
\begin{center}
\emph{University of California, Berkeley}\\
Department of Mathematics\\
12\textsuperscript{th} April, 2013, 12:10-12:55 pm\\
{\bf MATH 53 - Test \#3}\\
\end{center}
\skipline \skipline  \noindent \skipline
Last Name:\underline{\hspace{350pt}}\\
\skipline
First Name:\underline{\hspace{348pt}}\\
\skipline
Student Number:\underline{\hspace{322pt}}

\skipline
\noindent What is your discussion section number (201-215)?\points{1} \underline{\hspace{144pt}}

\skipline
\noindent What is the name of your GSI?\points{1} \underline{\hspace{245pt}} \\

\vspace{0.5in}


\begin{quote}
 {\bf Record your answers below each question in the space provided.    Left-hand pages may be used as scrap paper for rough work.  If you want any work on the left-hand pages to be graded, please indicate so on the right-hand page.
 
 \bigskip
 
Partial credit will be awarded for partially correct work, so be sure to show your work, and include all necessary justifications needed to support your arguments. 

There is a list of potentially useful formulas available on the last page of the exam.}
\end{quote}


\vspace{0.5in}

For grader's use only:

\begin{table}[hbt]
\begin{center}
\begin{tabular}{|l|r|} \hline
Page&Grade\\
\hline \hline
\cline{1-2} 1 & \enspace\enspace\enspace\enspace\enspace\enspace/2\\
\cline{1-2} 2 & \enspace\enspace\enspace\enspace\enspace\enspace/14\\
\cline{1-2} 3 & \enspace\enspace\enspace\enspace\enspace\enspace/12\\
\cline{1-2} 4 & \enspace\enspace\enspace\enspace\enspace\enspace/12\\
\cline{1-2} Total & \enspace\enspace\enspace\enspace\enspace\enspace/40\\
\hline
\end{tabular}

\skipline

\skipline

\skipline

B
\end{center}
\end{table}
\newpage


\begin{enumerate}
\item Evaluate the integral $\di \int_{-2}^2\int_0^{\sqrt{4-y^2}}\sqrt{4-x^2}\,dx\,dy$. \points{6}

(You can do it without reversing the order of integration, but it's not recommended.)


\vspace{3.5in}

\item The integral below computes the area of a region. Sketch the area, and compute it by converting to polar coordinates. \points{6}
\[
\int_{-2}^{-1}\int_{-y}^{\sqrt{8-y^2}}\,dx\,dy+\int_{-1}^{\sqrt{2}}\int_{\sqrt{2-y^2}}^{\sqrt{8-y^2}}\,dx\,dy + \int_{\sqrt{2}}^{2\sqrt{2}}\int_0^{\sqrt{8-y^2}}\,dx\,dy
\]
\newpage
\item Evaluate the integral $\di \int_0^4\int_{-\sqrt{4x-x^2}}^0\sqrt{x^2+y^2}\,dy\,dx$ by converting to polar coordinates. \points{6}

\vspace{3in}

\item Find the centroid (geometric center) of the triangle with vertices $(0,0)$, $(-1,-3)$, and $(1,3)$. \points{8}

\newpage

\item Set up, but do not evaluate, the integral $\di \iiint_E x^2\cos(yz)\,dV$, where $E$ is the tetrahedron with vertices $(1,0,0)$, $(0,3,0)$, and $(0,0,6)$. \points{6}

\vspace{3.5in}

\item Let $E\subseteq \R^3$ be the region bounded below by the cone $z=\sqrt{x^2+y^2}$ and above by the sphere $x^2+y^2+z^2=2z$. Express the volume of $E$ as a triple integral in {\bf both} cylindrical and spherical coordinates. You do not have to compute the volume. \points{6}
\end{enumerate}
\newpage
\begin{center}
List of potentially useful facts and formulas
\end{center}
\begin{itemize}
\item Fubini's Theorem: if $f$ is continuous on the rectangle $R=[a,b]\times[c,d]$ then
\[
\iint_R f(x,y)\,dA = \int_c^d\int_a^bf(x,y)\,dx\,dy = \int_a^b\int_c^d f(x,y)\,dy\,dx.
\]
\item For a Type I region $D$ given by $a\leq x\leq b$, $g(x)\leq y\leq h(x)$,
\[
\iint_D f(x,y)\,dA = \int_a^b\int_{g(x)}^{h(x)}f(x,y)\,dy\,dx.
\]
\item For a Type II region $D$ given by $g(y)\leq x\leq h(y)$, $c\leq y\leq d$,
\[
\iint_D f(x,y)\,dA = \int_c^d\int_{g(y)}^{h(y)}f(x,y)\,dx\,dy.
\]
\item If $D = D_1\cup D_2$, where $D_1,D_2$ intersect along a continuous curve, then
\[
\iint_Df(x,y)\,dA = \iint_{D_1}f(x,y)\,dA + \iint_{D_2}f(x,y)\,dA.
\]
\item Polar coordinates: $x=r\cos\theta$, $y=r\sin\theta$, and
\[
\iint_D f(x,y)\,dA = \int_\alpha^\beta\int_{r_1(\theta)}^{r_2(\theta)}f(r\cos\theta,r\sin\theta)r\,dr\,d\theta.
\]
\item Center of mass: for lamina occupying a region $D$ with a density $\rho(x,y)$,
\[
m = \iint_D\rho(x,y)\,dA,\quad \overline{x} = \frac{1}{m}\iint_D x\rho(x,y)\,dA,\quad \overline{y} = \frac{1}{m}\iint_Dy\rho(x,y)\,dA.
\]
\item Triple integrals: like double integrals, but with one more variable. (Fubini still applies.)
\item Cylindrical coordinates: $x=r\cos\theta$, $y=r\sin\theta$, $z=z$, $dV = r\,dz\,dr\,d\theta$.
\item Spherical coordinates: $x=\rho\cos\theta\sin\phi$, $y=\rho\sin\theta\sin\phi$, $z=\rho\cos\phi$, $dV = \rho^2\sin\phi\,d\rho\,d\phi\,d\theta$.
\item Average value: $\di f_{\mathrm{av}} = \frac{1}{A(D)}\iint_Df(x,y)\,dA$ or $\di f_{\mathrm{av}} = \frac{1}{V(E)}\iiint_Ef(x,y,z)\,dV$, where $A(D)$ and $V(E)$ denote the area of $D$ and volume of $E$, respectively.
\item $\sin^2\theta+\cos^2\theta=1$.
\end{itemize}
\end{document}