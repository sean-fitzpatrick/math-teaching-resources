\documentclass[12pt]{article}
\usepackage{amsmath}
\usepackage{amssymb}
\usepackage{fullpage}
\usepackage{fancyhdr}
\usepackage{lastpage}
%\input quizstyle.tex

\reversemarginpar

\pagestyle{fancy}
\addtolength{\headheight}{\baselineskip}

\lhead{{\bf Date:} 9th October, 2008 }
\chead{{\bf Time:} 9:10 pm}
\rhead{MAT294H1Y}
\cfoot{Page \thepage \ of \pageref{LastPage}}
\rfoot{{\bf Total Marks:} 50}

%Redefine the plain page style in fancyhdr package
\fancypagestyle{plain}{%  
\fancyhead{}
\fancyfoot{}
\fancyfoot[C]{Page \thepage \ of \pageref{LastPage}}
\renewcommand{\headrulewidth}{0pt}}

\newcounter{probnum}
\newcounter{subprobnum}

\DeclareMathOperator{\im}{im}
\DeclareMathOperator{\spn}{span}
\DeclareMathOperator{\col}{col}
\DeclareMathOperator{\rank}{rank}
\DeclareMathOperator{\diag}{diag}
\DeclareMathOperator{\adj}{adj}
\DeclareMathOperator{\nll}{null}
\DeclareMathOperator{\tr}{tr}

\newenvironment{problems}{
\begin{list}{\arabic{probnum}.}{\usecounter{probnum}}
}{
\end{list}
}

\newenvironment{subproblem}{ % start for subprob
\begin{list}{ % first arg for list
(\alph{subprobnum})
}{ % second arg for list
\usecounter{subprobnum}
\setlength{\topsep}{0in}
} % end of list def
}{ % end for subprob
\end{list}
}

\newcommand{\skipline}{\vspace{12pt}}
%\input local.tex
\renewcommand{\labelenumi}{(\roman{enumi})}

\begin{document}
\thispagestyle{plain}
%Supresses the headers on the front page

\centerline {\bf FACULTY OF APPLIED SCIENCE AND ENGINEERING}
\centerline {\bf University of Toronto}
\medskip
\centerline {\bf MAT294H1Y}
\centerline {\bf Calculus and Differential Equations}
\medskip
\centerline {Term Test \#1 - Early sitting}
\centerline {Duration: 110 minutes}
\bigskip
\bigskip

\noindent {\bf NO AIDS ALLOWED.} \hfill {\bf Total: 50 marks}
\vglue .25truein
\begin{tabular}{ll}
Family Name: &\underbar {\hskip 4.5in} \\
   &{\hskip 2truein } {\footnotesize (Please Print)}\\
[15pt]
Given Name(s): &\underbar {\hskip 4.5in} \\
    &{\hskip 2truein } {\footnotesize (Please Print)}\\
[15pt]
Please sign here: &\underbar {\hskip 4.5in}\\
[25pt]
Student ID Number: &\underbar {\hskip 4.5in}\\
\end{tabular}
\bigskip


%\vspace{1in}
\begin{quote}
{\large \bf You may not use calculators, cell phones, or PDAs during
the test.  Partial credit will be given for partially correct work.
Please read through the entire test before starting, and take note of
how many points each question is worth.}
\end{quote}

\vspace{.25in}
\begin{center}
\begin{tabular}{|l|r|}
\hline \hline
\multicolumn{2}{|c|}
{\rule[-3mm]{0mm}{8mm}
FOR MARKER'S USE ONLY} \\
\hline
Problem 1: & \hspace{.5in}  /10 \\ [3pt]
\hline
Problem 2: & \hspace{.5in}  /15 \\ [3pt]
\hline
Problem 3: & \hspace{.5in}  /8\phantom{1} \\ [3pt]
\hline
Problem 4: & \hspace{.5in}  /9\phantom{1} \\ [3pt]
\hline
Problem 5: & \hspace{.5in}  /8\phantom{1} \\ [3pt]
\hline
\hline 
  {\rule[-3mm]{0mm}{8mm} TOTAL:}  & /50  \\
\hline
\end{tabular}
\end{center}

\newpage
\vglue1pt
%vglue adds space between headers and text

\begin{problems}
\item
\begin{subproblem}
\item Calculate the first-order partial derivatives of the following functions:
\begin{enumerate}
\item $f(x,y,z) = z\sin(x-y)$\marginpar{[2]}
\vspace{1.4in}
\item $f(x,y) = x^2y e^{xy}$\marginpar{[2]}
\vspace{1.3in}
\item $\displaystyle f(x,y,z) = \frac{\ln z}{xy}$\marginpar{[2]}
\end{enumerate}
\vspace{1.4in}

\item Find all second-order partial derivatives of the function \marginpar{[4]}
\begin{equation*}
f(x,y) = x^2\tan y + y\ln x.
\end{equation*}
\end{subproblem}

\newpage
\vglue1pt

\item  Let $f(x,y) = x^3-3xy-y^3$. 
\begin{subproblem}
\item Locate all critical points of $f$. \marginpar{[3]}
\vspace{3in}
\item Classify any critcal points found in part (a) as local maxima, local minima, or saddle points. \marginpar{[4]}
\newpage
\vglue1pt
\item Find the maximum and minimum of $f(x,y)$ subject to the constraint $x+2y-1 = 0$.\marginpar{[5]}
\vspace{4.2in}
\item Find the absolute maximum and minumum of $f(x,y)$ on the region bounded by the coordinate axes and the line $x+2y=1$.\marginpar{[3]}

\noindent {\em Hint:} The only thing you still have to check is the value of $f(x,y)$ along the axes.
\end{subproblem}
\newpage
\vglue1pt
\item Let $f(x,y) = \sqrt{x^2+y^2}$.
\begin{subproblem}
\item Find $\nabla f(x,y)$.\marginpar{[2]}
\vspace{1.5in}
\item Find the equation of the tangent plane to the surface $z=\sqrt{x^2+y^2}$ at the point $(3,4)$.\marginpar{[3]}
\vspace{2.7in}
\item Use the differential $df$ to approximate the value of $\sqrt{(2.97)^2+(3.04)^2}$.\marginpar{[3]}

\noindent {\em Note:} $3/5 = 0.6$ and $4/5 = 0.8$.
\end{subproblem}
\newpage
\vglue1pt

\item Let $f(x,y,z) = \ln(x+y+z)$ and let $\vec{r}(t) = \cos^2 t\,\hat{\imath} + \sin^2 t\,\hat{\jmath} + t^2\,\hat{k}$.
\begin{subproblem}
\item Write the chain rule formula for the derivative $\displaystyle \frac{d}{dt}(f(\vec{r}(t)))$. \marginpar{[2]}
\vspace{3in}
\item Use your formula from (a) to evaluate $\displaystyle \frac{d}{dt}(f(\vec{r}(t)))$ when $t=\pi/4$.\marginpar{[2]}
\newpage
\vglue1pt
\item Verify your calculation in part (b) by first making the substitutions $x=\cos^2 t$, $y=\sin^2 t$ and $z=t^2$ in $f$ and then differentiating with respect to $t$. \marginpar{[2]}
\vspace{2.5in}
\item Find the derivative of $f$ in the direction of the curve $\vec{r}(t)$ at the point $(1/2, 1/2, \pi^2/16)$.
\marginpar{[3]}

\noindent {\em Hint:} You've done most of the work for this problem already!
\end{subproblem}

\newpage
\vglue1pt
\item
\begin{subproblem}
\item Show that the limit \marginpar{[3]}
\begin{equation*}
f(x,y) = \lim_{(x,y)\rightarrow (0,0)}\frac{x^2-y^2}{x^2+y^2}
\end{equation*}
does not exist by making the substitution $y=mx$, where $m$ can be any real number.
\newpage
\vglue1pt
\item Recall that $f(\vec{x})$ is differentiable at $\vec{a}$ if and only if
\begin{equation*}
\lim_{\vec{h}\rightarrow \vec{0}}\frac{f(\vec{a}+\vec{h})-f(\vec{a}) - \nabla f(\vec{a})\cdot \vec{h}}{||\vec{h}||} = 0.
\end{equation*}
Show that the statement, ``$f$ is continuous at any point at which it is differentiable'' holds for functions of more than one variable.

\noindent {\em Hint:} Note that $\lim\limits_{\vec{x}\rightarrow \vec{a}}f(\vec{x}) = \lim\limits_{\vec{h}\rightarrow \vec{0}}f(\vec{a} + \vec{h})$.
\end{subproblem}
\newpage
\vglue1pt

Extra space for rough work. Do {\bf not} tear out this page.
\end{problems}
\end{document}


