\documentclass[12pt]{article}
\usepackage{amsmath}
\usepackage{amssymb}
\usepackage[letterpaper,margin=0.85in,centering]{geometry}
\usepackage{fancyhdr}
\usepackage{enumerate}
\usepackage{lastpage}
\usepackage{multicol}
\usepackage{graphicx}

\reversemarginpar

\pagestyle{fancy}
\cfoot{Page \thepage \ of \pageref{LastPage}}\rfoot{{\bf Total Points: 36}}
\chead{MATH 53}\lhead{Test \# 1}\rhead{Friday, 5\textsuperscript{th} October, 2012}

\newcommand{\points}[1]{\marginpar{\hspace{24pt}[#1]}}
\newcommand{\skipline}{\vspace{12pt}}
\newcommand{\vv}{\mathbf{v}}
\newcommand{\dotp}{\boldsymbol{\cdot}}
\newcommand{\len}[1]{\lVert #1\rVert}

%\renewcommand{\headrulewidth}{0in}
\headheight 30pt

\newcommand{\di}{\displaystyle}
\newcommand{\R}{\mathbb{R}}
\begin{document}

\author{Instructor: Sean Fitzpatrick}
\thispagestyle{plain}
\begin{center}
\emph{University of California, Berkeley}\\
Department of Mathematics\\
5\textsuperscript{th} October, 2012, 12:10-12:55 pm\\
{\bf MATH 53 - Test \#1}\\
\end{center}
\skipline \skipline \skipline \noindent \skipline
Last Name:\underline{\hspace{350pt}}\\
\skipline
First Name:\underline{\hspace{348pt}}\\
\skipline
Discussion Section: \underline{\hspace{307pt}}\\
\skipline
Name of GSI: \underline{\hspace{336pt}}\\

\vspace{0.5in}


\begin{quote}
 {\bf Record your answers below each question in the space provided.    Left-hand pages may be used as scrap paper for rough work.  If you want any work on the left-hand pages to be graded, please indicate so on the right-hand page.
 
 \bigskip
 
Partial credit will be awarded for partially correct work, so be sure to show your work, and include all necessary justifications needed to support your arguments.}
\end{quote}


\vspace{0.5in}

For grader's use only:

\begin{table}[hbt]
\begin{center}
\begin{tabular}{|l|r|} \hline
Page&Grade\\
\hline \hline
\cline{1-2} 1 & \enspace\enspace\enspace\enspace\enspace\enspace/12\\
\cline{1-2} 2 & \enspace\enspace\enspace\enspace\enspace\enspace/12\\
\cline{1-2} 3 & \enspace\enspace\enspace\enspace\enspace\enspace/12\\
\cline{1-2} Total & \enspace\enspace\enspace\enspace\enspace\enspace/36\\
\hline
\end{tabular}

\skipline

\skipline

\skipline

\skipline

A
\end{center}
\end{table}
\newpage


\begin{enumerate}
\item Find the equation of the tangent line to the curve $C$ represented by the vector-valued function $\mathbf{r}(t) = \langle e^{2t}, t^2, \sin t\rangle$ at the point $(1,0,0)$.\points{4}

\bigskip

The point $(1,0,0)$ corresponds to the parameter value $t=0$, and $\mathbf{r}'(t) =\langle 2e^{2t}, 2t, \cos t\rangle$, so the tangent vector at $(1,0,0)$ is $\mathbf{r}'(0) = \langle 2, 0, 1\rangle$. The tangent line is thus the line through $(1,0,0)$ in the direction of $\langle 2,0,1\rangle$, so the equation of the line is
\[
 \langle x,y,z\rangle = \langle 1,0,0\rangle +t\langle 2,0,1\rangle.
\]

\bigskip

\bigskip

\item Find the area of one loop of the 4-leaved rose $r=\cos 2\theta$. \points{5}

\bigskip

The right-hand loop of the 4-leaved rose corresponds to the angles $-\dfrac{\pi}{4}\leq \theta\dfrac{\pi}{4}$, so the area is
\begin{align*}
 A & = \int_{-\pi/4}^{\pi/4}\frac{1}{2}\cos^2 2\theta d\theta\\
& = \frac{1}{4}\int_{-\pi/4}^{\pi/4}\left(1+\cos 4\theta\right))d\theta\\
& = \frac{1}{4}\left[\theta - \frac{1}{4}\sin 4\theta\right]_{-\pi/4}^{\pi/4}\\
& = \frac{\pi}{8}.
\end{align*}

\bigskip

\item Show that $\di \lim_{(x,y)\to (0,0)}\frac{xy}{x^2+y^2}$ does not exist. \points{3}

\bigskip

If we evaluate the limit by letting $(x,y)\to (0,0)$ along one of the coordinate axes, we immediately get 0 for the limit. However, if we let $(x,y)\to (0,0)$ along the line $x=y$, we get $\dfrac{xx}{x^2+x^2} = \frac{1}{2}$, so we get a limit of $1/2\neq 0$. Since we get different values along different paths, the limit does not exist.

\newpage

\item Consider the two lines in $\R^3$ given by
\begin{align*}
\mathbf{r}_1(t) & = \langle 2,0,-2\rangle + t\langle 1,0,-2\rangle\\
\mathbf{r}_2(s) & = \langle -2,1,0\rangle +s\langle 3,-1,0\rangle.
\end{align*}
\begin{enumerate}
\item Verify that the two lines intersect at the point $(1,0,0)$. \points{2}

\bigskip

We see by direct computation that 
\[
\mathbf{r}_1(-1) = \langle 2,0,-2\rangle - \langle 1,0,-2\rangle = \langle 1,0,0\rangle 
\]
 and 
\[
 \mathbf{r}_2(1) = \langle -2,1,0\rangle +\langle 3,-1,0\rangle = \langle 1,0,0\rangle.
\]

\bigskip

\item Find the cosine of the angle between the two lines. \points{3}

\bigskip

The directions of the two lines are given by the vectors $\mathbf{v}_1 = \langle 1,0,-2\rangle$ and $\mathbf{v}_2 = \langle 3,-1, 0\rangle$, so the angle between the lines at their point of intersection is given by
\[
 \cos\theta = \frac{\vv_1\dotp\vv_2}{\len{\vv_1}\len{\vv_2}} = \frac{3+0+0}{\sqrt{5}\sqrt{10}} = \frac{3}{\sqrt{50}}.
\]

\bigskip

\item Find the equation of the plane that contains the two lines. \points{4}

\bigskip

We know that the two lines intersect at the point $(1,0,0)$, which must therefore lie on the plane, and a normal vector is given by
\[
 \mathbf{n} = \vv_1\times\vv_2 = \langle 0(0)-(-2)(-1),-2(3)-1(0), 1(-1)-0(3)\rangle = \langle -2, -6, -1\rangle.
\]
The equation of the plane is therefore $-2(x-1)-6y-z=0$, or $2x+6y+z-2=0$.

\bigskip

\item Find the distance between the point $P(2,-1,3)$ and the plane from part (c).\points{3}

\bigskip

The distance from a point $P(x_1,y_1,z_1)$ to a plane $ax+by+cz+d=0$ is given by $D = \dfrac{\lvert ax_1+by_1+cz_1+d\rvert}{\sqrt{a^2+b^2+c^2}}$. Thus,
\[
 D = \frac{\lvert 2(2)+6(-1)+3-2\rvert}{\sqrt{2^2+6^2+1^2}} = \frac{1}{\sqrt{41}}.
\]


\end{enumerate}
\newpage


\item \begin{enumerate}
\item Find the equation of the tangent plane to the surface $z=\sqrt{x^2+y^3}$ at the point $(1,2,3)$. \points{5}

\bigskip

Letting $f(x,y) = \sqrt{x^2+y^2}$, we have $f_x(x,y) = \dfrac{x}{\sqrt{x^2+y^3}}$ and $f_y(x,y) = \dfrac{3y^2}{2\sqrt{x^2+y^3}}$, so the partial derivatives of $f$ at the point $(1,2)$ are $f_1(1,2) = \dfrac{1}{3}$ and $f_2(1,2) = \dfrac{3(2^2)}{2(3)} = 2$. Thus, the equation of the tangent plane to $z=f(x,y)$ at $(1,2,3)$ is
\[
 z = 3 + \frac{1}{3}(x-1)+2(y-2).
\]

\bigskip

\bigskip

\item Use the result from part (a) to approximate the value of $\sqrt{(1.03)^2+(2.05)^3}$. \points{2}

\bigskip

Since the point $(1.03, 2.05)$ is close to the point $(1,2)$, we use the linear approximation $f(x,y)\approx L(x,y)$ (where $z=L(x,y)$ is the equation of the tangent plane) to compute
\[
 \sqrt{(1.03)^2+(2.05)^3}\approx 3+\frac{1}{3}(0.03)+2(0.05) = 3+0.01+0.1 = 3.11.
\]

\end{enumerate} 

\bigskip


\item Use the chain rule to compute $\dfrac{\partial f}{\partial u}$ and $\dfrac{\partial f}{\partial v}$ if $f(x,y,z) = x^2y+y^2z^3$, where $x = v^2$, $y=u^2$, and $z=u^2v^2$. \points{5}

\bigskip

We have
\begin{align*}
 \frac{\partial f}{\partial u} & = \frac{\partial f}{\partial x}\frac{\partial x}{\partial u} + \frac{\partial f}{\partial y}\frac{\partial y}{\partial u} + \frac{\partial f}{\partial z}\frac{\partial z}{\partial u}\\
& = 2xy(0)+ (x^2+2yz^3)(2u) + 3y^2z^2(2uv^2)\\
& = 2uv^4+4u^8v^6+6u^9v^6,
\end{align*}
and
\begin{align*}
 \frac{\partial f}{\partial v} & = \frac{\partial f}{\partial x}\frac{\partial x}{\partial v} + \frac{\partial f}{\partial y}\frac{\partial y}{\partial v} + \frac{\partial f}{\partial z}\frac{\partial z}{\partial v}\\
 & = 2xy(2v)+(x^2+2yz^3)(0)+3y^2z^2(2u^2v)\\
 & = 4u^2v^3+6u^{10}v^5.
\end{align*}




\end{enumerate}



\end{document}