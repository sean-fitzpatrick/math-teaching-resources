\documentclass[12pt]{article}
\usepackage{amsmath}
\usepackage{amssymb}
\usepackage[legalpaper,margin=1in,centering]{geometry}
\usepackage{fancyhdr}
\usepackage{enumerate}
\usepackage{lastpage}

\reversemarginpar

\pagestyle{fancy}
\cfoot{Page \thepage \ of \pageref{LastPage}}\rfoot{{\bf Total Marks: 50}}
\chead{MATH2111}\lhead{Test \# 2}\rhead{Monday, 9\textsuperscript{th} November}
\newcommand{\skipline}{\vspace{12pt}}
%\renewcommand{\headrulewidth}{0in}
\headheight 30pt

\newcommand{\di}{\displaystyle}
\begin{document}

\author{Sean Fitzpatrick}
\thispagestyle{plain}
\begin{center}
\emph{Mount Allison University}\\
Department of Mathematics and Computer Science\\
9\textsuperscript{th} November, 2009\\
{\bf MATH2111 - Test \#2}\\
Time allowed: 45 minutes
\end{center}
\skipline \skipline \skipline \noindent \skipline
Last Name:\underline{\hspace{350pt}}\\
\skipline
First Name:\underline{\hspace{348pt}}\\
\\

\vspace{2in}


\begin{quote}
 {\bf Record your answers below each question in the space provided.    Left-hand pages may be used as scrap paper for rough work.  If you want any work on the left-hand pages to be graded, please indicate so on the right-hand page.
 
 \bigskip
 
Partial credit will be awarded for partially correct work. Be sure to show your work, and include all necessary justifications needed to support your arguments.}
\end{quote}

\vspace{2in}

For grader's use only:

\begin{table}[hbt]
\begin{center}
\begin{tabular}{|l|l|} \hline
Q&Mark\\
\hline \hline
\cline{1-2} 1 & \enspace\enspace\enspace\enspace\enspace\enspace/10\\
\cline{1-2} 2 & \enspace\enspace\enspace\enspace\enspace\enspace/8\\
\cline{1-2} 3 & \enspace\enspace\enspace\enspace\enspace\enspace/12\\
\cline{1-2} 4 & \enspace\enspace\enspace\enspace\enspace\enspace/10\\
\cline{1-2} 5 & \enspace\enspace\enspace\enspace\enspace\enspace/10\\
\cline{1-2} Total & \enspace\enspace\enspace\enspace\enspace\enspace/50\\
\hline
\end{tabular}
\end{center}
\end{table}
\newpage
\begin{enumerate}
\item Compute the first-order partial derivatives of the following functions:
\begin{enumerate}
\item $\di f(x,y) = xe^{3y}$\marginpar{[2]}


\vspace{2in}

\item $\di f(x,y) = \frac{xy+1}{x+y}$\marginpar{[4]}

\vspace{2in}



\item $\di f(x,y) = \sqrt{x^2-y^2}$\marginpar{[4]}

\vspace{2in}
\end{enumerate}
\item Compute all second-order partial derivatives of the function $\di f(x,y) = e^{xy}+xy$.\marginpar{[8]}
\newpage
\item Let $f(x,y) = \dfrac{xy}{x^2+y^2}$.
\begin{enumerate}
 \item Show that $\di \lim_{(x,y)\to (0,0)}f(x,y)$ does not exist.\marginpar{[4]}

\vspace{4in}

 \item If $z=f(x,y)$, compute the differential $dz$. \marginpar{[4]}

\vspace{3in}

 \item Find the linear approximation to $f(x,y)$ at the point $(2,1)$.\marginpar{[4]}

\end{enumerate}
\newpage
\item Find the derivative of $f(x,y)$ at the point $P(a,b)$ in the direction of the given vector $\vec{\bf v}$:
\begin{enumerate}
 \item $\di f(x,y) = x^2e^y$, $P(1,1)$, $\di \vec{\bf v} = \left<\frac{1}{\sqrt{2}},\frac{1}{\sqrt{2}}\right>$.\marginpar{[5]}

\vspace{2in}
 \item $\di f(x,y) = x\cos(y^2)$, $P(-4,0)$, $\di \vec{\bf v} = \left<3,-4\right>$.\marginpar{[5]}

\vspace{2.5in}
\end{enumerate}
\item Let $g(x,y,z) = 4x^2-y^2+9z^2$.
\begin{enumerate}
 \item Compute the gradient $\nabla g(x,y,z)$ of $g$.\marginpar{[3]}

\vspace{1in}

 \item Find the equation of the tangent plane to the level surface $g(x,y,z)=36$ at the point $(1,2,2)$ \marginpar{[4]}

\vspace{1.5in}

 \item Sketch the level surface $g(x,y,z)=36$. \marginpar{[3]}
\end{enumerate}


\end{enumerate}
\end{document}