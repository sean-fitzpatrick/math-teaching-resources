\documentclass[letter, 12pt]{article}
\usepackage{amsmath}
\usepackage{amssymb}
\usepackage[in]{fullpage}
\usepackage{fancyhdr}
\usepackage{lastpage}
%\input quizstyle.tex

\reversemarginpar

\pagestyle{fancy}
\addtolength{\headheight}{\baselineskip}

\lhead{{\bf Date:} 10$^{\text{th}}$ December, 2008 }
\chead{{\bf Time:} 9:30 am}
\rhead{MAT294H1F}
\cfoot{Page \thepage \ of \pageref{LastPage}}
\rfoot{{\bf Total Marks:} 100}

%Redefine the plain page style in fancyhdr package
\fancypagestyle{plain}{%  
\fancyhead{}
\fancyfoot{}
\fancyfoot[C]{Page \thepage \ of \pageref{LastPage}}
\renewcommand{\headrulewidth}{0pt}}

\newcounter{probnum}
\newcounter{subprobnum}

\DeclareMathOperator{\im}{im}
\DeclareMathOperator{\spn}{span}
\DeclareMathOperator{\col}{col}
\DeclareMathOperator{\rank}{rank}
\DeclareMathOperator{\diag}{diag}
\DeclareMathOperator{\adj}{adj}
\DeclareMathOperator{\nll}{null}
\DeclareMathOperator{\tr}{tr}

\newenvironment{problems}{
\begin{list}{\arabic{probnum}.}{\usecounter{probnum}}
}{
\end{list}
}

\newenvironment{subproblem}{ % start for subprob
\begin{list}{ % first arg for list
(\alph{subprobnum})
}{ % second arg for list
\usecounter{subprobnum}
\setlength{\topsep}{0in}
} % end of list def
}{ % end for subprob
\end{list}
}

\newcommand{\skipline}{\vspace{12pt}}
%\input local.tex
\renewcommand{\labelenumi}{(\roman{enumi})}

\begin{document}
\thispagestyle{plain}
%Supresses the headers on the front page

\centerline {\bf University of Toronto}
\centerline {\bf FACULTY OF APPLIED SCIENCE AND ENGINEERING}

\bigskip

\centerline {FINAL EXAMINATION, December 2008}
\centerline {DURATION: $2\frac{1}{2}$ hours}

\medskip

\centerline {Second Year - Materials Science and Engineering}

\medskip

\centerline {\bf MAT294H1F - Calculus and Differential Equations}
 
\medskip

\centerline {Calculator type: 3}
\centerline {Exam type: B}

\medskip

\centerline {Examiner: Sean Fitzpatrick}

\bigskip

\noindent {\bf Total: 100 marks}
\vglue .25truein
\begin{tabular}{ll}
Family Name: &\underbar {\hskip 4.5in} \\
   &{\hskip 2truein } {\footnotesize (Please Print)}\\
[15pt]
Given Name(s): &\underbar {\hskip 4.5in} \\
    &{\hskip 2truein } {\footnotesize (Please Print)}\\
[15pt]
Please sign here: &\underbar {\hskip 4.5in}\\
[25pt]
Student ID Number: &\underbar {\hskip 4.5in}\\
\end{tabular}
\bigskip


\vspace{.5in}
\begin{quote}
{\large \bf No aids, electronic or otherwise, are permitted, with the exception of faculty-approved calculators.  
Partial credit will be given for partially correct work.
Please read through the entire exam before beginning, and take note of
how many points each question is worth.

The second page of the exam contains a list of formulas that may be helpful in the completion of the exam.}
\end{quote}
\newpage

\thispagestyle{empty}
\vspace{.25in}

\begin{tabular}{|l|r|}
\hline
\multicolumn{2}{|c|}
{\rule[-3mm]{0mm}{8mm}
FOR MARKER'S USE ONLY} \\
\hline
Problem 1: & \hspace{.5in}  /12 \\ [3pt]
\hline
Problem 2: & \hspace{.5in}  /10 \\ [3pt]
\hline
Problem 3: & \hspace{.5in}  /18 \\ [3pt]
\hline
Problem 4: & \hspace{.5in}  /15 \\ [3pt]
\hline
Problem 5: & \hspace{.5in}  /15 \\ [3pt]
\hline
Problem 6: & \hspace{.5in}  /15 \\ [3pt]
\hline
Problem 7: & \hspace{.5in}  /15 \\ [3pt]
\hline
\hline 
  {\rule[-3mm]{0mm}{8mm} TOTAL:}  & /100  \\
\hline
\end{tabular}


\addtocounter{page}{-1}
\newpage
{\bf List of relevant formulas}

\bigskip

Coordinate systems:
\begin{align*}
Cylindrical & & Spherical &\\
x &= r\cos\theta & x &=\rho\sin\phi\cos\theta\\
y &= r\sin\theta & y &=\rho\sin\phi\sin\theta\\
z &= z & z &=\rho\cos\phi
\end{align*}

\bigskip

Jacobian:
\[J_T(u,v) = \det\begin{pmatrix}x_u & x_v\\y_u & y_v\end{pmatrix}\]

\bigskip

Integrating factor:
\[u(x) = e^{\int P(x)\,dx}\]

\bigskip

Normal vector:
\[\vec{N}(u,v) = \frac{\partial(y,z)}{\partial(u,v)}\hat{\imath} + \frac{\partial(z,x)}{\partial(u,v)}\hat{\jmath} + \frac{\partial(x,y)}{\partial(u,v)}\hat{k}\]

\bigskip

Fundamental theorem for line integrals:
\[\int_C \nabla f\cdot d\vec{r} = f(\vec{r}(b)) - f(\vec{r}(a))\]

\bigskip

Stokes' theorem:
\[\iint\limits_S(\nabla\times\vec{F})\cdot\vec{n}\,dS = \oint_C \vec{F}\cdot d\vec{r}\]

\bigskip

Divergence theorem:
\[\iiint\limits_T \nabla\cdot\vec{F}\,dV = \int\limits_{\phantom{l}S} \hspace{-.8em} \int
\vec{F}\cdot\vec{n}\,dS\]

\bigskip

Solution of the $n^{th}$-order system $\vec{r}\,'(t) = A\,\vec{r}(t)$:
\[\vec{r}(t) = c_1\vec{X}_1 e^{\lambda_1 t} + c_2\vec{X}_2 e^{\lambda_2 t} + \cdots + c_n\vec{X}_n e^{\lambda_n t}\]

		

\newpage
\vglue1pt
%vglue adds space between headers and text

\begin{problems}
\item Consider the function $f(x,y) = 4xy -2x^4-y^2$.
\begin{subproblem}
\item Calculate $\nabla f(x,y)$. \marginpar{[3]}
\vspace{1.5in}

\item At which points $(x,y)$ is the plane tangent to the surface $z=f(x,y)$ horizontal? \marginpar{[3]}
\vspace{1.5in}

\item Which (if any) of the above points are local maxima? \marginpar{[3]}
\vspace{2in}

\item Set up (but do not solve) \marginpar{[3]}the three equations that would let us find the maximum value of  $f(x,y)$ subject to the constraint $2x^2+3y^2 = 6$.  
\end{subproblem}

\newpage
\vglue1pt

\item  Let $R$ be the region in the first quadrant of the $xy$-plane bounded by the curves $y=x^2$, $y=2x^2$, $x=y^2$ and $x=4y^2$.
\begin{subproblem}
\item Sketch the region $R$. \marginpar{[2]}
\vspace{3in}
\item Determine a transformation $(x,y) = T(u,v)$ such that $R$ is the image under $T$ of a rectangle \marginpar{[2]}
of the form $a\leq u\leq b$, $c\leq v\leq d$, for some $a,b,c,d$.
\newpage
\vglue1pt
\item Compute the Jacobian of the transformation $T$.\marginpar{[3]}
\vspace{3in}
\item Evaluate the integral\marginpar{[3]}
\begin{equation*}
 \iint\limits_R\left(\frac{y^2}{x^4} + \frac{x^2}{y^4}\right)\,dx\,dy.
\end{equation*}
\end{subproblem}
\newpage
\vglue1pt
\item Consider the following two first-order differential equations:
\begin{equation*}
 \text{(I)}\:(-2x+3y)\,dx + x\,dy = 0 \qquad \text{(II)} (3x^2y + e^y)\,dx + (x^3 + xe^y - 2y)\,dy = 0.
\end{equation*}
\begin{subproblem}
\item Classify each of the above equations as either sepearable, linear, or exact.\marginpar{[4]}
\vspace{2in}
\item Find the general solution to equation (I).\marginpar{[4]}
\newpage
\vglue1pt

\item Find the general solution to equation (II).\marginpar{[4]}
\vspace{4in}

\item Evaluate the line integral $\displaystyle \int_C (3x^2y + e^y)\,dx + (x^3 + xe^y - 2y)\,dy$ \marginpar{[6]}, where $C$ is the curve given by $(x(t), y(t)) = (2t^2, -3t)$, for $-1\leq t\leq 1$.\\
You do {\bf not} have to simplify your answer.
\end{subproblem}
\newpage
\vglue1pt

\item Consider the surface $S$ given by $z = 9-x^2-y^2$, for $5\leq z\leq 9$..
\begin{subproblem}
\item Describe $S$ as a parametric surface $\vec{r}(u,v)$. \marginpar{[4]}
\vspace{3in}
\item Using the parameterization from part (a), compute the normal vector $\vec{N}(u,v)$ to the surface $S$.\marginpar{[5]}
\newpage
\vglue1pt
\item Compute the surface area of $S$. \marginpar{[6]}
\end{subproblem}

\newpage
\vglue1pt
\item Let $\vec{F}(x,y,z) = \left<yz,\,-xz,\,z^3\right>$, and let $S$ be the surface given by the part of the cone $z = \sqrt{x^2+y^2}$ between the planes $z=1$ and $z=3$.
\begin{subproblem}
\item Sketch the surface $S$.\marginpar{[3]}
\vspace{2.5in}

\item Calculate $\nabla\times \vec{F}$. \marginpar{[2]}
\newpage
\vglue1pt

\item Use Stokes' theorem to evaluate \marginpar{[6]}
\begin{equation*}
\iint\limits_S\left(\nabla\times\vec{F}\right)\cdot\vec{n}\,dS.
\end{equation*}
\noindent {\bf Hint:} This problem is easiest if you apply Stokes' theorem twice in a clever way to obtain two much simpler surface integrals.  The boundary of $S$ has two seperate parts so be sure to keep track of orientation!
\vspace{4in}

\newpage
\vglue1pt

\item In the special case that $\vec{F} = P\hat{\imath} + Q\hat{\jmath}$ \marginpar{[4]} and that $S$ is a region $R$ in the $xy$-plane, show that Stokes' theorem reduces to Green's theorem.
\end{subproblem}
\newpage
\vglue1pt

\item Let $\vec{F}(x,y,z) = ||\vec{r}||\vec{r}$, where $\vec{r} = x\hat{\imath} + y\hat{\jmath} + z\hat{k}$, and let $S$ be the spherical surface $x^2+y^2+z^2 = 9$.
\begin{subproblem}
\item Evaluate the surface integral $\displaystyle \iint\limits_S \vec{F}\cdot\vec{n}\,dS$ by direct computation. \marginpar{[6]}
\newpage
\vglue1pt

\item Compute $\nabla\cdot\vec{F}$. \marginpar{[3]}
\vspace{2in}

\item Evaluate $\displaystyle \iint\limits_S \vec{F}\cdot\vec{n}\,dS$ using the divergence theorem. \marginpar{[6]}
\end{subproblem}
\newpage
\vglue1pt

\item Solve the following system of linear first order differential equations:
\begin{equation*}
 \vec{r}\,'(t) = (4x+5y)\hat{\imath} + (-2x + 6y)\hat{\jmath}
\end{equation*}
\noindent {\bf Note:} Your solution should involve complex numbers.\\
A correct final answer that still involves complex numbers is worth 10/15.\\
To score 15/15 you must give the correct answer in terms of {\em real numbers only}.
\newpage
\vglue1pt

Extra space for rough work. Do {\bf not} tear out this page.
\end{problems}
\end{document}


