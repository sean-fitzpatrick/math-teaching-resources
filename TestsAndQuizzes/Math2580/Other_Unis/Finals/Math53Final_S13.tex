\documentclass[12pt]{article}
\usepackage{amsmath}
\usepackage{amssymb}
\usepackage[margin=1in, letterpaper]{geometry}
\usepackage{fancyhdr}
\usepackage{lastpage}
\usepackage{enumerate}
%\input quizstyle.tex

%\reversemarginpar

\pagestyle{fancy}
\addtolength{\headheight}{\baselineskip}

\lhead{{\bf Date:} May 15$^{\text{th}}$, 2013 }
\chead{{\bf Time:} 3:00-6:00 pm}
\rhead{MATH 53}
\cfoot{Page \thepage \ of \pageref{LastPage}}
\rfoot{{\bf Total Marks:} 100}

%Redefine the plain page style in fancyhdr package
\fancypagestyle{plain}{%  
\fancyhead{}
\fancyfoot{}
\fancyfoot[C]{Page \thepage \ of \pageref{LastPage}}
\renewcommand{\headrulewidth}{0pt}}

\newcounter{probnum}
\newcounter{subprobnum}

\newcommand{\points}[1]{\marginpar{\hspace{24pt}[#1]}}
\newcommand{\R}{\mathbb{R}}
\renewcommand{\S}{\mathbf{S}}
\renewcommand{\r}{\mathbf{r}}
\newcommand{\dotp}{\boldsymbol{\cdot}}
\renewcommand*{\thefootnote}{\fnsymbol{footnote}}
\newcommand{\di}{\displaystyle}
\newcommand{\F}{\mathbf{F}}
\newcommand{\G}{\mathbf{G}}
\newcommand{\pd}[2]{\frac{\partial #1}{\partial #2}}

\DeclareMathOperator{\im}{im}
\DeclareMathOperator{\spn}{span}
\DeclareMathOperator{\col}{col}
\DeclareMathOperator{\rank}{rank}
\DeclareMathOperator{\diag}{diag}
\DeclareMathOperator{\adj}{adj}
\DeclareMathOperator{\nll}{null}
\DeclareMathOperator{\tr}{tr}
\DeclareMathOperator{\curl}{curl}
\DeclareMathOperator{\Div}{div}


\newenvironment{problems}{
\begin{list}{\arabic{probnum}.}{\usecounter{probnum}}
}{
\end{list}
}

\newenvironment{subproblem}{ % start for subprob
\begin{list}{ % first arg for list
(\alph{subprobnum})
}{ % second arg for list
\usecounter{subprobnum}
\setlength{\topsep}{0in}
} % end of list def
}{ % end for subprob
\end{list}
}

\newcommand{\skipline}{\vspace{12pt}}
%\input local.tex
%\renewcommand{\labelenumi}{(\roman{enumi})}

\begin{document}
\thispagestyle{plain}
%Supresses the headers on the front page

\centerline {\bf University of California, Berkeley}

\bigskip

\centerline {FINAL EXAMINATION, Spring 2013}
\centerline {DURATION: $3$ hours}

\medskip

\centerline {Department of Mathematics}

\medskip

\centerline {{\bf MATH 53 LEC 002} Multivariable Calculus}
 
\medskip

\centerline {Examiner: Sean Fitzpatrick}

\bigskip

\bigskip

\begin{tabular}{ll}
Family Name: &\underbar {\hskip 4.2in} \\
   &{\hskip 2truein } {\footnotesize (Please Print)}\\
[12pt]
Given Name(s): &\underbar {\hskip 4.2in} \\
    &{\hskip 2truein } {\footnotesize (Please Print)}\\
[12pt]
Please sign here: &\underbar {\hskip 4.2in}\\
[12pt]
Student ID Number: &\underbar {\hskip 4.2in}\\
[12pt]
Discussion section: &\underbar {\hskip 4.2in}\\
[12pt]
Name of GSI: &\underbar {\hskip 4.2in}
\end{tabular}
\bigskip


\vspace{.15in}
\begin{quote}
{\large  No aids, electronic or otherwise, are permitted, with the exception of the formula sheet provided with your exam.  
Partial credit will be given for partially correct work. 
Please read through the entire exam before beginning, and take note of
how many points each question is worth.}
\end{quote}
\begin{center}
{\bf Good Luck!}
\end{center}
\vspace{.25in}

\begin{center}
\begin{tabular}{|l|r|}
\hline
\multicolumn{2}{|c|}
{\rule[-3mm]{0mm}{8mm}
FOR GRADER'S USE ONLY} \\
\hline
Problem 1: & \hspace{.5in}  /12 \\ [3pt]
\hline
Problem 2: & \hspace{.5in}  /14 \\ [3pt]
\hline
Problem 3: & \hspace{.5in}  /12 \\ [3pt]
\hline
Problem 4: & \hspace{.5in}  /12 \\ [3pt]
\hline
Problem 5: & \hspace{.5in}  /12 \\ [3pt]
\hline
Problem 6: & \hspace{.5in}  /14 \\ [3pt]
\hline
Problem 7: & \hspace{.5in}  /12 \\ [3pt]
\hline
Problem 8: & \hspace{.5in}  /12 \\ [3pt]
\hline
\hline 
  {\rule[-3mm]{0mm}{8mm} TOTAL:}  & /100  \\
\hline
\end{tabular}
\end{center}


%\newpage
%\vglue1pt
%vglue adds space between headers and text

\begin{enumerate}
\item For parts (a)-(c), let $f(x,y,z) = x^3\sin (yz)$.
\begin{enumerate}
\item Compute the gradient of $f$. \points{3}

\vspace{2in}

\item Compute the directional derivative of $f$ at the point $(2,1,0)$, in the direction of $\mathbf{v} = \langle -2,1,2\rangle$.\points{3}

\vspace{2in}

\item Compute $\di \pd{f}{u}$ and $\di \pd{f}{v}$ if $x = u^2$, $y=v^2$, and $z=u+v$.\points{6}
Your answer should be entirely in terms of $u$ and $v$ but does not have to be simplified.
\end{enumerate}
\newpage

\item \begin{enumerate}
\item What is the geometric meaning of the Lagrange multiplier equations\\ $\nabla f(x,y) = \lambda \nabla g(x,y)$?\points{3}

\vspace{1.5in}

\item Find the maximum and minimum of $f(x,y)=x^2-y^2$ subject to the constraint $x^2/9+y^2/4=1$. \points{7}

\vspace{3in}

\item Show how your answer from (b) illustrates part (a) by sketching the constraint curve from (b), along with a contour plot of $f$ that includes the level curves corresponding to the maximum and minimum values. \points{4}
\end{enumerate}
\newpage

\item Evaluate the following double integrals by either reversing the order of integration, or converting to polar coordinates.
\begin{enumerate}
\item $\di \int_0^4\int_{\sqrt{x}}^2\frac{1}{y^3+1}\,dy\,dx$ \points{6}

\vspace*{3.75in}

\item $\di \int_0^1\int_y^{\sqrt{2-y^2}}(x+y)\,dx\,dy$ \points{6}
\end{enumerate}
\newpage

\item \begin{enumerate}
\item Find the equation of the tangent line to the curve of intersection of the paraboloid $x=y^2+z^2$ with the ellipsoid $x^2+4y^2+z^2=9$ at the point $(2,1,1)$.\points{6}

{\em Hint}: You don't need to find the curve itself in order to determine the tangent line.

\vspace{3.75in}

\item Let $C$ denote the set of points in the intersection of two smooth level surfaces $f(x,y,z)=c$ and $g(x,y,z)=d$. In general, $C$ may not be a smooth curve. \points{6}

What condition on $f$ and $g$ (or the corresponding surfaces) will guarantee that $C$ is a smooth curve?
\end{enumerate}
\newpage

\item Evaluate the integral $\di \iint_D xy\,dA$, where $D$ is the region in the first quadrant bounded by the circles $x^2+y^2=4$ and $x^2+y^2=9$, and the hyperbolas $x^2-y^2=1$ and $x^2-y^2=4$. \points{12}

{\em Hint:} Use an appropriate change of variables. You might find it especially convenient in this problem to use the fact that the Jacobians for a transformation and its inverse are related by $J_T(u,v) = \dfrac{1}{J_{T^{-1}}(x(u,v),y(u,v))}$.

\newpage

\item Let $f(x,y,z)$ be a continuously differentiable function, and let $\F(x,y,z)$ be a continuously differentiable vector field.
\begin{enumerate}
\item Show that $\nabla (f^n) = nf^{n-1}\nabla f$. (Here $f^n$ denotes $f$ raised to the power of $n$.)\points{5}

\vspace*{2.5in}

\item Show that $\nabla\dotp (f\F) = \nabla f\dotp \F+f\nabla\dotp \F$. \points{5}

\vspace*{2.5in}

\item Show that $\nabla\dotp (\rho^n\r)=(n+3)\rho^n$, where $\r(x,y,z) = \langle x,y,z\rangle$ and $\rho(x,y,z) = \sqrt{x^2+y^2+z^2} = \lVert\r(x,y,z)\rVert$. \points{4}
\end{enumerate}

\newpage

\item Let $E$ be the region in $\R^3$ bounded by the sphere $S$ given by $x^2+y^2+z^2=R^2$, and let $\F$ be the vector field defined in problem 6(c), for $n\geq 0$. 
Verify the Divergence Theorem by computing both $\iint_S \F\dotp\,d\S$ and $\iiint_E (\nabla\dotp\F)\,dV$ and confirming that they're equal. \points{12}

\newpage

\item A hot air balloon known as the TARDIS (for Tethered Aerial Release Developed In Style) has the shape of the surface $S$ given by the part of ellipsoid $2x^2+2y^2+z^2=9$ with $-1\leq z\leq 3$. The hot gases that the balloon uses to fly have a velocity vector field given by $\mathbf{v} = \nabla\times \F$, where $\F(x,y,z) = \langle -y, x, xy+z^2\rangle$. The rate at which the gases escape from the balloon is equal to the flux of $\mathbf{v}$ across the surface of the balloon, given by $\di \iint_S \mathbf{v}\dotp\,d\S$. 
 
Sketch the surface\footnote{If you can't figure out how to do the problem, I'll award 3/12 for a correct sketch of the surface $S$, together with the basket underneath and at least one person inside.}, and then use Stokes' Theorem to calculate the rate at which the gases escape from the balloon. \points{12}
\end{enumerate}
\end{document}


