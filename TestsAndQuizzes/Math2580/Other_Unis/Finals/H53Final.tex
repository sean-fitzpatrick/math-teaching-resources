\documentclass[12pt]{article}
\usepackage{amsmath}
\usepackage{amssymb}
\usepackage[margin=1in, letterpaper]{geometry}
\usepackage{fancyhdr}
\usepackage{lastpage}
%\input quizstyle.tex

\reversemarginpar

\pagestyle{fancy}
\addtolength{\headheight}{\baselineskip}

\lhead{{\bf Date:} 11$^{\text{th}}$ May, 2012 }
\chead{{\bf Time:} 7:00-10:00 pm}
\rhead{MATH H53}
\cfoot{Page \thepage \ of \pageref{LastPage}}
\rfoot{{\bf Total Marks:} 100}

%Redefine the plain page style in fancyhdr package
\fancypagestyle{plain}{%  
\fancyhead{}
\fancyfoot{}
\fancyfoot[C]{Page \thepage \ of \pageref{LastPage}}
\renewcommand{\headrulewidth}{0pt}}

\newcounter{probnum}
\newcounter{subprobnum}

\newcommand{\points}[1]{\marginpar{\hspace{24pt}[#1]}}
\newcommand{\R}{\mathbb{R}}

\DeclareMathOperator{\im}{im}
\DeclareMathOperator{\spn}{span}
\DeclareMathOperator{\col}{col}
\DeclareMathOperator{\rank}{rank}
\DeclareMathOperator{\diag}{diag}
\DeclareMathOperator{\adj}{adj}
\DeclareMathOperator{\nll}{null}
\DeclareMathOperator{\tr}{tr}

\newenvironment{problems}{
\begin{list}{\arabic{probnum}.}{\usecounter{probnum}}
}{
\end{list}
}

\newenvironment{subproblem}{ % start for subprob
\begin{list}{ % first arg for list
(\alph{subprobnum})
}{ % second arg for list
\usecounter{subprobnum}
\setlength{\topsep}{0in}
} % end of list def
}{ % end for subprob
\end{list}
}

\newcommand{\skipline}{\vspace{12pt}}
%\input local.tex
\renewcommand{\labelenumi}{(\roman{enumi})}

\begin{document}
\thispagestyle{plain}
%Supresses the headers on the front page

\centerline {\bf University of California, Berkeley}

\bigskip

\centerline {FINAL EXAMINATION, Spring 2012}
\centerline {DURATION: $3$ hours}

\medskip

\centerline {Department of Mathematics}

\medskip

\centerline {{\bf MATH H53} Honors Multivariable Calculus}
 
\medskip

\centerline {Examiner: Sean Fitzpatrick}

\bigskip

\noindent {\bf Total: 100 points}
\vglue .25truein
\begin{tabular}{ll}
Family Name: &\underbar {\hskip 4.2in} \\
   &{\hskip 2truein } {\footnotesize (Please Print)}\\
[15pt]
Given Name(s): &\underbar {\hskip 4.2in} \\
    &{\hskip 2truein } {\footnotesize (Please Print)}\\
[15pt]
Please sign here: &\underbar {\hskip 4.2in}\\
[25pt]
Student ID Number: &\underbar {\hskip 4.2in}\\
\end{tabular}
\bigskip


\vspace{.15in}
\begin{quote}
{\large  No aids, electronic or otherwise, are permitted, with the exception of the formula sheet provided with your exam.  
Partial credit will be given for partially correct work.
Please read through the entire exam before beginning, and take note of
how many points each question is worth.}
\end{quote}
\begin{center}
{\bf Good Luck!}
\end{center}
\vspace{.25in}

\begin{center}
\begin{tabular}{|l|r|}
\hline
\multicolumn{2}{|c|}
{\rule[-3mm]{0mm}{8mm}
FOR GRADER'S USE ONLY} \\
\hline
Problem 1: & \hspace{.5in}  /7 \\ [3pt]
\hline
Problem 2: & \hspace{.5in}  /12 \\ [3pt]
\hline
Problem 3: & \hspace{.5in}  /10 \\ [3pt]
\hline
Problem 4: & \hspace{.5in}  /12 \\ [3pt]
\hline
Problem 5: & \hspace{.5in}  /10 \\ [3pt]
\hline
Problem 6: & \hspace{.5in}  /15 \\ [3pt]
\hline
Problem 7: & \hspace{.5in}  /10 \\ [3pt]
\hline
Problem 8: & \hspace{.5in}  /14 \\ [3pt]
\hline
Problem 9: & \hspace{.5in}  /10 \\ [3pt]
\hline
\hline 
  {\rule[-3mm]{0mm}{8mm} TOTAL:}  & /100  \\
\hline
\end{tabular}
\end{center}


\newpage
\vglue1pt
%vglue adds space between headers and text

\begin{problems}
\item Suppose you need to know an equation of the tangent plane to a surface $S$ at the point $(2,1,3)$. You don't have an equation for the surface $S$, but you know that the curves \points{7}
\begin{align*}
\vec{r}_1(t) & = \langle 2+3t,1-t^2,3-4t+t^2\rangle\\
\vec{r}_2(u) & = \langle 1+u^2, 2u^3-1, 2u+1\rangle
\end{align*}
both lie in $S$. Find an equation of the tangent plane to $S$ at $(2,1,3)$.

\newpage
\vglue1pt

\item Consider the function $f(x,y) = 4xy -2x^4-y^2$.
\begin{subproblem}
\item Find and classify the critical points of $f$. \points{6}

\newpage
\vglue1pt

\item Find the absolute maximum and minimum of $f$ subject to the constraint $2x-y=1$, if they exist. If either the maximum or the minimum does not exist, explain why. \points{6}
\end{subproblem}

\newpage
\vglue1pt

\item Let $T$ be the region in $\R^3$ bounded by the surfaces $z=\sqrt{x^2+y^2}$ and $z=\sqrt{2-x^2-y^2}$. Sketch the region, then compute its volume using whichever coordinate system seems best to you.\points{10}

\newpage
\vglue1pt


\item  Let $D$ be the region in the first quadrant of the $xy$-plane bounded by the curves $y=x^2$, $y=2x^2$, $x=y^2$ and $x=4y^2$.
\begin{subproblem}
\item Sketch the region $D$. \points{3}
\vspace{2in}
\item Evaluate the integral \points{9}
\begin{equation*}
 \iint\limits_D\left(\frac{y^2}{x^4} + \frac{x^2}{y^4}\right)\,dx\,dy.
\end{equation*}
\end{subproblem}
\newpage
\vglue1pt

\item Let $\vec{F} = (2xz+y^2)\hat{\imath}+2xy\hat{\jmath}+(x^2+z^3)\hat{k}$.
\begin{subproblem}
\item Show that $\vec{F}$ is conservative, and find a potential function for $\vec{F}$. \points{5}
\vspace{3in}
\item Compute $\displaystyle \int_C\vec{F}\cdot d\vec{r}$, where $C$ is given by $\vec{r}(t) = \langle t^2,t+1,2t-1\rangle$, with $t\in [0,1]$.\points{5}
\end{subproblem}


\newpage
\vglue1pt

\item Let $\vec{F}$ be a vector field with continuously differentiable components.
\begin{subproblem}
\item Let $S_1$ and $S_2$ be two surfaces with a common boundary $C=\partial S_1=\partial S_2$.\points{5}
Explain, with sketches, how $S_1$ and $S_2$ must be oriented in order to ensure that
\[
\iint_{S_1}(\nabla\times\vec{F})\cdot d\vec{S} = \iint_{S_2}(\nabla\times\vec{F})\cdot d\vec{S}.
\]

\vspace{3.8in}

\item Explain why we should expect that $\displaystyle \iint_S(\nabla\times\vec{F})\cdot d\vec{S}=0$ if $S$ is a closed surface.\points{3}

\newpage

\item Evaluate $\displaystyle \iint_S(\nabla\times \vec{F})\cdot d\vec{S}$ if $S$ is the hemisphere given by $x^2+y^2+z^2=1$, $z\geq 0$, and $\vec{F} = y\hat{\imath}-x\hat{\jmath}+zx^3y^2\hat{k}$.\points{7}
\end{subproblem}
\newpage
\vglue1pt

\item Verify the Divergence Theorem for $\vec{F}(x,y,z) = ||\vec{r}||\vec{r}$, where $\vec{r} = x\hat{\imath} + y\hat{\jmath} + z\hat{k}$, and $S$ is the spherical surface $x^2+y^2+z^2 = 9$.\points{10}
\newpage
\vglue1pt



\item Recall (or look up on your formula sheet) the definition of differentiability discussed at length in class.
\begin{subproblem}
\item Prove that $f(x,y) = x^2+y^2$ is differentiable for all $(x,y)\in\R^2$.\points{5}

\vspace{3in}

\item Discuss the meaning of the definition of the differentiability. \points{9} In particular, how is the derivative related to the original function? How would you describe it geometrically? Explain what you think is most significant, and why. 

\end{subproblem}

\newpage
\vglue1pt



\item A hypocycloid is the curve traced out by a marked point $P$ on a circle of radius $b$ as it rolls without slipping along the interior of a second circle with center $O$ and radius $a>b$.
\begin{subproblem}
\item Determine parametric equations for the hypocycloid. \points{5}

(Hint: Let $O=(0,0)$, and let the initial position of $P$ be $(a,0)$. There should be an obvious choice of parameter.)

\vspace{3.5in}


\item Find the length of the hypocycloid (for one trip around the big circle) in the case $a=4$ and $b=1$.\points{5}
\end{subproblem}
\end{problems}
\end{document}


