\documentclass[12pt]{article}
\usepackage{amsmath}
\usepackage{amssymb}
\usepackage[legalpaper,margin=1in,centering]{geometry}
\usepackage{fancyhdr}
\usepackage{enumerate}
\usepackage{lastpage}

\reversemarginpar

\pagestyle{fancy}
\cfoot{Page \thepage \ of \pageref{LastPage}}\rfoot{{\bf Total Marks: 50}}
\chead{MATH2111}\lhead{Test \# 3}\rhead{Monday, 30\textsuperscript{th} November}
\newcommand{\skipline}{\vspace{12pt}}
%\renewcommand{\headrulewidth}{0in}
\headheight 30pt

\newcommand{\di}{\displaystyle}
\begin{document}

\author{Sean Fitzpatrick}
\thispagestyle{plain}
\begin{center}
\emph{Mount Allison University}\\
Department of Mathematics and Computer Science\\
30\textsuperscript{th} November, 2009, 8:35-9:20 am\\
{\bf MATH2111 - Test \#3}\\
\end{center}
\skipline \skipline \skipline \noindent \skipline
Last Name:\underline{\hspace{350pt}}\\
\skipline
First Name:\underline{\hspace{348pt}}\\
\\

\vspace{2in}


\begin{quote}
 {\bf Record your answers below each question in the space provided.    Left-hand pages may be used as scrap paper for rough work.  If you want any work on the left-hand pages to be graded, please indicate so on the right-hand page.
 
 \bigskip
 
Partial credit will be awarded for partially correct work. Be sure to show your work, and include all necessary justifications needed to support your arguments.}
\end{quote}

\vspace{1in}
Some potentially useful results you may have forgotten from Calc II:
\[
 \int \cos\theta\,d\theta = \sin \theta +c,\,\, \int\sin\theta\,d\theta = -\cos\theta +c,\,\, \sin^2\theta+\cos^2\theta = 1
\]
\vspace{1in}

For grader's use only:

\begin{table}[hbt]
\begin{center}
\begin{tabular}{|l|l|} \hline
Q&Mark\\
\hline \hline
\cline{1-2} 1 & \enspace\enspace\enspace\enspace\enspace\enspace/8\\
\cline{1-2} 2 & \enspace\enspace\enspace\enspace\enspace\enspace/10\\
\cline{1-2} 3 & \enspace\enspace\enspace\enspace\enspace\enspace/16\\
\cline{1-2} 4 & \enspace\enspace\enspace\enspace\enspace\enspace/16\\
\cline{1-2} Total & \enspace\enspace\enspace\enspace\enspace\enspace/50\\
\hline
\end{tabular}
\end{center}
\end{table}
\newpage
\begin{enumerate}
\item Evaluate the following double integral:\marginpar{[8]}
\[
 \iint\limits_D x^2y\, dA, \text{ where } D=\{(x,y)| 0\leq x\leq 1 \text{ and } x\leq y\leq 2-x\}
\]
\vspace{3.6in}
\item For the following double integral,
\begin{enumerate}
 \item Sketch the region $D$ of integration.
 \item Reverse the order of integration by describing $D$ as a vertically simple (rather than horizontally simple) region.\marginpar{[10]}
 \item Evaluate the resulting integral. 
\end{enumerate}
\[
 \int_0^1\int_{3y}^{3}e^{x^2}\,dx\,dy
\]
\newpage
\item Evaluate the integral by converting to polar coordinates:
\begin{enumerate}
 \item \[
        \int_{-2}^{2}\int_0^{\sqrt{4-x^2}}(x+y)\,dy\,dx.
       \]\marginpar{[8]}

 \vspace{4in}

 \item \[
        \int_0^2\int_0^{\sqrt{2x-x^2}}\sqrt{x^2+y^2}\,dy\,dx.
       \] \marginpar{[8]}

\end{enumerate}
\newpage
\item Given a differentiable function $f(x,y)$, we know that if $f(a,b)$ is a maximum or minimum value subject to a constraint $g(x,y)=c$, then
\[
 \nabla f(a,b) = \lambda \nabla g(a,b),\,\,\text{ for some } \lambda\in\mathbb{R}.
\]
\begin{enumerate}
 \item Explain the meaning of the above equation in geometic terms.  \marginpar{[4]} In particular, if $f(a,b)=M$ is the maximum, what can you say about the curves $f(x,y)=M$ and $g(x,y)=c$ at the point $(a,b)$?

\vspace{4in}

 \item Let $f(x,y)=x^2-y^2$.
\begin{enumerate}
 \item Sketch the level curves $f(x,y)=c$, for $c=-2,-1,0,1,2$ on one set of axes. \marginpar{[3]} 

\newpage

 \item On a new set of axes, sketch the curve $x^2+4y^2=4$. \marginpar{[1]}

 \item On the same set of axes used in part (ii), sketch the level curves $f(x,y)=m$ and $f(x,y)=M$ corresponding to the minimum $m$ and maximum $M$ of $f(x,y)$ subject to the constraint $x^2+4y^2=4$. \marginpar{[4]}
\end{enumerate}

\medskip

Hint: no calculation is required.  The geometric meaning (from part (a)) of the Lagrange multiplier condition should tell you where these curves have to intersect the constraint curve $x^2+4y^2=4$, and how they should intersect.

\vspace{5in} 

 \item Either by reading the answer off of your sketch above, or by explicitly solving the Lagrange multiplier equations, give the minimum value $m$ and maximum value $M$ of $f(x,y)$ subject to $x^2+4y^2=4$, and the points at which they occur.\marginpar{[4]}


\end{enumerate}


\end{enumerate}
\end{document}