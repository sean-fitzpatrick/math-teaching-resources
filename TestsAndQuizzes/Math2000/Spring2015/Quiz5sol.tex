\documentclass[12pt]{article}
\usepackage{amsmath}
\usepackage{amssymb}
\usepackage[letterpaper,margin=0.85in,centering]{geometry}
\usepackage{fancyhdr}
\usepackage{enumerate}
\usepackage{lastpage}
\usepackage{multicol}
\usepackage{graphicx}

\reversemarginpar

\pagestyle{fancy}
\cfoot{}
\lhead{Math 2000B}\chead{Quiz \# 5}\rhead{Thursday, 12\textsuperscript{th} February, 2015}
\rfoot{Total: 10 points}
%\chead{{\bf Name:}}
\newcommand{\points}[1]{\marginpar{\hspace{24pt}[#1]}}
\newcommand{\skipline}{\vspace{12pt}}
%\renewcommand{\headrulewidth}{0in}
\headheight 30pt

\newcommand{\di}{\displaystyle}
\newcommand{\R}{\mathbb{R}}
\newcommand{\Z}{\mathbb{Z}}
\newcommand{\aaa}{\mathbf{a}}
\newcommand{\bbb}{\mathbf{b}}
\newcommand{\ccc}{\mathbf{c}}
\newcommand{\modd}[3]{#1\equiv #2\,\pmod{#3}}
\newcommand{\dotp}{\boldsymbol{\cdot}}
\renewcommand{\div}[2]{#1\, |\,#2}

\begin{document}
{\bf Name: Solutions}
%\author{Instructor: Sean Fitzpatrick}
\thispagestyle{fancy}
%\noindent{{\bf Name and student number:}}

\bigskip
Choose {\bf one} of the following two problems:
\begin{enumerate}
 \item Prove the following proposition: For each integer $a$, if $a^2=a$, then $a=0$ or $a=1$.
 
\bigskip

{\bf Solution:} Let $a$ be any integer such that $a^2=a$. We consider two cases: $a=0$ or $a\neq 0$.\footnote{Recall from the Law of the Excluded Middle that $P\vee \neg P$ is a tautology. Thus, in particular, either $a=0$ is true, or its negation, $a\neq 0$ is true. This tautology is frequently used to obtain a proof by cases.}

If $a=0$, then we're done, since we want to show that either $a=0$ or $a=1$. Thus, if $a\neq 0$, it remains to show that we must have $a=1$. But if $a\neq 0$, then we can divide by $a$, and
\[
 a^2=a \quad \Rightarrow \quad \frac{1}{a}(a^2) = \frac{1}{a}(a) \quad  \Rightarrow \quad a = 1,
\]
which is what we needed to show.

\bigskip

\item Use cases based on congruence modulo 3 and properties of congruence\footnote{In particular, you will want the property we proved on Tuesday: if $a\equiv b \pmod{3}$ and $c\equiv d\pmod{3}$, then $ac\equiv bd\pmod{3}$.} to prove that for each integer $n$, $n^3\equiv n \pmod{3}$.

\bigskip

{\bf Solution:} Let $n$ be any integer. From the division algorithm, we know that we must have $\modd{n}{r}{3}$ for exactly one $r\in \{0,1,2\}$.

Case 1: Suppose $\modd{n}{0}{3}$. Then $\modd{n^3}{0^3}{3}$, but $0^3=0$, so $\modd{n^3}{0}{3}$ and $\modd{0}{n}{3}$, from which it follows that $\modd{n^3}{n}{3}$. 

\medskip

Case 2: Suppose $\modd{n}{1}{3}$. Then $\modd{n^3}{1^3}{3}$, but $1^3=1$, so $\modd{n^3}{1}{3}$ and $\modd{1}{n}{3}$, and it follows that $\modd{n^3}{n}{3}$.

\medskip

Case 3: Finally, suppose that $\modd{n}{2}{3}$. Then $\modd{n^3}{2^3}{3}$, and $2^3=8 = 2+3(2)$, so $\modd{2^3}{2}{3}$. Thus $\modd{n^3}{2}{3}$ and $\modd{2}{n}{3}$, and we have $\modd{n^3}{n}{3}$, as required.

\end{enumerate}



\end{document}