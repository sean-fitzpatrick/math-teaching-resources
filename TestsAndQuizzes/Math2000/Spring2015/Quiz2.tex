\documentclass[12pt]{article}
\usepackage{amsmath}
\usepackage{amssymb}
\usepackage[letterpaper,margin=0.85in,centering]{geometry}
\usepackage{fancyhdr}
\usepackage{enumerate}
\usepackage{lastpage}
\usepackage{multicol}
\usepackage{graphicx}

\reversemarginpar

\pagestyle{fancy}
\cfoot{}
\lhead{Math 2000B}\chead{Quiz \# 2}\rhead{Thursday, 22\textsuperscript{nd} January, 2015}
\rfoot{Total: 10 points}
%\chead{{\bf Name:}}
\newcommand{\points}[1]{\marginpar{\hspace{24pt}[#1]}}
\newcommand{\skipline}{\vspace{12pt}}
%\renewcommand{\headrulewidth}{0in}
\headheight 30pt

\newcommand{\di}{\displaystyle}
\newcommand{\R}{\mathbb{R}}
\newcommand{\Z}{\mathbb{Z}}
\newcommand{\aaa}{\mathbf{a}}
\newcommand{\bbb}{\mathbf{b}}
\newcommand{\ccc}{\mathbf{c}}
\newcommand{\dotp}{\boldsymbol{\cdot}}
\begin{document}
{\bf Name:}
%\author{Instructor: Sean Fitzpatrick}
\thispagestyle{fancy}
%\noindent{{\bf Name and student number:}}

 \begin{enumerate}
 \item  Use previously established logical equivalences to prove the following:\points{6}
\[
 P\to (Q\wedge R)\equiv (P\to Q)\wedge (P\to R)
\]

\vspace{4in}

\item For each of the following sets, describe the set in English, and then list the elements of the set using the ``roster method'':
\begin{align*}
 A &= \{x\in \mathbb{Z} : -3\leq x\leq 5\}  & B = \{x\in\R : x^2=4\}\\
 C &= \{2k+1 : k\in \Z\} & D = \{k\in\mathbb{Z} : k \text{ is even}\}
\end{align*}

 \end{enumerate}
\end{document}