\documentclass[12pt]{article}
\usepackage{amsmath}
\usepackage{amssymb}
\usepackage[letterpaper,margin=0.85in,centering]{geometry}
\usepackage{fancyhdr}
\usepackage{enumerate}
\usepackage{lastpage}
\usepackage{multicol}
\usepackage{graphicx}

\reversemarginpar

\pagestyle{fancy}
\cfoot{}
\lhead{Math 2000B}\chead{Quiz \# 8}\rhead{Thursday, 19\textsuperscript{th} March, 2015}
\rfoot{Total: 10 points}
%\chead{{\bf Name:}}
\newcommand{\points}[1]{\marginpar{\hspace{24pt}[#1]}}
\newcommand{\skipline}{\vspace{12pt}}
%\renewcommand{\headrulewidth}{0in}
\headheight 30pt

\newcommand{\di}{\displaystyle}
\newcommand{\R}{\mathbb{R}}
\newcommand{\Z}{\mathbb{Z}}
\newcommand{\aaa}{\mathbf{a}}
\newcommand{\bbb}{\mathbf{b}}
\newcommand{\ccc}{\mathbf{c}}
\newcommand{\dotp}{\boldsymbol{\cdot}}
\renewcommand{\div}[2]{#1\, |\,#2}

\begin{document}
{\bf Name: Solutions}
%\author{Instructor: Sean Fitzpatrick}
\thispagestyle{fancy}
%\noindent{{\bf Name and student number:}}

\bigskip

\begin{enumerate}
 \item Let $T$, $A$, and $B$ be subsets of some universal set $U$. \\
 Prove that if $T\subseteq A$, then $T\times B\subseteq A\times B$. \points{5}

\bigskip

{\bf Proof: } Suppose that $T\subseteq A$. We want to show that $T\times B\subseteq A\times B$. Thus, suppose that $(a,b)\in T\times B$, where $a\in T$ and $b\in B$. Since $a\in T$ and $T\subseteq A$, we must have $a\in A$. But then $a\in A$ and $b\in B$, which implies that $(a,b)\in A\times B$.

Since $(a,b)\in T\times B$ was aribtrary, it follows that $T\times B\subseteq A\times B$.

\bigskip


 \item For each natural number $n$, let $A_n=\{n,n+1,n+2,n+3\}$. Determine the elements of the set $\displaystyle\bigcup_{i=2}^4 A_i$. \points{5}


\bigskip

We need to compute $\displaystyle\bigcup_{i=2}^4 A_i = A_2\cup A_3\cup A_4$. Since
\begin{align*}
 A_2 & = \{2,3,4,5\}\\
 A_3 & = \{3,4,5,6\}\\
 A_4 & = \{4,5,6,7\}
\end{align*}
we see that $\displaystyle\bigcup_{i=2}^4 A_i = \{2,3,4,5,6,7\}$.
\end{enumerate}


\end{document}