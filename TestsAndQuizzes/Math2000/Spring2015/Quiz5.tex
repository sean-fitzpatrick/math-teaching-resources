\documentclass[12pt]{article}
\usepackage{amsmath}
\usepackage{amssymb}
\usepackage[letterpaper,margin=0.85in,centering]{geometry}
\usepackage{fancyhdr}
\usepackage{enumerate}
\usepackage{lastpage}
\usepackage{multicol}
\usepackage{graphicx}

\reversemarginpar

\pagestyle{fancy}
\cfoot{}
\lhead{Math 2000B}\chead{Quiz \# 5}\rhead{Thursday, 12\textsuperscript{th} February, 2015}
\rfoot{Total: 10 points}
%\chead{{\bf Name:}}
\newcommand{\points}[1]{\marginpar{\hspace{24pt}[#1]}}
\newcommand{\skipline}{\vspace{12pt}}
%\renewcommand{\headrulewidth}{0in}
\headheight 30pt

\newcommand{\di}{\displaystyle}
\newcommand{\R}{\mathbb{R}}
\newcommand{\Z}{\mathbb{Z}}
\newcommand{\aaa}{\mathbf{a}}
\newcommand{\bbb}{\mathbf{b}}
\newcommand{\ccc}{\mathbf{c}}
\newcommand{\dotp}{\boldsymbol{\cdot}}
\renewcommand{\div}[2]{#1\, |\,#2}

\begin{document}
{\bf Name:}
%\author{Instructor: Sean Fitzpatrick}
\thispagestyle{fancy}
%\noindent{{\bf Name and student number:}}

\bigskip
Choose {\bf one} of the following two problems:
\begin{enumerate}
 \item Prove the following proposition: For each integer $a$, if $a^2=a$, then $a=0$ or $a=1$.
 \item Use cases based on congruence modulo 3 and properties of congruence\footnote{In particular, you will want the property we proved on Tuesday: if $a\equiv b \pmod{3}$ and $c\equiv d\pmod{3}$, then $ac\equiv bd\pmod{3}$.} to prove that for each integer $n$, $n^3\equiv n \pmod{3}$.

\end{enumerate}



\end{document}