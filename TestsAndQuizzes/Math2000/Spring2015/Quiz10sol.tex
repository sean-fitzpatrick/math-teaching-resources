\documentclass[12pt]{article}
\usepackage{amsmath}
\usepackage{amssymb}
\usepackage[letterpaper,margin=0.85in,centering]{geometry}
\usepackage{fancyhdr}
\usepackage{enumerate}
\usepackage{lastpage}
\usepackage{multicol}
\usepackage{graphicx}

\reversemarginpar

\pagestyle{fancy}
\cfoot{}
\lhead{Math 2000B}\chead{Quiz \# 10}\rhead{Thursday, 2\textsuperscript{nd} April, 2015}
\rfoot{Total: 10 points}
%\chead{{\bf Name:}}
\newcommand{\points}[1]{\marginpar{\hspace{24pt}[#1]}}
\newcommand{\skipline}{\vspace{12pt}}
%\renewcommand{\headrulewidth}{0in}
\headheight 30pt

\newcommand{\di}{\displaystyle}
\newcommand{\R}{\mathbb{R}}
\newcommand{\Z}{\mathbb{Z}}
\newcommand{\aaa}{\mathbf{a}}
\newcommand{\bbb}{\mathbf{b}}
\newcommand{\ccc}{\mathbf{c}}
\newcommand{\dotp}{\boldsymbol{\cdot}}
\renewcommand{\div}[2]{#1\, |\,#2}

\begin{document}
{\bf Name: Solutions}
%\author{Instructor: Sean Fitzpatrick}
\thispagestyle{fancy}
%\noindent{{\bf Name and student number:}}

\bigskip


\begin{enumerate}
 \item For each of the following, given an example\footnote{Arrow diagrams are acceptable, as long as they clearly indicate (a) what the sets $A,B,C$ are, and (b) how the functions are defined.} of functions $f:A\to B$ and $g:B\to C$ that satisfy the stated conditions, or explain why no such example is possible:
\begin{enumerate}
 \item The function $f$ is a surjection, but the function $g\circ f$ is not a surjection. \points{3}

\bigskip

Finte example: take $A=B=\{1\}$, and $C=\{1,2\}$. Define $f:A\to B$ by $f(1)=1$ and $g:B\to C$ by $g(1)=1$. Then $f$ is a surjection, since $\operatorname{range}f = \{1\}=B$. However, we have $g\circ f(1) = g(f(1))=g(1)=1$, so the range of $g\circ f$ is $\{1\}\neq C$, and thus $g\circ f$ is not a surjection.

\medskip

Real-valued example (the one from the back of the textbook): take $A=B=C=\R$ and define $f(x)=x$ and $g(x)=x^2$. Then $f$ is a surjection, but $g\circ f(x) = x^2$ is not, since, for example $-1$ is not in the range of $g\circ f$.

\bigskip

 \item The function $f$ is an injection, but the function $g\circ f$ is not an injection. \points{3}

 \bigskip

Finite example: take $A=B=\{1,2\}$ and $C=\{1\}$. Define $f:A\to B$ by $f(1)=1$ and $f(2)=2$, and define $g:B\to C$ by $g(1)=g(2)=1$. Then $f$ is clearly an injection, but $g\circ f(1) = g\circ f(2) = 1$, so $g\circ f$ is not an injection.

\medskip

Real-valued example: use the same one as above (also from the back of the textbook).

\bigskip

 \end{enumerate}
\item Let $f:A\to B$ and $g:B\to A$ be functions, and let $I_B:B\to B$ denote the identity function on $B$. Prove that if $f\circ g=I_B$, then $f$ is a surjection.\points{4}

\bigskip

{\bf Proof:} Suppose $f\circ g=I_B$, and let $b\in B$ be arbitrary. We need to show that there exists some $a\in A$ such that $f(a)=b$. Since $g$ is a function from $B$ to $A$, we can set $a=g(b)$. Then
\[
 f(a) = f(g(b)) = f\circ g(b) = I_B(b) = b,
\]
which is what we needed to show.

\end{enumerate}


\end{document}