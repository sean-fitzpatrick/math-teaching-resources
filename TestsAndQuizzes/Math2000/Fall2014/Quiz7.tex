\documentclass[12pt]{article}
\usepackage{amsmath}
\usepackage{amssymb}
\usepackage[letterpaper,margin=0.85in,centering]{geometry}
\usepackage{fancyhdr}
\usepackage{enumerate}
\usepackage{lastpage}
\usepackage{multicol}
\usepackage{graphicx}

\reversemarginpar

\pagestyle{fancy}
\cfoot{}
\lhead{Math 2000}\chead{Quiz \# 7}\rhead{Thursday, 30\textsuperscript{th} October, 2014}
\rfoot{Total: 10 points}

\newcommand{\points}[1]{\marginpar{\hspace{24pt}[#1]}}
\newcommand{\skipline}{\vspace{12pt}}
%\renewcommand{\headrulewidth}{0in}
\headheight 30pt

\newcommand{\di}{\displaystyle}
\newcommand{\R}{\mathbb{R}}
\newcommand{\Z}{\mathbb{Z}}
\newcommand{\N}{\mathbb{N}}
\newcommand{\modd}[3]{#1\equiv #2 \pmod{#3}}

\begin{document}

%\author{Instructor: Sean Fitzpatrick}
\thispagestyle{fancy}
\noindent{{\bf Name and student number:}}

 \begin{enumerate}
 \item Let $A=\{a,b,c,d\}$, $B=\{a,b,c\}$, and $C=\{s,t,u,v\}$.
\begin{enumerate}
 \item Create a function $f:A\to C$ whose range is the set $\{u,v\}$, or explain why it is not possible to do so. \points{2}

\vspace{4in}

 \item Create a function $f:B\to C$ whose range is the entire set $C$, or explain why it is not possible to do so. \points{2}
\end{enumerate}
\newpage
\item In each part, you're given sets $A$ and $B$, and a function $f:A\to B$. Determine which functions are one-to-one.
\begin{enumerate}
 \item $A=\{1,2,3\}, B=\{1,2,3,4\}$, and $f(1)=3, f(2)=2, f(3)=1$.\points{1}

\vspace{1.5in}

 \item $A=B=\{1,2,3,4\}$, and $f(1)=2, f(2)=1, f(3)=2, f(4)=1$.\points{1}

\vspace{1.5in}

 \item $A=B=\Z$, and $f(m)=-m$.\points{2}

\vspace{2in}

 \item $A=B=\N$, and $f(n)=n-1$ if $n$ is even, and $f(n)=n+1$, if $n$ is odd.\points{2}

\end{enumerate}
 \end{enumerate}
\end{document}