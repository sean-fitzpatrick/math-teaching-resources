\documentclass[letterpaper,12pt]{article}

\usepackage{ucs}
\usepackage[utf8x]{inputenc}
\usepackage{amsmath}
\usepackage{amsfonts}
\usepackage{amssymb}
\usepackage[margin=1in]{geometry}
\usepackage{multicol}

\newcommand{\N}{\mathbb{N}}
\newcommand{\Z}{\mathbb{Z}}
\newcommand{\R}{\mathbb{R}}
\newcommand{\Q}{\mathbb{Q}}
\newcommand{\modd}[3]{#1\equiv #2 \pmod{#3}}
\newcommand{\abs}[1]{\lvert #1\rvert}

\title{Practice Problems for Quiz 10\\Math 2000A}
\date{}
\begin{document}
 \maketitle
\vspace{-0.5in}

Quiz \#10 will take place in class on Thursday, November 20th. 
As usual, solving the problems on this sheet will significantly improve your chances of getting a high score on the quiz.

{\bf Note to help session tutors}: It's 100\% OK for you to help my students solve these questions.
\begin{enumerate}
\item Let $A$ be a subset of some universal set $U$. Prove that for any $x\in U$, $A\times\{x\}\approx A$.

(Recall that $A\approx B$ means there exists a bijection $f:A\to B$. Note that $A$ is not assumed finite in this problem.)

\item Recall that we defined the notation $\N_k = \{1,2,\ldots, k\}$ for each $k\in\N$.
\begin{enumerate}
 \item Prove that if $k\leq l$, then there is a one-to-one function $f:\N_k\to\N_l$.
 \item Prove that if $A$ and $B$ are finite sets and $\abs{A}\leq \abs{B}$, then there is a one-to-one function $f:A\to B$.
\end{enumerate}
\item Prove the following:
\begin{enumerate}
 \item If $k\geq l$, then there exists an onto function $f:\N_k\to\N_l$.
 \item If $A$ and $B$ are finite sets and $\abs{A}\geq \abs{B}$, then there exists an onto function $f:A\to B$.
\end{enumerate}

\item Prove that each of the following sets is countably infinite:
\begin{enumerate}
 \item $\{n\in \Z : 5|n\}$
 \item $\{m\in \Z : \modd{m}{2}{3}\}$.
 \item $\N\setminus\{4,5,6\}$.
\end{enumerate}
\item Prove that if $A$ is countably infinite, then $A\approx B$ for some proper subset $B\subseteq A$. 

(Hint: let $f:\N\to A$ be a bijection and note that there are many proper subsets $C\subseteq \N$ with $C\approx\N$.)
\item Let $A$ and $B$ be subsets of some universal set $U$.
\begin{enumerate}
 \item Prove that $A\cap B$, $A\setminus B$, and $B\setminus A$ are disjoint.
 \item Prove that $A = (A\cap B)\cup (A\setminus B)$.
 \item Prove that $A\cup B = (A\setminus B)\cup (A\cap B) \cup (B\setminus A)$
 \item Prove that $\abs{A\cup B} = \abs{A}+\abs{B}-\abs{A\cap B}$
 \item Prove that $\abs{A\cup B\cup C} = \abs{A}+\abs{B}+\abs{C}-\abs{A\cap B}-\abs{B\cap C}-\abs{A\cap C} + \abs{A\cap B\cap C}$.

(Hint for part (e): use the fact that $A\cup B\cup C = A\cup (B\cup C)$, and part (d).)
\end{enumerate}
\item If $A$ and $B$ are finite sets, then $\abs{A\times B} = \abs{A}\cdot\abs{B}$. It's possible to show that this result extends to products of finitely many sets: if $A_1, A_2, \ldots, A_n$ are finite sets, then
\[
 \abs{A_1\times A_2\times \cdots \times A_n} = \abs{A_1}\cdot \abs{A_2}\cdot \cdots \cdot \abs{A_n},
\]
where the product $A_1\times A_2\times \cdots \times A_n$ is defined as the set of ``ordered $n$-tuples''
\[
 A_1\times A_2\times \cdots \times A_n = \{(a_1,a_2,\ldots, a_n) | a_i\in A_i \text{ for } i=1,2,\ldots, n\}.
\]
This is a basic counting technique known as the {\em Multiplication Principle}. It can be re-phrased as follows:

Suppose an experiment consists of selecting $n$ objects in order, such that
\begin{itemize}
 \item The first object can be selected in $m_1$ different ways.
 \item For each selction of the first object, the second object can be selected in $m_2$ different ways.
 \item Once the first two objects have been chosen, the third object can be selected in $m_3$ different ways.
\[
 \vdots\hspace{2in}\vdots\hspace{2in}\vdots
\]
 \item Once the first $n-1$ objects have been selected, the last object can be chosen in $m_n$ different ways.
\end{itemize}
Then the total number of different outcomes for the experiment (that is, the total number of ways of choosing the $n$ objects) is equal to $m_1\cdot m_2\cdot\cdots\cdot m_n$.

Based on the above discussion, determine how many ways a sequence of four cards can be chosen from a standard deck of 52 cards if
\begin{enumerate}
 \item Each card is returned to the deck once it's been chosen.
 \item Each card is removed from the deck once it's been chosen.
\end{enumerate}
\item Suppose that every student at University X has a first name, middle name, and last name. How many students must the university have to guarantee that at least two students have the same set of three initials?
\item Prove the strong form of the Pigeonhole Principle: Let $n$, $r$, and $d$ be positive integers.  If $n(r-1)+d$ objects are placed into $n$ boxes, then some box contains at least $r$ objects.

Hint: Use proof by contradiction, and the multiplication principle. If each of the $n$ boxes contains at most $r-1$ objects, then how many objects are there in total?
\item Suppose that in a class of $n$ students, the average on a test exceeded 70\%. Prove that there was some student whose grade was at least 71\%. (Use the result of the previous problem.)

\end{enumerate}



\end{document}
 
