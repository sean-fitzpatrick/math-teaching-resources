\documentclass[letterpaper,12pt]{article}

\usepackage{ucs}
\usepackage[utf8x]{inputenc}
\usepackage{amsmath}
\usepackage{amsfonts}
\usepackage{amssymb}
\usepackage[margin=1in]{geometry}

\title{Practice Problems for Quiz 2\\Math 2000A}
\author{Sean Fitzpatrick}


\begin{document}
 \maketitle

Quiz \#2 will take place in class on Thursday, September 18. The rules for the quiz are the same as for Quiz \#1.

\begin{enumerate}
 \item Give a two-column proof of the following deductions:
 \begin{enumerate}
  \item $A\leftrightarrow (B\wedge C)$, $A\wedge (\neg B\vee D)$, $\therefore (B\wedge D)\vee (A\leftrightarrow C)$
  \item $A\vee B$, $A\to C$, $B\to (\neg C\vee D)$, $\therefore C\vee D$
  \item $(P\vee \neg R)\to R$, $\therefore P\to R$
 \end{enumerate}
 \item From the textbook, Exercise 3.12 (parts 1-6).
 \item Recall that an integer is defined to be {\em even} if it is a multiple of 2; i.e., $n$ is an even integer if $n=2k$ for some integer $k$. Give a two-column proof of the following assertion: 
 
 If $n$ is even and $m$ is any integer, then $nm$ is also even. 

You do not need to use formal logic; instead, each of your justifications should be a hypothesis, a definition, or a known property of integer arithmetic. (For example, you may use the fact that multiplication of integers is {\em associative}: for any integers $a,b,c$ we have $a(bc)=(ab)c$.)
  \item Use proof by contradiction to establish the following:
 \begin{enumerate}
  \item Deduction: $A\to C$, $B\to \neg C$, $\therefore \neg (A\wedge B)$
  \item Claim: there is no smallest rational number. 
  
  (Rational numbers are those that can be written as fractions $a/b$. You may use the following ``theorem'' as one of your justifications: If $r$ is a rational number, then $r/2$ is rational as well.)
 \end{enumerate} 
 \item Two well-known (and yet far too common) logical fallacies are {\em affirming the consequent} ($P\to Q$, $Q$, $\therefore P$) and {\em denying the antecedent} ($P\to Q$, $\neg P$, $\therefore \neg Q$). Show that these deductions are invalid by means of a counterexample.

\end{enumerate}
\end{document}
 
