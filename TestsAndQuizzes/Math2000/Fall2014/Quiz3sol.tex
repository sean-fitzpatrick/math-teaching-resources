\documentclass[12pt]{article}
\usepackage{amsmath}
\usepackage{amssymb}
\usepackage[letterpaper,margin=0.85in,centering]{geometry}
\usepackage{fancyhdr}
\usepackage{enumerate}
\usepackage{lastpage}
\usepackage{multicol}
\usepackage{graphicx}

\reversemarginpar

\pagestyle{fancy}
\cfoot{}
\lhead{Math 2000}\chead{Quiz \# 3}\rhead{Thursday, 2\textsuperscript{nd} October, 2014}
\rfoot{Total: 10 points}

\newcommand{\points}[1]{\marginpar{\hspace{24pt}[#1]}}
\newcommand{\skipline}{\vspace{12pt}}
%\renewcommand{\headrulewidth}{0in}
\headheight 30pt

\newcommand{\di}{\displaystyle}
\newcommand{\R}{\mathbb{R}}
\newcommand{\aaa}{\mathbf{a}}
\newcommand{\bbb}{\mathbf{b}}
\newcommand{\ccc}{\mathbf{c}}
\newcommand{\dotp}{\boldsymbol{\cdot}}
\begin{document}

%\author{Instructor: Sean Fitzpatrick}
\thispagestyle{fancy}
\noindent{{\bf Name and student number:  Solutions}}

 \begin{enumerate}
 \item  Let $A = \{0,3,7\}$ and $B = \{1,2,3,5\}$ be subsets of the universal set $U = \{0,1,2,3,4,5,6,7,8\}$.
\begin{enumerate}
 \item What is $A\cup B$? \points{1}

\bigskip

{\bf Solution}: Since $A\cup B$ is the set of all elements of $U$ that belong to either $A$ or $B$, we have
\[
 A\cup B = \{0,3,7,1,2,5\}.
\]

\bigskip

\item What are the complements $A^c$ and $B^c$? \points{1}

\bigskip

{\bf Solution}: The complement of a set $A\subseteq U$ consists of all elements of $U$ that do not belong to $A$. Thus, we have
\[
 A^c =\{1,2,4,5,6,8\}\quad\text{ and }\quad B^c = \{0,4,6,7,8\}.
\]

\bigskip

\item What is the intersection $A^c\cap B^c$? \points{1}

\bigskip

{\bf Solution}: Since $x\in A^c\cap B^c$ if and only if $x\in A^c$ and $x\in B^c$, $A^c\cap B^c$ consists of all elements of $U$ that are common to the two sets $A^c$ and $B^c$ from part (b). Thus, we have
\[
 A^c\cap B^c = \{4,6,8\}.
\]

\bigskip

\item What is the relationship between your answers in (a) and (c)? \points{1}

\bigskip

{\bf Solution}: We have that $A\cup B = \{0,1,2,3,5,7\}$ and $A^c\cap B^c = \{4,6,8\}$. Thus, $A^c\cap B^c$ consists of all the elements of $U$ that do not belong to $A\cup B$. In other words, $A^c\cap B^c = (A\cup B)^c$.

\bigskip

\item Give {\bf two} subsets of $U$ that are subsets of both $A$ and $B$. \points{1}

\bigskip

{\bf Solution}: We note that $A\cap B = \{3\}$, and thus $\{3\}$ is a subset of both $A$ and $B$. Another subset that is common to both is the empty set $\emptyset$, which is a subset of every set.

\end{enumerate}
\newpage

\item Recall that one of de Morgan's laws for logic is that $\neg (P\vee Q) \equiv \neg P\wedge \neg Q$.
\begin{enumerate}
 \item What is the corresponding de Morgan's law for sets?\points{1}

\bigskip

{\bf Solution}: Let $A$ and $B$ be subsets of some universal set $U$, and let $x$ be some element of $U$. If we let $P$ represent the assertion $x\in A$, and $Q$ represent $x\in B$, then $P\vee Q$ is the assertion that $x\in A$ or $x\in B$, or in other words, $x\in A\cup B$. Thus, $\neg (P\vee Q)$ is the assertion $x\notin (A\cup B)$, or equivalently, $x\in (A\cup B)^c$.

On the other hand, $\neg P\wedge \neg Q$ means $x\notin A$ and $x\notin B$, so $x\in A^c$ and $x\in B^c$, and thus $x\in A^c\cap B^c$ by definition of intersection. Since we are claiming that these two statements are equivalent, we must have $x\in (A\cup B)^c \leftrightarrow x\in A^c\cap B^c$, or $(A\cup B)^c = A^c\cap B^c$.

 \bigskip

 \item Prove that $(A\cup B)^c \subseteq A^c\cup B^c$ ($\overline{A\cup B}\subseteq \overline{A}\cap \overline{B}$ in the textbook notation). \points{3}

Reminder: to prove $U\subseteq V$, assume that $x\in U$ and deduce $x\in V$. A two-column proof is acceptable but not required.

\bigskip

{\bf Solution}: Suppose that $x\in (A\cup B)^c$. Then $x\notin A\cup B$, so it is not the case that $x\in A$ or $x\in B$; that is, $x$ belongs to neither $A$ nor $B$, so $x\notin A$ and $x\notin B$. But this means that $x\in A^c$ and $x\in B^c$, so $x\in A^c\cap B^c$ as required.

\bigskip

 \item What remains to be proved in order to establish your claim in part (a)? \points{1}

\bigskip

{\bf Solution}: Since we've proved $(A\cup B)^c \subseteq A^c\cap B^c$, and two sets are equal if and only if each is a subset of the other, we still have to prove that $A^c\cap B^c$ is a subset of $(A\cup B)^c$.
\end{enumerate}

\end{enumerate}
\end{document}