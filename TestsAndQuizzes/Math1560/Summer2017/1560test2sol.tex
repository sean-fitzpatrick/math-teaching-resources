\documentclass[12pt]{article}
\usepackage{amsmath}
\usepackage{amssymb}
\usepackage[letterpaper,margin=0.85in,centering]{geometry}
\usepackage{fancyhdr}
\usepackage{enumerate}
\usepackage{lastpage}
\usepackage{multicol}
\usepackage{graphicx}

\reversemarginpar

\pagestyle{fancy}
\cfoot{}
\lhead{Math 1560}\chead{Test \# 2 Solutions}\rhead{May 25th, 2017}
%\rfoot{Total: 10 points}
%\chead{{\bf Name:}}
\newcommand{\points}[1]{\marginpar{\hspace{24pt}[#1]}}
\newcommand{\skipline}{\vspace{12pt}}
%\renewcommand{\headrulewidth}{0in}
\headheight 30pt

\newcommand{\di}{\displaystyle}
\newcommand{\abs}[1]{\lvert #1\rvert}
\newcommand{\len}[1]{\lVert #1\rVert}
\renewcommand{\i}{\mathbf{i}}
\renewcommand{\j}{\mathbf{j}}
\renewcommand{\k}{\mathbf{k}}
\newcommand{\R}{\mathbb{R}}
\newcommand{\aaa}{\mathbf{a}}
\newcommand{\bbb}{\mathbf{b}}
\newcommand{\ccc}{\mathbf{c}}
\newcommand{\dotp}{\boldsymbol{\cdot}}
\newcommand{\bbm}{\begin{bmatrix}}
\newcommand{\ebm}{\end{bmatrix}}                   
                  
\begin{document}


%\author{Instructor: Sean Fitzpatrick}
\thispagestyle{fancy}
%\noindent{{\bf Name and student number:}}

 \begin{enumerate}
 \item  Compute the derivatives of the following functions:
\begin{enumerate}
 \item $f(x) = e^x\cos(x)$ \points{2}

\bigskip

\begin{align*}
f'(x) &= \left(\frac{d}{dx}e^x\right)\cos(x)+e^x\left(\frac({d}{dx}\cos(x)\right)\\
& = e^x\cos(x)-e^x\sin(x).
\end{align*}

\bigskip


 \item $g(x) = \dfrac{\sin(x)}{x^2+1}$ \points{2}

\bigskip

\begin{align*}
 g'(x) &= \frac{\left(\frac{d}{dx}\sin(x)\right)(x^2+1)-\sin(x)\left(\frac{d}{dx}(x^2+1)\right)}{(x^2+1)^2}\\
 & = \frac{\cos(x)(x^2+1)-2x\sin(x)}{(x^2+1)^2}.
\end{align*}

\bigskip



 \item $h(x) = \tan^3(x)$ \points{2}

\bigskip

\begin{align*}
  h'(x) & = 3\tan^2(x)\frac{d}{dx}(\tan(x))\\
 & = 3\tan^2(x)\sec^2(x).
\end{align*}

\bigskip

 \item $r(x) = (x^2+1)^x$ \points{2}


\bigskip

Using the fact that $e^{\ln y} = y$ for any $y$, we write $r(x) = e^{\ln(x^2+1)^x} = e^{x\ln(x^2+1)}$. The Chain Rule then gives us

\begin{align*}
 r'(x) & = e^{x\ln(x^2+1)}\frac{d}{dx}(x\ln(x^2+1))\\
 & = (x^2+1)^x\left((1)\ln(x^2+1)+x\left(\frac{2x}{x^2+1}\right)\right).
\end{align*}

\end{enumerate}
\newpage

\item Compute the derivative of $\di f(x)=\ln\left(\sqrt[3]{\frac{x^2(x-3)^3}{(x^4+4x)(2x-1)^4}}\right)$. \points{3}

(Hint: there is an easy way and a hard way.)

\bigskip

Using properties of logarithms, we have
\[
 f(x) = \frac{1}{3}\left(2\ln(x)+3\ln(x-3)-\ln(x^4+4x)-4\ln(2x-1)\right).
\]
Thus, 
\[
 f'(x) = \frac{1}{3}\left(\frac{2}{x}+\frac{3}{x-3}-\frac{4x^3+4}{x^4+4x}-\frac{4(2)}{2x-1}\right).
\]

\bigskip

\item Find the equation of the tangent line to the curve $(x^2+y^2)^2=4xy$ at the point $(1,1)$. \points{3}

(Suggestion: use implicit differentiation.)

\bigskip

Computing the derivative of both sides of the given equation with respect to $x$ (and assuming that $y$ is defined implicitly as a function of $x$), we have

\begin{align*}
 \frac{d}{dx}((x^2+y^2)^2) & = \frac{d}{dx} (4xy)\\
 2(x^2+y^2)\frac{d}{dx}(x^2+y^2) & = 4(1)y + 4x\frac{dy}{dx}\\
 2(x^2+y^2)\left(2x+2y\frac{dy}{dx}\right) & = 4y + 4x\frac{dy}{dx}\\
 \frac{dy}{dx}(4y(x^2+y^2)-4x) & = 4y-4x(x^2+y^2)\\
 \frac{dy}{dx} & = \frac{4y-4x^3-4xy^2}{4yx^2+4y^3-4x}.
\end{align*}
When $x=1$ and $y=1$, we find $\frac{dy}{dx} = \frac{4-4-4}{4+4-4} = -1$. The equation of the tangent line is therefore
\[
 y-1 = -1(x-1).
\]




\item Write down an example of a function that is continuous everywhere, \points{1} but not differentiable everywhere. (Just give the funciton. You don't have to show that it's a valid example.)

\medskip

There are many examples, but the standard one is $f(x)=\abs{x}$. This function is continuous at all points, including $x=0$, but $f'(0)$ is undefined. (Note that $f'(x)=-1$ for $x<0$ and $f'(x)=+1$ for $x>0$.)
\end{enumerate}





\end{document}