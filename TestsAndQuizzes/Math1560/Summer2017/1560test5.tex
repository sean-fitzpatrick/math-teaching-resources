\documentclass[12pt]{article}
\usepackage{amsmath}
\usepackage{amssymb}
\usepackage[letterpaper,margin=0.85in,centering]{geometry}
\usepackage{fancyhdr}
\usepackage{enumerate}
\usepackage{lastpage}
\usepackage{multicol}
\usepackage{graphicx}

\reversemarginpar

\pagestyle{fancy}
\cfoot{}
\lhead{Math 1560}\chead{Test \# 5}\rhead{June 15th, 2017}
%\rfoot{Total: 10 points}
%\chead{{\bf Name:}}
\newcommand{\points}[1]{\marginpar{\hspace{24pt}[#1]}}
\newcommand{\skipline}{\vspace{12pt}}
%\renewcommand{\headrulewidth}{0in}
\headheight 30pt

\newcommand{\di}{\displaystyle}
\newcommand{\abs}[1]{\lvert #1\rvert}
\newcommand{\len}[1]{\lVert #1\rVert}
\renewcommand{\i}{\mathbf{i}}
\renewcommand{\j}{\mathbf{j}}
\renewcommand{\k}{\mathbf{k}}
\newcommand{\R}{\mathbb{R}}
\newcommand{\aaa}{\mathbf{a}}
\newcommand{\bbb}{\mathbf{b}}
\newcommand{\ccc}{\mathbf{c}}
\newcommand{\dotp}{\boldsymbol{\cdot}}
\newcommand{\bbm}{\begin{bmatrix}}
\newcommand{\ebm}{\end{bmatrix}}                   
                  
\begin{document}
{\bf \large Name:} \hspace{2.5in} 

\bigskip

{\bf Note:} Use of scrap paper and/or a basic calculator is permitted.

%\author{Instructor: Sean Fitzpatrick}
\thispagestyle{fancy}
%\noindent{{\bf Name and student number:}}

 \begin{enumerate}
 \item  Calculate the degree 3 Taylor polynomials, centred at $a=0$, for the following functions:
\begin{enumerate}
 \item $f(x) = \ln(x+1)$ \points{3}

\vspace{2in}

 \item $g(x) = e^{2x}$ \points{3}

\vspace{2in}

\end{enumerate}

\item Let $f(x) = 2x^3+\sin(x)-\dfrac{1}{\sqrt{1-x^2}}$. \points{2}\\
Determine the antiderivative $F(x)$ of $f(x)$ such that $F(0)=7$. 

\newpage

\item Estimate the area under the graph of $f(x) = \dfrac{x}{x^2+1}$ between $x=1$ and $x=4$ using 3 rectangles of equal width, if the height of each rectangle is computed using the left endpoint of each interval. \points{3}

\vspace{3.5in}

\item Compute the derivative of $\di f(x) = x\int_1^{x} \sin(t^3+1)\,dt$. \points{2}

\vspace{2in}

\item Evaluate the integral $\di \int_0^4 \left(2x+\frac{1}{\sqrt{x}}\right)\,dx$. \points{3}
\end{enumerate}




\end{document}