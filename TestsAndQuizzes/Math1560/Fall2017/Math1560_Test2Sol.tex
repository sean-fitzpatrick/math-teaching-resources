\documentclass[12pt]{article}
\usepackage{amsmath}
\usepackage{amssymb}
\usepackage[letterpaper,top=1.5in,bottom=1.25in,left=0.75in,right=0.75in,centering]{geometry}
\usepackage{fancyhdr}
\usepackage{enumerate}
\usepackage{lastpage}
\usepackage{multicol}
\usepackage{graphicx}
\usepackage{vwcol}
\reversemarginpar

\pagestyle{fancy}
\cfoot{Page \thepage \ of \pageref{LastPage}}\rfoot{}
\chead{MATH 1560}\lhead{Test \#2 Solutions}\rhead{October 3rd, 2017}

\newcommand{\points}[1]{\marginpar{\hspace{24pt}[#1]}}
\newcommand{\skipline}{\vspace{12pt}}
%\renewcommand{\headrulewidth}{0in}
\headheight 30pt

\newcommand{\di}{\displaystyle}
\newcommand{\abs}[1]{\lvert #1\rvert}
\newcommand{\R}{\mathbb{R}}
\newcommand{\C}{\mathbb{C}}
\renewcommand{\P}{\mathcal{P}}
\DeclareMathOperator{\nul}{null}
\DeclareMathOperator{\range}{range}
\DeclareMathOperator{\spn}{span}
\newcommand{\len}[1]{\lVert #1\rVert}
\newcommand{\Q}{\mathbb{Q}}
\newcommand{\N}{\mathbb{N}}
\renewcommand{\L}{\mathcal{L}}
\newcommand{\dotp}{\boldsymbol{\cdot}}
\newenvironment{amatrix}[1]{%
  \left[\begin{array}{@{}*{#1}{c}|c@{}}
}{%
  \end{array}\right]
}
\newcommand{\bam}{\begin{amatrix}}
\newcommand{\eam}{\end{amatrix}}
\newcommand{\bbm}{\begin{bmatrix}}
\newcommand{\ebm}{\end{bmatrix}}

\begin{document}


 \begin{enumerate}
 \item  Evaluate the limit: \points{3}\\
 \\
\begin{align*}  \lim_{x\to \infty}\,\frac{5+2x-3x^3}{5x^3-4x^2+7}&= \lim_{x\to\infty}\,\frac{x^3(5/x^3+2/x^2-3)}{x^3(5-4/x+7/x^3)}\\
& =\lim_{x\to \infty}\frac{5/x^3+2/x^2-3}{5-4/x+7/x^3}\\
& = \frac{0+0-3}{5+0+0} = -\frac{3}{5}.
\end{align*}
 
 \bigskip
 
 \item Is the function \points{3}
 \[
 f(x) = \begin{cases} 5x-x^2, & \text{ if } x<2\\ 4x-2, & \text{ if } x\geq 2\end{cases}
 \]
 continuous at $x=2$? Why or why not? 
 
\bigskip

By definition, $f$ is continuous at 2 if $\di \lim_{x\to 2}f(x) = f(2)$.

From the formula for $f(x)$ we see that $f(2) =4(2)-2=6$. For the limit, we must consider left and right hand limits. For the left-hand limit, we have
\[
\lim_{x\to 2^-}f(x) = \lim_{x\to 2^-}(5x-x^2) = 5(2)-2^2 = 6.
\]
For the right-hand limit,
\[
\lim_{x\to 2^+}f(x) = \lim_{x\to 2^+}(4x-2) = 4(2)-2 = 6.
\]
Since the left and right hand limits are equal, we can conclude that
\[
\lim_{x\to 2}f(x) = 6 = f(2),
\]
and thus $f$ is continuous at 2.

\bigskip
 
 \item Let $f(x) = \sqrt{x^2+1}$. Write down, but do not evaluate, a limit that computes $f'(0)$ \textit{according to the definition of the derivative}. \points{2}
 
\bigskip

The derivative at a point $x=a$ is defined by 
\[
f'(a) = \lim_{h\to 0}\frac{f(a+h)-f(a)}{h}.
\]
In our case, $a=0$, and $f(x) = \sqrt{x^2+1}$, so $f(0+h) = f(h) = \sqrt{h^2+1}$, while $f(0) = \sqrt{0^2+1}=1$. Thus, we have
\[
f'(0) = \lim_{x\to 0}\frac{\sqrt{h^2+1}-1}{h}.
\]

\item Compute $f'(x)$ for each function $f(x)$ below. You do \textbf{not} need to simplify your answers.
\begin{enumerate}
\item $f(x) = 4x^5-2x^3+\sqrt{2}x-3^4$\points{2}

\bigskip

\[
f'(x) = 20x^4-6x^2+\sqrt{2}.
\]

\bigskip

\item $f(x) = x^3\sin(x)$ \points{2}

\bigskip

\[
f'(x) = \frac{d}{dx}(x^3)\cdot \sin(x) + x^3\frac{d}{dx}(\sin(x)) = 3x^2\sin(x)+x^3\cos(x).
\]

\bigskip

\item $f(x) = \dfrac{x^3-\sqrt{x}}{x^2}$ \points{3}

\bigskip

Notice that we can simplify the function first. Dividing each term in the numerator by $x^2$, we have 
\[
f(x) = \frac{x^3}{x^2}-\frac{x^{1/2}}{x^2} = x-x^{-3/2}.
\]
Thus,
\[
f'(x) = 1+\frac{3}{2}x^{-5/2}.
\]
If you didn't notice this, then the quotient rule gives
\[
f'(x) = \frac{\frac{d}{dx}(x^3-\sqrt{x})\cdot x^2 - (x^3-\sqrt{x})\frac{d}{dx}(x^2)}{(x^2)^2} = \frac{(3x^2-\frac{1}{2}x^{-1/2})x^2-(x^3-\sqrt{x})(2x)}{x^4}.
\]

\item $\di f(x) = e^{\sqrt{x^2+1}}$ \points{3}

\bigskip

We have $f(x) = g(h(k(x)))$ with $g(x)=e^x$, $h(x) = \sqrt{x}$, and $k(x)=x^2+1$. The Chain Rule gives us
\begin{align*}
f'(x) &= g'(h(k(x)))\frac{d}{dx}(h(k(x)))=g'(h(k(x)))h'(k(x))k'(x)\\
& = e^{\sqrt{x^2+1}}(\frac{1}{2}(x^2+1)^{-1/2})(2x).
\end{align*}
If you prefer the Leibniz notation, $y=e^u$, where $u=\sqrt{v}$ and $v=x^2+1$, and
\begin{align*}
\frac{dy}{dx} &= \frac{dy}{du}\frac{du}{dv}\frac{dv}{dx}\\
 & = e^u(\frac{1}{2}v^{-1/2})(2x)\\
 & = e^{\sqrt{x^2+1}}(\frac{1}{2}(x^2+1)^{-1/2})(2x).
\end{align*}
\end{enumerate}

\newpage

\item (Extra group question!) Suppose $f$ and $g$ are continuous functions on an interval $[a,b]$, and you know that $f(a)<g(a)$, and $f(b)>g(b)$. \points{2}

Show that there must be some number $c\in (a,b)$ such that $f(c)=g(c)$.

\medskip

\textit{Hint:} Apply the Intermediate Value Theorem to $h(x)=f(x)-g(x)$. Be sure to justify your work.

\bigskip

Following the hint, we let $h(x)=f(x)-g(x)$. Since $h$ is the difference of two continuous functions, we know that $h$ is continuous on $[a,b]$ as well. Since $f(a)<g(a)$, we see that 
\[
h(a) = f(a)-g(a)<0,
\]
while
\[
h(b) = f(b)-g(b)>0,
\]
since $f(b)>g(b)$. Since $h$ is continuous on $[a,b]$, and $h(a)<0$ while $h(b)>0$, it follows from the Intermediate Value Theorem that there must exist some $c\in (a,b)$ such that 
\[
h(c) = 0 = f(c)-g(c),
\]
and thus $f(c)=g(c)$, as required.
\end{enumerate}
\end{document}