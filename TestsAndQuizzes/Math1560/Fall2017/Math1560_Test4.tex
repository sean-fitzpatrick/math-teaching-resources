\documentclass[12pt]{article}
\usepackage{amsmath}
\usepackage{amssymb}
\usepackage[letterpaper,top=1.35in,bottom=0.75in,left=0.75in,right=0.75in,centering]{geometry}
\usepackage{fancyhdr}
\usepackage{enumerate}
\usepackage{lastpage}
\usepackage{multicol}
\usepackage{graphicx}
\usepackage{vwcol}
\reversemarginpar

\pagestyle{fancy}
\cfoot{Page \thepage \ of \pageref{LastPage}}\rfoot{{\bf Total Points: 20}}
\lhead{\hspace*{2.2in}\underline{MATH 1560: Test 4}}

\newcommand{\points}[1]{\marginpar{\hspace{24pt}[#1]}}
\newcommand{\skipline}{\vspace{12pt}}
\renewcommand{\headrulewidth}{0in}
\headheight 20pt

\newcommand{\di}{\displaystyle}
\newcommand{\abs}[1]{\lvert #1\rvert}
\newcommand{\R}{\mathbb{R}}
\newcommand{\C}{\mathbb{C}}
\renewcommand{\P}{\mathcal{P}}
\DeclareMathOperator{\nul}{null}
\DeclareMathOperator{\range}{range}
\DeclareMathOperator{\spn}{span}
\newcommand{\len}[1]{\lVert #1\rVert}
\newcommand{\Q}{\mathbb{Q}}
\newcommand{\N}{\mathbb{N}}
\renewcommand{\L}{\mathcal{L}}
\newcommand{\dotp}{\boldsymbol{\cdot}}
\newenvironment{amatrix}[1]{%
  \left[\begin{array}{@{}*{#1}{c}|c@{}}
}{%
  \end{array}\right]
}
\newcommand{\bam}{\begin{amatrix}}
\newcommand{\eam}{\end{amatrix}}
\newcommand{\bbm}{\begin{bmatrix}}
\newcommand{\ebm}{\end{bmatrix}}

\begin{document}


\author{Instructor: Sean Fitzpatrick}
\thispagestyle{plain}
\begin{center}
\emph{University of Lethbridge}\\
Department of Mathematics and Computer Science\\
31 October, 2017\\
{\bf MATH 1560 - Test \#4 -- Individual Stage}\\
Examiner: Sean Fitzpatrick
\end{center}
%\skipline \skipline \skipline \noindent \skipline
%Last Name:\underline{\hspace{350pt}}\\
%\skipline
%First Name:\underline{\hspace{348pt}}\\
%\skipline
%Student Number:\underline{\hspace{322pt}}\\
%\skipline



\vspace{0.1in}

\vspace*{\fill}

\begin{quote}
Print your name and student number clearly in the space above. You may remove this cover page, and use the back for scrap paper. If you want any work on the back of this page to be graded, you must clearly indicate this on the page containing the corresponding question.

\medskip

Answer the questions in the space provided. Show all work and necessary justification. Partial credit may be awarded for partially correct work.
 
\medskip

No outside aids are permitted, with the exception of a basic calculator. 
\end{quote}



\newpage

This page left intentionally blank. You may use this page for rough work.
\newpage

 \begin{enumerate}
 \item  Find and classify the critical points of \points{5}
 \[
 f(x) = x^3(1-x)^2.
 \]
 
 \vspace{7cm}
 
 \item A street light is at the top of a 18 foot tall pole. A 6 foot tall circus bear (who has learned to walk upright) walks away from the pole with a speed of 4 ft/sec along a straight path. 
\begin{enumerate}
\item Draw a diagram of the situation. \points{1}
\item At what rate is the length of her shadow increasing when she is 30 feet from the pole?\points{2}
\item How fast is the tip of her shadow moving when she is 30 feet from the pole? \points{2}
\end{enumerate} 
 \newpage
 
 \item Let $f(x) = x^4-4x^3$.
 \begin{enumerate}
 \item Determine a sign diagram for $f(x)$. \points{2} State the domain of $f$, and list any intercepts or asymptotes.
 \item Determine a sign diagram for $f'(x)$. \points{3} State the intervals on which $f$ is increasing or decreasing.
 \item Determine a sign diagram for $f''(x)$.\points{3} State the intervals on which the graph of $f$ is concave up or concave down.
 \item Sketch the graph of $f$.\points{2} Be sure to label all intercepts, critical points, and inflection points.
 \end{enumerate}
\end{enumerate}
\end{document}