\documentclass[12pt]{article}
\usepackage{amsmath}
\usepackage{amssymb}
\usepackage[letterpaper,margin=0.85in,centering]{geometry}
\usepackage{fancyhdr}
\usepackage{enumerate}
\usepackage{lastpage}
\usepackage{multicol}
\usepackage{graphicx}

\reversemarginpar

\pagestyle{fancy}
\cfoot{}
\lhead{Math 3410A}\chead{Quiz \# 6}\rhead{Friday, 6\textsuperscript{th} March, 2015}
\rfoot{Total: 10 points}
%\chead{{\bf Name:}}
\newcommand{\points}[1]{\marginpar{\hspace{24pt}[#1]}}
\newcommand{\skipline}{\vspace{12pt}}
%\renewcommand{\headrulewidth}{0in}
\headheight 30pt

\newcommand{\di}{\displaystyle}
\newcommand{\R}{\mathbb{R}}
\newcommand{\C}{\mathbb{C}}
\newcommand{\vv}{\mathbf{v}}
\newcommand{\aaa}{\mathbf{a}}
\newcommand{\bbb}{\mathbf{b}}
\newcommand{\ccc}{\mathbf{c}}
\newcommand{\dotp}{\boldsymbol{\cdot}}
\DeclareMathOperator{\range}{range}

\begin{document}

%\author{Instructor: Sean Fitzpatrick}
\thispagestyle{fancy}
%\noindent{{\bf Name and student number:}}
{\bf Name: Solutions}

\bigskip

Solve {\bf both} of the following two questions:
 \begin{enumerate}
 \item Suppose that $S,T\in\mathcal{L}(V)$ are such that $ST=TS$. Prove that $\range S$ is invariant under $T$. \points{5}

\bigskip

{\bf Solution:} Let $w\in\range S$. Then $w = Sv$ for some $v\in V$. Thus, assuming that $ST=TS$, we have
\[
 Tw = T(Sv) = S(Tv) \in\range S,
\]
which shows that $\range S$ is invariant under $T$.

\bigskip

 \item Let $T\in\mathcal{L}(\R^2)$ be defined by $T(x,y)=(-3y,x)$.
\begin{enumerate}
 \item What are the eigenvalues of $T$? \points{3}

\bigskip

{\bf Solution:} $\lambda$ is an eigenvalue of $T$ provided that $T(x,y)=(-3y,x)=\lambda (x,y)$ for some nonzero vector $v=(x,y)$. Thus, we must have $-3y = \lambda x$ and $x=\lambda y$, which gives us the system
\begin{align*}
\lambda x + 3y = 0\\
 x - \lambda y = 0
\end{align*}
To have a non-trivial solution the second equation must be a multiple of the first equation.
If we multiply the second equation by $\lambda$ we have $\lambda x - \lambda^2y=0$, which tells us that we must have
\[
 \lambda ^2 = -3.
\]
However, this is impossible if $\lambda$ is a real number, so $T$ does not have any real eigenvalues.

\bigskip

 
 \item Does your answer change if we view $T$ as an operator on $\C^2$ rather than $\R^2$? \points{2}

\bigskip

{\bf Solution:} Yes -- over the complex numbers, the argument above shows that we have the eigenvalues $\lambda = \pm i\sqrt{3}$.
\end{enumerate}

 \end{enumerate}
\end{document}