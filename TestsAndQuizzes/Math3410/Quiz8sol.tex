\documentclass[12pt]{article}
\usepackage{amsmath}
\usepackage{amssymb}
\usepackage[letterpaper,margin=0.85in,centering]{geometry}
\usepackage{fancyhdr}
\usepackage{enumerate}
\usepackage{lastpage}
\usepackage{multicol}
\usepackage{graphicx}

\reversemarginpar

\pagestyle{fancy}
\cfoot{}
\lhead{Math 3410A}\chead{Quiz \# 8}\rhead{Friday, 27\textsuperscript{th} March, 2015}
\rfoot{Total: 10 points}
%\chead{{\bf Name:}}
\newcommand{\points}[1]{\marginpar{\hspace{24pt}[#1]}}
\newcommand{\skipline}{\vspace{12pt}}
%\renewcommand{\headrulewidth}{0in}
\headheight 30pt

\newcommand{\di}{\displaystyle}
\newcommand{\R}{\mathbb{R}}
\newcommand{\C}{\mathbb{C}}
\newcommand{\vv}{\mathbf{v}}
\newcommand{\aaa}{\mathbf{a}}
\newcommand{\bbb}{\mathbf{b}}
\newcommand{\ccc}{\mathbf{c}}
\newcommand{\dotp}{\boldsymbol{\cdot}}
\DeclareMathOperator{\range}{range}
\DeclareMathOperator{\nul}{null}
\newcommand{\len}[1]{\lVert #1\rVert}

\begin{document}

%\author{Instructor: Sean Fitzpatrick}
\thispagestyle{fancy}
%\noindent{{\bf Name and student number:}}
{\bf Name: Solutions}

\bigskip

Solve the following {\bf two} questions. 
 \begin{enumerate}
 \item Suppose that $u,v\in V$ are such that $\len{u}=3$, $\len{u+v}=4$, and $\len{u-v}=6$. What is the value of $\len{v}$? \points{5}

\bigskip

We note that
\begin{align*}
 \len{u+v}^2+\len{u-v}^2 & = \langle u+v,u+v\rangle + \langle u-v,u-v\rangle\\
& = \len{u}^2+\langle u,v\rangle+\langle v,u\rangle+\len{v}^2+\len{u}^2-\langle u,v\rangle-\langle v,u\rangle +\len{v}^2\\
& = 2\len{u}^2+2\len{v}^2.
\end{align*}
Subsitutiting $\len{u}=3$, $\len{u+v}=4$, and $\len{u-v}=6$, we find
\[
 4^2+6^2 = 2(3^2)+2\len{v}^2,
\]
which gives $2\len{v}^2 = 16+36-18 = 34$, so $\len{v}^2=17$, and thus $\len{v}=\sqrt{17}$.

\bigskip

 \item Prove that for all positive numbers $a,b,c,d\in\R$, we have \points{5}
\[
 16\leq (a+b+c+d)\left(\frac{1}{a}+\frac{1}{b}+\frac{1}{c}+\frac{1}{d}\right).
\]

\bigskip

Consider the vectors $u=(\sqrt{a},\sqrt{b},\sqrt{c},\sqrt{d})$ and $v = \left(\dfrac{1}{\sqrt{a}},\dfrac{1}{\sqrt{b}},\dfrac{1}{\sqrt{c}},\dfrac{1}{\sqrt{d}}\right)$ in $\R^4$. We have $\langle u,v\rangle = 1+1+1+1=4$, $\len{u}^2=a+b+c+d$, and $\len{v}^2 = \dfrac{1}{a}+\dfrac{1}{b}+\dfrac{1}{c}+\dfrac{1}{d}$.

Since the Cauchy-Schwarz inequality guarantees that $\langle u,v\rangle^2\leq \len{u}^2\len{v}^2$, the result follows.

\bigskip

\item ({\bf Bonus}) Suppose that $V$ is a real inner product space. Prove that \points{5}
\[
 \langle u,v\rangle = \frac{\len{u+v}^2-\len{u-v}^2}{4}
\]
for all $u,v\in V$.

\bigskip

For any $u,v\in V$, we have (noting that $\langle u,v\rangle = \langle v,u\rangle$, since $V$ is a real inner product space)
\begin{align*}
 \len{u+v}^2-\len{u-v}^2 & = \langle u+v,u+v\rangle - \langle u-v,u-v\rangle\\
& = \langle u,u\rangle+ 2\langle u,v\rangle + \langle v,v\rangle -(\langle u,u\rangle - 2\langle u,v\rangle +\langle v,v\rangle)\\
& = 4\langle u,v\rangle,
\end{align*}
and the result follows upon dividing both sides of the equality by 4.
 \end{enumerate}
\end{document}