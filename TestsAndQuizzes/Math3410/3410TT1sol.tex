\documentclass[12pt]{article}
\usepackage{amsmath}
\usepackage{amssymb}
\usepackage[letterpaper,margin=0.85in,centering]{geometry}
\usepackage{fancyhdr}
\usepackage{enumerate}
\usepackage{lastpage}
\usepackage{multicol}
\usepackage{graphicx}

\reversemarginpar

\pagestyle{fancy}
\cfoot{Page \thepage \ of \pageref{LastPage}}\rfoot{{\bf Total Points: 30}}
\chead{MATH 3410}\lhead{Test \# 1}\rhead{Friday, 13\textsuperscript{th} February, 2015}

\newcommand{\points}[1]{\marginpar{\hspace{24pt}[#1]}}
\newcommand{\skipline}{\vspace{12pt}}
%\renewcommand{\headrulewidth}{0in}
\headheight 30pt

\newenvironment{amatrix}[1]{%
  \left[\begin{array}{@{}*{#1}{c}|c@{}}
}{%
  \end{array}\right]
}

\newcommand{\di}{\displaystyle}
\newcommand{\R}{\mathbb{R}}
\newcommand{\aaa}{\mathbf{a}}
\newcommand{\bbb}{\mathbf{b}}
\newcommand{\ccc}{\mathbf{c}}
\newcommand{\dotp}{\boldsymbol{\cdot}}
\newcommand{\abs}[1]{\lvert #1\rvert}
\newcommand{\len}[1]{\lVert #1\rVert}
\newcommand{\ivec}{\,\boldsymbol{\hat{\imath}}}
\newcommand{\jvec}{\,\boldsymbol{\hat{\jmath}}}
\newcommand{\kvec}{\,\boldsymbol{\hat{k}}}
\DeclareMathOperator{\comp}{comp}
\DeclareMathOperator{\spn}{span}
\DeclareMathOperator{\nul}{null}
\DeclareMathOperator{\range}{range}

\begin{document}

\author{Instructor: Sean Fitzpatrick}
\thispagestyle{plain}
\begin{center}
\emph{University of Lethbridge}\\
Department of Mathematics and Computer Science\\
13\textsuperscript{th} February, 2015, 3:00 - 3:50 pm\\
{\bf MATH 3410 - Test \#1}\\
\end{center}
\skipline \skipline \skipline \noindent \skipline
Last Name:\underline{\hspace{50pt}{\bf Solutions}\hspace{248pt}}\\
\skipline
First Name:\underline{\hspace{50pt}{\bf The}\hspace{275pt}}\\
\skipline
Student Number:\underline{\hspace{323pt}}\\
\skipline



\vspace{0.5in}


\begin{quote}
 {\bf Record your answers below each question in the space provided.    Left-hand pages may be used as scrap paper for rough work.  If you want any work on the left-hand pages to be graded, please indicate so on the right-hand page.
 
 \bigskip
 
Partial credit will be awarded for partially correct work, so be sure to show your work, and include all necessary justifications needed to support your arguments.}
\end{quote}


\vspace{0.5in}

For grader's use only:

\begin{table}[hbt]
\begin{center}
\begin{tabular}{|l|r|} \hline
Page&Grade\\
\hline \hline
\cline{1-2} 2 & \enspace\enspace\enspace\enspace\enspace\enspace/12\\
\cline{1-2} 3 & \enspace\enspace\enspace\enspace\enspace\enspace/8\\
\cline{1-2} 4 & \enspace\enspace\enspace\enspace\enspace\enspace/10\\
\cline{1-2} Total & \enspace\enspace\enspace\enspace\enspace\enspace/30\\
\hline
\end{tabular}




\end{center}
\end{table}
\newpage


\begin{enumerate}
\item True/False: For each of the statements below, state whether it is true or false, and give a {\bf brief} explanation supporting your choice.
 \begin{enumerate}
\item The set $U=\{(x,y,xy)\,|\,x,y\in\R\}$ is a subspace of $\R^3$.\points{3}

\bigskip

{\bf False}.

For example, $(1,1,1)\in U$, but $2(1,1,1)=(2,2,2)\notin U$, since $2\cdot 2=4\neq 2$. Thus, $U$ is not closed under scalar multiplication, and therefore cannot be a subspace.

\bigskip

\item If a vector space $V$ can be written as a direct sum $V=U\oplus W$, and for some $v\in V$ we have $v\notin U$, then $v\in W$.\points{3}

\bigskip

{\bf False}.

For example, take $V=\R^2$, $U=\spn\{(1,0)\}$, and $W=\spn\{(0,1)\}$. Then the vector $v=(1,1)$ belongs to neither $U$ nor $W$. (In general, any vector of the form $v=u+w$, where $u\in U$ and $w\in W$ are nonzero vectors, will do the job.)

\bigskip

\item For any subspace $U\subseteq V$, where $V$ is finite-dimensional, there exists a subspace $W\subseteq V$ such that $V=U\oplus W$.\points{3}

\bigskip

{\bf True}.

Let $B_U=\{u_1,\ldots, u_k\}$ be any basis for $U$. As we know from class, $B_U$ can be extended to a basis $B_V = \{u_1,\ldots, u_k,w_1,\ldots, w_m\}$, and letting $W=\spn\{w_1,\ldots, w_m\}$ provides the desired subspace.

\bigskip

\item If $T:V\to W$ is a linear transformation, and we know $\dim V=4$ and $\dim W=3$, then $T$ cannot be one-to-one.\points{3}

\bigskip

{\bf True}.

Since $\range T\subseteq W$, we know that $\dim\range T\leq \dim W=3$. Thus,
\[
 \dim\nul T = \dim V - \dim \range T \geq 4-3=1.
\]
This shows that $\nul T\neq \{0\}$, and therefore, $T$ cannot be one-to-one.
\end{enumerate}
\newpage
Please provide a solution to {\bf one} of the two problems on this page: 
\item Suppose that the vectors $v_1,v_2,v_3,v_4$ form a basis for $V$. Prove that the vectors \points{8}
\[
 v_1+v_2,v_2+v_3,v_3+v_4,v_4
\]
also form a basis for $V$.

\bigskip

{\bf Solution}: Since $\{v_1,v_2,v_3,v_4\}$ is a basis for $V$, we know that $\dim V=4$. Since we are given four vectors, it suffices to show that {\em either} they're linearly indepenent,  {\em or} they span.

To see that the vectors $ v_1+v_2,v_2+v_3,v_3+v_4,v_4$ are linearly indepenent, suppose that we have
\[
 c_1(v_1+v_2)+c_2(v_2+v_3)+c_3(v_3+v_4)+c_4v_4=0
\]
for some scalars $c_1,c_2,c_3,c_4$. Then we have
\[
 0 = c_1v_1+(c_1+c_2)v_2+(c_2+c_3)v_3+(c_3+c_4)v_4,
\]
and since the vectors $v_1,v_2,v_3,v_4$ are linearly independent, we have
\[
 c_1=0, c_1+c_2=0, c_2+c_3=0, c_3+c_4=0.
\]
But putting $c_1=0$ into the second equation gives $c_2=0$, which in turn gives $c_3=0$ in the third equation, and then $c_4=0$ in the fourth equation. Thus, we've shown that the four vectors $v_1+v_2, v_2+v_3,v_3+v_4,v_4$ are linearly independent, and therefore form a basis for $V$.

\medskip

At this point you're done, but if you want to also show that the vectors $v_1+v_2, v_2+v_3,v_3+v_4,v_4$ span $V$ (or you did both) then note that we can write
\begin{align*}
 v_1 & = 1(v_1+v_2)+(-1)(v_2+v_3)+(1)(v_3+v_4)+(-1)v_4\\
 v_2 & = 1(v_2+v_3)+(-1)(v_3+v_4)+(1)v_4\\
 v_3 & = 1(v_3+v_4)+(-1)v_4\\
 v_4 & = v_4.
\end{align*}
Since any vector in $V$ can be written in terms of the vectors $v_1,v_2,v_3,v_4$, since these vectors form a basis and therefore span $V$, and each of these four basis vectors can be written in terms of the vectors $v_1+v_2, v_2+v_3,v_3+v_4,v_4$, it follows that these latter vectors span $V$. To see this directly, note that any $v\in V$ can be written as
\begin{align*}
 v & = c_1v_1+c_2v_2+c_3v_3+c_4v_4\\
 & = c_1[(v_1+v_2)+(-1)(v_2+v_3)+(1)(v_3+v_4)+(-1)v_4]\\
 & \quad\quad\quad+c_2[(v_2+v_3)+(-1)(v_3+v_4)+(1)v_4]\\
 &\quad\quad\quad\quad\quad\quad +c_3[(v_3+v_4)+(-1)v_4]+c_4v_4\\
 & = c_1(v_1+v_2)+(c_2-c_1)(v_2+v_3)+(c_3-c_2+c_1)(v_3+v_4)+(c_4-c_3+c_2-c_1)v_4
\end{align*}
for some scalars $c_1,c_2,c_3,c_4$.


\newpage

\item Determine whether or not the vector $v=(1,3,-4)$ belongs to the span of the vectors $(2,0,1), (0,3,-4)$, and  $(4,-3,9)$. \points{8}

\bigskip

{\bf Solution:} We need to determine whether or not there exist scalars $x,y,z\in\R$ such that
\[
 (1,3,-4) = x(2,0,1)+y(0,3,-4)+z(4,-3,9) = (2x+4z,3y-3z,x-4y+9z).
\]
Since two vectors in $\R^3$ are equal if and only if their components are all equal, this yields the system of linear equations
\[
 \begin{array}{ccccccc}
  2x& &  &+&4y&=&1\\
    & &3y&-&3z&=&3\\
   x&-&4y&+&9z&=&-4
 \end{array}
\]
We solve the system by setting up an augmented matrix and reducing, using Gaussian elimination:
\begin{align*}
 \begin{amatrix}{3}
  2&0&4&1\\
  0&3&-3&3\\
  1&-4&9&-4
 \end{amatrix}&\xrightarrow[R_2\to \frac{1}{3}R_2]{R_1\leftrightarrow R_2}
 \begin{amatrix}{3}
  1&-4&9&-4\\
  0& 1&-1&1\\
  2&0&4&1
 \end{amatrix}\\
&\xrightarrow[]{R_3\to R_3-2R_1}
\begin{amatrix}{3}
 1&-4&9&-4\\0&1&-1&1\\0&8&-14&9
\end{amatrix}\\
&\xrightarrow[R_3\to R_3-8R_2]{R_1\to R_1+4R_2}
 \begin{amatrix}{3}
  1&0&5&0\\0&1&-1&1\\0&0&-6&1
 \end{amatrix}.
\end{align*}
Since the rank of the coefficient matrix is equal to 3, which is also the number of variables, we know that we have a unique solution, which we can read off using back-substitution. The third row gives us $z=-1/6$; the second row yields $y-z=1$, and plugging in $z=-1/6$ gives $y=5/6$. Finally, the first row tells us $x+5z=0$, and substituting $z=-1/6$ gives us $x=5/6$.

Thus, we have shown that
\[
 (1,3,-4) = \frac{5}{6}(2,0,1)+\frac{5}{6}(0,3,-4)-\frac{1}{6}(4,-3,9),
\]
which shows that $(1,3,-4)$ is in the span of the given vectors.


\newpage
Please provide a solution to {\bf one} of the two problems on this page:
\item Suppose $T:V\to W$ is injective, and the vectors $v_1,\ldots, v_n$ are linearly independent in $V$. Prove that the vectors $Tv_1,\ldots, Tv_n$ are linearly independent in $W$. \points{10}

\bigskip

{\bf Solution}: Let $T:V\to W$ be an injective linear transformation, and let $v_1,\ldots, v_n$ be linearly independent vectors in $V$. Suppose that we have
\[
 c_1Tv_1+c_2Tv_2+\cdots +c_nTv_n = 0
\]
for some scalars $c_1,c_2,\ldots, c_n$. It follows that
\[
 0 = c_1Tv_1+c_2Tv_2+\cdots +c_nTv_n = T(c_1v_1+c_2v_2+\cdots +c_nv_n),
\]
which implies that $c_1v_1+c_2v_2+\cdots +c_nv_n\in \nul T$. Since $T$ is injective, $\nul T = \{0\}$, and thus $c_1v_1+c_2v_2+ \cdots +c_nv_n=0$. Since the vectors $v_1,v_2,\ldots, v_n$ are linearly independent, we must have $c_1=c_2=\cdots =c_n=0$ as the only solution. Therefore, the vectors $Tv_1,Tv_2,\ldots, Tv_n$ are linearly independent.

\newpage

\item Let $V=\R^{3,1} = \left\{\begin{bmatrix}x\\y\\z\end{bmatrix} : x,y,z\in\R\right\}$, and let $T:V\to V$ be the linear transformation given by\points{10}
\[
T\left(\begin{bmatrix}x\\y\\z\end{bmatrix}\right) = \begin{bmatrix}2&-1&3\\-1&0&4\\4&-1&-5\end{bmatrix}\begin{bmatrix}
x\\y\\z
\end{bmatrix}=\begin{bmatrix}
2x-y+3z\\-x+4z\\4x-y-5z
\end{bmatrix}.
\]
Determine the null space and range of $T$.

\bigskip

{\bf Solution}: For both the null space and range, it therefore suffices to consider the system of equations
\[
 \begin{array}{ccccccc}
  2x&-&y&+&3z&=&a\\
  -x& & &+&4z&=&b\\
  4x&-&y&-&5z&=&c
 \end{array}.
\]
The null space corresponds to all solutions of the system with $a=b=c=0$, while the range consists of all values of $a, b$, and $c$ for which a solution exists. Reducing the corresponding augmented matrix, we find
\begin{align*}
 \begin{amatrix}{3}
  2&-1&3&a\\-1&0&4&b\\4&-1&-5&c
 \end{amatrix} & \xrightarrow{R_1\leftrightarrow R_2} 
\begin{amatrix}{3}
-1&0&4&b\\2&-1&3&a\\4&-1&-5&c                                                       
\end{amatrix}\xrightarrow[R_3\to R_3+4R_1]{R_2\to R_2+2R_1}
\begin{amatrix}{3}
 -1&0&4&b\\
 0&-1&11&a+2b\\
 0&-1&11&c+4b
\end{amatrix}\\
&\xrightarrow[R_3\to R_3-R_2]{R_1\to -R_1}
\begin{amatrix}{3}
 1&0&-4&-b\\
 0&-1&11&a+2b\\
 0&0&0&(c+4b)-(a+2b)\\
\end{amatrix}\\
&\xrightarrow{R_2\to-R_2}
\begin{amatrix}{3}
 1&0&-4&-b\\
 0&1&-11&a+2b\\
 0&0&0&-a+2b+c
\end{amatrix}.
\end{align*}
From this, we see that to have $T\left(\begin{bmatrix}x\\y\\z\end{bmatrix}\right)=\begin{bmatrix}0\\0\\0\end{bmatrix}$ we must have $x-4z=0$ and $y-11z=0$. Thus,
\[
 \begin{bmatrix}x\\y\\z\end{bmatrix}\in\nul T \quad \Leftrightarrow  \quad \begin{bmatrix}x\\y\\z\end{bmatrix} = \begin{bmatrix}4z\\11z\\z\end{bmatrix} = z\begin{bmatrix}4\\11\\1\end{bmatrix},
\]
so that $\nul T = \spn\left\{\begin{bmatrix}4\\11\\1\end{bmatrix}\right\}$.

We also see that in order for the system to be consistent, we must have $-a+2b+c=0$, or $c=a-2b$. Thus, the range of $T$ consists of all vectors of the form
\[
 \begin{bmatrix}a\\b\\c\end{bmatrix} = \begin{bmatrix}a\\b\\a-2b\end{bmatrix} = a\begin{bmatrix}1\\0\\1\end{bmatrix}+b\begin{bmatrix}0\\1\\-2\end{bmatrix},
\]
so that $\range T = \spn\left\{\begin{bmatrix}1\\0\\1\end{bmatrix},\begin{bmatrix}0\\1\\-2\end{bmatrix}\right\}$.

\end{enumerate}

\end{document}