\documentclass[12pt]{article}
\usepackage{amsmath}
\usepackage{amssymb}
\usepackage[letterpaper,margin=0.85in,centering]{geometry}
\usepackage{fancyhdr}
\usepackage{enumerate}
\usepackage{lastpage}
\usepackage{multicol}
\usepackage{graphicx}

\reversemarginpar

\pagestyle{fancy}
\cfoot{Page \thepage \ of \pageref{LastPage}}\rfoot{{\bf Total Points: 40}}
\chead{MATH 3410}\lhead{Test \# 2}\rhead{Friday, 20\textsuperscript{th} March, 2015}

\newcommand{\points}[1]{\marginpar{\hspace{24pt}[#1]}}
\newcommand{\skipline}{\vspace{12pt}}
%\renewcommand{\headrulewidth}{0in}
\headheight 30pt

\newcommand{\di}{\displaystyle}
\newcommand{\R}{\mathbb{R}}
\newcommand{\aaa}{\mathbf{a}}
\newcommand{\bbb}{\mathbf{b}}
\newcommand{\ccc}{\mathbf{c}}
\newcommand{\dotp}{\boldsymbol{\cdot}}
\newcommand{\abs}[1]{\lvert #1\rvert}
\newcommand{\len}[1]{\lVert #1\rVert}
\newcommand{\ivec}{\,\boldsymbol{\hat{\imath}}}
\newcommand{\jvec}{\,\boldsymbol{\hat{\jmath}}}
\newcommand{\kvec}{\,\boldsymbol{\hat{k}}}
\DeclareMathOperator{\comp}{comp}
\DeclareMathOperator{\nul}{null}
\DeclareMathOperator{\range}{range}

\begin{document}

\author{Instructor: Sean Fitzpatrick}
\thispagestyle{plain}
\begin{center}
\emph{University of Lethbridge}\\
Department of Mathematics and Computer Science\\
20\textsuperscript{th} March, 2015, 3:00 - 3:50 pm\\
{\bf MATH 3410 - Test \#2}\\
\end{center}
\skipline \skipline \skipline \noindent \skipline
Last Name:\underline{\hspace{50pt}{\bf Solutions}\hspace{248pt}}\\
\skipline
First Name:\underline{\hspace{50pt}{\bf The}\hspace{275pt}}\\
\skipline
Student Number:\underline{\hspace{323pt}}\\
\skipline



\vspace{0.5in}


\begin{quote}
 {Record your answers below each question in the space provided.    Left-hand pages may be used as scrap paper for rough work.  If you want any work on the left-hand pages to be graded, please indicate so on the right-hand page.
 
 \bigskip
 
Partial credit will be awarded for partially correct work, so be sure to show your work, and include all necessary justifications needed to support your arguments.}

 \bigskip

You must solve all problems on pages 2, 3, and 4, but you only need to do either page 5 or page 6. {\bf Do not complete both page 5 and page 6.}
\end{quote}


\vspace{0.5in}

For grader's use only:

\begin{table}[hbt]
\begin{center}
\begin{tabular}{|l|r|} \hline
Page&Grade\\
\hline \hline
\cline{1-2} 2 & \enspace\enspace\enspace\enspace\enspace\enspace/8\\
\cline{1-2} 3 & \enspace\enspace\enspace\enspace\enspace\enspace/8\\
\cline{1-2} 4 & \enspace\enspace\enspace\enspace\enspace\enspace/12\\
\cline{1-2} 5/6 & \enspace\enspace\enspace\enspace\enspace\enspace/12\\
\cline{1-2} Total & \enspace\enspace\enspace\enspace\enspace\enspace/40\\
\hline
\end{tabular}




\end{center}
\end{table}
\newpage


\begin{enumerate}
\item Provide definitions for the following terms:
\begin{enumerate}
 \item What it means for a linear map $T:V\to W$ to be {\bf invertible}. \points{2}

\bigskip

An operator $T:V\to W$ is {\bf invertible} if there exists a linear map $T^{-1}:W\to V$ such that $T^{-1}T = I_V$ and $TT^{-1}=I_W$, where $I_V$ and $I_W$ denote the identity operators on $V$ and $W$, respectively.

\bigskip
 
 \item An {\bf invariant subspace} for an operator $T:V\to V$. \points{2}

\bigskip

A subspace $U\subseteq V$ is an {\bf invariant subspace} for an operator $T\in\mathcal{L}(V)$ if for all $u\in U$ we have $Tu\in U$.

\bigskip

 \item What it means for a linear operator $T:V\to V$ to be {\bf diagonalizable}.\points{2}

\bigskip

An operator $T\in\mathcal{L}(V)$ is {\bf diagonalizable} if there exists a basis $B$ of $V$ with respect to which the matrix of $T$ is diagonal.

\bigskip

 \item The {\bf eigenspace} $E(\lambda, T)$ of an operator $T:V\to V$ and scalar $\lambda$.\points{2}

\bigskip

For any $T\in\mathcal{L}(V)$ and any $\lambda\in\mathbb{F}$, we define the {\bf eigenspace} $E(\lambda T)$ by $E(\lambda, T) = \nul(T-\lambda I)$, where $I$ is the identity operator on $V$.
\end{enumerate}
\newpage

\item Short answer: provide a brief answer to the questions below. You do not have to explain your answers.
\begin{enumerate}
 \item If $V$ and $W$ are finite-dimensional vector spaces, what is $\dim\mathcal{L}(V,W)$?\points{1}

\bigskip

\[
\dim\mathcal{L}(V,W) = \dim V\cdot \dim W. 
\]

\bigskip

 \item What is the matrix (with respect to the standard bases) of the linear map $T:\R^3\to\R^2$ given by \points{3}
\[
 T(x,y,z) = (2x-3y+z,-x+2y+4z)?
\]

\bigskip

Since $T(1,0,0) = (2,-1)$, $T(0,1,0)=(-3,2)$, and $T(0,0,1) = (1,4)$, we have
\[
 \mathcal{M}(T) = \begin{bmatrix}2&-3&1\\-1&2&4\end{bmatrix}.
\]

\bigskip

\item If $T$ is the operator on $\R^{2,1}$ given by
\[
 T\left(\begin{bmatrix}x\\y\end{bmatrix}\right) = \begin{bmatrix}-3&-2\\2&5\end{bmatrix}\begin{bmatrix}x\\y\end{bmatrix}
\]
 and $p(x) = 2x^2-3x+5$, determine the operator $p(T)$. \points{4}

\bigskip

Let $A=\begin{bmatrix}-3&-2\\2&5\end{bmatrix}$, so that $T(X)=AX$, and thus $p(T)X = p(A)X$. Since 
\[
 A^2 = \begin{bmatrix}-3&-2\\2&5\end{bmatrix}\begin{bmatrix}-3&-2\\2&5\end{bmatrix}=\begin{bmatrix}5&-4\\4&21\end{bmatrix},
\]
we have
\[
p(A) = 2A^2-3A+5I_2 = 2\begin{bmatrix}5&-4\\4&21\end{bmatrix}-3\begin{bmatrix}-3&-2\\2&5\end{bmatrix}+5\begin{bmatrix}1&0\\0&1\end{bmatrix}=\begin{bmatrix}24&-2\\2&32\end{bmatrix},
\]
and thus $p(T)\begin{bmatrix}x\\y\end{bmatrix}=\begin{bmatrix}24&-2\\2&32\end{bmatrix}\begin{bmatrix}x\\y\end{bmatrix}$.

\end{enumerate}
\newpage

Please solve {\bf both} problems on this page.

\bigskip

\item Let $S,T\in\mathcal{L}(V)$, where $V$ is finite-dimensional. Prove that the operator $ST$ is invertible if and only if $S$ and $T$ are invertible. \points{6}

\bigskip

If $S$ and $T$ are invertible, then $ST$ is invertible, with inverse $T^{-1}S^{-1}$, since
\[
 ST(T^{-1}S^{-1}) = S(TT^{-1})S^{-1}=S(I_VS^{-1})=SS^{-1}=I_V,
\]
and similarly $(T^{-1}S^{-1})(ST) = I_V$.

Conversely, suppose that $ST$ is invertible. Then $ST$ is a bijection, and therefore $S$ must be surjective and $T$ must be injective. (As proved in Math 2000 and discussed in class and mentioned on the review sheet.) Since $V$ is finite-dimensional, either injectivity or surjectivity implies bijectivity, and thus $S$ and $T$ are both bijections, and therefore invertible.

\bigskip

\item Suppose that $S,T\in \mathcal{L}(V)$ satisfy $ST=TS$. Prove that $\operatorname{null}S$ is invariant under $T$. \points{6}

\bigskip

Suppose that $ST=TS$, and let $v\in \nul S$. We need to show that $Tv\in\nul S$; that is, we need to show that $S(Tv) = 0$. But we're assuming that $ST=TV$, so
\[
 S(Tv) = T(Sv) = T(0)=0,
\]
where we've used the fact that $Sv=0$, since $v\in \nul S$. Thus, $Tv\in \nul S$, and since $v$ was arbitrary, the result follows.

\newpage

You may either solve both problems on this page, or leave it blank, and move on to the next page.

\item Let $V$ be finite-dimensional, and let $P\in\mathcal{L}(V)$. Prove that if $P^2=P$, then\\ $V=\nul P\oplus\range P$.\points{6}

{\em Hint:} $\dim V = \dim \nul P+\dim \range P$, so it suffices to show that $\nul P\cap \range P=\{0\}$.

\bigskip

Suppose that $P^2=P$. By the hint, we need to show that $\nul P\cap \range P=\{0\}$. Thus, suppose that $w\in\nul P\cap\range P$. Then we have that $Pw=0$, and $w=Pv$ for some $v\in V$. But we're assuming that $P^2=P$, and thus
\[
 0 = Pw = P^2v = Pv = w,
\]
so we must have $w=0$, and this completes the proof.

\bigskip

\item Suppose that $\dim V = n$, $T\in\mathcal{L}(V)$ has $n$ distinct eigenvalues, and $S\in\mathcal{L}(V)$ has the same eigenvectors as $T$ (but not necessarily the same eigenvalues). Prove that $ST=TS$. \points{6}

\bigskip

Suppose that $T\in\mathcal{L}(V)$ has $n$ distinct eigenvalues $\lambda_1,\ldots, \lambda_n$, where $n=\dim V$. Since eigenvectors corresponding to distinct eigenvalues are linearly independent, it follows that we can choose a basis $B=\{v_1,\ldots, v_n\}$ of eigenvectors of $T$, where $Tv_i=\lambda_iv_i$ for $i=i,\ldots, n$.

Now suppose that the vectors in $B$ are also eigenvectors for $S$. Thus, $Sv_i = \mu_iv_i$, for some scalars $\mu_i$, for $i=1,\ldots, n$. We wish to show that $ST=TS$; that is, that $S(Tv)=T(Sv)$ for all $v\in V$. Thus, let us choose an arbitrary element $v\in V$.

Since $B$ is a basis, there exist unique scalars $c_1,\ldots, c_n$ such that $v=c_1v_1+\cdots + c_nv_n$, and thus
\begin{align*}
 S(Tv) & = S(T(c_1v_1+\cdots + c_nv_n))\\
& = S(c_1\lambda_1v_1+\cdots + c_n\lambda_nv_n)\\
& = c_1\lambda_1\mu_1v_1+\cdots + c_n\lambda_n\mu_nv_n\\
& = c_1\mu_1\lambda_1v_1+\cdots + c_n\mu_n\lambda_nv_n\\
& = T(c_1\mu_1v_1+\cdots + c_n\mu_nv_n)\\
& = T(S(c_1v_1+\cdots +c_nv_n))\\
& = T(Sv).
\end{align*}




\newpage

If you solved the two problems on the previous page, then leave this page blank. If you skipped the last page, then please solve the following:

\item Let $T:\R^2\to\R^2$ be the operator $T(x,y) = (5x-2y,7x-4y)$.
\begin{enumerate}
\item Compute the matrix $\mathcal{M}(T)$ of $T$ with respect to the standard basis of $\R^2$. \points{2}

\bigskip

$T(1,0)=(5,7)$ and $T(0,1)=(-2,-4)$, so $\mathcal{M}(T)=\begin{bmatrix}5&-2\\7&-4\end{bmatrix}$.

\bigskip

\item Find the eigenvalues of $T$. \points{4}

\bigskip

Since $\lambda$ is an eigenvalue of $T$ if and only if $\mathcal{M}(T)-\lambda I_2$ is not invertible, we must have
\[
 0 = \begin{vmatrix}5-\lambda&-2\\7&-4\lambda\end{vmatrix} = (5-\lambda)(-4-\lambda)+14 = \lambda^2-\lambda-6 = (\lambda-3)(\lambda+2).
\]
The eigenvalues of $T$ are therefore $\lambda=3$ and $\lambda = -2$.


\bigskip

\item Find a basis of $\R^2$ consisting of eigenvectors of $\R^2$. \points{4}

\bigskip

For $\lambda = 3$ we have $\mathcal{M}(T)-3I_2 = \begin{bmatrix}2&-2\\7&-7\end{bmatrix}$, and since $\begin{bmatrix}2&-2\\7&-7\end{bmatrix}\begin{bmatrix}1\\1\end{bmatrix} = \begin{bmatrix}0\\0\end{bmatrix}$, we have that $v_1 = (1,1) \in \nul(T-3I)$.

For $\lambda = -2$ we have $\mathcal{M}(T)+2I_2 = \begin{bmatrix}7&-2\\7&-2\end{bmatrix}$, and since $\begin{bmatrix}7&-2\\7&-2\end{bmatrix}\begin{bmatrix}2\\7\end{bmatrix} = \begin{bmatrix}0\\0\end{bmatrix}$, it follows that $v_2 = (2,7)\in \nul(T+2I)$.

Since $v_1$ and $v_2$ are not parallel, they are linearly independent, and therefore form a basis of $\R^2$.

\bigskip

\item Is the operator $T$ diagonalizable? Why or why not? If it is, give a matrix $P$ such that $P^{-1}\mathcal{M}(T)P$ is diagonal. (You don't have to verify it's diagonal.) \points{2}

\bigskip

Since we found a basis of eigenvectors of $T$ in part (c), we know that $T$ is diagonalizable. Since the columns of $P$ are given by the column vectors for $v_1$ and $v_2$, we have
\[
 P = \begin{bmatrix}1&2\\1&7\end{bmatrix}
\]
Although it was not required, you may wish to verify that
\[
 P^{-1}AP = \begin{bmatrix}7/5&-2/5\\-1/5&1/5\end{bmatrix}\begin{bmatrix}5&-2\\7&-4\end{bmatrix}\begin{bmatrix}1&2\\1&7\end{bmatrix} = \begin{bmatrix}3&0\\0&-2\end{bmatrix}.
\]

\end{enumerate}
\end{enumerate}
\end{document}