\documentclass[12pt]{article}
\usepackage{amsmath}
\usepackage{amssymb}
\usepackage[letterpaper,margin=0.85in,centering]{geometry}
\usepackage{fancyhdr}
\usepackage{enumerate}
\usepackage{lastpage}
\usepackage{multicol}
\usepackage{graphicx}

\reversemarginpar

\pagestyle{fancy}
\cfoot{Page \thepage \ of \pageref{LastPage}}\rfoot{{\bf Total Points: 35}}
\chead{MATH 4310}\lhead{Midterm}\rhead{Wednesday, 15\textsuperscript{th} October, 2014}

\newcommand{\points}[1]{\marginpar{\hspace{24pt}[#1]}}
\newcommand{\skipline}{\vspace{12pt}}
%\renewcommand{\headrulewidth}{0in}
\headheight 30pt

\newcommand{\di}{\displaystyle}
\newcommand{\R}{\mathbb{R}}
\newcommand{\N}{\mathbb{N}}
\newcommand{\Z}{\mathbb{Z}}
\newcommand{\abs}[1]{\lvert #1\rvert}

\begin{document}

\author{Instructor: Sean Fitzpatrick}
\thispagestyle{plain}
\begin{center}
\emph{University of Lethbridge}\\
Department of Mathematics and Computer Science\\
15\textsuperscript{th} October, 2014, 5:00-5:50 pm\\
{\bf Math 4310 - Term Test I}\\
\end{center}
\skipline \skipline \skipline \noindent \skipline
Last Name:\underline{\hspace{350pt}}\\
\skipline
First Name:\underline{\hspace{348pt}}\\
\skipline
Student Number:\underline{\hspace{322pt}}\\


\vspace{0.5in}


\begin{quote}
 {\bf Record your answers below each question in the space provided.    Left-hand pages may be used as scrap paper for rough work.  If you want any work on the left-hand pages to be graded, please indicate so on the right-hand page.
 
 \bigskip
 
Partial credit will be awarded for partially correct work, so be sure to show your work, and include all necessary justifications needed to support your arguments. 

There is a list of potentially useful formulas available on the last page of the exam.}
\end{quote}


\vspace{0.5in}

For grader's use only:

\begin{table}[hbt]
\begin{center}
\begin{tabular}{|l|r|} \hline
Page&Grade\\
\hline \hline
\cline{1-2} 2 & \enspace\enspace\enspace\enspace\enspace\enspace/12\\
\cline{1-2} 3 & \enspace\enspace\enspace\enspace\enspace\enspace/8\\
\cline{1-2} 4 & \enspace\enspace\enspace\enspace\enspace\enspace/9\\
\cline{1-2} 5 & \enspace\enspace\enspace\enspace\enspace\enspace/6\\
\cline{1-2} Total & \enspace\enspace\enspace\enspace\enspace\enspace/35\\
\hline
\end{tabular}

\skipline

\skipline

\skipline


\end{center}
\end{table}
\newpage


\begin{enumerate}
\item For each of the following, give an example, or explain why no such example exists:
\begin{enumerate}
\item A subset of a topological space that is both open and closed. \points{3}

\vspace{1.6in}

\item A continuous function $f:X\to Y$, if $X$ is equipped with the indiscrete topology. \points{3}

\vspace{1.8in}

\item An interior point that is not a limit point. \points{3}

\vspace{1.8in}

\item A metric space that is not Hausdorff. \points{3}
\end{enumerate}
\newpage

\item Let $X=\displaystyle l^1(\mathbb{R}) = \left\{\sum_{n=1}^\infty a_n \,\left|\: \sum_{n=1}^\infty\abs{a_n}<\infty\right.\right\}$ be the space of absolutely convergent sequences of real numbers. Prove that the function $d:X\times X\to\R$ given by \points{8}
\[
d\left(\sum a_n, \sum b_n\right) = \sum_{n=1}^\infty\abs{a_n-b_n}
\]
is well-defined (i.e. that $d(x,y)$ is finite for all $x,y\in X$) and makes $X$ into a metric space.

\newpage

\item \begin{enumerate}
\item Define what it means for a set $\mathcal{B}$ of subsets of a set $X$ to be a {\bf basis} for a topology on $X$. \points{3}

(Either of the two definitions we discussed is acceptable.)

\vspace{2in}

\item Let $X$ and $Y$ be topological spaces, and let $\mathcal{B}$ be a basis for the topology on $Y$. Prove that a function $f:X\to Y$ is continuous if and only if $f^{-1}(U)$ is open in $X$ for every $U\in\mathcal{B}$. \points{6}
\end{enumerate}
\newpage

\item Solve {\bf one} of the following two problems:\points{6}
\begin{enumerate}
\item Let $X, Y$, and $Z$ be topological spaces, and equip $X\times Y$ with the product topology. Show that a map $f:Z\to X\times Y$ is continuous if and only if the maps $\pi_X\circ f:Z\to X$ and $\pi_Y\circ f:Z\to Y$ are continuous.  \\(Hint: one direction is easy. For the other, use 3(b).)

\item Given a topological space $X$, let $X_0$ denote the space with the same underlying set as $X$, but with the cofinite topology. Show that the identity map $I:X\to X_0$ (given by $I(x)=x$) is continuous if and only if $X$ is a $T_1$ space.

Hint: $X$ is $T_1$ if and only if finite point sets are closed.
\end{enumerate}
\end{enumerate}



\end{document}