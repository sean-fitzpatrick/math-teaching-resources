\documentclass[12pt]{article}
\usepackage{amsmath}
\usepackage{amssymb}
\usepackage[letterpaper,margin=0.85in,centering]{geometry}
\usepackage{fancyhdr}
\usepackage{enumerate}
\usepackage{lastpage}
\usepackage{multicol}
\usepackage{graphicx}

\reversemarginpar

\pagestyle{fancy}
\cfoot{Page \thepage \ of \pageref{LastPage}}\rfoot{{\bf Total Points: 35}}
\chead{MATH 4310}\lhead{Midterm}\rhead{Wednesday, 15\textsuperscript{th} October, 2014}

\newcommand{\points}[1]{\marginpar{\hspace{24pt}[#1]}}
\newcommand{\skipline}{\vspace{12pt}}
%\renewcommand{\headrulewidth}{0in}
\headheight 30pt

\newcommand{\di}{\displaystyle}
\newcommand{\R}{\mathbb{R}}
\newcommand{\N}{\mathbb{N}}
\newcommand{\Z}{\mathbb{Z}}
\newcommand{\B}{\mathcal{B}}
\newcommand{\abs}[1]{\lvert #1\rvert}

\begin{document}

\author{Instructor: Sean Fitzpatrick}
\thispagestyle{plain}
\begin{center}
\emph{University of Lethbridge}\\
Department of Mathematics and Computer Science\\
15\textsuperscript{th} October, 2014, 5:00-5:50 pm\\
{\bf Math 4310 - Term Test I}\\
\end{center}
\skipline \skipline \skipline \noindent \skipline
Last Name:\underline{\hspace{50pt}{\bf SOLUTIONS}\hspace{50pt}}\\
\skipline
First Name:\underline{\hspace{50pt}{\bf THE}\hspace{100pt}}\\
\skipline
Student Number:\underline{\hspace{322pt}}\\


\vspace{0.5in}


\begin{quote}
 {\bf Record your answers below each question in the space provided.    Left-hand pages may be used as scrap paper for rough work.  If you want any work on the left-hand pages to be graded, please indicate so on the right-hand page.
 
 \bigskip
 
Partial credit will be awarded for partially correct work, so be sure to show your work, and include all necessary justifications needed to support your arguments. 

There is a list of potentially useful formulas available on the last page of the exam.}
\end{quote}


\vspace{0.5in}

For grader's use only:

\begin{table}[hbt]
\begin{center}
\begin{tabular}{|l|r|} \hline
Page&Grade\\
\hline \hline
\cline{1-2} 2 & \enspace\enspace\enspace\enspace\enspace\enspace/12\\
\cline{1-2} 3 & \enspace\enspace\enspace\enspace\enspace\enspace/8\\
\cline{1-2} 4 & \enspace\enspace\enspace\enspace\enspace\enspace/9\\
\cline{1-2} 5 & \enspace\enspace\enspace\enspace\enspace\enspace/6\\
\cline{1-2} Total & \enspace\enspace\enspace\enspace\enspace\enspace/35\\
\hline
\end{tabular}

\skipline

\skipline

\skipline


\end{center}
\end{table}
\newpage


\begin{enumerate}
\item For each of the following, give an example, or explain why no such example exists:
\begin{enumerate}
\item A subset of a topological space that is both open and closed. \points{3}

\bigskip

\noindent{\bf Solution}: Let $X=\R$ with the Euclidean topology. Then $\emptyset\in\R$ is both open and closed (as it is in any topological space).

\bigskip

\item A continuous function $f:X\to Y$, if $X$ is equipped with the indiscrete topology. \points{3}


\bigskip

\noindent{\bf Solution}: We know that constant functions are continuous in any topology. To see this, note that if $f(x)=a$ for all $x\in\R$, for some $a\in Y$, then for any open subset $U\subseteq Y$ (in fact, for any subset), $f^{-1}(U)=X$ if $a\in U$, and $f^{-1}(U)=\emptyset$ if $a\notin U$. Thus, the topology $\{\emptyset, X\}$ is sufficient for $f$ to be continuous.

\bigskip


\item An interior point that is not a limit point. \points{3}


\bigskip

\noindent{\bf Solution}: Let $X$ have the discrete topology, let $x\in X$, and let $A=\{x\}$. Then $x$ is in the interior of $A$, since $A$ itself is an open neighbourhood of $x$ contained in $A$. However, $x$ cannot be a limit point of $A$ since every neighbourhood of $x$ would have to contain some $a\in A$ with $a\neq x$, and this is impossible if $A=\{x\}$.

Note: I'm pretty sure this is the only topology in which an interior point can fail to be a limit point. 

\bigskip


\item A metric space that is not Hausdorff. \points{3}

\bigskip

\noindent{\bf Solution}: No such space can exist. Given any two points $x\neq y$ in a metric space $(X,d)$, let $\epsilon = d(x,y)/2$. It follows that the open neighbourhoods $N_\epsilon(x)$ and $N_\epsilon(y)$ are disjoint, since if not, there exists some $z\in N_\epsilon(x)$ with $d(z,y)<\epsilon$. But then we have
\[
 d(x,y)\leq d(x,z)+d(z,y) <\epsilon+\epsilon = d(x,y),
\]
and this is impossible.

\bigskip

\end{enumerate}
\newpage

\item Let $X=\displaystyle l^1(\mathbb{R}) = \left\{\sum_{n=1}^\infty a_n \,\left|\: \sum_{n=1}^\infty\abs{a_n}<\infty\right.\right\}$ be the space of absolutely convergent sequences of real numbers. Prove that the function $d:X\times X\to\R$ given by \points{8}
\[
d\left(\sum a_n, \sum b_n\right) = \sum_{n=1}^\infty\abs{a_n-b_n}
\]
is well-defined (i.e. that $d(x,y)$ is finite for all $x,y\in X$) and makes $X$ into a metric space.

\bigskip

\noindent{\bf Solution}: Let $x=\sum a_n$, $y=\sum b_n$ and $z=\sum c_n$ be absolutely convergent series. Since $\sum\abs{a_n}$ and $\sum\abs{b_n}$ converge and $\abs{a_n-b_n}\leq \abs{a_n}+\abs{b_n}$ for all $n\in \N$, we see that $d(x,y)=\sum\abs{a_n-b_n}$ converges by comparison.

Thus, we obtain a well-defined function $d:X\times X\to\R$. We now verify that $d$ is a metric:
\begin{itemize}
 \item Since $\abs{a_n-b_n}\geq 0$ for all $n\in \N$, it follows that $d(x,y)\geq 0$ for all $x,y\in X$, and if $\di\sum_{n=1}\infty\abs{a_n-b_n}=0$ then we must have $\abs{a_n-b_n}=0$ for all $n\in\N$, and thus $x=y$.
 \item Since $\abs{a_n-b_n} = \abs{b_n-a_n}$ for all $n\in\N$, it follows that $d(x,y)=d(y,x)$ for all $x,y\in X$.
 \item Since $\abs{a_n-b_n} = \abs{a_n-c_n+c_n-b_n} \leq \abs{a_n-c_n}+\abs{c_n-b_n}$ for all $n\in \N$, it follows that for any $N\in \N$ we have
\[
 \sum_{i=1}^N\abs{a_i-b_i}\leq \sum_{i=1}^N\abs{a_i-c_i}+\sum_{i=1}^N\abs{c_i-b_i} \leq \sum_{i=1}^\infty\abs{a_i-c_i}+\sum_{i=1}^\infty\abs{c_i-b_i}.
\]
Thus, $d(x,z)+d(z,y)$ is an upper bound for the increasing sequence $s_N = \sum_{i=1}^N\abs{a_i-b_i}$, and since the limit of this sequence is $d(x,y)$, it follows that $d(x,y)\leq d(x,z)+d(z,y)$.
\end{itemize}


\bigskip

\newpage

\item \begin{enumerate}
\item Define what it means for a set $\mathcal{B}$ of subsets of a set $X$ to be a {\bf basis} for a topology on $X$. \points{3}

(Either of the two definitions we discussed is acceptable.)


\bigskip

\noindent{\bf Solution}: A collection $\B\subseteq \mathcal{P}(X)$ is a {\bf basis} for a topology on $X$ if
\begin{enumerate}
 \item $\di X\subseteq \bigcup_{B\in \B}B$
 \item For any $B_1,B_2\in\B$, if $x\in B_1\cap B_2$, then there exists $B_3\in\B$ such that $x\in B_3\subseteq B_1\cap B_2$.
\end{enumerate}

Alternatively, if we are given a topology $\mathcal{T}_X$ for $X$, then a basis for $\mathcal{T}_X$ is a collection $\B\subseteq \mathcal{T}_X$ such that any $U\subseteq T_X$ can be written as a union of basic open subsets $B\in\B$.

\bigskip


\item Let $X$ and $Y$ be topological spaces, and let $\mathcal{B}$ be a basis for the topology on $Y$. Prove that a function $f:X\to Y$ is continuous if and only if $f^{-1}(U)$ is open in $X$ for every $U\in\mathcal{B}$. \points{6}


\bigskip

\noindent{\bf Solution}: If $f$ is continuous and $U\in\B$, then $U$ is open in $Y$, so $f^{-1}(U)$ is open in $X$. Conversely, suppose that $f^{-1}(U)$ is open for all $U\in\B$, and let $V$ be any open subset of $Y$. Then there exists a collection $\{B_\alpha:\alpha\in I\}\subseteq \B$ such that $\di V = \bigcup_{\alpha\in I}B_\alpha$.

(Using the first definition above, we declare $V\subseteq Y$ to be open if for each $y\in Y$ there exists some $B_y\in\B$ with $y\in B_y\subseteq V$, and it follows that we can write $\di V = \bigcup_{y\in V}B_y$, but with either definition you can just state without justification that $V$ is a union of basic open sets.)

It follows that $\di f^{-1}(V) f^{-1}\left(\bigcup_{\alpha\in I}B_\alpha\right) = \bigcup_{\alpha\in I}f^{-1}(B_\alpha)$ is a union of open subsets of $X$, and therefore is open.

\bigskip

\end{enumerate}
\newpage

\item Solve {\bf one} of the following two problems:\points{6}
\begin{enumerate}
\item Let $X, Y$, and $Z$ be topological spaces, and equip $X\times Y$ with the product topology. Show that a map $f:Z\to X\times Y$ is continuous if and only if the maps $\pi_X\circ f:Z\to X$ and $\pi_Y\circ f:Z\to Y$ are continuous.  \\(Hint: one direction is easy. For the other, use 3(b).)

\bigskip

\noindent{\bf Solution}: If $f$ is continuous, then so are $\pi_X\circ f$ and $\pi_Y\circ f$, since they are the composition of continuous functions. Conversely, suppose that $\pi_X\circ f$ and $\pi_Y\circ f$ are continuous. We wish to show that $f$ is continuous. By 3(b), it suffices to prove that $f^{-1}(U\times V)$ is open in $Z$ whenever $U$ is open in $X$ and $V$ is open in $Y$. Letting $U$ and $V$ be open subsets of $X$ and $Y$, respectively, we have
\begin{align*}
 f^{-1}(U\times V) & = f^{-1}((U\times Y)\cap (X\times V)\\
& = f^{-1}(U\times Y)\cap f^{-1}(X\times V)\\
& = f^{-1}(\pi_x^{-1}(U))\cap f^{-1}(\pi_Y^{-1}(V))\\
& = (\pi_X\circ f)^{-1}(U)\cap (\pi_Y\circ f)^{-1}(V),
\end{align*}
which is open in $Z$, since it's the intersection of open sets, due to the assumption that $pi_x\circ f$ and $\pi_Y\circ f$ are continuous.


\bigskip

\item Given a topological space $X$, let $X_0$ denote the space with the same underlying set as $X$, but with the cofinite topology. Show that the identity map $I:X\to X_0$ (given by $I(x)=x$) is continuous if and only if $X$ is a $T_1$ space.

Hint: $X$ is $T_1$ if and only if finite point sets are closed.

\bigskip

\noindent{\bf Solution}: Since $I$ is the identity map, we have $f^{-1}(A) = A$ for any $A\subseteq X_0$. (Note $X=X_0$ as sets.) Suppose $I$ is continuous. Then $I^{-1}(F)=F$ is closed in $X$ whenever $F$ is closed in $X_0$. But the closed sets of $X_0$ are the finite subsets, so every finite subset of $X$ must be closed. Thus, $X$ is $T_1$. Conversely, if $X$ is $T_1$ and $F\subseteq X_0$ is closed, then $F$ is finite, and $I^{-1}(F)=F$ is finite and therefore closed, so that $I$ must be continuous.

\bigskip

\end{enumerate}
\end{enumerate}



\end{document}