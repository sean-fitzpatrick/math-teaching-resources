\documentclass[12pt]{article}
\usepackage{amsmath}
\usepackage{amssymb}
\usepackage[letterpaper,margin=0.85in,centering]{geometry}
\usepackage{fancyhdr}
\usepackage{enumerate}
\usepackage{lastpage}
\usepackage{multicol}
\usepackage{graphicx}

\reversemarginpar

\pagestyle{fancy}
\cfoot{Page \thepage \ of \pageref{LastPage}}\rfoot{{\bf Total Points: 40}}
\chead{MATH 1410A}\lhead{Test \# 1}\rhead{Monday, 9\textsuperscript{th} February, 2015}

\newcommand{\points}[1]{\marginpar{\hspace{24pt}[#1]}}
\newcommand{\skipline}{\vspace{12pt}}
%\renewcommand{\headrulewidth}{0in}
\headheight 30pt

\newcommand{\di}{\displaystyle}
\newcommand{\R}{\mathbb{R}}
\newcommand{\aaa}{\mathbf{a}}
\newcommand{\bbb}{\mathbf{b}}
\newcommand{\ccc}{\mathbf{c}}
\newcommand{\dotp}{\boldsymbol{\cdot}}
\newcommand{\abs}[1]{\lvert #1\rvert}
\newcommand{\len}[1]{\lVert #1\rVert}
\newcommand{\ivec}{\,\boldsymbol{\hat{\imath}}}
\newcommand{\jvec}{\,\boldsymbol{\hat{\jmath}}}
\newcommand{\kvec}{\,\boldsymbol{\hat{k}}}
\DeclareMathOperator{\comp}{comp}

\begin{document}

\author{Instructor: Sean Fitzpatrick}
\thispagestyle{plain}
\begin{center}
\emph{University of Lethbridge}\\
Department of Mathematics and Computer Science\\
9\textsuperscript{th} February, 2015, 11:00 - 11:50 am\\
{\bf MATH 1410A - Test \#1 - EARLY SITTING}\\
\end{center}
\skipline \skipline \skipline \noindent \skipline
Last Name:\underline{\hspace{353pt}}\\
\skipline
First Name:\underline{\hspace{350pt}}\\
\skipline
Student Number:\underline{\hspace{323pt}}\\
\skipline
Tutorial Section: \underline{\hspace{320pt}}\\


\vspace{0.5in}


\begin{quote}
 {\bf Record your answers below each question in the space provided.    Left-hand pages may be used as scrap paper for rough work.  If you want any work on the left-hand pages to be graded, please indicate so on the right-hand page.
 
 \bigskip
 
Partial credit will be awarded for partially correct work, so be sure to show your work, and include all necessary justifications needed to support your arguments.

\bigskip

No external aids are allowed, with the exception of a 5-function calculator.}
\end{quote}


\vspace{0.5in}

For grader's use only:

\begin{table}[hbt]
\begin{center}
\begin{tabular}{|l|r|} \hline
Page&Grade\\
\hline \hline
\cline{1-2} 2 & \enspace\enspace\enspace\enspace\enspace\enspace/10\\
\cline{1-2} 3 & \enspace\enspace\enspace\enspace\enspace\enspace/10\\
\cline{1-2} 4 & \enspace\enspace\enspace\enspace\enspace\enspace/10\\
\cline{1-2} 5 & \enspace\enspace\enspace\enspace\enspace\enspace/10\\
\cline{1-2} Total & \enspace\enspace\enspace\enspace\enspace\enspace/40\\
\hline
\end{tabular}

\skipline

\skipline

\skipline

A
\end{center}
\end{table}
\newpage


\begin{enumerate}
\item SHORT ANSWER: For each of the questions below, please provide a short (one line) answer.
 \begin{enumerate}
\item What is the rank of a matrix? \points{2}

\vspace{1in}

\item What does it mean to say that an $n\times n$ matrix $A$ is invertible? \points{2}

\vspace{1in}

\item The matrix $E=\begin{bmatrix}
1&0&-3\\0&1&0\\0&0&1
\end{bmatrix}$ is an elementary matrix. If $A$ is any other $3\times 3$ matrix, what elementary row operation would let us obtain $EA$ from $A$? \points{2}

\vspace{1in}

\item Do the values $x=3$, $y=-2$, $z=4$ provide a solution to the system of equations below? Why or why not?\points{2}
\[
\begin{array}{ccccccc}
x&+&y&+&z&=&5\\
2x&+&4y&-&z&=&-6\\
-3x&-&5y&+&z&=&3
\end{array}
\]

\vspace{1in}

\item Identify the matrices below as symmetric, antisymmetric, or neither: \points{2}
\[
\begin{bmatrix}
1&3\\2&4
\end{bmatrix}\hspace{1in} \begin{bmatrix}
0&4\\-4&0
\end{bmatrix}\hspace{1in} \begin{bmatrix}
1&2\\2&-3
\end{bmatrix}
\]
\end{enumerate}
\newpage

\item Find the general solution to the following system of linear equations: \points{10}
\[
\begin{array}{ccccccccc}
x&-&2y& &  &+&w&=&0\\
 & &-y&+&2z&-&w&=&1\\
x&-&3y&+&2z& & &=&1 
\end{array}
\]

\newpage

\item Suppose $A$, $B$, and $X$ are $2\times 2$ matrices.
\begin{enumerate}
\item Given that $3A+X^T=B$, solve for $X$ in terms of $A$ and $B$.\points{3}

\vspace{1.25in}

\item If $A=\begin{bmatrix}1&-2\\4&1\end{bmatrix}$ and $B=\begin{bmatrix}0&2\\3&0\end{bmatrix}$ and $X$ is as in part (a), determine the entries of $X$. \points{3}

\vspace{2.25in}

\end{enumerate}
\item Suppose that $A,B,C,$ and $D$ are $n\times n$ matrices, with $A,B,$ and $C$ {\em invertible}. Given that \points{4}
\[
AB^{-1}XBC^T = AD,
\]
solve for $X$ in terms of $A,B,C,$ and $D$.
\newpage

\item Let $A=\begin{bmatrix}2&-4\\-4&9\end{bmatrix}$.
\begin{enumerate}
\item Find $A^{-1}$. \points{5}

\vspace{3in}

\item Write $A^{-1}$ as a product of elementary matrices. \points{3}

\vspace{2.5in}

\item Write $A$ as a product of elementary matrices. \points{2}
\end{enumerate}
\end{enumerate}
\end{document}