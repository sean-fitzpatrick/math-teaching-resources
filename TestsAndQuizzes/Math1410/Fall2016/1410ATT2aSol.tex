\documentclass[12pt]{article}
\usepackage{amsmath}
\usepackage{amssymb}
\usepackage[letterpaper,margin=0.85in,centering]{geometry}
\usepackage{fancyhdr}
\usepackage{enumerate}
\usepackage{lastpage}
\usepackage{multicol}
\usepackage{graphicx}

\reversemarginpar

\pagestyle{fancy}
\cfoot{Page \thepage \ of \pageref{LastPage}}\rfoot{{\bf Total Points: 50}}
\chead{MATH 1410A}\lhead{Test \# 2}\rhead{Monday, 31\textsuperscript{st} October, 2016}

\newcommand{\points}[1]{\marginpar{\hspace{24pt}[#1]}}
\newcommand{\skipline}{\vspace{12pt}}
%\renewcommand{\headrulewidth}{0in}
\headheight 30pt

\newcommand{\di}{\displaystyle}
\newcommand{\R}{\mathbb{R}}
\newcommand{\aaa}{\mathbf{a}}
\newcommand{\bbb}{\mathbf{b}}
\newcommand{\ccc}{\mathbf{c}}
\newcommand{\dotp}{\boldsymbol{\cdot}}
\newcommand{\abs}[1]{\lvert #1\rvert}
\newcommand{\len}[1]{\lVert #1\rVert}
\newcommand{\ivec}{\,\boldsymbol{\hat{\imath}}}
\newcommand{\jvec}{\,\boldsymbol{\hat{\jmath}}}
\newcommand{\kvec}{\,\boldsymbol{\hat{k}}}
\newcommand{\bvm}{\begin{vmatrix}}
\newcommand{\evm}{\end{vmatrix}}
\newcommand{\bbm}{\begin{bmatrix}}
\newcommand{\ebm}{\end{bmatrix}}

\newenvironment{amatrix}[1]{%
  \left[\begin{array}{@{}*{#1}{c}|c@{}}
}{%
  \end{array}\right]
}
\newenvironment{aamatrix}[1]{%
  \left[\begin{array}{@{}*{#1}{c}|*{#1}{c}@{}}
}{%
  \end{array}\right]
}
\DeclareMathOperator{\comp}{comp}
\newcommand{\bam}{\begin{amatrix}}
\newcommand{\eam}{\end{amatrix}}

\begin{document}

\author{Instructor: Sean Fitzpatrick}
\thispagestyle{plain}
\begin{center}
\emph{University of Lethbridge}\\
Department of Mathematics and Computer Science\\
31\textsuperscript{st} October, 2016, 9:00 - 9:50 am\\
{\bf MATH 1410A - Test \#2}\\
\end{center}
\skipline \skipline \skipline \noindent \skipline
Last Name:\underline{\hspace{50pt}{\bf Solutions}\hspace{248pt}}\\
\skipline
First Name:\underline{\hspace{50pt}{\bf The}\hspace{275pt}}\\
\skipline
Student Number:\underline{\hspace{323pt}}\\
\skipline
Tutorial Section: \underline{\hspace{320pt}}\\


\vspace{0.5in}


\begin{quote}
 {\bf Record your answers below each question in the space provided.    Left-hand pages may be used as scrap paper for rough work.  If you want any work on the left-hand pages to be graded, please indicate so on the right-hand page.
 
 \bigskip
 
Partial credit will be awarded for partially correct work, including intermediate steps. (You should show your work if you want to earn part marks.) Unless otherwise indicated, failure to justify your work may result in loss of marks, even for a correct answer. 

\bigskip

No external aids are allowed, with the exception of a 5-function calculator.}
\end{quote}


\vspace{0.5in}

For grader's use only:

\begin{table}[hbt]
\begin{center}
\begin{tabular}{|l|r|} \hline
Page&Grade\\
\hline \hline
\cline{1-2} 2 & \enspace\enspace\enspace\enspace\enspace\enspace/12\\
\cline{1-2} 3 & \enspace\enspace\enspace\enspace\enspace\enspace/12\\
\cline{1-2} 4 & \enspace\enspace\enspace\enspace\enspace\enspace/10\\
\cline{1-2} 5 & \enspace\enspace\enspace\enspace\enspace\enspace/8\\
\cline{1-2} 6 & \enspace\enspace\enspace\enspace\enspace\enspace/8\\
\cline{1-2} Total & \enspace\enspace\enspace\enspace\enspace\enspace/50\\
\hline
\end{tabular}

\skipline

\skipline



A
\end{center}
\end{table}
\newpage


\begin{enumerate}
\item Assume that $A$, $B$, and $X$ are matrices of the same size.
\begin{enumerate}
 \item Solve for $X$ in terms of $A$ and $B$, given that $2A-3X=B$.\points{3}

\bigskip

Adding $3X-B$ to both sides of the equation, we get $2A-B = 3X$. Multiplying both sides by $\frac{1}{3}$, we find that
\[
 X = \frac{2}{3}A-\frac{1}{3}B.
\]

\medskip

 \item Determine the entries of the matrix $X$ from part (a) if $A=\bbm 2&-3\\0&4\ebm$ and $B=\bbm 1&2\\-4&5\ebm$.\points{3}
\end{enumerate}

\bigskip

Plugging the given values into our expression from part (a), we find 
\[
 X = \frac{2}{3}\bbm 2&-3\\0&4\ebm-\frac{1}{3}\bbm 1&2\\-4&5\ebm = \bbm 4/3 & -2\\0&8/3\ebm+\bbm -1/3 & -2/3\\4/3 & -5/3\ebm = \bbm 1 & -8/3\\4/3 & 1\ebm.
\]

\medskip

\item Let $T:\R^3\to \R^2$ be a matrix transformation such that
\[
 T\left(\bbm 1\\0\\0\ebm\right) = \bbm 2\\-3\ebm,\quad T\left(\bbm 0\\1\\0\ebm\right) = \bbm -1\\1\ebm, \quad \text{ and } \quad T\left(\bbm 0\\0\\1\ebm\right) = \bbm -5\\4\ebm.
\]
\begin{enumerate}
 \item Determine the matrix of $T$. (That is, find the matrix $A$ such that $T(\vec{x})=A\vec{x}$ for any vector $\vec{x}$ in $\R^3$.) \points{3}

\bigskip

Since the values of $T$ on the standard basis vectors determine the columns of $A$, in order, we have
\[
 A = \bbm 2&-1&-5\\-3&1&4\ebm.
\]

\medskip

 \item Compute $T\left(\bbm 2\\-1\\3\ebm\right)$. \points{3}

\bigskip

Using the matrix from part (a), we have
\[
 T\left(\bbm 2\\-1\\3\ebm\right) = \bbm 2&-1&-5\\-3&1&4\ebm\bbm 2\\-1\\3\ebm = \bbm -10\\5\ebm.
\]

\end{enumerate}

\newpage

\item Determine vectors $\vec{u}$ and $\vec{v}$ such that $U=\operatorname{span}\{\vec{u},\vec{v}\}$, where $U$ is the subspace \points{4}
\[
 U = \left\{\left.\bbm 3a-2b\\b-3a\\5a+4b\ebm \,\right|\, a,b \text{ are real numbers}\right\}.
\]
(Recall that $\operatorname{span}\{\vec{u},\vec{v}\} = \{a\vec{u}+b\vec{v}\,|\, a,b\in\R\}$.)

\bigskip

If $\vec{w}$ is any vector in $U$, then for some real numbers $a$ and $b$ we have
\[
 \vec{w} = \bbm 3a-2b\\b-3a\\5a+4b\ebm = \bbm 3a\\-3a\\5a\ebm + \bbm -2b\\b\\4b\ebm = a\bbm 3\\-3\\5\ebm + b\bbm -2\\1\\4\ebm.
\]
Thus, $\vec{w} = a\vec{u}+b\vec{v}$, where $\vec{u} =\bbm 3\\-3\\5\ebm$ and $\vec{v} = \bbm -2\\1\\4\ebm$, so $U=\operatorname{span}\left\{\bbm 3\\-3\\5\ebm,\bbm -2\\1\\4\ebm\right\}$.

\bigskip

\item Verify that $x=-\dfrac{7}{2}, y=2, z=\dfrac{1}{2}$ is a solution to the system \points{4} \hspace{12pt} $\arraycolsep=1pt
 \begin{array}{ccccccc}
  2x&+&y& & &=&-5\\x& & &+&3z&=&-2\\ & &-y&+&6z&=&1
 \end{array}
$

\bigskip

Plugging in the given values, we have

\begin{align*}
 2(-7/2)+2 = -7+2 &= -5, \tag*{so the first equation works}\\
 -7/2 + 3(1/2) = -7/2+3/2 = -4/2&=-2, \tag*{so the second equation works}\\
 -2+6(1/2) = -2+3 &= 1, \tag*{so the third equation works}
\end{align*}

\medskip

\item Let $T:\R^2\to\R^3$ be a linear transformation such that $T(\vec{u}) = \bbm 1\\-2\\4\ebm$ and $T(\vec{v}) = \bbm 0\\6\\-3\ebm$ for some vectors $\vec{u}, \vec{v}$ in $\R^2$. What is the value of \points{4} $T(5\vec{u}-3\vec{v})$?

\bigskip

Using the properties of lienar transformations, we have
\[
 T(5\vec{u}-3\vec{v}) = 5T(\vec{u})-3T(\vec{v}) = 5\bbm 1\\-2\\4\ebm-3\bbm 0\\6\\-3\ebm =\bbm 5\\-28\\29\ebm.
\]


\newpage

\item Each of the matrices below is in row-echelon form, and represents a system of linear equations in the variables $x$, $y$, and $z$. If the system has no solution, explain why. If it does, determine the solution using either back substitution, or by finding the reduced row-echelon form of the matrix.\points{10}
\begin{enumerate}
 \item $\begin{amatrix}{3}1&-2&1&4\\0&1&-1&2\\0&0&1&3\end{amatrix}$ 

\bigskip

Using back substitution, we have $z=3$ from Row 3; Row 2 gives us $y-z=2$, so $y=2+3=5$, and from Row 1, we have $x-2y+z=4$, so $x=4+2(5)-3 = 11$.

\medskip

Using row operations, we find

\hglue-48pt\begin{minipage}{\textwidth}
\[
 \begin{amatrix}{3}1&-2&1&4\\0&1&-1&2\\0&0&1&3\end{amatrix}\xrightarrow{\scriptsize{R_1-R_3\to R_1}} \begin{amatrix}{3}1&-2&0&1\\0&1&-1&2\\0&0&1&3\end{amatrix}
\xrightarrow{\scriptsize{R_2+R_3\to R_2}}\begin{amatrix}{3}1&-2&0&1\\0&1&0&5\\0&0&1&3\end{amatrix}\xrightarrow{\scriptsize{R_1+2R_2\to R_1}}
\begin{amatrix}{3}1&0&0&11\\0&1&0&5\\0&0&1&3\end{amatrix},                                                                        
\]
\end{minipage}

\medskip

from which we can read off the solution $x=11, y=5, z=3$ as before.

\bigskip

 \item $\begin{amatrix}{3}1&3&0&2\\0&1&-3&5\\0&0&0&0\end{amatrix}$ 

\bigskip

Here, the third column contains no leading 1, so $z=t$ is a free parameter. The equation $y-3z=5$ from  Row 2 then gives us $y=5+3t$. From Row 1 we have the equation $x+3y=2$. Solving for $x$ and plugging in $y=5+3t$, we have $x=2-3y=2-3(5+3t) = 2-15-9t = -13-9t$. 

\medskip

Using row operations instead, we have
\[
 \begin{amatrix}{3}1&3&0&2\\0&1&-3&5\\0&0&0&0\end{amatrix}\xrightarrow{\scriptsize{R_1-3R_2\to R_1}}\begin{amatrix}{3}1&0&9&-13\\0&1&0&5\\0&0&0&0\end{amatrix}.
\]
We can then read off the solution $x=-13-9t, y=5+3t, z=t$, where $t$ is a free parameter, as before.

\medskip



 \item $\begin{amatrix}{3}1&5&-4&2\\0&0&1&-3\\0&0&0&1\end{amatrix}$ 

\bigskip

The last row of the augmented matrix corresponds to the equation $0x+0y+0z=1$. Since $0x+0y+0z=0\neq 1$ for all possible values of $x$, $y$, and $z$, there is no solution to this system.
\end{enumerate}
 \newpage

\item Compute the matrix products $AB$ and $BA$, where $A = \begin{bmatrix}3&1&-2\\-4&1&3\end{bmatrix}$ and $B = \begin{bmatrix}-1&0\\3&2\\5&-3\end{bmatrix}$.\points{8}

\bigskip

We have

\[
 AB = \begin{bmatrix}3&1&-2\\-4&1&3\end{bmatrix}\begin{bmatrix}-1&0\\3&2\\5&-3\end{bmatrix} = \begin{bmatrix}-3+3-10&0+2+6\\4+3+15&0+2-9\end{bmatrix} = \begin{bmatrix}-10&8\\22&-7\end{bmatrix}
\]
and
\[
 BA = \begin{bmatrix}-1&0\\3&2\\5&-3\end{bmatrix}\begin{bmatrix}3&1&-2\\-4&1&3\end{bmatrix} = \begin{bmatrix}-3+0&-1+0&2+0\\9-8&3+2&-6+6\\15+12&5-3&-10-9\end{bmatrix} = \begin{bmatrix}-3&-1&2\\1&5&0\\27&2&-19\end{bmatrix}
\]


\pagebreak

\item Solve the system \hspace{12pt} $\arraycolsep=2pt \begin{array}{cccccccc}&3x&-&4y&-&5z&=&2\\
 &x&-&2y&-&z&=&4\\
-&2x&+&2y&+&4z&=&2\end{array}$ \points{8}

\bigskip

We set up the corresponding augmented matrix and reduce, as follows:

\begin{align*}
 \bam{3} 3&-4&-5&2\\1&-2&-1&4\\-2&2&4&2\eam \xrightarrow{R_1\leftrightarrow R_2} &\bam{3}1&-2&-1&4\\3&-4&-5&2\\-2&2&4&2\eam\\
\xrightarrow{R_2-2R_1\to R_2}&\bam{3}1&-2&-1&4\\0&2&-2&-10\\-2&2&4&2\eam\\
\xrightarrow{R_3+2R_1\to R_3}&\bam{3}1&-2&-1&4\\0&2&-2&-10\\0&-2&2&10\eam\\
\xrightarrow{R_3-R_2\to R_3}&\bam{3}1&-2&-1&4\\0&2&-2&-10\\0&0&0&0\eam\\
\xrightarrow{\frac{1}{2}R_2\to R_2}&\bam{3}1&-2&-1&4\\0&1&-1&-5\\0&0&0&0\eam\\
\xrightarrow{R_1+2R_2\to R_1}&\bam{3}1&0&-3&-6\\0&1&-1&-5\\0&0&0&0\eam
\end{align*}
This last matrix is in reduced row-echelon form. We see that there are leading ones in the first and second columns, so $x$ and $y$ are basic variables, while $z$ is a free variable, since there is no leading one in its column. Setting $z=t$, where $t$ can be any real number, we have the solution
\begin{align*}
 x&=-6+3t\\y&=-5+t\\z&=t.
\end{align*}

\end{enumerate}
\end{document}