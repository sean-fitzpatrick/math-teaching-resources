\documentclass[12pt]{article}
\usepackage{amsmath}
\usepackage{amssymb}
\usepackage[letterpaper,margin=0.85in,centering]{geometry}
\usepackage{fancyhdr}
\usepackage{enumerate}
\usepackage{lastpage}
\usepackage{multicol}
\usepackage{graphicx}

\reversemarginpar

\pagestyle{fancy}
\cfoot{Page \thepage \ of \pageref{LastPage}}\rfoot{{\bf Total Points: 50}}
\chead{MATH 1410A}\lhead{Test \# 2}\rhead{Monday, 31\textsuperscript{st} October, 2016}

\newcommand{\points}[1]{\marginpar{\hspace{24pt}[#1]}}
\newcommand{\skipline}{\vspace{12pt}}
%\renewcommand{\headrulewidth}{0in}
\headheight 30pt

\newcommand{\di}{\displaystyle}
\newcommand{\R}{\mathbb{R}}
\newcommand{\aaa}{\mathbf{a}}
\newcommand{\bbb}{\mathbf{b}}
\newcommand{\ccc}{\mathbf{c}}
\newcommand{\dotp}{\boldsymbol{\cdot}}
\newcommand{\abs}[1]{\lvert #1\rvert}
\newcommand{\len}[1]{\lVert #1\rVert}
\newcommand{\ivec}{\,\boldsymbol{\hat{\imath}}}
\newcommand{\jvec}{\,\boldsymbol{\hat{\jmath}}}
\newcommand{\kvec}{\,\boldsymbol{\hat{k}}}
\newcommand{\bvm}{\begin{vmatrix}}
\newcommand{\evm}{\end{vmatrix}}
\newcommand{\bbm}{\begin{bmatrix}}
\newcommand{\ebm}{\end{bmatrix}}

\newenvironment{amatrix}[1]{%
  \left[\begin{array}{@{}*{#1}{c}|c@{}}
}{%
  \end{array}\right]
}
\newenvironment{aamatrix}[1]{%
  \left[\begin{array}{@{}*{#1}{c}|*{#1}{c}@{}}
}{%
  \end{array}\right]
}
\DeclareMathOperator{\comp}{comp}

\begin{document}

\author{Instructor: Sean Fitzpatrick}
\thispagestyle{plain}
\begin{center}
\emph{University of Lethbridge}\\
Department of Mathematics and Computer Science\\
31\textsuperscript{st} October, 2016, 9:00 - 9:50 am\\
{\bf MATH 1410A - Test \#2}\\
\end{center}
\skipline \skipline \skipline \noindent \skipline
Last Name:\underline{\hspace{353pt}}\\
\skipline
First Name:\underline{\hspace{350pt}}\\
\skipline
Student Number:\underline{\hspace{323pt}}\\
\skipline
Tutorial Section: \underline{\hspace{320pt}}\\


\vspace{0.5in}


\begin{quote}
 {\bf Record your answers below each question in the space provided.    Left-hand pages may be used as scrap paper for rough work.  If you want any work on the left-hand pages to be graded, please indicate so on the right-hand page.
 
 \bigskip
 
Partial credit will be awarded for partially correct work, including intermediate steps. (You should show your work if you want to earn part marks.) Unless otherwise indicated, failure to justify your work may result in loss of marks, even for a correct answer. 

\bigskip

No external aids are allowed, with the exception of a 5-function calculator.}
\end{quote}


\vspace{0.5in}

For grader's use only:

\begin{table}[hbt]
\begin{center}
\begin{tabular}{|l|r|} \hline
Page&Grade\\
\hline \hline
\cline{1-2} 2 & \enspace\enspace\enspace\enspace\enspace\enspace/12\\
\cline{1-2} 3 & \enspace\enspace\enspace\enspace\enspace\enspace/12\\
\cline{1-2} 4 & \enspace\enspace\enspace\enspace\enspace\enspace/10\\
\cline{1-2} 5 & \enspace\enspace\enspace\enspace\enspace\enspace/8\\
\cline{1-2} 6 & \enspace\enspace\enspace\enspace\enspace\enspace/8\\
\cline{1-2} Total & \enspace\enspace\enspace\enspace\enspace\enspace/50\\
\hline
\end{tabular}

\skipline

\skipline



B
\end{center}
\end{table}
\newpage


\begin{enumerate}
\item Assume that $A$, $B$, and $X$ are matrices of the same size.
\begin{enumerate}
 \item Solve for $X$ in terms of $A$ and $B$, given that $4A+3X=B$.\points{3}

\vspace{1in}

 \item Determine the entries of the matrix $X$ from part (a) if $A=\bbm -3&2\\1&0\ebm$ and $B=\bbm 3&1\\-3&2\ebm$.\points{3}
\end{enumerate}

\vspace{2in}

\item Let $T:\R^3\to \R^2$ be a matrix transformation such that
\[
 T\left(\bbm 1\\0\\0\ebm\right) = \bbm 3\\-2\ebm,\quad T\left(\bbm 0\\1\\0\ebm\right) = \bbm 1\\-1\ebm, \quad \text{ and } \quad T\left(\bbm 0\\0\\1\ebm\right) = \bbm 4\\5\ebm.
\]
\begin{enumerate}
 \item Determine the matrix of $T$. (That is, find the matrix $A$ such that $T(\vec{x})=A\vec{x}$ for any vector $\vec{x}$ in $\R^3$.) \points{3}

\vspace{1.25in}

 \item Compute $T\left(\bbm 3\\1\\-2\ebm\right)$. \points{3}
\end{enumerate}

\newpage

\item Determine vectors $\vec{u}$ and $\vec{v}$ such that $U=\operatorname{span}\{\vec{u},\vec{v}\}$, where $U$ is the subspace \points{4}
\[
 U = \left\{\left.\bbm 2a-5b\\b-2a\\3a+6b\ebm \,\right|\, a,b \text{ are real numbers}\right\}.
\]
(Recall that $\operatorname{span}\{\vec{u},\vec{v}\} = \{a\vec{u}+b\vec{v}\,|\, a,b\in\R\}$.)

\vspace{2in}

\item Verify that $x=\dfrac{5}{2}, y=-10, z=-\dfrac{3}{2}$ is a solution to the system \points{4} \hspace{12pt} $\arraycolsep=1pt
 \begin{array}{ccccccc}
  2x&+&y& & &=&-5\\x& & &+&3z&=&-2\\ & &-y&+&6z&=&1
 \end{array}
$


\vspace{2in}

\item Let $T:\R^2\to\R^3$ be a linear transformation such that $T(\vec{u}) = \bbm 2\\-1\\3\ebm$ and $T(\vec{v}) = \bbm 4\\0\\-2\ebm$ for some vectors $\vec{u}, \vec{v}$ in $\R^2$. What is the value of \points{4} $T(3\vec{u}-5\vec{v})$?

\newpage

\item Each of the matrices below is in row-echelon form, and represents a system of linear equations in the variables $x$, $y$, and $z$. If the system has no solution, explain why. If it does, determine the solution using either back substitution, or by finding the reduced row-echelon form of the matrix.\points{10}
\begin{enumerate}
 \item $\begin{amatrix}{3}1&2&-2&5\\0&1&3&-4\\0&0&1&2\end{amatrix}$ 

\vspace{2.25in}

 \item $\begin{amatrix}{3}1&-2&1&5\\0&1&-2&4\\0&0&0&1\end{amatrix}$ 

\vspace{2.25in}

 \item $\begin{amatrix}{3}1&-2&5&-3\\0&1&3&5\\0&0&0&0\end{amatrix}$ 
\end{enumerate}
 \newpage

\item Compute the matrix products $AB$ and $BA$, where $A = \begin{bmatrix}-1&3\\2&4\\-5&2\end{bmatrix}$ and $B = \begin{bmatrix}3&0&-2\\-1&4&2\end{bmatrix}$.\points{8}

\pagebreak

\item Solve the system \hspace{12pt} $\arraycolsep=2pt \begin{array}{cccccccc}&2x&-&3y&+&z&=&-5\\
 &x&-&3y&+&2z&=&2\\
-&4x&+&9y&-&5z&=&1\end{array}$ \points{8}
\end{enumerate}
\end{document}