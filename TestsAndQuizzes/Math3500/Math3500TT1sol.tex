\documentclass[12pt]{article}
\usepackage{amsmath}
\usepackage{amssymb}
\usepackage[letterpaper,margin=0.85in,centering]{geometry}
\usepackage{fancyhdr}
\usepackage{enumerate}
\usepackage{lastpage}
\usepackage{multicol}
\usepackage{graphicx}

\reversemarginpar

\pagestyle{fancy}
\cfoot{Page \thepage \ of \pageref{LastPage}}\rfoot{{\bf Total Points: 30}}
\chead{MATH 3500}\lhead{Test \# 1}\rhead{Monday, 6\textsuperscript{th} October, 2014}

\newcommand{\points}[1]{\marginpar{\hspace{24pt}[#1]}}
\newcommand{\skipline}{\vspace{12pt}}
%\renewcommand{\headrulewidth}{0in}
\headheight 30pt

\newcommand{\di}{\displaystyle}
\newcommand{\R}{\mathbb{R}}
\newcommand{\Q}{\mathbb{Q}}
\newcommand{\N}{\mathbb{N}}
\newcommand{\abs}[1]{\lvert #1\rvert}

\begin{document}

\author{Instructor: Sean Fitzpatrick}
\thispagestyle{plain}
\begin{center}
\emph{University of Lethbridge}\\
Department of Mathematics and Computer Science\\
6\textsuperscript{th} October 2014, 2:00-2:50 pm\\
{\bf MATH 3500 - Test \#1}\\
\end{center}
\skipline \skipline \skipline \noindent \skipline
Last Name:\underline{\hspace{50pt}{\bf SOLUTIONS}\hspace{50pt}}\\
\skipline
First Name:\underline{\hspace{50pt}{\bf THE}\hspace{100pt}}\\
\skipline
Student Number:\underline{\hspace{322pt}}\\
\skipline

\vspace{0.5in}


\begin{quote}
{\bf For full credit you must answer three of the four problems on this test. If you include work for all four problems, make sure that you clearly indicate which three problems you wish to have graded. If you solve all four problems and do not indicate which problems should be graded, only the {\bf first three} problems will be graded.}

\bigskip
 
 {Record your answers below each question in the space provided.    Left-hand pages may be used as scrap paper for rough work.  If you want any work on the left-hand pages to be graded, please indicate so on the right-hand page.
 
 \bigskip
 
Partial credit will be awarded for partially correct work, so be sure to show your work, and include all necessary justifications needed to support your arguments. }

\end{quote}


\vspace{0.5in}

For grader's use only:

\begin{table}[hbt]
\begin{center}
\begin{tabular}{|l|r|} \hline
Problem&Grade\\
\hline \hline
\cline{1-2} 1 & \enspace\enspace\enspace\enspace\enspace\enspace/10\\
\cline{1-2} 2 & \enspace\enspace\enspace\enspace\enspace\enspace/10\\
\cline{1-2} 3 & \enspace\enspace\enspace\enspace\enspace\enspace/10\\
\cline{1-2} 4 & \enspace\enspace\enspace\enspace\enspace\enspace/10\\
\cline{1-2} Total & \enspace\enspace\enspace\enspace\enspace\enspace/30\\
\hline
\end{tabular}

\skipline

\skipline

\skipline

\end{center}
\end{table}
\newpage


\begin{enumerate}
\item \begin{enumerate}
       \item Find the supremum and infimum of each of the following sets, or explain why they do not exist. 
\begin{enumerate}
\item $A = \{x\in\R : \abs{x-2}<1\}$ \points{2}

\bigskip

{\bf Solution}: We note that
\[
 \abs{x-2}<1 \Leftrightarrow -1<x-2<1 \Leftrightarrow 1<x<3,
\]
so $A=(1,3)$ and thus $\inf A = 1$ and $\sup A = 3$.

\bigskip

\item $B = \left\{(-1)^n\left(1+\frac{1}{n}\right) : n\in\N\right\}$ \points{2}


\bigskip

{\bf Solution}: We write $B$ as a union of two sets corresponding to whether $n$ is even or odd. We have:
\[
 B = \{-2, -4/3, -6/5, -8/7, \ldots\}\cup\{3/2, 5/4, 7/6, 9/8,\ldots\}.
\]
With $a_n = (-1)^n/(n+1)$, we see that for $n$ odd, $-2\leq a_n<-1$, and for $n$ even, $1<a_n\leq 3/2$. Thus, we have that $\inf B = \min B = -2$ and $\sup B = \max B = 3/2$.

\bigskip

\item $C = \{x\in \R : x^2<0\}$ \points{2}


\bigskip

{\bf Solution}: Since $x^2\geq 0$ for all $x\in \R$, the set $C$ is the empty set. Thus, $\sup C$ and $\inf C$ do not exist.

\end{enumerate}
\newpage
\item Let $x\in\R$ and set $A = \{q\in\Q : q<x\}$. Prove that $x=\sup A$. \points{4}
      \end{enumerate}

\bigskip

{\bf Solution}: By definition, $x>q$ for all $q\in A$, so $x$ is an upper bound for $A$. To see that $x$ must be the least upper bound, we note that for any $y\in\R$ with $y<x$, there must exist some $p\in\Q$ such that $y<p<x$, since $\Q$ is dense in $\R$. Since $p\in Q$ and $p<x$, we have $p\in A$, and thus, $y$ cannot be an upper bound for $A$.

\newpage
\item Let $B = \left\{\dfrac{(-1)^nn}{n+1} : n\in\N\right\}$.
\begin{enumerate}
       \item What are the limit points of $B$? (You should explain your answer but a formal proof is not required.)\points{3}


\bigskip

{\bf Solution}: We note that $B$ can be written as $B = B_1\cup B_2$, where
\[
 B_1= \left\{-\frac{1}{2}, -\frac{3}{4}, -\frac{5}{6}, -\frac{7}{8},\ldots\right\} \text{ and } B_2= \left\{\frac{2}{3}, \frac{4}{5}, \frac{6}{7}, \frac{8}{9},\ldots\right\}.
\]
We see that $-1$ is a limit point for $B_1$, and that $1$ is a limit point for $B_2$, since we can make $n/(n+1)$ arbitrarily close to 1 by taking $n$ large enough, so every neighbourhood of 1 and $-1$ contains some point of $B$.

\bigskip

      \item Is $B$ open, closed, both, or neither? Explain.\points{2}

\bigskip

{\bf Solution}: Since a set is closed if and only if it contains its limit points, we see that $B$ is not closed, since 1 and $-1$ are limit points, but they are not elements of $B$. 

But $B$ is also not open, since none of the points of $B$ are interior points. Indeed, we note that every element of $B$ is rational, but every interval contains irrational numbers. Thus, no neighbourhood of the form $(b_n-\epsilon,b_n+\epsilon)$, where $b_n = (-1)^nn/(n+1)$ can be contained in $B$.

Thus, $B$ is neither open nor closed.

\newpage

      \item Which points of $B$ are isolated points? Why? \points{2}


\bigskip

{\bf Solution}: Every point of $B$ is isolated. Recall that an element $b\in B$ is isolated if it is not a limit point. Given $b_n = (-1)^nn/(n+1)\in B$, the next element of $B$ closest to $b_n$ is $b_{n+2}$, but
\[
 \abs{b_n-b_{n+2}} = \left|\frac{n}{n+1}-\frac{n+2}{n+3}\right| = \frac{2}{(n+1)(n+3)},
\]
so if we take $\epsilon = \dfrac{1}{(n+1)(n+3)}$ then the only element of $b$ in $(b_n-\epsilon,b_n+\epsilon)$ is $b_n$ itself, so $b_n$ cannot be a limit point.

\bigskip

      \item What are the closure, interior, and boundary of$B$? \points{3}


\bigskip

{\bf Solution}: The closure of $B$ is the union of $B$ and its limit points; thus,\\ $\overline{B} = B\cup\{-1,1\}$.

The interior $B$ is empty. As noted in part (b), no $b\in B$ can have a neighbourhood $(b-\epsilon,b+\epsilon)$ contained in $B$, so no point of $B$ can be an interior point. (You could also note that no isolated point can be an interior point, and every point is isolated, by part (c).)

Finally, the boundary of $B$ is actually equal to the closure: $\partial B = \overline{B}$. The limit points $\pm 1$ are boundary points, since any neighbourhood of $1$ contains $1\in B^c$, and must also contain an element of $B$, since 1 is a limit point, and a similar argument applies to -1. Also, any isolated point must be a boundary point, since any neighbourhood of an isolated point $b\in B$ contains  $b$ itself, which is an element of $b$, as well as points that are not in $B$.

\end{enumerate}
\newpage
\item \begin{enumerate}
       \item Define what it means for a point $x\in\R$ to be a {\em limit point} of a set $A\subseteq \R$. (A limit point is also known as an accumulation point.) \points{3}


\bigskip

{\bf Solution}: A point $x\in \R$ is a limit point of $A\subseteq \R$ if for any $\epsilon>0$, the $\epsilon$-neighbourhood $N_\epsilon(x) = (x-\epsilon, x+\epsilon)$ contains some point $a\in A$ with $a\neq x$.

\bigskip

      \item Prove that if a set $K\subseteq \R$ is compact, than any subset $E\subseteq K$ with infinitely many elements has a limit point that belongs to $K$. \points{7}


\bigskip

{\bf Solution}: Suppose that $K$ is compact, and that $E\subseteq K$ is infinite. Since $K$ is compact, it is closed and bounded, by the Heine-Borel theorem. Since $E$ is a subset of a bounded set, $E$ must itself be bounded, and since $E$ is infinite, it must have a limit point $x\in\R$, by the Bolzano-Weierstrass theorem.\\
({\bf Note}: it need not be the case that $x\in E$.)

Since $E\subseteq K$, $x$ must also be a limit point of $K$. (Every neighbourhood of $x$ contains some $y\in E$, $y\neq x$, but since $E\subseteq K$, $y\in K$ as well.) Since $K$ is closed, it contains its limit points. Thus, $x\in K$, which is what we needed to show.
      \end{enumerate}
\newpage

\item \begin{enumerate}
       \item Prove that $\di \lim_{n\to\infty}\frac{n}{n+1} = 1$ using the definition of convergence. \points{4}

\bigskip

{\bf Solution}: Let $\epsilon>0$ be given. Since $\R$ is Archimedean, there exists $N\in\N$ such that $1/N<\epsilon$. Thus, for all $n\geq N$ we have that
\[
 \left|\frac{n}{n+1}-1\right| = \left|\frac{-1}{n+1}\right| = \frac{1}{n+1}<\frac{1}{n}\leq \frac{1}{N}<\epsilon.
\]
Thus, $\lim\dfrac{n}{n+1} = 1$ by the definition of convergence.


\newpage

      \item Let $(a_n)$ be the sequence defined by $a_1=1$ and $a_{n+1} = \sqrt{2a_n+2}$ for $n\geq 1$. Using induction, one can prove that for all $n\in\N$ we have $a_n\leq a_{n+1} <3$. Explain why we can conclude that $(a_n)$ converges, and find the limit of the sequence. \points{3}


\bigskip

{\bf Solution}: We are given that $a_n\leq a_{n+1}$ and $a_n<3$ for all $n\in \N$, so the sequence $(a_n)$ is increasing, and bounded above. Thus, by the Monotone Convergence Theorem, $a=\lim a_n$ exists.

Using the limit laws, we have
\[
 a^2 = \lim a_{n+1}^2 = \lim (2a_n+2) = 2a+2.
\]
Thus $a^2=2a+2$, or $a^2-2a-2=0$. It follows from the quadratic formula that $a = 1\pm \sqrt{3}$. Since $a_n\geq 0$ for all $n\in\N$, we must have $a\geq 0$, and since $1-\sqrt{3}$ is negative, it follows that $a=1+\sqrt{3}$.

\bigskip

      \item Prove that if $(a_n)$ and $(b_n)$ are Cauchy sequences, then so is $c_n=\abs{a_n-b_n}$. \points{3}

Hint: triangle inequality.


\bigskip

{\bf Solution}: Let $\epsilon>0$ be given. Since $(a_n)$ is Cauchy, there exists some $N_1\in\N$ such that for all $m,n\geq N_1$, we have $\abs{a_n-a_m}<\epsilon/2$. Similarly, there exists some $N_2\in\N$ such that for all $m,n\geq N_2$, $\abs{b_n-b_m}<\epsilon/2$.

Let $N=\max\{N_1,N_2\}$, and suppose that $n,m\geq N$. Then we have that
\begin{align*}
 \abs{c_n-c_m} & = \abs{\abs{a_n-a_m}-\abs{b_n-b_m}}\\
&\leq \abs{(a_n-a_m)-(b_n-b_m)}\\
& = \abs{(a_n-a_m)+(b_m-b_n)}\\
&\leq \abs{a_n-a_m}+\abs{b_n-b_m} <\epsilon/2+\epsilon/2 = \epsilon.
\end{align*}

      \end{enumerate}



\end{enumerate}

\end{document}